%%-*-latex-*-

% ------------------------------------------------------------------------
% 
\begin{frame}
\frametitle{Deterministic finite automata}
 
An \textbf{automaton} is a very useful and pervasive concept in
software and telecommunication engineering.

\bigskip

Let us present first the simplest of the automata, called the
\textbf{deterministic finite automaton}, or \textbf{DFA}. The most
intuitive presentation of automata is by means of a graphic:
\begin{center}
\includegraphics[bb=48 694 253 715]{trie_hello}
\end{center}
One component is a circle or a double-circle, called \textbf{node},
with a name inside, called \textbf{state}. The other component is an
arrow which connect two circles with a letter: these are
\textbf{edges} and the letters are \textbf{labels}. The arrow which
has no originating circle is a marker to distinguish a state, just as
the double-circle denotes another special state: the former is an
\textbf{initial state} and the latter is a \textbf{final state}.

\end{frame}

% ------------------------------------------------------------------------
% 
\begin{frame}
\frametitle{DFA (cont)}
 
\label{trie_then}

It is possible to have several edges between the same pair of states
but they must carry different labels. There may be several final
states.

\bigskip

\begin{columns}

  \column{0.5\textwidth} For instance, consider
  \begin{center}
    \includegraphics[bb=48 644 218 730,scale=0.93]{trie_then}
  \end{center}

  \column{0.5\textwidth} Note that here
  \begin{enumerate}

    \item the \textbf{alphabet} of labels is the English alphabet,

    \item there are nodes without out-going edges carrying all possible
    letters: this DFA is not \textbf{complete}.

  \end{enumerate}

\end{columns}

\end{frame}

% ------------------------------------------------------------------------
% 
\begin{frame}
\frametitle{DFA (cont)}

The graphic representation of an automaton is called a
\textbf{transition diagram}.

\bigskip

Here is another transition diagram over
the alphabet \(\{0,1\}\), called binary alphabet.
\begin{center}
\includegraphics[bb=48 685 185 756]{dfa_01_first}
\end{center} 
Note that there are here some edges that connect a node to
itself: these edges are called \textbf{loops}.

\end{frame}

% ------------------------------------------------------------------------
% 
\begin{frame}
\frametitle{DFA (cont)}

\label{dfa_01}

It is possible to simplify the transition diagram by merging edges
which have the same pair of nodes and listing the labels, separated by
commas:
\begin{center}
\includegraphics[bb=48 710 185 760]{dfa_01}
\end{center} 

\end{frame}

% ------------------------------------------------------------------------
% 
\begin{frame}
\frametitle{DFA/Trie}

Before giving a formal definition of DFAs, we can explain on
transition diagrams how they are used on an example.

\bigskip

Let a short list of English words be composed of \textsf{then},
\textsf{they}, \textsf{thus} and \textsf{this}. Imagine we are given a
word and asked to check if this word belongs to our list. 

\bigskip

We can of course, compare the given word to each word in the list, one
after the other, until a match is found or the end of the list is
reached. But this is not efficient.

\end{frame}

% ------------------------------------------------------------------------
% 
\begin{frame}
\frametitle{DFA/Trie (cont)}

A better approach is to sort the words in the list in alphabetic
order. This way, if we check the words in order and the current word
is ``greater'' than the given word, we know it is not in the list.

\bigskip

This is the way we search a word in a dictionary (well, not exactly,
but the analogy is close enough).

\bigskip

Our dictionary is: \textsf{this}, \textsf{then}, \textsf{they} and
\textsf{thus} (order matters). Let us search the word \textsf{they}.

\end{frame}

% ------------------------------------------------------------------------
% 
\begin{frame}
\frametitle{DFA/Trie (cont)}

First, \textsf{they} is compared to \textsf{this}. The failure happens
at the third letter (i.e., after three letter comparisons) and we must
try the next word, \textsf{then}. 

\bigskip

Second, \textsf{they} is compared to \textsf{then}. But then, we
compare in turn the first letters \textsf{t} and \textsf{h}, i.e., we
start again from the beginning of the searched word.

\bigskip

There is a way to avoid this inefficiency by considering a model of
the dictionary that is not one-dimensional, like a list, but
two-dimensional, like a DFA --- called in this context a
\textbf{trie}.

\end{frame}

% ------------------------------------------------------------------------
% 
\begin{frame}
\frametitle{DFA/Trie (cont)}

The trie corresponding to our example dictionary is exactly the
transition diagram given page~\pageref{trie_then}. Let us assume here
that the word is followed by a special marker \textsf{\$}.

\bigskip

\begin{columns}
  \column{0.5\textwidth}
    \begin{center}
      \includegraphics[bb=48 644 218 730,scale=0.93]{trie_then}
    \end{center}

  \column{0.5\textwidth} At the beginning of the search, the current
  state is the initial state, \(q_0\), and the current letter in the
  searched word is \textsf{t}.

  \bigskip

  Then if there is an out-going edge from the current state that
  matches the current letter, we change the current state to the state
  pointed to by this edge and the current letter becomes the following
  one in the word or is \textsf{\$}.
\end{columns}

\end{frame}

% ------------------------------------------------------------------------
% 
\begin{frame}
\frametitle{DFA/Trie (cont)}

If the current letter is \textsf{\$}, i.e., all letters of the word
have been successfully matched, and the current state is a final state
(double-circled node), then the words belongs to the dictionary (in
other words, the trie accepts the word or recognises it).

\bigskip

Otherwise, the word is not in the dictionary.

\bigskip

In the worst case, a word of length \(n\) is in the dictionary and it
is found in exactly \(n\) letter comparisons.

\end{frame}

% ------------------------------------------------------------------------
% 
\begin{frame}
\frametitle{DFA/Formal definition}
 
Formally, a DFA \({\cal D}\) is a 5-tuple \({\cal D} = (Q, \Sigma,
\delta, q_0, F)\) where
\begin{enumerate}

  \item a finite set of \emph{states}, often noted \(Q\);

  \item an \emph{initial state} \(q_0 \in Q\);

  \item a set of \emph{final (\emph{or} accepting)
  states} \(F \subseteq Q\);

  \item a finite set of \emph{input symbols}, called \emph{alphabet},
    often noted \(\Sigma\);

  \item a \emph{transition function} \(\delta\) that takes 
  a state and an input symbol and returns a state: if \(q\) is a state
  with an edge labeled \(a\), the edge leads to state \(\delta(q,
  a)\).

\end{enumerate}

\end{frame}

% ------------------------------------------------------------------------
% 
\begin{frame}
\frametitle{DFA/Recognised words}

Independently of the interpretation of the states, we can define how a
given word is accepted (or recognised) or rejected by a given DFA.

\bigskip

The word \(a_1 a_2 \cdots a_n\), where \(a_i \in \Sigma\), is
recognised by the DFA \({\cal D} = (Q, \Sigma, \delta, q_0, F)\) if
\begin{itemize}

  \item for all \(0 \leqslant i \leqslant n-1\)

  \item there is a sequence of states \(q_i \in Q\) such that

  \item \(\delta (q_i, a_{i+1}) = q_{i+1}\) 

  \item and \(q_n \in F\).

\end{itemize}
The language recognised by \(\cal D\), noted \(L({\cal D})\) is the
set of words recognised by \(\cal D\).

\end{frame}

% ------------------------------------------------------------------------
% 
\begin{frame}
\frametitle{DFA/Recognised words/Example}

\begin{columns}

  \column{0.5\textwidth} For example, consider again the DFA
  page~\pageref{trie_then}:
  \begin{center}
    \includegraphics[bb=48 644 218 730,scale=0.93]{trie_then}
  \end{center}

  \column{0.5\textwidth} The word \textsf{then} is recognised because
  there is a sequence of states \((q_0, q_1, q_2, q_4, q_5)\)
  connected by edges which satisfies
  \begin{align*}
    \delta (q_0, \textsf{t}) &= q_1\\
    \delta (q_1, \textsf{h}) &= q_2\\
    \delta (q_2, \textsf{e}) &= q_4\\
    \delta (q_4, \textsf{n}) &= q_5
  \end{align*}
  and \(q_5 \in F\), i.e., \(q_5\) is a final state.

\end{columns}

\end{frame}

% ------------------------------------------------------------------------
% 
\begin{frame}
\frametitle{DFA/Recognised language}

It is easy to define formally \(L({\cal D})\).

\bigskip

Let \({\cal D} = (Q, \Sigma, \delta, q_0, F)\).

\bigskip

First, let us extend \(\delta\) to words and let us call this
extension \(\hat{\delta}\):
\begin{itemize}

  \item for all state \(q \in Q\), let \(\hat{\delta} (q,
  \varepsilon) = q\), where \(\varepsilon\) is the empty string;

  \item for all state \(q \in Q\), all word \(w \in \Sigma^{*}\), all
 input \(a \in \Sigma\), let \(\hat{\delta} (q, wa) = \delta
 (\hat{\delta}(q,w),a)\).

\end{itemize}
Then the word \(w\) is recognised by \(\cal D\) if \(\hat{\delta}(q_0,
w) \in F\). The language \(L({\cal D})\) recognised by \(\cal D\) is
defined as
\[
L({\cal D}) = \{w \in \Sigma^{*} \; \lvert \; \hat{\delta}(q_0, w) \in F\}
\]

\end{frame}

% ------------------------------------------------------------------------
% 
\begin{frame}
\frametitle{DFA/Recognised language (cont)}

For example, in our last example:
\[
\begin{aligned}
   \hat{\delta}(q_0, \epsilon)
&= q_0\\
   \hat{\delta}(q_0, \textsf{t})
&= \delta (\hat{\delta}(q_0, \epsilon), \textsf{t})
= \delta (q_0, \textsf{t}) 
= q_1\\
   \hat{\delta}(q_0, \textsf{th})
&= \delta (\hat{\delta}(q_0, \textsf{t}), \textsf{h})
= \delta (q_1, \textsf{h})
= q_2\\
   \hat{\delta}(q_0, \textsf{the})
&= \delta (\hat{\delta}(q_0, \textsf{th}), \textsf{e})
= \delta (q_2, \textsf{e})
= q_4\\
   \hat{\delta}(q_0, \textsf{then}) 
&= \delta (\hat{\delta}(q_0, \textsf{the}), \textsf{n})
= \delta (q_4, \textsf{n})
= q_5 \in F
\end{aligned}
\]

\end{frame}

% ------------------------------------------------------------------------
% 
\begin{frame}
\frametitle{DFA/Recognised language (cont)}

What is the language recognised by the previous DFA? The definition
states that this language is composed of all the words that are
recognised.

\bigskip

\begin{columns}

  \column{0.5\textwidth} Each recognised word corresponds to a
  \textbf{path}, i.e., a sequence of states connected by edges, from
  the initial state to a final state.

  \bigskip

  In our example, the language recognised is \(\{\textsf{then},
  \textsf{they}, \textsf{this}, \textsf{thus}\}\). 

  \column{0.5\textwidth} In other words:
  \begin{align*}
    \hat{\delta}(q_0, \textsf{then}) &\in F\\
    \hat{\delta}(q_0, \textsf{they}) &\in F\\
    \hat{\delta}(q_0, \textsf{this}) &\in F\\
    \hat{\delta}(q_0, \textsf{thus}) &\in F
  \end{align*}
  It is a finite language.

\end{columns}

\end{frame}

% ------------------------------------------------------------------------
% 
\begin{frame}
\frametitle{DFA/Recognised language (cont)}

What is the language recognised by the DFA page~\pageref{dfa_01}?
\begin{center}
\includegraphics[bb=48 710 185 760]{dfa_01}
\end{center} 
In English, this language can be defined as the set of strings of
\(0\) and \(1\) containing \(01\). 

\bigskip

Note that this an \emph{infinite} language.

\end{frame}

% ------------------------------------------------------------------------
% 
\begin{frame}
\frametitle{DFA/Transition diagrams}
 
We can also redefine transition diagrams in terms of the concept of
DFA.

\bigskip

A transition diagram for a DFA \({\cal D} = (Q, \Sigma, \delta, q_0,
F)\) is a graph defined as follows:
\begin{enumerate}

  \item for each state \(q\) in \(Q\) there is a \textbf{node}, i.e., a
    single circle with \(q\) inside;

  \item for each state \(q \in Q\) and each input symbol \(a \in
    \Sigma\), if \(\delta (q, a)\) exists, then there is an
    \textbf{edge}, i.e., an arrow, from the node denoting \(q\) to the
    node denoting \(\delta (q, a)\) labeled by \(a\); multiple edges
    can be merged into one and the labels are then separated by
    commas;

  \item there is an edge coming to the node denoting \(q_0\) without
    origin;

  \item nodes corresponding to final states (i.e., in \(F\)) are
    double-circled.

\end{enumerate}

\end{frame}

% ------------------------------------------------------------------------
% 
\begin{frame}
\frametitle{DFA/Transition table}

There is a compact textual way to represent the transition function of
a DFA: a \textbf{transition table}.

\bigskip

The rows of the table correspond to the states and the columns
correspond to the inputs (symbols). In other words, the entry for
the row corresponding to state \(q\) and the column corresponding to
input \(a\) is the state \(\delta (q, a)\):
\[
\begin{array}{c||c|c|c}
\delta & \ldots & a & \ldots\\
\hline\hline
\vdots & & &\\
\hline
q & & \delta (q, a)\\
\hline
\vdots & & &
\end{array}
\]

\end{frame}

% ------------------------------------------------------------------------
% 
\begin{frame}
\frametitle{DFA/Transition table/Example}

The transition table corresponding to the function \(\delta\) of our
last example is
\[
\begin{array}{r@{}l||c|c}
\multicolumn{2}{c||}{\cal D} & 0 & 1\\
\hline\hline
\rightarrow & q_0 & q_1 & q_0\\
            & q_1 & q_1 & q_2\\
         \# & q_2 & q_2 & q_2
\end{array}
\]
Actually, we added some extra information in the table: the initial
state is marked with \(\rightarrow\) and the final states are marked
with \(\#\).

\bigskip

Therefore, it is not only \(\delta\) which is defined by means of the
transition table here, but the whole DFA \(\cal D\).

\end{frame}

% ------------------------------------------------------------------------
% 
\begin{frame}
\frametitle{DFA/Example}

We want to define formally a DFA which recognises the language \(L\)
whose words contain an even number of 0's and an even number of 1's
(the alphabet is binary).

\bigskip

We should understand that the role of the states here is to
\textbf{not} to count the exact number of 0's and 1's that have been
recognised before but this number \textbf{modulo 2}.

\bigskip

Therefore, there are four states because there are four cases:
\begin{enumerate}

  \item there has been an even number of 0's and 1's (state \(q_0\));

  \item there has been an even number of 0's and an odd number of 1's
    (state \(q_1\));

  \item there has been an odd number of 0's and an even number of 1's
    (state \(q_2\));

  \item there has been an odd number of 0's and 1's (state \(q_3\)).

\end{enumerate}

\end{frame}

% ------------------------------------------------------------------------
% 
\begin{frame}
\frametitle{DFA/Example (cont)}

What about the initial and final states?
\begin{itemize}

  \item State \(q_0\) is the initial state because before considering
  any input, the number of 0's and 1's is zero and zero is even.

  \item State \(q_0\) is the lone final state because its definition
  matches exactly the characteristic of \(L\) and no other state
  matches.

\end{itemize}
We know now almost how to specify the DFA for language \(L\). It is
\[
{\cal D} = (\{q_0,q_1,q_2,q_3\}, \{0,1\}, \delta,q_0, \{q_0\})
\]
where the transition function \(\delta\) is described by the following
transition diagram.

\end{frame}

% ------------------------------------------------------------------------
% 
\begin{frame}
\frametitle{DFA/Example (cont)}

\begin{columns}
  \column{0.5\textwidth}
  \begin{center}
    \includegraphics[bb=48 653 138 737]{dfa_even01}
  \end{center}

  \column{0.5\textwidth} Notice how each input 0 causes the state to
  cross the horizontal line.
 
  \bigskip

  Thus, after seeing an even number of 0's we are always above the
  horizontal line, in state \(q_0\) or \(q_1\), and after seeing an
  odd number of 0's we are always below this line, in state \(q_2\) or
  \(q_3\).

  \bigskip

  There is a vertically symmetric situation for transitions on 1.
\end{columns}

\end{frame}

% ------------------------------------------------------------------------
% 
\begin{frame}
\frametitle{DFA/Example (cont)}

We can also represent this DFA by a transition table:
\[
\begin{array}{r@{}l||c|c}
\multicolumn{2}{c||}{\cal D} & 0 & 1\\
\hline\hline
\# \rightarrow & q_0 & q_2 & q_1\\
               & q_1 & q_3 & q_0\\
               & q_2 & q_0 & q_3\\
               & q_3 & q_1 & q_2
\end{array}
\]
We can use this table to illustrate the construction of
\(\hat{\delta}\) from \(delta\). Suppose the input is
\(110101\). Since this string has even numbers of 0's and 1's, it
belongs to \(L\), i.e., we expect \(\hat{\delta}(q_0,110101) = q_0\),
since \(q_0\) is the sole final state.

\end{frame}

% ------------------------------------------------------------------------
% 
\begin{frame}[containsverbatim]
\frametitle{DFA/Example (cont)}

We can check this by computing step by step
\(\hat{\delta}(q_0,\verb+110101+)\), from the shortest prefix to the
longest (which is the word \verb+110101+ itself):
\[
\begin{aligned}
  \hat{\delta} (q_0, \varepsilon) 
&= q_0\\
   \hat{\delta} (q_0, \verb+1+) 
&= \delta (\hat{\delta} (q_0, \varepsilon), \verb+1+) 
= \delta (q_0, \verb+1+)
= q_1\\
   \hat{\delta} (q_0, \verb+11+) 
&= \delta (\hat{\delta} (q_0, \verb+1+), \verb+1+) 
= \delta (q_1, \verb+1+) 
= q_0\\
   \hat{\delta} (q_0, \verb+110+) 
&= \delta (\hat{\delta} (q_0, \verb+11+), \verb+0+) 
= \delta (q_0, \verb+0+) 
= q_2\\
   \hat{\delta} (q_0, \verb+1101+) 
&= \delta (\hat{\delta} (q_0, \verb+110+), \verb+1+) 
= \delta (q_2, \verb+1+) 
= q_3\\
   \hat{\delta} (q_0, \verb+11010+) 
&= \delta (\hat{\delta} (q_0, \verb+1101+), \verb+0+) 
= \delta (q_3, \verb+0+) 
= q_1\\
   \hat{\delta} (q_0, \verb+110101+) 
&= \delta (\hat{\delta} (q_0, \verb+11010+), \verb+1+) 
= \delta (q_1, \verb+1+) 
= q_0 \in F
\end{aligned}
\]

\end{frame}

% ------------------------------------------------------------------------
% 
\begin{frame}
\frametitle{DFA/Dictionary}
 
Let us use DFAs for implementing a dictionary, that is, a trie as
presented at page~\pageref{trie_then}.

\bigskip
First, here is how to add a word \(w \in \Sigma^{+}\) to a trie
\({\cal D} = (Q, \Sigma, \delta, q_0, F)\). The transition function
\(\delta\) is considered here as an array which is modified in place.
\begin{codebox}
\(\proc{Add} (w, (Q, \Sigma, \delta, q_0, F))\)\\
\li \(p \gets q_0\); \(i \gets 1\)
\li \While \(\delta[p,w[i]]\) is defined
\li   \Do \(p \gets \delta[p,w[i]]\); \(i \gets i + 1\)
      \End
\li \For \(j \gets i\) \To \(\strlen{w}\)
\li \Do \(q \gets \proc{New-state}(Q)\);
     \(Q \gets Q \cup \{q\}\);
     \(\delta[p, w[j]] \gets q\);
     \(p \gets q\)
    \End
\li  \(F \gets F \cup \{p\}\)
\end{codebox}

\end{frame}
