%%-*-latex-*-

% ------------------------------------------------------------------------
% 
\begin{frame}
\frametitle{Non-deterministic finite automata}

\label{nfa_01_suffix}

A \textbf{non-deterministic finite automaton} (\textbf{NFA}) has the
same definition as a DFA except that \(\delta\) returns a set of
states instead of one state.

\bigskip

Consider
\begin{center}
\includegraphics[bb=48 711 188 758]{nfa_01_suffix}
\end{center}
There are two out-going edges from state \(q_0\) which are labeled
\(0\), hence two states can be reached when \(0\) is input: \(q_0\)
(loop) and \(q_1\).

\bigskip

This NFA recognises the language of words on the binary alphabet whose
suffix is 01.

\end{frame}

% ------------------------------------------------------------------------
% 
\begin{frame}
\frametitle{NFA (cont)}

Before describing formally what is a recognisable language by a NFA,
let us consider as an example the previous NFA and the input
\(00101\).

\bigskip

Let us represent each transition for this input by an edge in a tree
where nodes are states of the NFA.
\begin{center}
  \includegraphics[bb=71 651 395 732]{nfa_01_suffix_trace}
\end{center}

\end{frame}

% ------------------------------------------------------------------------
% 
\begin{frame}
\frametitle{NFA/Formal definitions}

\label{nfa_01_suffix_table}

A NFA is represented essentially like a DFA:
\(
{\cal N} = (Q_N, \Sigma, \delta_N, q_0, F_N)
\)
where the names have the same interpretation as for DFA, except
\(\delta_N\) which returns a subset of \(Q\) --- not an element of
\(Q\).

\bigskip

\begin{columns}

  \column{0.5\textwidth} For example, the NFA whose transition diagram
  is page~\pageref{nfa_01_suffix} can be specified formally as
  \[
    {\cal N} \!=\! (\{q_0, q_1, q_2\}, \{0, 1\}, \delta_N, q_0, \{q_2\})
  \]

  \column{0.5\textwidth} where the transition function \(\delta_N\) is
  given by the transition table:
  \[
  \begin{array}{@{}r@{}l||c|c@{}}
    \multicolumn{2}{c||}{\cal N} & 0 & 1\\
    \hline\hline
    \rightarrow & q_0 & \{q_0, q_1\} & \{q_0\}\\
                & q_1 & \varnothing  & \{q_2\}\\
    \#          & q_2 & \varnothing  & \varnothing
  \end{array}
  \]
\end{columns}

\end{frame}

% ------------------------------------------------------------------------
% 
\begin{frame}
\frametitle{NFA/Formal definitions (cont)}

Note that, in the transition table of a NFA, all the cells are
filled: there is no transition between two states if and only if the
corresponding cell contains \(\varnothing\). 

\bigskip

In case of a DFA, the cell would remain empty.

\bigskip

It is common also to set that in case of the empty word input,
\(\varepsilon\), both for the DFA and NFA, the state remains the
same:
\begin{itemize}

  \item for DFA: \(\forall q \in Q.\delta_D (q, \varepsilon) = q\)

  \item for NFA: \(\forall q \in Q.\delta_N (q, \varepsilon) = \{q\}\)

\end{itemize}

\end{frame}

% ------------------------------------------------------------------------
% 
\begin{frame}
\frametitle{NFA/Formal definitions (cont)}

As we did for the DFAs, we can \emph{extend the transition function}
\(\delta_N\) to accept words and not just letters (labels). The extended
function is noted \(\hat{\delta}_N\) and defined as follows:
\begin{itemize}

  \item for all state \(q \in Q\), let \(\hat{\delta}_N (q,
  \varepsilon) = \{q\}\)

  \item for all state \(q\ \in Q\), all words \(w \in \Sigma^{*}\),
  all input \(a \in \Sigma\), let \[\hat{\delta}_N (q, wa) =
  \bigcup_{q' \in \hat{\delta}_N (q, w)}{\delta_N  (q', a)}\]
\end{itemize}
The language \(L({\cal N})\) recognised by a NFA \(\cal N\) is
defined as
\[
L({\cal N}) = \{w \in \Sigma^{*} \; \lvert \; \hat{\delta}_N (q_0, w)
\cap F \neq \varnothing\}
\]
which means that the processing of the input stops successfully as
soon as \textbf{at least one current state belongs to \(F\)}.

\end{frame}

% ------------------------------------------------------------------------
% 
\begin{frame}[containsverbatim]
\frametitle{NFA/Example}

\label{nfa_01_suffix_formal}

Let us use \(\hat{\delta}_N\) to describe the processing of the input
\verb+00101+ by the NFA page~\pageref{nfa_01_suffix}:
\begin{enumerate}

  \item \(\hat{\delta}_N (q_0, \varepsilon) = q_0\)

  \item \(\hat{\delta}_N (q_0, \verb+0+) = \delta_N (q_0, \verb+0+) =
    \{q_0, q_1\}\) 

  \item \(\hat{\delta}_N (q_0, \verb+00+) = \delta_N (q_0, \verb+0+)
    \cup \delta_N (q_1, \verb+0+) = \{q_0, q_1\} \cup \varnothing =
    \{q_0, q_1\}\)

  \item \(\hat{\delta}_N (q_0, \verb+001+) = \delta_N (q_0, \verb+1+)
    \cup \delta_N (q_1, \verb+1+) = \{q_0\} \cup \{q_2\} = \{q_0,
    q_2\}\)

  \item \(\hat{\delta}_N (q_0, \verb+0010+) = \delta_N (q_0, \verb+0+)
    \cup \delta_N (q_2, \verb+0+) = \{q_0, q_1\} \cup \varnothing =
    \{q_0, q_1\}\)

  \item \(\hat{\delta}_N (q_0, \verb+00101+) = \delta_N (q_0,
    \verb+1+) \cup \delta_N (q_1, \verb+1+) = \{q_0\} \cup \{q_2\} =
    \{q_0, q_2\} \ni q_2\)

\end{enumerate}
Because \(q_2\) is a final state, actually \(F = \{q_2\}\), we get
\(\hat{\delta}_N (q_0, \verb+00101+) \cap F \neq \varnothing\) thus
the string \verb+00101+ is recognised by the NFA.

\end{frame}
