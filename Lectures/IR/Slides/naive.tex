%%-*-latex-*-

% ------------------------------------------------------------------------
% Naive search
%
\begin{frame}
\frametitle{Naive search}

Let \(t\) be a text and \(x\) a word such that \(\len{x} \leqslant
\len{t}\), both using the alphabet \(\Sigma\).

\bigskip

Formally, the problem of text search is stated as follows: determine
if \(x\) is a factor of \(t\) and, if so, give the position of the
first letter of \(x\) in \(y\).

\bigskip

The simplest algorithm, often called \textbf{naive}, consists in
comparing \(x\) to \(t[k+1 \dots k+\len{x}]\) for all the \(k \in \{0,
1, \dots, \len{t} - \len{x}\}\) in the worst case.

\bigskip

It helps to imagine the text is an array of characters and the word we
are looking up is sliding along this array until we find a match. This
is referred to as the \textbf{sliding window}, because we can think of
a frame sliding along the text.

\end{frame}

% ------------------------------------------------------------------------
%
\begin{frame}
\frametitle{Naive search/Algorithm}

The naive search algorithm we just described can be more formally
described as
{\footnotesize
\begin{codebox}
\(\proc{Naive} (x, t)\)\\
\li \(i \gets 1\); \(j \gets 1\)
\li \While \(i \leqslant \len{x}\) \LogAnd \(j \leqslant \len{t}\)
\li \Do \If \(t[j] = x[i]\)\>\>\>\>\>\>\>\>\Comment Compare a letter of \(x\) with a letter
of \(t\).
\li	  \Then \(j \gets j + 1\); \(i \gets i + 1\)\>\>\>\>\>\>\Comment Schedule comparison with next letter in \(x\).
\li	  \Else \(j \gets j - i + 2\); \(i \gets 1\)\>\>\>\>\>\>\Comment Failed: we slide \(x\) on \(t\) and start again.\End \End
\li \If \(i > \len{x}\) 
\li \Then \(\ldots\)\>\>\>\>\>\>\>\>\Comment Occurrence of \(x\) in \(t\) at position \(j - \len{x}\).
\li \Else \(\ldots\)\>\>\>\>\>\>\>\>\Comment No occurrences. \End
\end{codebox}
}

\end{frame}

% ------------------------------------------------------------------------
%
\begin{frame}
\frametitle{Naive search/Analysis}

How many comparisons does the naive algorithm performs in the worst
case?

\bigskip

In the worst case, \(x\) is not in \(t\) and the test at line \(3\)
always fails on the last letter of \(x\), leading to make \(x\)
slide of one position at line \(5\) the latest as possible. 

\bigskip

An example of positive match in the worst case is \(t =
\texttt{a}^{m-1}\texttt{b}\) and \(x = \texttt{a}^{n-1}\texttt{b}\).

\bigskip

There are \(n - m + 1\) positions in \(t\) to compare with the first
letter of \(x\), and since there are \(m\) letters in \(x\), it makes
a total of \((n - m + 1) m = nm - m^2 + m < \len{t} \times \len{x}\)
positions.

\end{frame}
