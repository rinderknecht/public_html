%%-*-latex-*-

% ------------------------------------------------------------------------
% Morris and Pratt
%
\begin{frame}
\frametitle{Morris-Pratt algorithm}

The slowness of the naive algorithm in the worst case is due to the
fact that, in case of failure, it starts again to compare the first
letters of \(x\) etc. (see line \(5\)), \emph{without using the
  information of the partial success of the previous attempt.}

\bigskip

If the failure occurred at the index \(i\) in the word \(x\) and index
\(j\) in the text, we have
\begin{align*}
   x[1 \dots i-1] 
&= t[j-i \dots j-1]\\
x[i] &\neq t[j]
\end{align*}

\end{frame}

% ------------------------------------------------------------------------
% Morris and Pratt
%
\renewcommand\arraystretch{0.95}

\begin{frame}
\frametitle{Morris-Pratt algorithm (cont)}

\vspace*{-25pt}
\[
\begin{array}{rcc@{\,}c@{}ccr@{}r@{\,}c@{}ccc@{}ccc}
  & & & \scriptstyle j-i 
  & & & 
  & & \multicolumn{2}{@{}l}{\scriptstyle j-i+p}
  & & 
  & \multicolumn{1}{r@{}}{\scriptstyle j-1}
  & \multicolumn{1}{c}{\scriptstyle j}\\
\cline{2-15}
    \multicolumn{1}{r|}{t:}
  & \multicolumn{2}{c}{\phantom{H}\ldots\phantom{H}}
  & \multicolumn{1}{|c|}{a}
  & \multicolumn{1}{c|}{b}
  & \multicolumn{1}{c|}{c}
  & \multicolumn{2}{c|}{\ldots}
  &
  & 
  &
  & \phantom{HHHH}
  & 
  & \multicolumn{1}{|>{\columncolor{mygrey}}c|}{\textcolor{white}{\texttt{b}}}
  & \phantom{HHHH}\\
\cline{2-15}
  &
  & 
  & \multicolumn{1}{:c}{\scriptstyle 1}
  &
  &
  & 
  & \multicolumn{1}{r@{\,}:}{\scriptstyle p}
  & 
  &
  &
  &
  & \multicolumn{1}{c@{}:}{\scriptstyle i-1}
  & \multicolumn{1}{c}{\scriptstyle i}\\ 
\cline{4-15}
    \multicolumn{1}{r}{x:}
  &
  &
  & \multicolumn{1}{|c|}{a}
  & \multicolumn{1}{c|}{b}
  & \multicolumn{1}{c|}{c}
  & \multicolumn{2}{c}{\ldots}
  &
  &
  & 
  & 
  &
  & \multicolumn{1}{|>{\columncolor{mygrey}}c|}{\textcolor{white}{\texttt{a}}}
  & \\
\cline{4-15}
  &&&&&&
  & 
  & \multicolumn{1}{:c}{\scriptstyle 1}
  &
  &
  & 
  & \multicolumn{1}{c:}{}\\
\cline{9-15}
  &&&&&
  & \text{slided}
  & \multicolumn{1}{c}{x:}
  & \multicolumn{1}{|c|}{a}
  & \multicolumn{1}{c|}{b}
  & \multicolumn{1}{c|}{c}
  & \multicolumn{2}{c}{\ldots}\\
\cline{9-15}
\end{array}
\]
Any subsequent search in the factor \(t[j-i \dots j-1]\) could use the
information that it is a prefix of \(x\) --- the first comparison
failure is coloured in grey.

\bigskip

If a new search starts in \(t\) at position \(j-i+p\), with \(2
\leqslant p \leqslant i-1\), it compares \(x[1 \dots]\) to \(t[j-i+p
  \dots]\). Since \(t[j-i+p \dots j-1]\) is a suffix of \(t[j-i \dots
  j-1]\), which is equal to \(x[1 \dots i-1]\), it compares \(x[1
  \dots]\) to a suffix of \(x[1 \dots i-1]\), i.e., to a part of
itself!

\end{frame}

% ------------------------------------------------------------------------
%
\begin{frame}
\frametitle{Morris-Pratt algorithm (cont)}

The important point here is that these comparisons are
\textbf{independent} of the text \(t\), since they apply to part of
the pattern itself, i.e., the word \(x\).

\bigskip

The strategy devised here is to compute all the comparisons on \(x\)
\emph{before} we start the search in the text, and then use this
information to avoid repeating the same comparisons at different
positions.

\bigskip

This \textbf{preprocessing} on \(x\) will indeed allow to recognise
the sole configurations where the search may succeed.

\bigskip

Therefore the subproblem now is to determine the positions \(i\) where
the word \(x[1 \dots i-1]\) ends with a prefix of \(x\).

\end{frame}

% ------------------------------------------------------------------------
%
\begin{frame}
\frametitle{Morris-Pratt algorithm/Maximum borders}

\label{MP_supply_def}

Let \(u\) be a non-empty word. A \textbf{border} of \(u\) is a word
different from \(u\) which is both prefix and suffix of \(u\). For
example, the word \(u = \texttt{abacaba}\) has the three borders
\(\epsilon\), \texttt{a} and \texttt{aba}.

\bigskip

Let us note \(\Border{}{x}\) the \textbf{longest border} of a non-empty
word \(x\). 

\bigskip

For a given word \(x\), let us define a function \(\MPsupplyName_{x}\) on
all its prefixes as
\[
  \MPsupply{x}{}{i} = \len{\Border{}{x[1 \dots i]}} 
  \qquad 1 \leqslant i \leqslant \len{x}
\]
or, equivalently
\[
\MPsupply{x}{}{\len{y}} = \len{\Border{}{y}} \qquad y \preccurlyeq x
\]

\end{frame}

% ------------------------------------------------------------------------
%
\begin{frame}
\frametitle{Morris-Pratt algorithm/Pattern preprocessing (cont)}

Here is the table of the maximum borders for the prefixes of the word
\texttt{abacabac}:
\[
\begin{array}{c|ccccccccc|}
x & \texttt{a} & \texttt{b} & \texttt{a} & \texttt{c} 
 & \texttt{a} & \texttt{b} & \texttt{a} & \texttt{c}\\
\hline
  i
& 1 & 2 & 3 & 4 & 5 & 6 & 7 & 8\\
  \Border{}{y}
& \epsilon & \epsilon & \texttt{a} 
& \epsilon & \texttt{a} & \texttt{ab} & \texttt{aba} & \texttt{abac}\\ 
  \MPsupply{x}{}{i}
& 0 & 0 & 1 & 0 & 1 & 2 & 3 & 4\\
\hline
\end{array}
\]
Each row corresponds to a prefix \(y\) of \(x\), not just to a
letter. So, at index \(7\), you should read \(
\Border{}{\underline{\texttt{aba}}\texttt{c}\underline{\texttt{aba}}}
= \texttt{aba} \), at index \(4\), \(\Border{}{\texttt{abac}} =
\varepsilon\) because \(\texttt{abac} =
\underline{\varepsilon}\texttt{abac}\underline{\varepsilon}\).

\bigskip

\noindent Also, note that borders can overlap, e.g.,. \(\texttt{aaaa}
= \underline{\texttt{aaa}}\texttt{a} =
\texttt{a}\underline{\texttt{aaa}}\), so \(\Border{}{\texttt{aaaa}} =
\texttt{aaa}\).

\end{frame}

% ------------------------------------------------------------------------
%
\begin{frame}
\frametitle{Morris-Pratt algorithm/Pattern preprocessing (cont)}

\label{morris-pratt-conf}

Coming back to the naive search:
\[\small
\begin{array}{rcc@{\,}c@{}ccr@{}r@{\,}c@{}ccc@{}ccc}
  & & & \scriptstyle j-i & & & 
  & & \multicolumn{2}{@{}l}{\scriptstyle j-i+p}
  & &
  & \multicolumn{1}{r@{}}{\scriptstyle j-1}
  & \multicolumn{1}{c}{\scriptstyle j}\\
\cline{2-15}
    \multicolumn{1}{r|}{t:}
  & \multicolumn{2}{c}{\phantom{l}\ldots\phantom{l}}
  & \multicolumn{3}{|c|}{\Border{}{x[1 \dots i-1]}}
  & \multicolumn{2}{c|}{\ldots}
  & \multicolumn{5}{c}{\Border{}{x[1 \dots i-1]}}
  & \multicolumn{1}{|>{\columncolor{mygrey}}c|}{\textcolor{white}{\texttt{b}}}
  & \phantom{ll}\\
\cline{2-15}
  &
  & 
  & \multicolumn{1}{:@{\,}l}{\scriptstyle 1}
  &
  & \multicolumn{1}{c:}{}
  & 
  & \multicolumn{1}{r@{\,}:}{\scriptstyle p}
  & 
  &
  &
  &
  & \multicolumn{1}{r@{}:}{\scriptstyle i-1}
  & \scriptstyle i\\ 
\cline{4-15}
    \multicolumn{1}{r}{x:}
  &
  &
  & \multicolumn{3}{|c|}{\Border{}{x[1 \dots i-1]}}
  & \multicolumn{2}{c}{\ldots}
  & \multicolumn{5}{|c}{\Border{}{x[1 \dots i-1]}}
  & \multicolumn{1}{|>{\columncolor{mygrey}}c|}{\textcolor{white}{\texttt{a}}}
  & \\
\cline{4-15}
  &&&&&&
  & 
  &\multicolumn{1}{:@{\,}l}{\scriptstyle 1}
  &&&& \multicolumn{1}{r@{}:}{\scriptstyle \MPsupply{x}{}{i-1}}\\
\cline{9-15}
  &&&&&
  & \text{slided}
  & \multicolumn{1}{c}{x:}
  & \multicolumn{5}{|c}{\Border{}{x[1 \dots i-1]}}
  & \multicolumn{1}{|c|}{\textbf{?}}\\
\cline{9-15}
\end{array}
\]
the difference here is that, in case of failure (in grey), we do
\textbf{not} decrement the position in \(t\) and we slide \(x\) of
\(p=i - 1 - \MPsupply{x}{}{i-1}\) positions with respect to \(t\),
i.e., the current position in \(x\) does not decrement to \(1\) but
becomes \(1 + \MPsupply{x}{}{i-1}\).

\end{frame}

% ------------------------------------------------------------------------
%
\begin{frame}
\frametitle{Morris-Pratt algorithm/Pattern preprocessing (cont)}

\label{beta_zero}

What happens in the extreme case \(i = 1\), that is, when the
comparison fails on the first letter of the pattern?

\bigskip

In this case, there is no optimisation left, we just can try our luck
by comparing the next position in \(t\) to the first letter of \(x\)
(again): this is exactly what the naive algorithm would do.

\bigskip

This means that the sliding offset must equal one when \(i=1\):
\[
i - 1 - \MPsupply{x}{}{i-1} = 1 \quad \text{where} \; i = 1
\]
which is equivalent to
\[
\MPsupply{x}{}{0} = -1
\]
Therefore we need to extend \(\MPsupplyName\) to be defined on \(0\) as
\(\MPsupply{x}{}{0} = -1\) for all words \(x\).

\end{frame}

% ------------------------------------------------------------------------
%
\begin{frame}
\frametitle{Morris-Pratt algorithm/Pattern preprocessing (cont)}

\label{supply}

In practice, it is convenient to use a function
\(
s_{x}: \{1, \dots, \len{x}\} \rightarrow \{0, \dots, \len{x}\}
\)
defined as 
\[
\sX{x}{}{i} = 1 + \MPsupply{x}{}{i-1} \qquad 1 \leqslant i \leqslant
\len{x}
\]
Then the sliding consists simply in replacing \(i\) by \(\sX{x}{}{i}\).

\bigskip

This function \(s_{x}\) is called the \textbf{failure function} of
pattern \(x\). For the word \texttt{abacabac}, it is
\[
\begin{array}{c|ccccccccc|}
x & & \texttt{a} & \texttt{b} & \texttt{a} & \texttt{c} & \texttt{a} 
 & \texttt{b} & \texttt{a} & \texttt{c}\\
\hline
  i
& 0 & 1 & 2 & 3 & 4 & 5 & 6 & 7 & 8\\
  \MPsupply{x}{}{i}
& -1 & 0 & 0 & 1 & 0 & 1 & 2 & 3 & 4\\
  \sX{x}{}{i}
& \boxempty & 0 & 1 & 1 & 2 & 1 & 2 & 3 & 4\\
\hline
\end{array}
\]

\end{frame}

% ------------------------------------------------------------------------
%
\begin{frame}
\frametitle{Morris-Pratt algorithm/Imperative pseudo-code}

{\small
\begin{codebox}
\(\proc{MP} (x, t)\)\\
\li \(s \gets \proc{MP-Fail} (x)\)\>\>\>\>\>\>\>\>\>\>\Comment The failure function as an array.
\li \(i \gets 1\); \(j \gets 1\)
\li \While \(i \leqslant \len{x}\) \LogAnd \(j \leqslant \len{t}\)
\li \Do \If \(i = 0\) \LogOr \(x[i] = t[j]\)\>\>\>\>\>\>\>\>\Comment Test on \(i\) because now \(i\) can be 0.
\li	  \Then \(i \gets i + 1\); \(j \gets j + 1\)\>\>\>\>\>\>\Comment Schedule next letter in \(x\).
\li	  \Else \(i \gets s[i]\)\>\>\>\>\>\>\Comment Failed: we slide \(x\) on \(t\). 
          \End 
    \End
\li \If \(\len{x} < i\) 
\li \Then \(\ldots\) \>\>\>\>\>\>\>\>\Comment Occurrence of \(x\) in \(t\) at \(j - \len{x}\).
\li \Else \(\ldots\)\>\>\>\>\>\>\>\>\Comment No occurrences. 
    \End
\end{codebox}
}

\end{frame}

% ------------------------------------------------------------------------
%
\begin{frame}
\frametitle{Morris-Pratt algorithm/Single-assignment pseudo-code}
 
\begin{codebox}
\(\proc{MP} (x, t)\)\\
\li \(s \gets \proc{MP-Fail} (x)\)
\li \((i, j) \gets \proc{Offsets} (s,x,t,1,1)\)
\li \If \(\len{x} < i\) 
\li \Then \(\ldots\) \textbf{else} \(\ldots\)
    \End
\end{codebox}
\begin{codebox}
\(\proc{Offsets} (s,x,t,i,j)\)\\
\li \If \(i \leqslant \len{x}\) \LogAnd \(j \leqslant \len{t}\)
\li \Then \If \(i = 0\) \LogOr \(x[i] = t[j]\)
\li	      \Then \(\proc{Offsets} \gets \proc{Offsets} (s, x, t, i+1, j+1)\)
\li       \Else \(\proc{Offsets} \gets \proc{Offsets} (s, x, t, s[i], j)\)
          \End
\li	\Else \(\proc{Offsets} \gets (i, j)\)
    \End
\end{codebox}

\end{frame}

% ------------------------------------------------------------------------
%
\begin{frame}
\frametitle{Morris-Pratt algorithm/Analysis and example}

The Morris-Pratt algorithm finds an occurrence of a word \(x\) in a
text \(t\), or fail to do so, in at most \(2\len{t} - 1\) letter
comparisons, if we already have the failure function on \(x\).

\bigskip

\textbf{Example.} Consider the following table for the search of the
word \(x=\texttt{abacabac}\) in the text
\(t=\texttt{babacacabacaab}\).
\[
\begin{array}{c|cccccccccccccc|}
j & 1 & 2 & 3 & 4 & 5 & 6 & 7 & 8 & 9 & 10 & 11 & 12 & 13 & 14\\
  & \texttt{b} & \texttt{a} & \texttt{b} & \texttt{a} & \texttt{c} 
  & \texttt{a} & \texttt{c} & \texttt{a} & \texttt{b} & \texttt{a} 
  & \texttt{c} & \texttt{a} & \texttt{a} & \texttt{b}\\
\hline
i & 1 & 1 & 2 & 3 & 4 & 5 & 6 & 1 & 2 & 3 & 4 & 5 & 6 & 2 \\
  & 0 &   &   &   &   &   & 2 &   &   &   &   &   & 2 &   \\
  &   &   &   &   &   &   & 1 &   &   &   &   &   & 1 &   \\
  &   &   &   &   &   &   & 0 &   &   &   &   &   &   &  
\end{array}
\]

\end{frame}

% ------------------------------------------------------------------------
%
\begin{frame}
\frametitle{Morris-Pratt algorithm/Example}

\label{mp_example}

\setlength\arraycolsep{3pt}

\vspace*{-11pt}
\[
\begin{array}{rc|c|c|c|c|c|c|c|c|c|c|c|c|c|c|c|c|c|c|c|}
  & \multicolumn{1}{c}{\scriptstyle 1}
  & \multicolumn{1}{c}{\scriptstyle 2}
  & \multicolumn{1}{c}{\scriptstyle 3}
  & \multicolumn{1}{c}{\scriptstyle 4}
  & \multicolumn{1}{c}{\scriptstyle 5}
  & \multicolumn{1}{c}{\scriptstyle 6}
  & \multicolumn{1}{c}{\scriptstyle 7}
  & \multicolumn{1}{c}{\scriptstyle 8}
  & \multicolumn{1}{c}{\scriptstyle 9}
  & \multicolumn{1}{c}{\scriptstyle 10}
  & \multicolumn{1}{c}{\scriptstyle 11}
  & \multicolumn{1}{c}{\scriptstyle 12}
  & \multicolumn{1}{c}{\scriptstyle 13}
  & \multicolumn{1}{c}{\scriptstyle 14}\\
\cline{2-15}
    \multicolumn{1}{r|}{t:}
  & \multicolumn{1}{>{\columncolor{mygrey}}c}{\textcolor{white}{\texttt{b}}}
  & \texttt{a} 
  & \texttt{b} 
  & \texttt{a} 
  & \texttt{c} 
  & \texttt{a} 
  & \multicolumn{1}{>{\columncolor{mygrey}}c}{\textcolor{white}{\texttt{c}}}
  & \texttt{a} 
  & \texttt{b} 
  & \texttt{a} 
  & \texttt{c} 
  & \texttt{a} 
  & \multicolumn{1}{>{\columncolor{mygrey}}c}{\textcolor{white}{\texttt{a}}}
  & \texttt{b}\\
\cline{2-15}
\\
\cline{2-9}
    \multicolumn{1}{r|}{x:}
  & \multicolumn{1}{>{\columncolor{mygrey}}c}{\textcolor{white}{\texttt{a}}}
  & \texttt{b} 
  & \texttt{a} 
  & \texttt{c} 
  & \texttt{a} 
  & \texttt{b} 
  & \texttt{a} 
  & \texttt{c}\\
\cline{2-10}
  &
  & \texttt{a} 
  & \texttt{b} 
  & \texttt{a} 
  & \texttt{c} 
  & \texttt{a} 
  & \multicolumn{1}{>{\columncolor{mygrey}}c}{\textcolor{white}{\texttt{b}}}
  & \texttt{a} 
  & \texttt{c}\\
\cline{3-14}
    \multicolumn{6}{c|}{}
  & \texttt{a} 
  & \multicolumn{1}{>{\columncolor{mygrey}}c}{\textcolor{white}{\texttt{b}}}
  & \texttt{a} 
  & \texttt{c} 
  & \texttt{a} 
  & \texttt{b} 
  & \texttt{a} 
  & \texttt{c}\\
\cline{7-15}
    \multicolumn{7}{c|}{}
  & \multicolumn{1}{>{\columncolor{mygrey}}c}{\textcolor{white}{\texttt{a}}}
  & \texttt{b} 
  & \texttt{a} 
  & \texttt{c} 
  & \texttt{a} 
  & \texttt{b} 
  & \texttt{a} 
  & \texttt{c}\\
\cline{8-16}
    \multicolumn{8}{c|}{}
  & \texttt{a} 
  & \texttt{b} 
  & \texttt{a} 
  & \texttt{c} 
  & \texttt{a} 
  & \multicolumn{1}{>{\columncolor{mygrey}}c}{\textcolor{white}{\texttt{b}}}
  & \texttt{a} 
  & \texttt{c}\\
\cline{9-20}
    \multicolumn{12}{c|}{}
  & \texttt{a} 
  & \multicolumn{1}{>{\columncolor{mygrey}}c}{\textcolor{white}{\texttt{b}}}
  & \texttt{a} 
  & \texttt{c} 
  & \texttt{a} 
  & \texttt{b} 
  & \texttt{a} 
  & \texttt{c}\\
\cline{13-21}
    \multicolumn{13}{c|}{}
  & \texttt{a} 
  & \texttt{b} 
  & \texttt{a} 
  & \texttt{c} 
  & \texttt{a} 
  & \texttt{b} 
  & \texttt{a} 
  & \texttt{c}\\
\cline{14-21}
\end{array}
\]

\end{frame}

% ------------------------------------------------------------------------
%
\begin{frame}
\frametitle{Morris-Pratt algorithm/Maximum borders (cont)}

\label{longest_borders_fig}

We need to show how to compute the function \(\MPsupplyName\) or,
equivalently, the failure function \(s\). Both rely on the computation
of maximum-length borders.

How do we compute the longest borders?
\[
\begin{array}{r@{}>{\,}cccccc@{\,}c}
  & {\scriptstyle 1}
  & 
  &&
  & 
  & \phantom{HHHHHI}{\scriptstyle \len{y}}
  &\\
\cline{2-8}
    \multicolumn{1}{r|}{y \cdot a:}
  & \multicolumn{2}{c|}{\Border{}{y}}
  & \multicolumn{1}{c|}{a?}
  & \phantom{HHHHHHHH}
  & \multicolumn{2}{|c}{\Border{}{y}}
  & \multicolumn{1}{|>{\columncolor{mygrey}}c|}{\textcolor{white}{a}}\\
\cline{2-8}
  & {\scriptstyle 1}
  & 
  &&
  & 
  & \phantom{HH}{\scriptstyle \len{\Border{}{y}}}
  &\\
\cline{2-8}
    \multicolumn{1}{r|}{\Border{}{y} \cdot a:}
  & \multicolumn{2}{c|}{\Border{2}{y}}
  & \multicolumn{1}{c|}{a?}
  & \phantom{HHHHHHHH}
  & \multicolumn{2}{|c}{\Border{2}{y}}
  & \multicolumn{1}{|>{\columncolor{mygrey}}c|}{\textcolor{white}{a}}\\
\cline{2-8}
  & {\scriptstyle 1}
  & 
  &&
  & 
  & \phantom{HH}{\scriptstyle \len{\Border{2}{y}}}
  &\\
\cline{2-8}
    \multicolumn{1}{r|}{\Border{2}{y} \cdot a:}
  & \multicolumn{2}{c|}{\Border{3}{y}}
  & \multicolumn{1}{c|}{a?}
  & \phantom{HHHHHHHH}
  & \multicolumn{2}{|c}{\Border{3}{y}}
  & \multicolumn{1}{|>{\columncolor{mygrey}}c|}{\textcolor{white}{a}}\\
\cline{2-8}
\end{array}
\]

\end{frame}

% ------------------------------------------------------------------------
%
\begin{frame}
\frametitle{Morris-Pratt algorithm/Maximum borders (cont)}

Let us find the longest border of a word \(ya\),
i.e., \(\Border{}{ya}\), where \(a\) is a letter. 

\bigskip

The idea is to, recursively, consider \(\Border{}{y} \cdot a\).

\bigskip

If \(\Border{}{y} \cdot a\) is a prefix of \(y\), then
\(\Border{}{ya} = \Border{}{y} \cdot a\).

\bigskip

Otherwise, we must consider the border of the border of \(y\),
i.e., \(\Border{2}{y} \cdot a\), instead of \(\Border{}{y} \cdot a\)
and repeat this process, until \(\Border{k}{y} \cdot a\) is a prefix
of \(y\) of \(\Border{k}{y}\) is empty.

\end{frame}

%% % ------------------------------------------------------------------------
%% %
%% \begin{frame}
%% \frametitle{Morris-Pratt algorithm/Maximum borders (cont)}

%% \raggedslides[0pt]

%% \label{decreasing_borders}

%% Let \(y\) be a non-empty word and let \(k\) be the smallest integer
%% such that \(\Border{k}{y} = \epsilon\). Then
%% \begin{enumerate}

%%   \item the borders of \(y\) are \(\{\Border{}{y}, \Border{2}{y},
%%     \ldots, \Border{k-1}{y}, \epsilon\}\); \label{border:1}

%%   \item let \(a\) be a letter, then \(\Border{}{ya}\) is the longest
%%     prefix of \(y\) which is in\\ \(\{\Border{}{y} \cdot a,
%%     \Border{2}{y} \cdot a, \dots, \Border{k-1}{y} \cdot a, a,
%%     \epsilon\}\) \label{border:2}

%% \end{enumerate}
%% Note that we ordered the borders by strictly decreasing lengths:
%% \[
%% \len{\Border{}{y} \cdot a} > \len{\Border{2}{y} \cdot a} >
%% \ldots \; > \len{\Border{k-1}{y} \cdot a} > \len{a} > 0
%% \]
%% and that \(1 \leqslant k\).

%% \end{frame}

% ------------------------------------------------------------------------
%
\begin{frame}
\frametitle{Morris-Pratt algorithm/Maximum borders (cont)}

This can be formally summarised as follows. For all word \(y \neq
\epsilon\) and all letter \(a\):
\[
\begin{aligned}
  \Border{}{a}
 &= \epsilon\\
\Border{}{y \cdot a} &= \left\{
\begin{aligned}
& \Border{}{y} \cdot a 
&& \text{if} \; \Border{}{y} \cdot a \preccurlyeq y\\
& \Border{}{\Border{}{y} \cdot a} 
&& \text{otherwise}
\end{aligned}
\right.
\end{aligned}
\]
For example, let \(y = \texttt{ababbb}\) and assume \(\Border{}{y}  =
\epsilon\).\\ Then \(\Border{}{y \cdot \texttt{a}} = \Border{}{y} \cdot
\texttt{a} = \texttt{a}\) because \(\texttt{a} \preccurlyeq y\). Other
examples:
\begin{align*}
  y             &= \texttt{babbaa}
& \Border{}{y}  &= \epsilon
& \Border{}{y \cdot \texttt{a}} &= \Border{}{\epsilon \cdot
    \texttt{a}} = \epsilon\\
  y             &= \texttt{abaaab}
& \Border{}{y}  &= \texttt{ab}
& \Border{}{y \cdot \texttt{a}} &= \Border{}{y} \cdot \texttt{a} =
  \texttt{aba}\\
  y             &= \texttt{abbaab}
& \Border{}{y}  &= \texttt{ab}
& \Border{}{y \cdot \texttt{a}} &= \Border{}{\texttt{ab} \cdot
    \texttt{a}} = \texttt{a}
\end{align*}

\end{frame}

% ------------------------------------------------------------------------
%
\begin{frame}
\frametitle{Morris-Pratt algorithm/Maximum borders (cont)}

\label{decreasing_borders_proof}

Let us prove that, when we take the border of a word, we get a prefix
of this word, i.e.,
\[
\Border{}{y} \prec y \qquad y \neq \varepsilon
\]
First, the property holds on words of length \(1\) since, by
definition,
\[
\Border{}{a} = \varepsilon \prec a
\]
Assume now that the property holds for all words of length lower or
equal to \(n\) and let us prove that it holds for words of length
\(n+1\).

\end{frame}

% ------------------------------------------------------------------------
%
\begin{frame}
\frametitle{Morris-Pratt algorithm/Maximum borders (cont)}

Let \(y\) be a word of length \(n\), \(1 \leqslant n\), and a letter
\(a\). 

\bigskip

By definition, we have 
\[
\Border{}{ya} = \left\{
\begin{aligned}
& \Border{}{y} \cdot a 
&& \text{if} \; \Border{}{y} \cdot a \preccurlyeq y\\
& \Border{}{\Border{}{y} \cdot a} 
&& \text{otherwise}
\end{aligned}
\right.
\]
So, if \(\Border{}{y} \cdot a \preccurlyeq y\), then 
\[
\Border{}{ya} = \Border{}{y} \cdot a \preccurlyeq y \prec ya 
\]
This is the property for words of length \(n+1\) such that
\(\Border{}{y} \cdot a \preccurlyeq y\).

\end{frame}

% ------------------------------------------------------------------------
%
\begin{frame}
\frametitle{Morris-Pratt algorithm/Maximum borders (cont)}

Otherwise, if \(\Border{}{y} \cdot a \npreccurlyeq y\) then
\[
\Border{}{ya} = \Border{}{\Border{}{y} \cdot a}.
\]
Since \(\Border{}{y} \prec y\) and \(\len{y} = n\) then 
\begin{align*}
\len{\Border{}{y}} < \len{y}
&\Leftrightarrow
\len{\Border{}{y}} < n
\Leftrightarrow
\len{\Border{}{y}} + 1 < n + 1\\
&\Leftrightarrow
\len{\Border{}{y}} + 1 \leqslant n
\end{align*}
Thus \(\len{\Border{}{y} \cdot a} = \len{\Border{}{y}} + 1 \leqslant
n\), and the inductive assumption holds:
\begin{align*}
  \Border{}{\Border{}{y} \cdot a} 
& \prec \Border{}{y} \cdot a\\
  \Border{}{ya}
& \preccurlyeq \Border{}{y} \prec y \prec ya
\end{align*}
This is the property for words of length \(n+1\) such that
\(\Border{}{y} \cdot a \npreccurlyeq y\), so this achieves the proof.

\end{frame}

% ------------------------------------------------------------------------
%
\begin{frame}
\frametitle{Morris-Pratt algorithm/Maximum borders (cont)}

Let us try now the following.

\bigskip

\(\Border{}{ya}\)
\begin{align*}
&= \Border{}{\Border{}{y} \cdot a}
                &  \Border{}{y} \cdot a \npreccurlyeq y\\
                &= \Border{}{\Border{2}{y} \cdot a}
                &  \Border{2}{y} \cdot a \npreccurlyeq \Border{}{y}\\
                &= \ldots
                & \ldots\\
                &= \Border{}{\Border{k-1}{y} \cdot a}
                & \Border{k-1}{y} \cdot a \npreccurlyeq \Border{k-2}{y}
\end{align*}
and \(\forall p \leqslant k-2.\Border{p}{y} \neq \varepsilon\).

\bigskip

\noindent Then there is a smallest \(k\) such that \(\Border{k}{y}
\cdot a \preccurlyeq \Border{k-1}{y}\) or \(\Border{k-1}{y} =
\varepsilon\).

\end{frame}

% ------------------------------------------------------------------------
%
\begin{frame}
\frametitle{Morris-Pratt algorithm/Pattern preprocessing (resumed)}

We proved page~\pageref{decreasing_borders_proof}, that
\[
\Border{}{y} \prec y \qquad y \neq \varepsilon
\]
Therefore
\[
\Border{k}{y} \prec \Border{k-1}{y} \prec y
\qquad 2 \leqslant k
\]
Then, 
\[
\Border{k}{y} \cdot a \preccurlyeq \Border{k-1}{y}
\Longleftrightarrow
\Border{k}{y} \cdot a \preccurlyeq y
\qquad 1 \leqslant k
\]
(Make a picture to convince yourself.)

\end{frame}

% ------------------------------------------------------------------------
%
\begin{frame}
\frametitle{Morris-Pratt algorithm/Pattern preprocessing (resumed)}

\label{border_def}

If \(\Border{k}{y} \cdot a \preccurlyeq y\) and \(\Border{k-1}{y} \neq
\varepsilon\) then, using the definition
\[
  \Border{}{ya}
= \Border{}{\Border{k-1}{y} \cdot a} 
= \Border{k}{y} \cdot a 
\]
Otherwise, if \(\Border{k-1}{y} = \varepsilon\) then
\[
  \Border{}{ya} 
= \Border{}{\Border{k-1}{y} \cdot a} 
= \Border{}{a} 
= \varepsilon
\]
and we cannot have \(\Border{k}{y} \cdot a \preccurlyeq y\) since
\(\Border{k}{y} = \Border{}{\Border{k-1}{y}}\) is then
undefined (the empty word has no border). Finally, we can summarise
all cases:
\[
\Border{}{ya} = 
\left\{
  \begin{aligned}
   & \Border{k}{y} \cdot a 
   && \text{if} \; \Border{k}{y} \cdot a \preccurlyeq y\\
   & \varepsilon 
   && \text{if} \; \Border{k-1}{y} = \varepsilon
  \end{aligned}
\right. 
\]

\end{frame}

% ------------------------------------------------------------------------
%
\begin{frame}
\frametitle{Morris-Pratt algorithm/Pattern preprocessing (resumed)}

\label{condition}

Let us give a computational definition of function
\(\MPsupplyName_{x}\). Let \(ya \preccurlyeq x\).
\begin{align*}
   \MPsupply{x}{}{\len{ya}}  
&= \len{\Border{}{ya}}
&& \text{by substitution of} \; y \; \text{by}
                             \; ya \; \text{in definition}\\
   \MPsupply{x}{}{\len{y}+1} 
&= \len{\Border{}{ya}}
&& \text{by property of length} \tag{1}
\end{align*}
From the definition of \(\MPsupplyName_{x}\)
page~\pageref{MP_supply_def}, we deduce the corollary
\[
\MPsupply{x}{k}{\len{y}} = \len{\Border{k}{y}} \quad y \preccurlyeq x \tag{2}
\]
which we can prove by induction on \(k\). The property is true for
\(k=1\) (even \(k=0\)), by definition. Then assume
\(\MPsupply{x}{k}{\len{y}} = \len{\Border{k}{y}}\) and do
\begin{align*}
  \MPsupply{x}{k+1}{\len{y}}
&= \MPsupply{x}{k}{\MPsupply{x}{}{\len{y}}} 
= \MPsupply{x}{k}{\len{\Border{}{y}}}
= \len{\Border{k}{\Border{}{y}}}\\
&= \len{\Border{k+1}{y}}
\end{align*} 
which proves that the property holds for all \(k\) such that \( 1
\leqslant k\).

\end{frame}

% ------------------------------------------------------------------------
%
\begin{frame}
\frametitle{Morris-Pratt algorithm/Pattern preprocessing (resumed)}

For the sake of clarity, let us mimic the setting of
figure page~\pageref{morris-pratt-conf} by letting \(\len{ya}=i\).

\bigskip

\(\Border{k}{y} \cdot a \preccurlyeq y\)
\begin{align*}
&\Longleftrightarrow y [\len{\Border{k}{y}} + 1] = a
&& \text{since} \; \Border{}{y} \prec y\\
&\Longleftrightarrow y [\MPsupply{x}{k}{\len{y}} + 1] = a
&& \text{by property} \; (2)\\
&\Longleftrightarrow x [\MPsupply{x}{k}{\len{y}} + 1] = x [\len{y} + 1]
&& \text{since} \; ya \preccurlyeq x\\
&\Longleftrightarrow x [\MPsupply{x}{k}{i-1} + 1] = x [i]
&& \text{since} \; \len{y} = i - 1 \tag{3}
\end{align*}

\end{frame}

% ------------------------------------------------------------------------
%
\begin{frame}
\frametitle{Pattern preprocessing (cont)}

Let us consider again the recursive definition of \textsc{Border} at
page~\pageref{border_def}:
\[
\Border{}{ya} = 
\left\{
  \begin{aligned}
   & \Border{k}{y} \cdot a 
   && \text{if} \; \Border{k}{y} \cdot a \preccurlyeq y\\
   & \varepsilon 
   && \text{if} \; \Border{k-1}{y} = \varepsilon
  \end{aligned}
\right. 
\]
where \(k\) is the smallest non-zero integer such that \(\Border{k}{y}
\cdot a \preccurlyeq y\) or \(\Border{k-1}{y} = \varepsilon\).

\bigskip

By taking the lengths of each side of the equations, it comes
\[
\len{\Border{}{ya}} = 
\left\{
  \begin{aligned}
   & \len{\Border{k}{y} \cdot a }
   && \text{if} \; \Border{k}{y} \cdot a \preccurlyeq y\\
   & \len{\varepsilon}
   && \text{if} \; \Border{k-1}{y} = \varepsilon
  \end{aligned}
\right. 
\]

\end{frame}

% ------------------------------------------------------------------------
%
\begin{frame}
\frametitle{Pattern preprocessing (cont)}

By equation \((1)\) we get
\[
\MPsupply{x}{}{\len{y} + 1} = 
\left\{
  \begin{aligned}
  & \len{\Border{k}{y}} + 1
  && \text{if} \; \Border{k}{y} \cdot a \preccurlyeq y\\
  & 0
  && \text{if} \; \Border{k-1}{y} = \varepsilon
\end{aligned}
\right.
\]
Using equations \((2)\) and \((3)\):
\[
\MPsupply{x}{}{\len{y} + 1} = 
\left\{
  \begin{aligned}
   & \MPsupply{x}{k}{\len{y}} + 1
   && \text{if} \; x [\MPsupply{x}{k}{i-1} + 1] = x [i]\\
   & 0
   && \text{if} \; \MPsupply{x}{k-1}{i-1} = 0
\end{aligned}
\right.
\]
\end{frame}

% ------------------------------------------------------------------------
%
\begin{frame}
\frametitle{Pattern preprocessing (cont)}

Finally, taking into account the value in \(0\) and \(1\), we have, for
\(\len{x} = i \geqslant 2\)
\[
\begin{aligned}
  \MPsupply{x}{}{0} &= -1 \qquad \MPsupply{x}{}{1} = 0\\
\MPsupply{x}{}{i} &=
\left\{
\begin{aligned}
& 1 + \MPsupply{x}{k}{i-1}
&& \text{if} \; x [1 + \MPsupply{x}{k}{i-1}] = x [i]\\
& 0
&& \text{if} \; \MPsupply{x}{k-1}{i-1} = 0
\end{aligned}
\right.
\end{aligned}
\]
where \(k\) is the smallest nonzero integer such that
\(x[1+\MPsupply{x}{k}{i-1}] = x[i]\) or \(\MPsupply{x}{k-1}{i-1} =
0\).

\end{frame}

% ------------------------------------------------------------------------
%
\begin{frame}
\frametitle{Pattern preprocessing (cont)}

\label{beta_def}

From that definition, it is clear that \(\MPsupplyName\) returns the value
\(-1\) only on \(0\). 

\bigskip

Thus
\begin{align*}
  \MPsupply{x}{k-1}{i-1} = 0
&\Longleftrightarrow
  \MPsupply{x}{}{\MPsupply{x}{k-1}{i-1}} = \MPsupply{x}{}{0}
\Longleftrightarrow
  \MPsupply{x}{k}{i-1} = -1\\
&\Longleftrightarrow
  1 + \MPsupply{x}{k}{i-1} = 0
\end{align*}
Then we can rewrite the previous definition of \(\MPsupplyName_{x}\) as
\[
\MPsupply{x}{}{0} = -1 \qquad \MPsupply{x}{}{i} = 1 + \MPsupply{x}{k}{i-1}
\quad 1 \leqslant i
\]
where \(k\) is the smallest non-zero integer such that
\begin{itemize}

  \item either \(1 + \MPsupply{x}{k}{i-1} = 0\)

  \item or \(x[1+\MPsupply{x}{k}{i-1}] = x[i]\)

\end{itemize}

\end{frame}

% ------------------------------------------------------------------------
%
\begin{frame}
\frametitle{Pattern preprocessing/Imperative}

It is then straightforward to write an algorithm for \(\MPsupplyName_{x}\),
assuming \(0 \leqslant i \leqslant \len{x}\):
\begin{codebox}
\(\proc{Beta} (x, i)\)\\
\li \If \(i = 0\)
\li \Then \(\proc{Beta} \gets -1\)\RComment \(\MPsupply{x}{}{0} = -1\)
\li \Else \(\id{offset} \gets i-1\) 
\li       \Repeat
\li         \(\id{offset} \gets \proc{Beta}(x,\id{offset})\)
\li       \Until \(\id{offset} = -1\) \LogOr \(x[1+\id{offset}] = x[i]\)
\li       \(\proc{Beta} \gets 1 + \id{offset}\)\RComment \(\MPsupply{x}{}{i}
= 1+\MPsupply{x}{k}{i-1}\)
    \End
\end{codebox}

\end{frame}

% ------------------------------------------------------------------------
%
\begin{frame}
\frametitle{Pattern preprocessing/Single-assignment pseudo-code}

\begin{codebox}
\(\proc{Beta} (x, i)\)\\
\li \If \(i = 0\)
\li \Then \(\proc{Beta} \gets -1\)
\li \Else \(\proc{Beta} \gets 1 + \proc{Offset} (x, \proc{Beta} (x,i-1))\)
    \End
\end{codebox}
\begin{codebox}
\(\proc{Offset}(x,\id{offset})\)\\
\li \If \(\id{offset} = -1\) \LogOr \(x[1+\id{offset}] = x[i]\)
\li \Then \(\proc{Offset} \gets \id{offset}\)
\li \Else \(\proc{Offset} \gets \proc{Offset} (x, \proc{Beta} (x,
            \id{offset}))\)
    \End
\end{codebox}

\end{frame}

% ------------------------------------------------------------------------
%
\begin{frame}
\frametitle{Pattern preprocessing (cont)}

The problem here is inefficiency: 
\begin{enumerate}

  \item successive calls with the same arguments trigger again the
    same computations,

  \item the recursive calls are expensive.

\end{enumerate}
It is much preferable to have \(\MPsupplyName\) as an array:
\begin{enumerate}

  \item the values of \(\MPsupplyName\) are computed only once each

  \item and are accessed in constant time instead of relying on
    function calls.

\end{enumerate}

\end{frame}

% ------------------------------------------------------------------------
%
\begin{frame}
\frametitle{Pattern preprocessing/Imperative}

The idea is to compute all the values of \(\MPsupply{x}{}{i}\) for
increasing values of \(i\):
\begin{codebox}
\(\proc{Make-Beta}(x)\)\\
\li \(\MPsupplyName_{x}[0] \gets -1\)
\li \For \(i \gets 1\) \To \(\len{x}\)
\li	\Do \(\id{offset} \gets i-1\) 
\li     \Repeat
\li       \(\id{offset} \gets \MPsupplyName_{x}[\id{offset}]\)
\li     \Until \(\id{offset} = -1\) \LogOr \(x[1+\id{offset}] = x[i]\)
\li     \(\MPsupplyName_{x}[i] \gets 1 + \id{offset}\)\RComment \(\MPsupply{x}{}{i}
= 1+\MPsupply{x}{k}{i-1}\)
    \End
\li \(\proc{Make-Beta} \gets \MPsupplyName_{x}\)
\end{codebox}

\end{frame}

% ------------------------------------------------------------------------
%
\begin{frame}
\frametitle{Pattern preprocessing /
  Single-assignment pseudo-code}

\begin{codebox}
\(\proc{Make-Beta}(x)\)\\
\li \(\MPsupplyName_{x}[0] \gets -1\)
\li \For \(i \gets 1\) \To \(\len{x}\)
\li	\Do \(\MPsupplyName_{x}[i] \gets 1 + \proc{Offset} (x, \MPsupplyName_{x}, \MPsupplyName_{x}[i-1])\)
    \End
\li \(\proc{Make-Beta} \gets \MPsupplyName_{x}\)
\end{codebox}
\begin{codebox}
\(\proc{Offset}(x, \MPsupplyName_{x}, \id{offset})\)\\
\li \If \(\id{offset} = -1\) \LogOr \(x[1+\id{offset}] = x[i]\)
\li \Then \(\proc{Offset} \gets \id{offset}\)
\li \Else \(\proc{Offset} \gets \proc{Offset} (x, \MPsupplyName_{x},
          \MPsupplyName_{x}[\id{offset}])\)
    \End
\end{codebox}

\end{frame}

% ------------------------------------------------------------------------
%
\begin{frame}
\frametitle{Pattern preprocessing (cont)}

In order to finish, we need to give the algorithm for the failure
function \(s_{x}\) (see page~\pageref{supply}). Recall that
\[
\sX{x}{}{i} = 1 + \MPsupply{x}{}{i-1} \quad 1 \leqslant i
\]
We could then keep \proc{Beta} and create a separate function
\proc{MP-Fail} simply by sticking to the above definition:
\begin{codebox}
\(\proc{MP-Fail} (\MPsupplyName_{x}, i)\)\\
\li \(\proc{MP-Fail} \gets 1 + \MPsupplyName_{x}[i-1]\)\>\>\>\>\>\>\>\>\Comment \(s_{x}[i] \gets 1 + \MPsupplyName_{x}[i-1]\)
\end{codebox}
But, again, this is not efficient: it would be better to precompute
and store all the values of \proc{MP-Fail} in an array.

\end{frame}

% ------------------------------------------------------------------------
%
\begin{frame}
\frametitle{Pattern preprocessing (cont)}

Instead, let us prove by induction on \(k\) that
\[
\sX{x}{k}{i} = 1 + \MPsupply{x}{k}{i-1} \quad 1 \leqslant i
\]
The case \(k=1\) is the definition of \(s\). Let us then assume that
the property is true for all values from \(1\) to \(k\). We have
\[
  \sX{x}{k+1}{i} 
= \sX{x}{k}{\sX{x}{}{i}} 
= 1 + \MPsupply{x}{k}{\sX{x}{}{i}-1} 
= 1 + \MPsupply{x}{k}{\MPsupply{x}{}{i-1}}
= 1 + \MPsupply{x}{k+1}{i-1}
\]
which proves the property for all \(1 \leqslant k\).

\end{frame}

% ------------------------------------------------------------------------
%
\begin{frame}
\frametitle{Pattern preprocessing (cont)}

Let us recall the definition of \(\MPsupplyName_{x}\) (page~\pageref{beta_def}):
\[
\MPsupply{x}{}{0} = -1 \qquad \MPsupply{x}{}{i} = 1 + \MPsupply{x}{k}{i-1}
\qquad 1 \leqslant i
\]
where \(k\) is the smallest non-zero integer such that
\begin{itemize}

  \item either \(1 + \MPsupply{x}{k}{i-1} = 0\) or
    \(x[1+\MPsupply{x}{k}{i-1}] = x[i]\)

\end{itemize}
This becomes
\[
\sX{x}{}{1} = 0 \qquad \sX{x}{}{1 + i} = 1 + \sX{x}{k}{i}
\qquad 1 \leqslant i
\]
where \(k\) is the smallest non-zero integer such that
\begin{itemize}

  \item either \(\sX{x}{k}{i} = 0\) or \(x[\sX{x}{k}{i}] = x[i]\)

\end{itemize}

\end{frame}

% ------------------------------------------------------------------------
%
\begin{frame}
\frametitle{Pattern preprocessing/Imperative}

\begin{codebox}
\(\proc{MP-Fail}(x)\)\\
\li \(s_{x}[1] \gets 0\)
\li \For \(i \gets 1\) \To \(\len{x}-1\)
\li	\Do \(\id{offset} \gets i\)
\li     \Repeat
\li     \(\id{offset} \gets s_{x}[\id{offset}]\)
\li     \Until \(\id{offset} = 0\) \LogOr \(x[\id{offset}] = x[i]\)
\li	    \(s_{x}[1 + i] \gets 1 + \id{offset}\)
    \End
\li \(\proc{MP-Fail} \gets s\)
\end{codebox}

\end{frame}

% ------------------------------------------------------------------------
%
\begin{frame}
\frametitle{Pattern preprocessing/Single-assignment pseudo-code}

\begin{codebox}
\(\proc{MP-Fail}(x)\)\\
\li \(s_{x}[1] \gets 0\)
\li \For \(i \gets 1\) \To \(\len{x}-1\)
\li \Do \(s_{x}[1 + i] \gets 1 + \proc{Offset} (x, s_{x}, s_{x}[i])\)
    \End
\li \(\proc{MP-Fail} \gets s_{x}\)
\end{codebox}
\begin{codebox}
\(\proc{Offset} (x, s_{x}, \id{offset})\)\\
\li \If \(\id{offset} = 0\) \LogOr \(x[\id{offset}] = x[i]\)
\li \Then \(\proc{Offset} \gets \id{offset}\)
\li \Else \(\proc{Offset} \gets \proc{Offset} (x, s_{x}, 
                                               s_{x}[\id{offset}])\)
    \End
\end{codebox}

\end{frame}

% ------------------------------------------------------------------------
%
\begin{frame}
\frametitle{Analysis}

The Morris-Pratt algorithm makes at most \(2 (\len{t} + \len{x} - 2)\)
comparisons to find a word \(x\) in a text \(t\) or to fail.

\bigskip

This is much better than the naive algorithm whose cost was \((\len{t}
- \len{x} + 1) \times \len{x}\).

\end{frame}
