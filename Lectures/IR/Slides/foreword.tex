%%-*-latex-*-

% ------------------------------------------------------------------------
%
\begin{frame}
\frametitle{Focus of this course}

The most popular application of Information Retrieval is searching
information on the World Wide Web, by means of \textbf{search engines}
as Naver, Google, Yahoo etc.

\bigskip

There are other applications, like \textbf{document databases}
(e.g., as found in libraries) and \textbf{text searches}, always
relying on \textbf{complex mathematical models} (e.g., based on
probability theory) to assess the \emph{relevance} of the retrieved
information or the \emph{efficiency} of the retrieving algorithms.

\bigskip

In this course, however, we shall focus exclusively on text search,
because it is the simplest way to start the study of this wide topics.

\end{frame}

% ------------------------------------------------------------------------
%
\begin{frame}[containsverbatim]
\frametitle{Focus of this course (cont)}

Text search, in its simplest form, refers to the search of the
occurrences of a word in a text, i.e. determine whether a word occurs
in a given text and, if so, where.

\bigskip

All text editors include such a feature, as well as scripting
languages as \Perl and \AWK.

\bigskip

There is a famous utility called \texttt{grep} under \textsc{UNIX}
which allows to search a set of strings in a text very efficiently.

\end{frame}

% ------------------------------------------------------------------------
%
\begin{frame}[containsverbatim]
\frametitle{Focus of this course (cont)}

A \textbf{text} is nothing else than a string of characters and a
word is a special case of a \textbf{pattern}.

\bigskip

In general, a pattern can describe a set of words, not only a single
word. For example, the pattern \verb+abra.*bra+ defines the set of
words starting by \verb+abra+ and ending by \verb+bra+.

\bigskip

The efficiency of a search algorithm is usually measured by
\textbf{the number of letter comparisons}: the lower, the better.

\end{frame}
