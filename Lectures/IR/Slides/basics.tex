%%-*-latex-*-

% ------------------------------------------------------------------------
% Basic definitions
%
\begin{frame}
\frametitle{Basic definitions}

An \textbf{alphabet} is a finite set whose elements are called
\textbf{letters}.

\bigskip

A \textbf{string} is a sequence of letters of an alphabet. A
\textbf{text} and a \textbf{word} are other names for strings,
depending on the context. The empty string is noted \(\epsilon\).

\bigskip

The \textbf{length} of a string \(x\) is the length of the associated
sequence and is noted \(\len{x}\).

\end{frame}


% ------------------------------------------------------------------------
% Basic definitions
%
\begin{frame}
\frametitle{Basic definitions (cont)}

The sequence of letters is written simply by enumerating in order the
letters, like \texttt{abraca}.

\bigskip

For \(i = 1, 2, \ldots, \len{x}\), we note \(x[i]\) the letter at
\textbf{index} (or position) \(i\) in \(x\). For example, if \(x =
\texttt{abba}\) then \(x[1] = x[4] = \texttt{a}\).

\bigskip

From the previous definitions, we have \(x = x[1] x[2] \ldots
x[\len{x}]\) for all non-empty string \(x\).

\end{frame}

% ------------------------------------------------------------------------
% Basic definitions
%
\begin{frame}
\frametitle{Basic definitions (cont)}

We can define the equality of two strings using the equality of two
letters: by definition \(x = y\) if \(x = y = \epsilon\) or \(\len{x}
= \len{y}\) and then for all \(i\) such that \(1 \leqslant i \leqslant
\len{x}\), we have \(x[i] = y[i]\).

\bigskip

The \textbf{product} or \textbf{concatenation} of two strings \(x\)
and \(y\) is the string composed of the letters of \(x\) followed by
the letters of \(y\). It is noted \(xy\) or \(x \cdot y\).

\bigskip

Property \(x \cdot \epsilon = \epsilon \cdot x = x\) holds for all
strings \(x\).

\end{frame}


% ------------------------------------------------------------------------
% Basic definitions
%
\begin{frame}
\frametitle{Basic definitions (cont)}

A word \(x\) is a \textbf{factor} of a word \(y\) if there exists two
words \(u\) and \(v\) such that \(y = uxv\). 

\bigskip

Then there exists a position \(i\) such that \(x = y[i] y[i+1] \dots
y[i + \len{x} - 1]\), noted more simply as \(x = y[i \dots i + \len{x}
  - 1]\). One says that \(x\) \textbf{occurs} in \(y\).

\bigskip

Also, when \(y = xv\), then \(x\) is a \textbf{prefix} of \(y\), noted
\(x \preccurlyeq y\). 

\bigskip

When \(y = ux\), then \(x\) is a \textbf{suffix} of \(y\). A factor
\(x\) of \(y\) such that \(x \neq y\) is a \textbf{proper factor}. We
note \(x \prec y\) if \(x\) is a proper prefix of \(y\).

\end{frame}

% ------------------------------------------------------------------------
% Basic definitions
%
\begin{frame}
\frametitle{Basic definitions (cont)}

The \textbf{\(n\)-th power} of word \(x\) is defined as \(x^0 =
\epsilon\) and \(x^{n+1} = x^{n} x\), for all \(n \geqslant 0\). This
notation denotes the repetition of a word. So, for example, if \(x =
\texttt{abb}\), then
\begin{align*}
x^0 &= \epsilon\\
x^1 &= x^0 x = \varepsilon x = x = \texttt{abb}\\
x^2 &= x^1 x = (x^0 x) x = xx = (\texttt{abb})(\texttt{abb}) =
\texttt{abbabb}\\
x^3 &= x^2 x = (x^1 x) x = ((x^0 x) x) x = xxx\\
    &= (\texttt{abb})(\texttt{abb})(\texttt{abb}) = \texttt{abbabbabb}
\end{align*}
It is good to remember a similar concept (power) and notation for
functions. If \(f\) is a function from a set onto itself, then, for
all \(n \geqslant 0\)
\[
f^0(x) = x \qquad f^{n+1}(x) = f^{n}(f(x))
\]

\end{frame}

% ------------------------------------------------------------------------
% Basic definitions
%
\begin{frame}
\frametitle{Basic definitions (cont)}

So, for example, if \(f(x) = x + 1\), then
\begin{align*}
f^0(x) &= x\\
f^1(x) &= f^0(f(x)) = f(x) = x + 1\\
f^2(x) &= f^1(f(x)) = f^0(f(f(x))) = f(f(x)) = f(x+1) = (x + 1) + 1\\
       &= x + 2
\end{align*}
Note that it was possible to define the power of a function \(f\) as
\[
f^0(x) = x \qquad f^{n+1}(x) = f(f^{n}(x))
\]
It is possible to compose two different functions too. If \(f\) and
\(g\) are two composable functions, then we define the composed
function \(f \circ g\) as
\[
(f \circ g)(x) = f(g(x))
\]

\end{frame}
