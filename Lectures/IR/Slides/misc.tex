%%-*-latex-*-


% ------------------------------------------------------------------------
% 
\begin{slide}
\titre{Subset construction (continued)}

\raggedslides[0pt]

Why is \(2^n\) the number of subsets of a finite set of cardinal
\(n\)?

Let us note \(\dbinom{n}{k}\) the number of \myblue{combinations} of
\(k\) elements taken among \(n\).

For instance, \(\dbinom{5}{3} = 10\) because, assuming \(\{a, b, c, d,
e\}\), we can form the subsets
\[abc, \quad adb, \quad abe, \quad acd, \quad ace, \quad ade, \quad
bcd, \quad bce, \quad bde, \quad cde
\]

In general, the choice of the first element can be made among \(n\),
so \[\dbinom{n}{k} = n \dots\]

\end{slide}

% ------------------------------------------------------------------------
% 
\begin{slide}
\titre{Subset construction (continued)}

\raggedslides[0pt]

The second element can be chosen among \(n-1\) remining ones, but the
choice of the second element does not depend on the choice of the
first one, so
\[\dbinom{n}{k} = n \times (n-1) \times \dots\]
Until we choose the \(k\)-th element:
\[\dbinom{n}{k} = n (n-1) \dots (n - k + 1) \dots\]
But since there is no order among the \(k\) elements (it is a set), we
have to not count the repeated \myblue{permutations}.

In the example above, \(abc, bca, cab\) should be counted as one
combination, not three.

\end{slide}

% ------------------------------------------------------------------------
% 
\begin{slide}
\titre{Subset construction (continued)}

\raggedslides[0pt]

How many permutations of \(k\) elements are they? The first element
can be chosen among \(k\), the second among the \(k-1\) remaining ones
etc. So the number of permutations of \(k\) elements is \(k (k-1)
\dots 1 = k!\)

So, finally, the number of combinations of \(k\) elements chosen among
\(n\), without order is:
\[
\dbinom{n}{k} = \frac{n (n-1) \dots (n - k + 1)}{k (k-1) \dots 1} =
  \frac{n!}{k! (n-k)!}
\quad \text{where} \; n \geqslant k \geqslant 0
\]
So the number of subsets of a set of size \(n\) is the number of
subsets of size 0 plus the number of subsets of size 1 plus
etc. Formally, it is
\[
\binom{n}{0} + \binom{n}{1} + \dots + \binom{n}{n}=
\sum_{k=0}^{n}{\binom{n}{k}}
\]

\end{slide}

