%%-*-latex-*-

% ------------------------------------------------------------------------
%
\begin{frame}
\frametitle{Petite histoire de Caml}

\begin{itemize}

  \item [\textcolor{blue}{1975}] Robin Milner propose ML comme
    méta-langage (langage de script) pour l'assistant de preuve
    LCF. Il devient rapidement un langage de programmation
    indépendant.

  \item [\textcolor{blue}{1981}] Premières implantations de ML.

  \item [\textcolor{blue}{1985}] Création de Caml à l'INRIA et de
    Standard ML of New Jersey (Bell Labs).

  \item [\textcolor{blue}{1990}] Réalisation de Caml Light par
    X. Leroy et D. Doligez (INRIA).

  \item [\textcolor{blue}{1995}] Compilation vers du code natif et
    ajout d'un langage de modules.

  \item [\textcolor{blue}{1996}] Objets et classes (OCaml).

  \item [\textcolor{blue}{2001}] Arguments étiquetés et optionels;
    variantes polymorphes (J. Garrigue).

  \item [\textcolor{blue}{2002}] Méthodes polymorphes, bibliothèques
    partagées.

\end{itemize}

\end{frame}


%--------------------------------------------------------------------
%
\begin{frame}
\frametitle{Premiers pas}

\centerline{\textbf{Mode compilé}}

\textcolor{blue}{Éditer le ficher source \textsf{hello.ml}}

\phrase{print\_string \str{Hello world!\symbol{92}n}}

\textcolor{blue}{On supposera dans ce cours un \emph{shell} Unix dont l'invite
est \shellprompt{}.}

\textcolor{blue}{Compiler et exécuter}

\shellin{ocamlc -o hello hello.ml}\\
\shellin{./hello}\\
\topout{Hello world!}

\end{frame}

% ------------------------------------------------------------------------
%
\begin{frame}

\centerline{\textbf{Exemple de session interactive}}

\shellin{ocaml}\\
\topout{Objective Caml version 3.06}\\
\topin{print\_string "Hello world!\symbol{92}n"}\\
\topout{-~:~unit = ()}\\
\topin{let euro x = floor~(100.0~*.~x /.~6.55957)~*.~0.01}\\
\topout{val euro~:~float $\rightarrow$ float = $\topclos$}\\
\topin{let baguette = 4.20}\\
\topout{val baguette~:~float = 4.2}\\
\topin{euro baguette}\\
\topout{-~:~float = 0.64}\\
\topin{\#quit}\\
\shellprompt{}\ \_

\end{frame}

% ------------------------------------------------------------------------
%
\begin{frame}
\frametitle{Usage du mode interprété}

\textcolor{blue}{Comme une calculette}

\textcolor{blue}{Mise au point de programmes}

Évaluation des phrases du programme en cours de construction
(utilisation conjointe d'un éditeur).

\textcolor{blue}{Comme langage de script}

\begin{tabular}{ll}
    \shellin{ocaml < hello.ml}
  & équivalent à la version interactive\\
    \shellin{ocaml hello.ml} 
  & idem, mais suppression des messages 
\end{tabular}

\textbf{Conseil} \emph{Le mode interactif aide à la mise au point, mais
l'évaluation incrémentale des phrases peut être déroutante.}

\end{frame}

% ------------------------------------------------------------------------
%
\begin{frame}[containsverbatim]
\frametitle{Usage du mode interprété (suite et fin)}

On peut également écrire un script directement exécutable en indiquant
le chemin absolu de \textsf{ocaml} précédé de \texttt{\#!} sur la
première ligne.

\textcolor{blue}{Contenu de \textsf{hello}}

\begin{semiverbatim}
#!/usr/bin/ocamlrun /usr/bin/ocaml
print_string "Hello world!"; print_newline ();;
\end{semiverbatim}

\textcolor{blue}{Sous le \emph{shell} Unix}

\shellin{chmod +x hello}\\
\shellin{./hello}

\end{frame}

% ------------------------------------------------------------------------
%
\begin{frame}[containsverbatim]
\frametitle{Mode Ocaml pour \textsf{emacs}}

\textcolor{blue}{Installation}

Insérer à la fin du fichier \verb+~+\textsf{/.emacs}:

\begin{verbatim}
; OCaml mode
(setq auto-mode-alist
      (cons '("\\.ml[iylp]?$" . caml-mode) auto-mode-alist))
(autoload 'caml-mode "caml" "Major mode for editing Caml code." t)
(autoload 'run-caml "inf-caml" "Run an inferior Caml process." t)
(if window-system (require 'caml-hilit))
\end{verbatim}

\end{frame}

% ------------------------------------------------------------------------
%
\begin{frame}
\frametitle{Mode Ocaml pour \textsf{emacs} (suite)}

\textcolor{blue}{Mode compilé}

Le mode OCaml est choisi automatiquement lors de l'ouverture d'un
fichier ayant pour suffixe \textsf{.mli} ou \textsf{.ml}.

Pour chaque compilation, sélectionnez \texttt{compile} dans le menu
\textsl{Caml} (modifier s'il le faut la commande proposée).

\textcolor{blue}{Mode interactif}

Ouvrir un fichier \textsf{hello.ml}, y écrire des phrases, puis
exécuter la commande \texttt{eval-phrase} dans le menu \textsl{Caml}
(éventuellement appeler \texttt{start-subshell});

\end{frame}


% ------------------------------------------------------------------------
%
\begin{frame}
\frametitle{Phrases et programmes}

Une \emph{phrase} est définie par les cas suivants (où $e$ dénote une
expression, $f$ une variable et $\ipat{p}$ un motif irréfutable), où
\Xlet et \Xrec sont des mots-clés:

\begin{tabular}{rll}
    $\bullet$ 
  & \textcolor{blue}{définition globale}
  & \phrase{$\Xlet \,\, \ipat{p} = e$}\\
    $\bullet$
  & \textcolor{blue}{définition globale récursive}
  & \phrase{$\Xlet \,\, \Xrec \,\, f = e$}
%  & \phrase{$\textsf{let} \,\, \Xrec \,\, f_1 = e_1 \,\,
%      \Xand \, \ldots \, \Xand \,\, f_n  = e_n$}
\end{tabular}

Les phrases sont terminées par «~\textsf{;;}~» et un \emph{programme}
est une suite de phrases.

La fonction d'une phrase est d'associer des expressions à des
variables (une expression par variable): on parle de \emph{liaison}
(p.ex. $f \mapsto e$). Un sous-programme définit donc un ensemble de
liaisons appelé \emph{environnement}.

%% {\small
%% \begin{verbatim}
%% (* Ceci est un commentaire (* et ceci un commentaire
%% imbriqué *) sur plusieurs lignes *)
%% \end{verbatim}
%% }

\end{frame}

% ------------------------------------------------------------------------
%
\begin{frame}[containsverbatim]
\frametitle{Les expressions}

Une expression $e$ est définie \emph{récursivement} par les cas
suivants:

%%  \\
%%     $\bullet$
%%   & \textcolor{blue}{filtre}
%%   & \textsf{match} $e$ \textsf{with} $p_1 \rightarrow e_1
%%     \mid \ldots \mid p_n \rightarrow e_n$\\
%%     $\bullet$
%%   & \textcolor{blue}{définition locale}
%%   & \textsf{let} $\ipat{p}_1 = e_1 \,\, \textrm{[}\Xand \,\,
%%     \ipat{p}_2 = e_2 \ldots\textrm{]}$ \textsf{in} $e$\\ 

\begin{tabular}{rll}
    $\bullet$
  & \textcolor{blue}{variable}
  & $f, g, h$ (fonctions) et $x, y, z$ (autres). \\
    $\bullet$
  & \textcolor{blue}{fonction} (ou \textcolor{blue}{abstraction})
  & $\Xfun \,\, \ipat{p} \rightarrow e$\\
    $\bullet$
  & \textcolor{blue}{appel} (ou \textcolor{blue}{application})
  & $e_1 \,\, e_2$ \\ 
    $\bullet$
  & \textcolor{blue}{constante}
  & \unit ou \Xtrue ou \textsf{0} ou \textsf{1.2}
    ou \textsf{'a'} ou \textsf{"abc"} etc.\\
    $\bullet$
  & \textcolor{blue}{n-uplet}
  & $e_1, \ldots, e_n$\\
    $\bullet$
  & \textcolor{blue}{conditionnelle}
  & \Xif $e_0$ \Xthen $e_1$ \Xelse $e_2$\\
    $\bullet$
  & \textcolor{blue}{parenthèse}
  & $(e)$\\
    $\bullet$
  & \textcolor{blue}{définition locale}
  & $\Xlet \,\, \ipat{p}_1 = e_1 \,\, \Xin \,\, e_2$
\end{tabular}

\begin{tabular}{ll}
    \remarques 
  & Ce que nous écrivons joliment $\rightarrow$ s'écrit \verb+->+ en
    \textsc{ascii}.\\
  & Les parenthèses sont recommandées autour des n-uplets.
\end{tabular}

\end{frame}

% ------------------------------------------------------------------------
%
\begin{frame}
\frametitle{Les motifs irréfutables}

Un \emph{motif irréfutable} $\ipat{p}$ est défini \emph{récursivement}
par les cas suivants:

\begin{tabular}{rll}
    $\bullet$
  & \textcolor{blue}{variable}
  & $f, g, h$ (fonctions) et $x, y, z$ (autres). \\
    $\bullet$
  & \textcolor{blue}{unité}
  & \unit\\
    $\bullet$
  & \textcolor{blue}{n-uplet}
  & $\ipat{p}_1, \ldots, \ipat{p}_n$\\
    $\bullet$
  & \textcolor{blue}{parenthèse}
  & $(\ipat{p})$\\
    $\bullet$
  & \textcolor{blue}{joker}
  & {\Large \_}
\end{tabular}

\remarques

\begin{itemize}

  \item Les parenthèses sont recommandées autour des n-uplets.

  \item Du point de vue syntaxique, mis à part le joker, les motifs
  irréfutables sont des cas particuliers d'expressions.

%%  \item Un joker est un cas spécial pour que la phrase ne crée pas
%%  de liaison.

\end{itemize}

\end{frame}

% ------------------------------------------------------------------------
%
\begin{frame}
\frametitle{Un programme correct}

\phrase{1+4}\\
\phrase{let x = 0}\\
\phrase{let id = fun x -> x}\\
\phrase{let rec fact = fun n -> if n = x then 1 else n * fact (n-1)}\\
\phrase{let (x) = 6}\\
\phrase{x+1}

%\remarques

\begin{itemize}

  \item Les variables commencent par une minuscule.

  \item L'avant-dernière phrase redéfinit la liaison de \ident{x} qui
  est finalement lié à \num{6} \emph{après cette phrase}. Avant
  celle-ci, il est lié à \num{0} (cf. \ident{fact}).

\end{itemize}

\end{frame}

% ------------------------------------------------------------------------
%
\begin{frame}
\frametitle{Évaluation des fonctions récursives}

Lorsque l'on évalue une fonction récursive, il faut changer un peu la
règle vue page~\pageref{semantique_let_in}. Elle devient (différence
en bleu):

\begin{tabular}{rll}
    $\bullet$
  & $\Xlet \,\, \Xrec \,\, f = e_1 \,\, \Xin \,\, e_2$
  & Évaluer $e_1$ en $v_1$ dans $\rho$;\\
  &
  & on doit avoir $v_1 = \langle\Xfun \,\, x \rightarrow
  e,\rho'\rangle$;\\ 
  & 
  & évaluer $e_2$ en $v$ dans $(f \mapsto v_1) \oplus \rho$.
\end{tabular}

XXXXX A REVOIR XXXXX

\end{frame}

% ------------------------------------------------------------------------
%
\begin{frame}
\frametitle{Les opérateurs}

Un opérateur est une fonction qui est représentée par un symbole
plutôt que par un nom, par exemple $+$ et $||$.

Les opérateurs sont \emph{infixes}, c.-à-d. qu'ils sont placés entre
leurs arguments, contrairement aux fonctions qui sont \emph{préfixes},
c.-à-d. placées avant leurs arguments. La nouvelle syntaxe des
expressions est alors:

\begin{tabular}{rll}
    $\bullet$
  & \textcolor{blue}{application complète d'opérateur}
  & $e_1 \,\, \textsf{\&\&} \,\, e_2$ \ \ ou \ \ $\textsf{not} \,\, e$
     \,\, etc.
\end{tabular}

Pour permettre un usage préfixe des opérateurs, nous étendons encore:

\begin{tabular}{rll}
    $\bullet$
  & \textcolor{blue}{application}
  & $f \,\, e$ \ \ ou \ \ $(r) \, e$
\end{tabular}

où $r$ est le symbole d'un opérateur. Ainsi \textcolor{blue}{\textsf{(+)} $e_1$
$e_2$} est équivalent à \textcolor{blue}{$e_1$ \textsf{+} $e_2$}. Mais quel est
le sens de \textcolor{blue}{$(+) \, e_1$}? Nous y reviendrons quand nous aborderons
l'évaluation.

\end{frame}

% ------------------------------------------------------------------------
%
\begin{frame}
\frametitle{Types, constantes et primitives (suite)}

\textcolor{blue}{Cas polymorphe}

\begin{tabular}{|l|l|c|}
\hline
     tableaux
   & \textsf{[|0;1;2|]}, \textsf{[|'a';'b'|]}
   & \textsf{a.(i)} \ \textsf{a.(i) $\leftarrow$ v}\\
    n-uplets
  & \textsf{(1,2)}, \textsf{(true,false,true)}
  & \\ %\textsf{fst} \, \textsf{snd}\\
    listes
  & \textsf{[1;2;3]}, \textsf{["a";"b"]}
  & \textsf{List.hd} \ \textsf{List.tl}\\
\hline
\end{tabular}

\end{frame}

% ------------------------------------------------------------------------
%
\begin{frame}
\frametitle{Exemples de fonctionnelle}

\textcolor{blue}{La composition de deux fonctions est une fonction}

\phrase{let compose f g = fun x -> f (g (x))}

\textcolor{blue}{Dérivation numérique}

\topin{let der dx f = fun x -> (f(x+.dx) -. f(x)) /. dx}\\
\topout{val der : float $\rightarrow$ (float $\rightarrow$ float)
$\rightarrow$ float $\rightarrow$ float= $\topclos$} 

\end{frame}

% ------------------------------------------------------------------------
%
\begin{frame}
\frametitle{Application}

\textbf{Exercice 1} \ Quelle différence entre $f \, x \, y$, $(f \, x) \,
y$, $f \, (x \, y)$ et $(f \, (x)) \, (y)$?

\textcolor{blue}{Application partielle}

On peut ne passer qu'un argument à une fonction qui en attend deux:

\topin{let plus x y = x + y}\\
\topout{val plus~:~int $\rightarrow$ int $\rightarrow$ int}\\
\topin{let incr = plus 1}\\
\topout{val incr~:~int $\rightarrow$ int}

\remarque

On aurait pu écrire \phrase{let plus = ( + )}\\
Mettre des espaces autour du symbole d'opérateur:\\
essayez \topin{let mult = (*)} \ et \ \topin{let mult = ( * )}

\end{frame}

% ------------------------------------------------------------------------
%
\begin{frame}[containsverbatim]
\frametitle{Les enregistrements}

Ce sont les \textsf{record} et \textsf{struct} de Pascal et C.

Il faut les déclarer avant de les utiliser.

\begin{verbatim}
type monnaie = {nom : string; valeur : float}
let x = {nom = "euro"; valeur = 6.55957}
let v = x.valeur
\end{verbatim}

\textcolor{blue}{Champs polymorphes}

\topin{type 'a info = \{nom : string; info : 'a\}}\\
\topout{type 'a info = \{nom : string; info : 'a\}}\\
\topin{let proj = fun x $\rightarrow$ x.valeur}\\
\topout{val proj : 'a info $\rightarrow$ 'a}

\end{frame}


% ---------------------------------------------------------------------
%
\begin{frame}[containsverbatim]
\frametitle{Les motifs}

Un \emph{motif} $p$ est défini récusivement par les cas:

\begin{tabular}{rll}
    $\bullet$
  & \textcolor{blue}{variable}
  & $x,y,z$\\
    $\bullet$
  & \textcolor{blue}{constantes simples}
  & \unit ou \Xtrue ou \textsf{0} ou \textsf{1.2}
    ou \textsf{'a'} ou \textsf{"abc"} etc.\\
    $\bullet$
  & \textcolor{blue}{n-uplets et listes}
  & $(p_1, \ldots, p_n)$ et $[p_1; \ldots; p_n]$\\
    $\bullet$
  & \textcolor{blue}{parenthèse}
  & $(p)$\\
    $\bullet$
  & \textcolor{blue}{joker}
  & \_
\end{tabular}

\end{frame}

% ------------------------------------------------------------------------
%
\begin{frame}[containsverbatim]
\frametitle{Listes}

Autre définition possible (avec un \textcolor{blue}{enregistrement}):

{\small
\begin{verbatim}
type 'a liste = Vide | Cellule of 'a cell
and 'a cell = {hd : 'a; tl : 'a liste}
\end{verbatim}
}

\textcolor{blue}{Listes modifiables en place}:

{\small
\begin{verbatim}
type 'a liste = Vide | Cellule of 'a cell
and 'a cell = {mutable hd : 'a; mutable tl : 'a liste}
\end{verbatim}
}

\end{frame}

% ------------------------------------------------------------------------
%
\begin{frame}[containsverbatim]
\frametitle{Le jeu de cartes}

Il combine types somme et types enregistrement.

{\small
\begin{verbatim}
type carte = Carte of ordinaire | Joker
and ordinaire = {couleur: couleur; figure: figure}
and couleur = Coeur | Carreau | Pique | Trefle
and figure = As | Roi | Reine | Valet | Simple of int
\end{verbatim}
}

Définition de fonctions auxiliaires:

\topin{let valet\_de\_pique = Carte \{couleur=Pique; figure=Valet\}}\\
\topout{val valet\_de\_pique : carte = \{couleur=Pique;
  figure=Valet\}}\\
\topin{let carte f c = Carte \{couleur=c;figure=f\}\\
let roi = carte Roi}\\
\topout{val carte : figure $\rightarrow$ couleur $\rightarrow$ carte =
  $\topclos$\\
val roi : couleur $\rightarrow$ carte = $\topclos$}
 
\end{frame}

% ------------------------------------------------------------------------
%
\begin{frame}[containsverbatim]
\frametitle{Filtrage des cartes}

{\small
\begin{verbatim}
let valeur = function
  Carte {figure = As}       -> 14
| Carte {figure = Roi}      -> 13
| Carte {figure = Reine}    -> 12
| Carte {figure = Valet}    -> 11
| Carte {figure = Simple k} -> k
| Joker                     -> 0
\end{verbatim}
}

Un motif peut capturer plusieurs cas:

{\small
\begin{verbatim}
let petite = function
  Carte {figure = Simple _} -> true
| _ -> false
\end{verbatim}
}

\end{frame}

% ------------------------------------------------------------------------
%
\begin{frame}
\frametitle{Les motifs}


Ils peuvent

\begin{itemize}
  \item être emboîtés, contenir des produits et des sommes;

  \item ne pas énumérer tous les champs d'un enregistrement;

  \item lier une valeur intermédiaire avec le mot-clé \textsf{as};
 
  \item n'imposer aucune contrainte (caractère \textsf{\_}).
\end{itemize}

\topin{function Carte (\{figure=Simple \_\} as z) -> z.couleur}\\
\error{Warning: this pattern-matching is not exhaustive.\\
Here is an example of a value that is not matched:\\
Joker\\
- : carte $\rightarrow$ couleur = $\topclos$}

\end{frame}

% ------------------------------------------------------------------------

%
\begin{frame}
\frametitle{Le type option}

\centerline{\textbf{À FAIRE}}

\end{frame}

------------------------------------------------------------------------
%
\begin{frame}
\frametitle{Exemple: la commande Unix \textsf{cat}}

Voici une version restreinte:

\begin{verbatim}
try
  while true do
    print_string (read_line()); 
    print_newline ()
  done
with End_of_file -> ()
\end{verbatim}

\end{frame}

% ------------------------------------------------------------------------
%
\begin{frame}
\frametitle{Solution de l'exercice~3}

\begin{verbatim}
let last = Array.length (Sys.argv) - 1 in
  let print_args (first) =
    begin
      if first < last then print_string Sys.argv.(first) else ();
      for i = first + 1 to last do
        print_char ' '; print_string Sys.argv.(i)
      done
    end
in if last > 0 && Sys.argv.(1) = "-n"
   then print_args (2)
   else begin print_args (1); print_newline () end
\end{verbatim}

\end{frame}

% ------------------------------------------------------------------------
%
\begin{frame}
\frametitle{Solution de l'exercice~4}

Programme \Xtrue: \phrase{exit 0}

Programme \Xfalse: \phrase{exit 1}

Pour les compiler:

\shellin{ocamlc -o true true.ml}\\
\shellin{ocamlc -o false false.ml}

Pour les tester:

\shellin{./true}\\
\shellin{echo \$?}\\
\shellin{./false}\\
\shellin{echo \$?}

\end{frame}

---------------------------------------------------------------------
%
\begin{frame}
\frametitle{Tableaux}

Les opérations sur les tableaux sont polymorphes, c'est-à-dire que
l'on peut créer des tableaux \insist{homogènes} d'entiers, de booléens
etc. 

\begin{center}
\begin{tabular}{rcl}
  \textsf{[|0 ; 1 ; 3|]} & : & \textsf{int array}\\
  \textsf{[| true ; false |]} & : & \textsf{bool array}
\end{tabular}
\end{center}

Les indices de tableaux varient de $0$ à $n-1$, où $n$ est la taille
du tableau.

Les projections sont polymorphes: elles opèrent aussi bien sur des
tableaux d'entiers que de booléens etc.

\begin{center}
\begin{tabular}{rcl}
    \textsf{fun x $\rightarrow$ x.(0)} 
  & :
  & \textsf{$\alpha$ array $\rightarrow$ $\alpha$}\\
    \textsf{fun t k x $\rightarrow$ t.(k) $\leftarrow$ x}
  & :
  & \textsf{$\alpha$ array $\rightarrow$ int $\rightarrow$ $\alpha$
       $\rightarrow$ unit}
\end{tabular}
\end{center}

Les tableaux sont toujours initialisés par \textsf{Array.create : int
$\rightarrow \alpha \rightarrow \alpha$ array} où le premier
argument est la taille et le second est la valeur initiale des cases.

\end{frame}


% ---------------------------------------------------------------------
%
\begin{frame}
\frametitle{Exemples de tableaux}

\topin{let chr = Array.create 5 ' '}\\
\topout{val chr : char array = [|' ';' ';' ';' ';' '|]}\\
\topin{for i = 0 to 4 do chr.(i) $\leftarrow$ Char.chr (Char.code '0' + i) done}\\
\topout{- : unit = ()}
\topin{let chiffre_trois = chr.(3)}
\topout{val chiffre_trois : char = '3'}

Polymorphisme

\topin{Array.set}\\
\topout{- : 'a array $\rightarrow$ int $\rightarrow$ 'a $\rightarrow$ unit = \topclos}\\
\topin{Array.set chr}\\
\topout{- : int $\rightarrow$ char $\rightarrow$ unit = \topclos}\\

\end{frame}

------------------------------------------------------------------------
%
\begin{frame}
\frametitle{\'{E}galité (suite)}

En réalité il existe un second opérateur d'égalité:

\textcolor{blue}{\'{E}galité physique} (notée \textsf{==}, et \textsf{!=} pour la
différence)

C'est l'égalité des emplacements mémoire.

\topin{[1] != [1] \&\& \str{non} != \str{non}\\ \&\& (let x = ref 1 in x
== (fun y -> y) x)}\\
\topout{- : bool = true}

Pour les valeurs non allouées (dans le tas), les deux égalités
coïncident:

\topin{1 == 1 \&\& 'a' == 'a'}\\
\topout{- : bool = true}

\end{frame}


------------------------------------------------------------------------
%
\begin{frame}
\frametitle{La ligne de commande}

\textcolor{blue}{Les arguments passés à un exécutable}

Ils sont placés dans le tableau \textsf{Sys.argv : string array}

Le nom de la commande est compté.

\textcolor{blue}{Exemple} \ La commande Unix \textsf{echo}:

\begin{verbatim}
for i = 1 to Array.length (Sys.argv) - 1 do
  print_string Sys.argv.(1); print_char ' '
done
print_newline ()
\end{verbatim}

\textcolor{blue}{Exercice 3} \ Corrigez le programme précédent pour qu'il accepte
l'option \str{-n} qui supprime l'impression de la fin de
ligne. Supprimez également l'espace finale.

\end{frame}

------------------------------------------------------------------------
%
\begin{frame}
\frametitle{Retourner un résultat}

\textcolor{blue}{La fonction} \textsf{exit : int $\rightarrow$ unit}

Elle arrête l'exécution du programme et retourne au système
d'exploitation la valeur entière reçue en argument (modulo 256). Par
défaut une exécution retourne 0 si elle se termine normalement, et une
valeur non nulle autrement. (Sous Unix on peut observer la valeur de
retour des commandes shell avec: \textsf{echo \$?})

\textcolor{blue}{Exercice 4} \ \'{E}crire deux programmes \Xtrue et
\Xfalse qui se comportent comme les commandes Unix
\textsf{/bin/true} et \textsf{/bin/false} qui ne font rien et
retournent respectivement 0 et 1.

\end{frame}

---------------------------------------------------------------------
%
\begin{frame}
\frametitle{Les champs mutables}

On peut déclarer des champs mutables dans les définitions
d'enregistrements. 

\textsf{type personne = \{nom : string; mutable age : int\}}

permet de modifier (en place) la seconde composante, mais pas la
première.

\topin{let p = \{nom="Paul"; age=23\}}\\
\topout{val p : personne = \{nom="Paul"; age=23\}}\\
\topin{let anniversaire x = x.age <- x.age + 1}\\
\topout{val anniversaire : personne $\rightarrow$ unit = \topclos}\\
\topin{p.nom $\leftarrow$ "Pierre"}\\
\error{Characters 0-17}\\
\error{The label nom is not mutable}

\end{frame}

---------------------------------------------------------------------
%
\begin{frame}
\frametitle{Le type option}

\textcolor{blue}{Les références sont toujours initialisées.} Parfois il est
nécessaire de créer des références qui seront vraiment initialisées
plus tard. Ce traitement doit être explicite.

\begin{verbatim}
type 'a option = None | Some of 'a  (* Prédéfini *)
let statut = ref None
\end{verbatim}

Plus tard on pourra définir 

\phrase{statut := Some 3}

\textbf{Mais} le filtrage de \texttt{statut} doit toujours inclure le cas
où il est indéfini (c-à-d. valant \texttt{None}):

\phrase{match !statut with Some n -> n | None -> 0}

\end{frame}

% ------------------------------------------------------------------------
%
\begin{frame}
\frametitle{Les entrées/sorties}

\begin{tabular}{|rl|}
\hline
\multicolumn{2}{|c|}{Canaux prédéfinis}\\
\hline
\textsf{stdin} & \textsf{: in\_channel}\\
\textsf{stdout} & \textsf{: out\_channel}\\
\textsf{stderr} & \textsf{: out\_channel}\\
\hline
\end{tabular} \hspace*{5mm}
\begin{tabular}{|rl|}
\hline
\multicolumn{2}{|c|}{Ouverture de canaux}\\
\hline
\textsf{open\_out} & \textsf{: string $\rightarrow$ out\_channel}\\
\textsf{open\_in} & \textsf{: string $\rightarrow$ in\_channel}\\
\textsf{close\_out} & \textsf{: out\_channel $\rightarrow$ unit}\\
\textsf{close\_in} & \textsf{: in\_channel $\rightarrow$ unit}\\
\hline
\end{tabular}

\begin{tabular}{|rl|}
\hline
\multicolumn{2}{|c|}{Lecture sur \textsf{stdin}}\\
\hline
\textsf{read\_line} & \textsf{: unit $\rightarrow$ string}\\
\textsf{read\_int} & \textsf{: unit $\rightarrow$ int}\\
\hline
\end{tabular} \hspace*{5mm}
\begin{tabular}{|rl|}
\hline
\multicolumn{2}{|c|}{\'{E}criture sur \textsf{stdout}}\\
\hline
\textsf{print\_string} & \textsf{: string $\rightarrow$ unit}\\
\textsf{print\_int} & \textsf{: int $\rightarrow$ unit}\\
\hline
\end{tabular}

\end{frame}

------------------------------------------------------------------------
%
\begin{frame}
\frametitle{Les entrées/sorties (suite)}

Pour plus de fonctionnalités, voir
\begin{itemize}

  \item la bibliothèque noyau (\emph{The Core Library});

  \item la bibliothèque \textsf{Printf} avec laquelle on écrira,
        comme en C:\\
        \excerpt{Printf.printf \str{\%s \%d\symbol{92}n} \str{Année} 2003}

\end{itemize}

\end{frame}

------------------------------------------------------------------------
%
\begin{frame}
\frametitle{Les champs mutables}

On peut déclarer des champs mutables dans les définitions
d'enregistrements.

\phrase{type personne = \{nom : string; mutable age : int\}}

\topin{let p = \{nom="Paul"; age=23\}}\\
\topout{val p : personne = \{nom="Paul"; age=23\}}\\
\topin{let anniversaire x = x.age <- x.age + 1}\\
\topout{val anniversaire : personne $\rightarrow$ unit = $\topclos$}\\
\topin{p.nom <- "Paul"}\\
\error{Characters 0-17\\ The label nom is not mutable}

\end{frame}

------------------------------------------------------------------------
%
\begin{frame}
\frametitle{Les références}

Le passage d'arguments se fait toujours par valeur.

Il n'existe pas de constructions telles que \verb+&x+ ou \verb+*x+ en
C qui retournent l'adresse à laquelle se trouve une valeur ou la
valeur qui se trouve à une adresse: \insist{la valeur d'une variable
n'est pas modifiable.}


Mais on peut modifier les champs \emph{mutables} d'un
enregistrement.

\phrase{type $\alpha$ ref = \{mutable contents : $\alpha$\} (* Prédéfini *)}

\begin{center}
\begin{tabular}{l|l|l}
  Langage C & OCaml & OCaml raccourci\\
  \hline 
  \verb+int x = 0;+ & \verb+let x = {contents=0}+ & \verb+let x = ref 0+\\
  \verb|x = x + 1;| & \verb|x.contents <- x.contents + 1| &
  \verb|x := !x + 1|
\end{tabular}
\end{center}

\insist{Peu de valeurs ont besoin d'être des références en Caml.}

\end{frame}
