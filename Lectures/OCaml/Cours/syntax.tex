%%-*-latex-*-

% ------------------------------------------------------------------------
%
\begin{frame}
\frametitle{Représentation arborescente des programmes mini-ML}
\label{arbres_de_prog}

Comme dans n'importe quel langage de programmation, avant d'aborder
l'exécution des programmes mini-ML, nous devons préciser la notion de
\emph{portée des variables}, c.-à-d. à quoi se refère une variable
donnée. Pour cela, une représentation graphique des programmes (les
expressions suffisent) sous forme d'arbres est très commode.
\begin{center}
\begin{tabular}{l|c}
    Expression & Arbre\\
 \hline
    $x$
  & $x$\\
    $\Xfun \,\, x \rightarrow e$
  & \pstree[nodesep=2pt,levelsep=15pt,treesep=10pt]{\TR{\Xfun}}{
      \TR{$x$}
      \TR{$e$}
    } \\
    $e_1 \,\, e_2$ 
  & \pstree[nodesep=2pt,levelsep=15pt,treesep=10pt]{\TR{\$}}{
      \TR{$e_1$}
      \TR{$e_2$}
    }
\end{tabular}
\end{center}

\end{frame}

% ------------------------------------------------------------------------
%
\begin{frame}
\frametitle{Représentation arborescente des programmes mini-ML (suite et fin)}

\begin{center}
\begin{tabular}{l|c}
    Expression & Arbre\\
    \hline
    $e_1$ \texttt{+} $e_2$ etc.
  & \pstree[nodesep=2pt,levelsep=20pt,treesep=10pt]{\TR{\texttt{+}}}{
      \TR{$e_1$}
      \TR{$e_2$}
    }\\
    \textsf{0} ou \textsf{1} ou \textsf{2} etc.
  & \textsf{0} ou \textsf{1} ou \textsf{2} etc.\\
    $\lpar{e}\rpar$
  & $e$\\
    $\Xlet \,\, x = e_1 \,\, \Xin \,\, e_2$
  & \pstree[nodesep=2pt,levelsep=20pt,treesep=10pt]{\TR{\Xlet}}{
      \TR{$x$}
      \TR{$e_1$}
      \TR{$e_2$}
    }
\end{tabular}
\end{center}

\end{frame}

% ------------------------------------------------------------------------
%
\begin{frame}
\frametitle{Construction des arbres de programme}

Intuitivement, la méthode générale consiste d'abord à parenthèser
complètement l'expression qui fait le programme.

\bigskip

\noindent Chaque parenthèse correspond à une sous-expression et chaque
sous-expression correspond à un sous-arbre.

\bigskip

On construit l'arbre des feuilles vers la racine en parcourant les
sous-expressions parenthèsées les plus imbriquées vers les plus
externes.

\end{frame}

% ------------------------------------------------------------------------
%
\begin{frame}[containsverbatim]
\frametitle{Exemples d'arbres de programmes}

L'expression \verb|(1+2)*(5/1)| se représente
    {\small
    \begin{center}
      \pstree[nodesep=2pt,levelsep=18pt]{\TR{\texttt{*}}}{
        \pstree{\TR{\texttt{+}}}{
          \TR{\num{1}}
          \TR{\num{2}}
        }
        \pstree{\TR{\texttt{/}}}{
          \TR{\num{5}}
          \TR{\num{1}}
          }
      }
    \end{center}
    }
et \verb|let x = 1 in (1+2)*(5/1)| devient (attention: \verb+x+ et $x$
sont différents) 
    {\small
    \begin{center}
      \pstree[nodesep=2pt,levelsep=18pt]{\TR{\Xlet}}{
        \TR{\ident{x}}
        \TR{\num{1}}
        \pstree{\TR{\texttt{*}}}{
          \pstree{\TR{\texttt{+}}}{
            \TR{\num{1}}
            \TR{\num{2}}
          }
          \pstree{\TR{\texttt{/}}}{
            \TR{\num{5}}
            \TR{\num{1}}
            }
        }
      }
    \end{center}
    }

\end{frame}

% ------------------------------------------------------------------------
%
\begin{frame}[containsverbatim]
\frametitle{Exemples d'arbres de programmes (suite)}
\label{exemples_arbres_suite}

L'expression \verb|let x = 1 in ((let x = 2 in x) + x)| est
{\small
\begin{center}
\pstree[nodesep=2pt,levelsep=18pt]{\TR{\Xlet}}{
  \TR{\ident{x}}
  \TR{\num{1}}
  \pstree{\TR{\texttt{+}}}{
    \pstree{\TR{\Xlet}}{
      \TR{\ident{x}}
      \TR{\num{2}}
      \TR{\ident{x}}
    }
    \TR{\ident{x}}
  }
}
\end{center}
}
Qu'en est-il de \verb|fun y -> x + (fun x -> x) y|? Et de
{\small
\begin{verbatim}
let x = 1 in
  let f = fun y -> x + y in
  let x = 2
in f(x)
\end{verbatim}
}
Quid du programme page~\pageref{un_autre_programme}?

\end{frame}

% ------------------------------------------------------------------------
%
\begin{frame}[containsverbatim]
\frametitle{Liaison statique et environnement}
\label{exemple_liaisons}

Une phrase associe une expression $e$ à une variable $x$: on parle de
\emph{liaison}, notée $x \mapsto e$. Un sous-programme définit donc un
ensemble de liaisons appelé \emph{environnement}. 

\bigskip

Une liaison est \emph{statique} si l'on peut déterminer à la
compilation (c.-à-d. en examinant le code source) à quelle expression
une variable donnée fait référence. Par exemple dans {\small
\begin{verbatim}
let x = 0 in
  let id = fun x -> x in
  let y = id (x) in
  let x = (fun x -> fun y -> x + y) 1 2 
in x+1;;
\end{verbatim}
}
à quelle expression fait référence \verb+x+ dans \verb|x+1|?

\end{frame}

% ------------------------------------------------------------------------
%
\begin{frame}[containsverbatim]
\frametitle{Liaison statique et environnement (suite et fin)}

Les liaisons sont ordonnées dans l'environnement \emph{par ordre de
définition}. Ainsi
\begin{enumerate}

  \item l'environnement est initialement vide: \{\}

  \item après {\small \verb|let x = 0 in |} il vaut $\{\ident{x} \mapsto
  0\}$

  \item après {\small \verb|let id = fun x -> x in |} il vaut
  $\{\ident{id} \mapsto \Xfun \,\, \ident{x} \rightarrow \ident{x}; \ident{x}
  \mapsto 0\}$

  \item après {\small \verb|let y = id (x) in |} il vaut $\{\ident{y}
  \mapsto \ident{id} \, \lpar\ident{x}\rpar; \ident{id} \mapsto \Xfun \,\,
  \ident{x} \rightarrow \ident{x}; \ident{x} \mapsto 0\}$

  \item après {\small \verb|let x = ...|} il vaut $\{\textcolor{blue}{\ident{x}
  \mapsto \ldots}; \ident{y} \mapsto \ident{id} \, \lpar\ident{x}\rpar; \ident{id}
  \mapsto \Xfun \,\, \ident{x} \rightarrow \ident{x}; \textcolor{blue}{\ident{x} \mapsto
  0}\}$

\end{enumerate}

\textcolor{blue}{La liaison $\ident{x} \mapsto 0$ est donc cachée, ou hors de
portée, dans \texttt{x+1}.}

\bigskip

On notera $\rho(x)$ la première liaison de $x$ dans l'environnement
$\rho$ (si elle existe).

\end{frame}

% ------------------------------------------------------------------------
%
\begin{frame}
\frametitle{Variables libres et représentation graphique des liaisons dans une expression}

La définition locale \Xlet{} $x$ = $e_1$ \Xin{} $e_2$ lie $e_1$ à $x$,
noté $x \mapsto e_1$, dans $e_2$. Il se peut que dans $e_2$ une autre
définition locale lie la même variable...

\bigskip

Pour y voir plus clair on applique le procédé suivant sur l'arbre de
programme. À partir de chaque occurrence de variable, remontons vers
la racine. Si nous trouvons un premier \Xlet{} liant cette variable,
créons un arc entre son occurrence et ce \Xlet. Si, à la racine, aucun
\Xlet{} n'a été trouvé, la variable est dite \emph{libre} dans
l'expression. On notera ${\cal L} (e)$ l'ensemble des variables libres
de $e$.  {\small
\begin{center}
\pstree[nodesep=2pt,levelsep=18pt]{\TR[name=outer-let]{\Xlet}}{
  \TR{\ident{x}}
  \TR{\num{1}}
  \pstree{\TR{\texttt{+}}}{
    \pstree{\TR[name=inner-let]{\Xlet}}{
      \TR{\ident{x}}
      \TR{\num{2}}
      \TR[name=inner-var]{\ident{x}}
    }
    \TR[name=outer-var]{\ident{x}}
  }
}
\ncarc[linecolor=blue,nodesepA=1pt,nodesepB=1pt,arcangle=-40]{->}{inner-var}{inner-let}
\ncarc[linecolor=blue,nodesepA=1pt,nodesepB=1pt,arcangle=-40]{->}{outer-var}{outer-let}
\end{center}
}

\end{frame}

% ------------------------------------------------------------------------
%
\begin{frame}[containsverbatim]
\frametitle{Variables libres d'une abstraction}

Une situation similaire se pose avec les fonctions \Xfun{} $x
\rightarrow e$: dans leur corps $e$ le paramètre $x$ cache une
éventuelle variable $x$ liée plus haut dans l'arbre. Il nous faut
alors considérer que \Xfun{} est un lieur comme \Xlet.

\bigskip

Reprenons \verb|fun y -> x + (fun x -> x) y|:
{\small
\begin{center}
\pstree[nodesep=2pt,levelsep=18pt]{\TR[name=outer-fun]{\Xfun}}{
  \TR{\ident{y}}
  \pstree{\TR{\texttt{+}}}{
    \textcolor{red}{
      \TR{\ident{x}}
    }
    \pstree{\TR{\$}}{
      \pstree{\TR[name=inner-fun]{\Xfun}}{
        \TR{\ident{x}}
        \TR[name=inner-var]{\ident{x}}
      }
      \TR[name=outer-var]{\ident{y}}
    }
  }
}
\end{center}
}
\nccurve[linecolor=blue,nodesepA=1pt,nodesepB=1pt,angleA=45]{->}{inner-var}{inner-fun}
\ncarc[linecolor=blue,nodesepA=1pt,nodesepB=1pt,arcangle=-40]{->}{outer-var}{outer-fun}
Quid des programmes pages~\pageref{un_autre_programme}
et~\pageref{exemples_arbres_suite}?

\end{frame}

% ------------------------------------------------------------------------
%
\begin{frame}[containsverbatim]
\frametitle{Expressions closes et évaluation}

Une expression \emph{close} est une expression sans variables
libres. Seul un programme clos peut être évalué (exécuté). En effet,
quel serait la valeur du programme réduit à la simple expression
\verb|x|?

C'est pourquoi la première analyse statique des compilateurs consiste
à déterminer les variables libres des expressions. Si le programme
n'est pas clos, il est rejeté. Dans le cas de \verb|x|, le compilateur
OCaml imprimerait 
\begin{center}
\texttt{Unbound value x}
\end{center}
(c.-à-d. «~Valeur \texttt{x} non liée~») et s'arrêterait. L'intérêt
est que cette expression non close est rejetée à la compilation et ne
provoque donc pas une erreur à l'exécution.

\end{frame}
