%-*-latex-*-

% ------------------------------------------------------------------------
%
\begin{frame}
\frametitle{Les types}

Un type $t$ est défini \emph{récursivement} par les cas:

\begin{tabular}{rll}
    $\bullet$
  & \textcolor{blue}{simple}
  & \textsf{char}, \textsf{bool}, \textsf{int}, \textsf{string},
    \textsf{float}, \textsf{unit}\\
    $\bullet$
  & \textcolor{blue}{produit cartésien}
  & $t_1 \times \ldots \times t_n$\\
    $\bullet$
  & \textcolor{blue}{fonctionnel}
  & $t_1 \rightarrow t_2$\\
    $\bullet$
  & \textcolor{blue}{parenthésé}
  & $(t)$\\
    $\bullet$
  & \textcolor{blue}{variable libre}
  & $\alpha$, $\beta$, $\gamma$ etc. \\
    $\bullet$
  & \textcolor{blue}{type paramétré}
  & $\alpha$ \textsf{list}
\end{tabular}

\bigskip

\remarque 

Jusqu'à présent, nous n'avons pas rencontré de valeurs de type
\textsf{float}, \textsf{char} ou \textsf{string}.

\end{frame}


% ------------------------------------------------------------------------
%
\begin{frame}
\frametitle{Règles syntaxiques sur les types}

\begin{itemize}

  \item Nous notons $\times$ ce qui s'écrit \textsf{*} en
  \textsc{ascii}.

   \item Nous notons $\alpha$, $\beta$ etc. ce qui s'écrit
     respectivement \textsf{'a}, \textsf{'b} etc. en \textsc{ascii}.

  \item Le produit cartésien est n-aire, et non binaire comme en
    mathématiques, car $\times$ n'est pas associatif en OCaml: $t_1
    \times t_2 \times t_3 \neq (t_1 \times t_2) \times t_3 \neq t_1
    \times (t_2 \times t_3)$

  \item La flèche est utilisée aussi dans les expressions. Elle
    associe à droite: $t_1 \rightarrow t_2 \rightarrow \ldots
    \rightarrow t_n$ \ \ \ équivaut à \ \ \ $t_1 \rightarrow
    \textcolor{blue}{(}t_2 \rightarrow \textcolor{blue}{(} \ldots
    \textcolor{blue}{(}t_{n-1} \rightarrow t_n\textcolor{blue}{))
      \ldots )}$

  \item Le produit cartésien est prioritaire sur la flèche:\\ $t_1
    \times t_2 \rightarrow t_3$ \ \ \ équivaut à
    \ \ \ $\textcolor{blue}{(}t_1 \times t_2\textcolor{blue}{)}
    \rightarrow t_3$

\end{itemize}

\end{frame}

% ------------------------------------------------------------------------
%
\begin{frame}[containsverbatim]
\frametitle{Types, constantes simples et primitives}

Les compilateur associe un type à chaque expression du programme: on
parle d'\emph{inférence de types statique}. Pour les constantes
simples, nous avons
\begin{center}
\begin{tabular}{|l|l|c|}
\hline
    \textsf{unit}
  & \unit
  & \\
    \textsf{bool}
  & \Xtrue \ \Xfalse
  & \texttt{\&\&} \ \texttt{||} \ \texttt{not} \\
    \textsf{int}
  & \textsf{1} \, \textsf{2} \ \textsf{max\_int} \ etc.
  & \texttt{+} \ \texttt{-} \ \texttt{*} \ \texttt{/} \ etc.\\
    \textsf{float}
  & \textsf{1.0} \, \textsf{2.} \ \textsf{1e4} \ etc.
  & \texttt{+.} \ \texttt{-.} \ \texttt{*.} \ \texttt{/.} \
    \texttt{cos} \ etc.\\
    \textsf{char}
  & \textsf{'a'} \ \textsf{'}\verb+\+\textsf{n'} \
    \textsf{'}\verb+\+\textsf{097'} \ etc.
  & \textsf{Char.code} \, \textsf{Char.chr} \ etc.\\
    \textsf{string}
  & \textsf{"a}\verb+\+\textsf{tb}\verb+\+\textsf{010c}\verb+\+\textsf{n"}
    \ etc.
  & \verb+^+ \, \textsf{s.[i]} \, \textsf{s.[i] $\leftarrow$ c} \ etc.\\
\hline
\end{tabular}
\end{center}
Les opérations sur les flottants sont notées différemment de leurs
homologues sur les entiers. Ce que nous notons joliment $\leftarrow$
s'écrit \verb+<-+ en \textsc{ascii}.

\end{frame}

% ------------------------------------------------------------------------
%
\begin{frame}
\frametitle{L'évaluation des opérateurs booléens}


\begin{itemize}

  \item Les opérateurs booléens sont \emph{séquentiels},
  c.-à-d. qu'ils n'évaluent leurs arguments que si c'est nécessaire,
  l'évaluation se faisant de la gauche vers la droite. 

\end{itemize}

\end{frame}

% ------------------------------------------------------------------------
%
\begin{frame}
\frametitle{Extension de la syntaxe des types et des phrases}

On étend la syntaxe des phrases pour permettre de lier un type à un
nom, comme on peut le faire pour les expressions.

\begin{tabular}{rll}
    $\bullet$
  & \textcolor{blue}{liaison de type (ou alias)}
  & \phrase{$\textsf{type} \,\, q = t$}
\end{tabular}

où $q$ dénote une variable de type (commençant par une minuscule).

\begin{tabular}{rll}
    $\bullet$
  & \textcolor{blue}{types récursifs}
  & \phrase{$\textsf{type} \,\, q_1 = t_1 \,\, \textrm{[}\Xand
     \,\, q_2 = t_2 \,\, \ldots\textrm{]}$}
\end{tabular}

Pour utiliser ces variables, il faut étendre la syntaxe des types:

\begin{tabular}{rll}
    $\bullet$
  & \textcolor{blue}{variable}
  & $q$
\end{tabular}

On peut maintenant écrire 

\phrase{type abscisse = float}

\phrase{type ordonnée = float}

\phrase{type point = abscisse * ordonnée}

\end{frame}

% ------------------------------------------------------------------------
%
\begin{frame}
\frametitle{Inférence de types}

Les $n$-uplets sont homogènes et leur arité est fixée par leur type:
\begin{center}
\begin{tabular}{lrcl}
    une paire
  & \textsf{(1,2)}
  & de type
  & \textsf{int} $\times$ \textsf{int}\\
    et un triplet
  & \textsf{(1,2,3)}
  & de type
  & \textsf{int} $\times$ \textsf{int} $\times$ \textsf{int}
\end{tabular}
\end{center}
sont incompatibles.

\topin{let milieu x y = (x+y)/2}

\topout{val milieu~:~int $\rightarrow$ int $\rightarrow$ int}

\topin{let milieu (x,y) = (x+y)/2}

\topout{val milieu~:~int $\times$ int $\rightarrow$ int}

\end{frame}

% ------------------------------------------------------------------------
%
\begin{frame}[containsverbatim]
\frametitle{Polymorphisme}

\textcolor{blue}{$n$-uplets}

Les projections sont polymorphes sur les $n$-uplets de \emph{même arité}:
\begin{center}
\begin{tabular}{rcl}
    \Xfun $(x, y, z) \rightarrow x$
  & a pour type
  & $(\alpha \times \beta \times \gamma) \rightarrow \alpha$
\end{tabular}
\end{center}

\bigskip

\textcolor{blue}{Fonction puissance}

\begin{semiverbatim}
\textcolor{blue}{\#} let rec power f n =
    if n <= 0 then fun x -> x
              else compose f (power f (n-1))\textcolor{blue}{;;}
\end{semiverbatim}
\topout{val power~:~($\alpha \rightarrow \alpha$) $\rightarrow$ int
  $\rightarrow$ ($\alpha \rightarrow \alpha$) = $\topclos$}

\end{frame}

% ------------------------------------------------------------------------
%
\begin{frame}
\frametitle{Polymorphisme (suite)}

\topin{let compose f g = fun x -> f (g (x))}

\topout{val compose~: $(\alpha \rightarrow \beta) \rightarrow (\gamma
\rightarrow \alpha) \rightarrow \gamma \rightarrow \beta = \topclos$}

\bigskip

Le type de la fonction \textsf{compose} se construit ainsi:
\begin{itemize}

  \item le premier argument \textsf{f} est une fonction quelconque,
  donc de type $\alpha \rightarrow \beta$; 

  \item le second argument \textsf{g} est une fonction dont le
  résultat doit être passé en argument à \textsf{f}, donc de type
  $\alpha$; 

  \item le domaine de \textsf{g} est quelconque, donc \textsf{g} est
  de type $\gamma \rightarrow \alpha$; 

  \item la fonction \textsf{compose} prend un argument \texttt{x} qui
  doit être passé à \textsf{g}, donc du type $\gamma$; finalement, le
  résultat de \textsf{compose} est retouné par \textsf{f}, donc de
  type $\beta$.

\end{itemize}

\end{frame}

% ------------------------------------------------------------------------
%
\begin{frame}
\frametitle{Égalité structurelle}

L'opérateur d'égalité est polymorphe et ne peut être défini en OCaml:

\bigskip

\topin{( = )}

\topout{- : $\alpha \rightarrow \alpha \rightarrow \texttt{bool} = \topclos$}

\bigskip

Donc attention à ce qu'il coïncide avec \emph{votre} notion d'égalité.

C'est l'égalité mathématique: deux valeurs sont égales si elles ont la
même structure et si leurs parties respectives sont égales. Ne marche
pas avec les expressions fonctionnelles (problème indécidable).

\bigskip

\topin{1 = 1 \&\& \str{oui} = \str{oui}}

\topout{- : bool = true}

\topin{(fun x -> x) = (fun x -> x)}

\error{Exception: Invalid\_argument "equal: functional value".}

\bigskip

On note \texttt{<>} la négation de l'égalité.

\end{frame}
