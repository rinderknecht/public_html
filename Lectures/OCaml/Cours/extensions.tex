%%-*-latex-*-

% ------------------------------------------------------------------------
%
\begin{frame}
\frametitle{Extension de mini-ML}

Ajoutons les expressions suivantes à mini-ML:

\begin{tabular}{rll} 
    $\bullet$
  & \textcolor{blue}{constante booléenne}
  & \Xtrue{} ou \Xfalse\\
    $\bullet$
  & \textcolor{blue}{opérateurs booléens}
  & \texttt{\&\&} ou \texttt{||} ou \textsf{not}\\
    $\bullet$
  & \textcolor{blue}{$n$-uplet}
  & $e_1, \ldots, e_n$\\
    $\bullet$
  & \textcolor{blue}{conditionnelle}
  & \Xif{} $e_0$ \Xthen{} $e_1$ \Xelse{} $e_2$\\
    $\bullet$
  & \textcolor{blue}{liaison locale récursive}
  & \Xlet{} \Xrec{} $f$ = $e_1$ \Xin{} $e_2$
\end{tabular}

%\bigskip
%
%\remarque \ \ Les parenthèses sont recommandées autour des $n$-uplets.

\bigskip

De plus, généralisons la méta-variable $x$ après \Xlet{} et \Xfun{}
pour en faire des \emph{motifs irréfutables}, que nous notons
$\ipat{p}$:

\begin{tabular}{rll}
    $\bullet$
  & \textcolor{blue}{fonction}
  & $\Xfun \,\, \ipat{p} \rightarrow e$\\
    $\bullet$
  & \textcolor{blue}{définition locale}
  & $\Xlet \,\, \textrm{[}\Xrec\textrm{]} \,\, \ipat{p} = e_1 \,\,
    \Xin \,\, e_2$
\end{tabular}

\end{frame}

% ------------------------------------------------------------------------
%
\begin{frame}
\frametitle{Les motifs irréfutables}

Un \emph{motif irréfutable} $\ipat{p}$ est défini \emph{récursivement}
par les cas suivants:

\begin{tabular}{rll}
    $\bullet$
  & \textcolor{blue}{variable}
  & $f, g, h$ (fonctions) et $x, y, z$ (autres). \\
    $\bullet$
  & \textcolor{blue}{unité}
  & \unit\\
    $\bullet$
  & \textcolor{blue}{$n$-uplet}
  & $\ipat{p}_1, \ldots, \ipat{p}_n$\\
    $\bullet$
  & \textcolor{blue}{parenthèse}
  & $\lpar\ipat{p}\rpar$\\
    $\bullet$
  & \textcolor{blue}{joker}
  & {\Large \_}
\end{tabular}

\bigskip

\remarques

\begin{itemize}

%  \item Les parenthèses sont recommandées autour des $n$-uplets.

  \item Du point de vue syntaxique, les motifs irréfutables sont des
  cas particuliers d'expressions --- hormis le joker.

   \item Un joker est un cas spécial pour que la phrase ne crée pas
   de liaison.

\end{itemize}

\end{frame}

% ------------------------------------------------------------------------
%
\begin{frame}
\frametitle{Exemples de phrases correctes (suite)}

\phrase{let x, y = 5, ('a',())}

\phrase{let y = f x}

\phrase{let z = f x y}

\phrase{let \_ = 5}

\phrase{let \_, (\_,\_) = x, (y,z)}

\phrase{let \_ = (x,(y,z))}

\phrase{let () = print\_string \str{Hello world!}}

\phrase{let rec fact = fun n -> if n = 1 then 1 else n * fact(n-1)}

\end{frame}

% ------------------------------------------------------------------------
%
\begin{frame}[containsverbatim]
\frametitle{Règles syntaxiques supplémentaires sur les expressions}

\begin{itemize}

  \item La virgule est prioritaire sur la flèche:\\
        $\Xfun \,\, x \rightarrow x, y$ \ \ équivaut à \ \ 
        $\Xfun \,\, x \rightarrow \textcolor{blue}{(}x,y\textcolor{blue}{)}$

  \item Pour alléger la notation $\Xfun \,\, \ipat{p}_1 \rightarrow
  \Xfun \,\, \ipat{p}_2 \rightarrow \ldots \rightarrow \Xfun \,\,
  \ipat{p}_n \rightarrow e$\\
  on définit les constructions équivalentes

        \begin{tabular}{rll}
            $\bullet$
          & \phrase{$\Xlet \,\,
            \textrm{[}\Xrec\textrm{]} \,\, f 
            = \Xfun \,\, \ipat{p}_1 \, \ipat{p}_2 \, \textcolor{blue}{\ldots
            \, \ipat{p}_n} \rightarrow e$}
          & (nouvelle expression)\\
            $\bullet$
          & \phrase{$\Xlet \,\,
            \textrm{[}\Xrec\textrm{]} \,\, f 
            \,\, \ipat{p}_1 \, \ipat{p}_2 \, \textcolor{blue}{\ldots \,
            \ipat{p}_n} = e$}
          & (nouvelle phrase)
        \end{tabular}

  \item Exemple:
  {\small
   \begin{verbatim}
let f = fun x y -> x + y in
  let g x y = x + y in
  let rec fact n = if n = 1 then 1 else n * fact(n-1)
in f 1 2 - g 3 4;;
  \end{verbatim}
  }
  
\end{itemize}

\end{frame}

% ------------------------------------------------------------------------
%
\begin{frame}
\frametitle{Extensions de mini-ML (suite)}

Nous étendons la syntaxe pour alléger certaines expressions.

\bigskip

Ainsi, par définition

\begin{tabular}{rl}
    $\bullet$
  & \phrase{$\Xlet \,\, \ipat{p}_1 = e_1 \,\,
    \textcolor{blue}{\Xand \,\, \ipat{p}_2 = e_2 \, \ldots \,\,
    \Xand \,\, \ipat{p}_n = e_n} \,\, \Xin \,\, e$}\\
  & équivaut à\\
    $\bullet$
  & \phrase{$\Xlet \,\, \ipat{p}_1, \ldots, \ipat{p}_n =
    e_1, \ldots, e_n \,\, \Xin \,\, e$}
\end{tabular}

\bigskip

De même nous introduisons les définitions mutuellement récursives:
\begin{itemize}

  \item \phrase{$\Xlet \,\, \Xrec \,\, \ipat{p}_1 = e_1 \,\,
    \textcolor{blue}{\Xand \,\, \ipat{p}_2 = e_2 \, \ldots \,\, \Xand
      \,\, \ipat{p}_n = e_n} \,\, \Xin \,\, e$}

\end{itemize}
De plus, la phrase \ \ \phrase{\textcolor{blue}{$e$}} \ \, équivaut à
\ \ \phrase{$\Xlet \,\, \_ = e$}

\end{frame}

% ------------------------------------------------------------------------
% 
%
\begin{frame}
\frametitle{Les expressions parallèles}
\label{expressions_paralleles}

Considérons le cas où les motifs irréfutables sont des variables

\centerline{\Xlet{} $x$ \equal{} $e_1$ \Xand{} $y$ \equal{}
$e_2$ \Xin{} $e$ \ où $x \neq y$}
Si $x \in {\cal L} (e_2)$, nous la définissons comme étant équivalente à

\begin{center}
\begin{minipage}{0.3\linewidth}
\begin{tabbing}
\textcolor{blue}{\Xlet} \= \textcolor{blue}{$z$ \equal{} $x$ \Xin}\\
\> \Xlet{} $x$ \equal{} $e_1$ \Xin\\
\> \Xlet{} $y$ \equal{} \textcolor{blue}{\Xlet{} $x$ \equal{} $z$ \Xin} $e_2$\\
\Xin $e$
\end{tabbing}
\end{minipage}
\end{center}
où $z \not\in {\cal L} (e_1) \cup {\cal L} (e_2) \cup {\cal L} (e)$,
pour n'être capturé ni par $e_1$, ni par $e_2$, ni par $e$.

\bigskip

Ce n'est donc pas une construction élémentaire.

\end{frame}

% ------------------------------------------------------------------------
% 
%
\begin{frame}
\frametitle{Les expressions mutuellement récursives}
\label{expressions_mutuellement_recursives}

Le \Xlet{} \Xrec{} multiple (avec \Xand) peut toujours se ramener à un
\Xlet{} \Xrec{} simple (avec \Xin) en paramétrant l'une des
définitions par rapport à l'autre. Posons que
\begin{equation*}
\Xlet \; \Xrec \; x \; \equal \; e_1 \; \Xand \; y \; \equal \; e_2 \;
\Xin \; e
\end{equation*}
où $x \neq y$, est équivalent à
\begin{center}
\begin{minipage}{0.3\linewidth}
\begin{tabbing}
\Xlet{} \= \Xrec{} $x$ \equal{} \textcolor{blue}{\Xfun{} $y$
  $\rightarrow$} $e_1$ \Xin\\
\> \Xlet{} \Xrec{} $y$ \equal{} \textcolor{blue}{\Xlet{} $x$ \equal{} $x \, y$ \Xin} $e_2$ \Xin\\
\> \textcolor{blue}{\Xlet{} $x$ \equal{} $x \, y$}\\
\textcolor{blue}{\Xin} $e$
\end{tabbing}
\end{minipage}
\end{center}
Ce n'est donc pas une construction élémentaire.

\end{frame}

% ------------------------------------------------------------------------
%
\begin{frame}
\frametitle{D'autres exemples de phrases correctes}

\phrase{let x = 5 and y = ()}

\phrase{let id x = x}

\phrase{let rec even n = (n=0) || odd (n-1)}

\phrase{and odd n = if n = 0 then false else even(n-1)}

\phrase{let x = 5 and y = 'a' and z = ()}
%\phrase{let rec fact (n) = if n = 0 then 1 else n * fact (n-1)}\\

\end{frame}

% ------------------------------------------------------------------------
% 
\begin{frame}
\frametitle{Extensions de la syntaxe des valeurs}

L'ajout de nouvelles expressions au langage nous oblige à étendre les
valeurs qui sont maintenant définies par

\bigskip

\begin{tabular}{rll}
    $\bullet$
  & \textcolor{blue}{unité ou constantes}
  & \unit \ ou \textsf{0} ou \Xtrue{} etc.\\
    $\bullet$
  & \textcolor{blue}{fermeture}
  & $\langle\Xfun \,\, x \rightarrow e, \rho\rangle$ où $\rho$ est un
    environnement.\\
  & 
  & Pour les opérateurs: $\langle\lpar{\texttt{+}}\rpar, \rho\rangle$
    etc.\\
    $\bullet$
  & \textcolor{blue}{$n$-uplet}
  & $v_1, \ldots, v_n$
\end{tabular}

\end{frame}

% ------------------------------------------------------------------------
% 
\begin{frame}
\frametitle{Fonctions curryfiées}

Une fonction est dite \emph{curryfiée} (du nom du logicien Curry) si
elle retourne une fonction. Cela permet d'effectuer des applications
partielles (cf. page~\pageref{application_partielle}).

\bigskip

En passant, n'oublions pas qu'une fonction OCaml prend toujours un
seul argument.

\bigskip

Si l'on souhaite le passage simultané de plusieurs valeurs il faut
alors employer une structure de donnée, par exemple un
$n$-uplet. Ainsi

\topin{let add x y = x + y}

\topout{val add~:~int $\rightarrow$ int $\rightarrow$ int}

\topin{let add' (x,y) = x + y}

\topout{val add'~:~int $\times$ int $\rightarrow$ int}

\bigskip

La fonction \texttt{\small add} est curryfiée et \texttt{\small add'}
ne l'est pas.

\end{frame}

% ------------------------------------------------------------------------
% 
\begin{frame}
\frametitle{Les termes ouverts revus}

Nous avons présenté une analyse statique qui nous donne les variables
libres d'une expression. Nous avons vu qu'une expression close ne peut
échouer par absence de liaison. Tous les compilateurs (comme OCaml)
rejettent les programmes ouverts (c.-à-d. non-clos), mais, du coup,
rejettent d'innocents programmes, comme \Xif{} \Xtrue{} \Xthen{}
\num{1} \Xelse{} \ident{x}.

\bigskip

Pour accepter ce type d'exemple (ouvert), il faudrait pouvoir prédire
le flot de contrôle (ici, quelle branche de la conditionnelle est
empruntée pour toutes les exécutions). Dans le cas ci-dessus cela est
trivial, mais en général le problème est indécidable, et ce ne peut
donc être une analyse statique (car la compilation doit toujours
terminer).

\end{frame}
