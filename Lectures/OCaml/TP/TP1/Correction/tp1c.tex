%%-*-latex-*-

\documentclass[a4paper]{article}

\usepackage[francais]{babel}
\usepackage[T1]{fontenc}
\usepackage[latin1]{inputenc}

\input{trace}

\title{Corrig� du TP 1 de programmation fonctionnelle en Objective Caml}
\author{Christian Rinderknecht}

\begin{document}

\maketitle

\section{Nombres}

\begin{verbatim}
let p x = 2 * x*x + 3 * x - 2
let q x = 2.0 *. x*.x +. 3.0 *. x -. 2.0
\end{verbatim}

\section{Expressions conditionnelles}

\begin{verbatim}
# if -3 < 0 then 3 else 3;;
- : int = 3
# let abs_int n = if n < 0 then -n else n;;
val abs_int : int -> int = <fun>
\end{verbatim}

\section{Evaluation d'une expression}

Pas de questions.

\section{D\'eclarations}

\subsection{Variables globales}

\begin{verbatim}
# let an = "2003";;
val an : string = "2003"
# let x = int_of_string(an);;
val x : int = 2003
# let nouvel_an = string_of_int(x+1);;
val nouvel_an : string = "2004"
\end{verbatim}

\subsection{Variables locales}

\begin{verbatim}
# let x = 3 in let b = x < 10 in if b then 0 else 10;;
- : int = 0
# let a = 3.0 and b = 4.0 in sqrt(a*.a+.b*.b);;
- : float = 5.
# let solve (a,b,c) =
    let delta = b *. b -. 4.0 *. a *. c
  in ((-. b -. sqrt (delta))/. (2.0 *. a),
      (-. b +. sqrt (delta))/. (2.0 *. a));;
val solve : float * float * float -> float * float = <fun>
\end{verbatim}

\section{Produit cart�sien}

Pas de questions.

\section{R\'ecursivit\'e}

Pas de questions.

\section{Expressions fonctionnelles}

\begin{verbatim}
# let compose f g x = f (g(x));;
val compose : ('a -> 'b) -> ('c -> 'a) -> 'c -> 'b = <fun>
# let add1 = function x -> x + 1;;
val add1 : int -> int = <fun>
# let mult5 = function x -> 5 * x;;
val mult5 : int -> int = <fun>
# compose add1 mult5 3;;
- : int = 16
# compose mult5 add1;;
- : int -> int = <fun>
# let deuxfois f x = f(f(x));;
val deuxfois : ('a -> 'a) -> 'a -> 'a = <fun>
# let deuxfois f x = compose f f x;;
val deuxfois : ('a -> 'a) -> 'a -> 'a = <fun>
\end{verbatim}

\section{Port�e statique des variables}

\begin{verbatim}
# let p = 10;;
p : int = 10
# let k x = x + p;;
k : int -> int = <fun>
# let p = p+1;;
p : int = 11
# p;;
- : int = 11
# k 0;;
- : int = 10
\end{verbatim}

La variable \texttt{p} qui appara�t dans la d�finition de la fonction
\texttt{k}, est li�e \emph{d�finitivement} � la valeur \texttt{10}. La
troisi�me phrase red�finit une nouvelle variable \texttt{p}, rendant
inaccessible la premi�re variable \texttt{p}, \emph{sauf pour la
fonction \texttt{k}}. En effet, une fonction est mod�lis�e par une
\emph{fermeture}, c'est-�-dire une paire constitu�e du code de la
fonction et de l'environnement statique au lieu de d�finition.

\end{document}




\end{document}
