%% -*-latex-*-

\documentclass[a4paper]{article}
 
\usepackage[francais]{babel}
\usepackage[OT1]{fontenc}
\usepackage[latin1]{inputenc}
\usepackage{amsmath,amssymb}

\input{trace}

\title{Quiz de programmation fonctionnelle en Objective Caml}
\author{Christian Rinderknecht}

\begin{document}

\maketitle

\begin{itemize}

\item \'{E}crire la fonction qui calcule $u_n$ en fonction de $u_0\in
      \mathbb{N}$ et de $n$, la suite $(u_n)_{n\in N}$ �tant d�finie par
\[ \begin{array}{ll}
   u_n = & \left\{ \begin{array}{ll}
                  1 & \mbox{si $u_{n-1}$ vaut $1$} \\
                  u_{n-1}/2 & \mbox{si $u_{n-1}$ est pair} \\
                  3 u_{n-1}+1 & \mbox{si $u_{n-1}$ est impair}
                 \end{array}
        \right.
   \end{array}
\]

\item �crire une fonction qui calcule $v_n$ en fonction de $n$, o�
      $(v_n)_{n\in N}$ est la suite de Fibonacci.

\item Si la fonction ci-dessus a �t� �crite de la fa�on sugg�r�e par
      la d�finition de la suite, quel est le nombre d'additions
      effectu�es pour calculer $v_{10}$? (On peut �crire une fonction
      Caml pour r�aliser ce calcul!) Cette m�thode est-elle efficace?
      �crire une fonction qui, �tant donn� $n$, calcule $v_n$ et
      $v_{n+1}$.

\item �crire une fonction qui, �tant donn�s deux entiers $x$ et $y$,
      calcule leur PGCD par la m�thode d'Euclide. On supposera dans un
      premier temps $x \leq y$, puis on �crira une deuxi�me fonction
      qui ne fait pas cette supposition.

\item On repr�sente une fraction $p \over q$ par le couple d'entiers
      $(p, q)$. Ecrire une fonction qui r�duit une fraction
      donn�e. Ecrire une fonction qui additionne deux fractions.

\end{itemize}

\end{document}
