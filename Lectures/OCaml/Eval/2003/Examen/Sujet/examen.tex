%%-*-latex-*-

\documentclass[a4paper]{article}

\usepackage[francais]{babel}
\usepackage[T1]{fontenc}
\usepackage[latin1]{inputenc}
\usepackage{amsmath}
\usepackage{graphicx}
\usepackage{ae,aecompl}

\input{trace}

\title{Examen de programmation fonctionnelle en Objective Caml}
\author{Christian Rinderknecht}
\date{Mardi 22 avril 2003}

\begin{document}

\maketitle


Les exercices peuvent \^etre trait\'es de fa\c{c}on ind\'ependante.
Bar\^eme: A:3, B:3, C:10, D:4.

\smallskip

\textbf{Exercice A:}
Donner le type de chacune des expressions suivantes:

\noindent
\textbf{A.1}: \verb"fun (x,y) -> (y+1, not x)"\\
\textbf{A.2}: \verb"fun p -> (snd p, fst p)"\\
\textbf{A.3}: \verb"fun f -> fun (x,y) -> f(x+1) + f y"\\
\textbf{A.4}: \verb"fun f -> fun (x,y) -> (f(x+1), f y)"\\
\textbf{A.5}: \verb"fun f -> fun (x,y) -> (f x, f y)"\\
\textbf{A.6}: \verb"fun f -> fun (x,y) -> (f x + 1, f y)"
%A.7: \verb"fun (f,g) -> fun x -> (f x, g x)"\\

\medskip

\textbf{Exercice B:}
Pour chacun des types suivants, trouver une expression Caml
poss\'edant ce type:

\noindent
%B.1: \verb"(int * int) -> bool"\\
\textbf{B.1}: \verb"int -> bool -> int"\\
\textbf{B.2}: \verb"int -> (bool * int)"\\
\textbf{B.3}: \verb"int list -> bool list"\\
\textbf{B.4}: \verb"('a * 'b) list -> 'a list"\\
\textbf{B.5}: \verb"('a * 'b) list -> 'b -> 'a"\\
\textbf{B.6}: \verb"'a list -> 'b list -> ('a * 'b) list"

\medskip

\textbf{Exercice C:}

Un arbre peut repr\'esenter une collection de mots de sorte qu'il soit
tr\`es efficace de v\'erifier si une s\'equence donn\'ee de
caract\`eres est un mot valide ou pas. Dans ce type d'arbre, appel\'e
un {\em{trie}}, les arcs ont une lettre associ\'ee, chaque n{\oe}ud
poss�de une indication si la lettre de l'arc entrant est une fin de
mot et la suite des mots partageant le m\^eme d\'ebut. La figure
suivante montre le \emph{trie} des mots �~le~�, �~les~�, �~la~� et
�~son~�, l'ast�risque marquant la fin d'un mot:
\begin{center}
\includegraphics[scale=0.8]{trie.eps}
\end{center}
\noindent
On utilisera le type Caml suivant pour implanter les {\em{tries}}:

\begin{verbatim}
type trie = 
  { mot_complet : string option; suite : (char * trie) list }
\end{verbatim}

Le type pr�d�fini \verb+type 'a option = None | Some of 'a+ sert �
repr�senter une valeur �ventuellement absente. Chaque n{\oe}ud d'un
\emph{trie} contient les informations suivantes:

\begin{itemize}
  \item Si le n{\oe}ud marque la fin d'un mot {\tt s} (ce qui
        correspond aux n{\oe}uds �toil�s de la figure), alors le champ
        {\tt mot\_complet} contient {\tt Some s}, sinon ce champ
        contient {\tt None}.

  \item Le champ {\tt suite} contient une liste qui associe les
        caract�res aux n{\oe}uds.
\end{itemize}

\medskip

\noindent
\textbf{C.1}: �crire la valeur Caml de type \verb+trie+ correspondant
� la figure ci-dessus.\\
\textbf{C.2}: �crire une fonction \texttt{compte\_mots} qui compte le
nombre de mots dans un \emph{trie}.\\
\textbf{C.3}: �crire une fonction \texttt{select} qui prend en
argument un \emph{trie} et une lettre, et renvoie le \emph{trie}
correspondant aux mots commen�ant par cette lettre. Si ce \emph{trie}
n'existe pas (parce qu'aucun mot ne commence par cette lettre), la
fonction devra lancer une exception {\tt Absent}, que l'on
d�finira. On utilisera la fonction pr�d�finie \texttt{List.assoc} pour
effectuer la recherche dans la liste des sous-arbres \verb+suite+.\\
\textbf{C.4}: �crire une fonction \texttt{recherche} qui v\'erifie si
une cha\^{\i}ne de caract\`eres est un mot dans un {\em trie}
donn\'e. La fonction devra prendre un argument suppl�mentaire {\tt i},
qui repr�sente la position dans le mot associ�e au n{\oe}ud courant.
Le i\ieme{} caract�re d'une cha�ne {\tt s} s'obtient en �crivant {\tt
s.[i]}, le premier caract�re �tant num�rot� $0$. La longueur de la
cha�ne {\tt s} s'�crit {\tt String.length s}. On emploiera la fonction
\texttt{select}.

\medskip

\textbf{Exercice D:}

\noindent
On s'int\'eresse aux expressions bool\'eennes \'ecrites \`a l'aide des
connecteurs \texttt{or} (�~ou~� bool\'een), \texttt{and} (�~et~�
bool\'een), \texttt{not} (n\'egation bool\'eenne), des constantes
\texttt{true} et \texttt{false}, et de variables.  Par exemple: $(x
~or~ y) ~and~ not(x ~and~ y)$.

\noindent
\textbf{D.1}: D\'efinir un type Caml \texttt{bool\_exp} pour ces
expressions.

\noindent
\textbf{D.2}: �crire une fonction \texttt{eval} permettant d'\'evaluer
de telles expressions.Cette fonction devra prendre en param\`etre un
environnement associant les noms de variables � leur valeur.

\noindent
\textbf{D.3}: 
Les connecteurs bool\'eens consid\'er\'es satisfont en particulier
les identit\'es
\begin{align*}
not(a ~or~ b)  & = not(a) ~and~ not(b)\\
not(a~ and~ b) & = not(a)~ or~ not(b)\\
not(not(a))    & = a\\
not(true)      & = false\\
not(false)     & = true
\end{align*}

\noindent
Ces identit\'es permettent de transformer toute expression bool\'eenne
en une expression \'equivalente o\`u les n\'egations ne sont
appliqu\'ees qu'\`a des variables. Par exemple, l'expression $not(x
~and~ (y~ or~ not(z)))$ peut \^etre transform\'ee en $not(x) ~or~
not(y ~or ~not(z)))$ puis en $not(x) ~or~ (not(y)~ and ~not(not(z))))$
et enfin en $not(x) ~or~ (not(y) ~and~ z)$.

\noindent
�crire une fonction Caml \texttt{normalise} qui r\'ealise cette
transformation.

\end{document}
