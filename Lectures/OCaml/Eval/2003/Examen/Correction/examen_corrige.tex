%%-*-latex-*-

\documentclass[a4paper]{article}

\usepackage[francais]{babel}
\usepackage[T1]{fontenc}
\usepackage[latin1]{inputenc}
\usepackage{amsmath}
\usepackage{graphicx}
\usepackage{ae,aecompl}

\input{trace}

\title{Corrig� de l'examen de programmation fonctionnelle en Objective
  Caml}
\author{Christian Rinderknecht}
\date{Mardi 22 avril 2003}

\begin{document}

\maketitle

\noindent
\textbf{Exercice A:}

\begin{verbatim}
# fun (x,y) -> (y+1, not x);;
- : bool * int -> int * bool = <fun>
# fun p -> (snd p, fst p);;
- : 'a * 'b -> 'b * 'a = <fun>
# fun f -> fun (x,y) -> f(x+1) + f y;;
- : (int -> int) -> int * int -> int = <fun>
# fun f -> fun (x,y) -> (f(x+1), f y);;
- : (int -> 'a) -> int * int -> 'a * 'a = <fun>
# fun f -> fun (x,y) -> (f x, f y);;
- : ('a -> 'b) -> 'a * 'a -> 'b * 'b = <fun>
# fun f -> fun (x,y) -> (f x + 1, f y);;
- : ('a -> int) -> 'a * 'a -> int * int = <fun>
\end{verbatim}
%A.7: \verb"fun (f,g) -> fun x -> (f x, g x)"

\bigskip

\noindent
{\bf  Exercice B:}

\begin{verbatim}
# fun n b -> if b then n+1 else n-1;;
- : int -> bool -> int = <fun>
# fun n -> (true, n+1);;
- : int -> bool * int = <fun>
# List.map (fun n -> if n mod 2 = 0 then true else false);;
- : int list -> bool list = <fun>
# fun l -> List.map fst l;;
- : ('a * 'b) list -> 'a list = <fun>
# fun l x -> List.assoc x (List.map (fun (x,y) -> (y,x)) l);;
- : ('a * 'b) list -> 'b -> 'a = <fun>
# List.combine;;
- : 'a list -> 'b list -> ('a * 'b) list = <fun>
\end{verbatim}
%B.7: \verb"(int * int) -> bool"

\bigskip

\noindent
\textbf{Exercice C:}

\emph{Un arbre peut repr\'esenter une collection de mots de sorte qu'il soit
tr\`es efficace de v\'erifier si une s\'equence donn\'ee de
caract\`eres est un mot valide ou pas. Dans ce type d'arbre, appel\'e
un {\em{trie}}, les arcs ont une lettre associ\'ee, chaque n{\oe}ud
poss�de une indication si la lettre de l'arc entrant est une fin de
mot et la suite des mots partageant le m\^eme d\'ebut. La figure
suivante montre le \emph{trie} des mots �~le~�, �~les~�, �~la~� et
�~son~�, l'ast�risque marquant la fin d'un mot:}
\begin{center}
\includegraphics[scale=0.8]{trie.eps}
\end{center}
\noindent
\emph{On utilisera le type Caml suivant pour implanter les
{\em{tries}}}:

\begin{verbatim}
type trie = 
  { mot_complet : string option; suite : (char * trie) list }
\end{verbatim}

\emph{Le type pr�d�fini \texttt{type 'a option = None | Some of 'a}
sert � repr�senter une valeur �ventuellement absente. Chaque n{\oe}ud
d'un \emph{trie} contient les informations suivantes:}

\begin{itemize}
  \item \emph{Si le n{\oe}ud marque la fin d'un mot {\tt s} (ce qui
        correspond aux n{\oe}uds �toil�s de la figure), alors le champ
        {\tt mot\_complet} contient {\tt Some s}, sinon ce champ
        contient {\tt None}.}

  \item \emph{Le champ {\tt suite} contient une liste qui associe les
        caract�res aux n{\oe}uds.}
\end{itemize}

\medskip

\noindent
\textbf{C.1}: \emph{�crire la valeur Caml de type \texttt{trie}
correspondant � la figure ci-dessus.}

\smallskip

\begin{verbatim}
{mot_complet=None;
 suite=[('l',{mot_complet=None;
              suite=[('e',{mot_complet=Some "le";
                           suite=[('s',{mot_complet=Some "les"; 
                                        suite=[]})]});
                     ('a',{mot_complet=Some "la";
                           suite=[]})]}); 
        ('s',{mot_complet=None;
              suite=[('o', {mot_complet=None;
                            suite=[('n', {mot_complet=Some "son";
                                           suite=[]})]})]})]}
\end{verbatim}

\noindent
\textbf{C.2}: \emph{�crire une fonction \texttt{compte\_mots} qui
compte le nombre de mots dans un \emph{trie}}.

\smallskip

\begin{verbatim}
let rec compte_mots {mot_complet=m; suite=l} =
    (match m with Some _ -> 1 | None -> 0) 
  + List.fold_left (fun n (_,t) -> n + compte_mots t) 0 l
\end{verbatim}

\smallskip

\textbf{C.3}: \emph{�crire une fonction \texttt{select} qui prend en
argument un \emph{trie} et une lettre, et renvoie le \emph{trie}
correspondant aux mots commen�ant par cette lettre. Si ce \emph{trie}
n'existe pas (parce qu'aucun mot ne commence par cette lettre), la
fonction devra lancer une exception {\tt Absent}, que l'on
d�finira. On utilisera la fonction pr�d�finie \texttt{List.assoc} pour
effectuer la recherche dans la liste des sous-arbres \texttt{suite}.}

\smallskip

\begin{verbatim}
exception Absent

let select lettre trie =
  try
    List.assoc lettre trie.suite
  with Not_found -> raise Absent
\end{verbatim}

\smallskip

\noindent
\textbf{C.4}: \emph{�crire une fonction \texttt{recherche} qui
v\'erifie si une cha\^{\i}ne de caract\`eres est un mot dans un {\em
trie} donn\'e. La fonction devra prendre un argument suppl�mentaire
{\tt i}, qui repr�sente la position dans le mot associ�e au n{\oe}ud
courant.  Le i\ieme{} caract�re d'une cha�ne {\tt s} s'obtient en
�crivant {\tt s.[i]}, le premier caract�re �tant num�rot� $0$. La
longueur de la cha�ne {\tt s} s'�crit {\tt String.length s}. On
emploiera la fonction \texttt{select}.}

\smallskip

\begin{verbatim}
let rec recherche chaine index trie =
  if index = String.length chaine 
  then match trie.mot_complet with
         Some _ -> true
       | None -> false
  else try
         let fils = select chaine.[index] trie 
         in recherche chaine (index+1) fils
       with Absent -> false
\end{verbatim}

\smallskip

\noindent
\textbf{Exercice D:}

\noindent
\emph{On s'int\'eresse aux expressions bool\'eennes \'ecrites \`a
l'aide des connecteurs \texttt{or} (�~ou~� bool\'een), \texttt{and}
(�~et~� bool\'een), \texttt{not} (n\'egation bool\'eenne), des
constantes \texttt{true} et \texttt{false}, et de variables.  Par
exemple: $(x ~or~ y) ~and~ not(x ~and~ y)$.}

\smallskip

\noindent
\textbf{D.1}: \emph{D\'efinir un type Caml \texttt{bool\_exp} pour ces
expressions.}

\smallskip

\begin{verbatim}
type bool_exp = 
  Or  of bool_exp * bool_exp
| And of bool_exp * bool_exp
| Not of bool_exp
| Var of string
| True
| False
\end{verbatim}

\smallskip

\noindent
\textbf{D.2}: \emph{�crire une fonction \texttt{eval} permettant
d'\'evaluer de telles expressions. Cette fonction devra prendre en
param\`etre un environnement associant les noms de variables � leur
valeur.}

\smallskip

\begin{verbatim}
let rec eval env = function
  Or (e1, e2)  -> eval env e1 || eval env e2
| And (e1, e2) -> eval env e1 && eval env e2
| Not (e)      -> not (eval env e)
| Var (s)      -> List.assoc s env
| x            -> x
\end{verbatim}

\smallskip

\noindent
\textbf{D.3}: \emph{Les connecteurs bool\'eens consid\'er\'es
satisfont en particulier les identit\'es}

\begin{align*}
not(a ~or~ b)  & = not(a) ~and~ not(b)\\
not(a~ and~ b) & = not(a)~ or~ not(b)\\
not(not(a))    & = a\\
not(true)      & = false\\
not(false)     & = true
\end{align*}

\noindent
\emph{Ces identit\'es permettent de transformer toute expression
bool\'eenne en une expression \'equivalente o\`u les n\'egations ne
sont appliqu\'ees qu'\`a des variables. Par exemple, l'expression
$not(x ~and~ (y~ or~ not(z)))$ peut \^etre transform\'ee en $not(x)
~or~ not(y ~or ~not(z)))$ puis en $not(x) ~or~ (not(y)~ and
~not(not(z))))$ et enfin en $not(x) ~or~ (not(y) ~and~ z)$.}

\noindent
\emph{�crire une fonction Caml \texttt{normalise} qui r\'ealise cette
transformation.}

\smallskip

\begin{verbatim}
let rec normalise = function
  Or (e1, e2)        -> Or (normalise e1, normalise e2)
| And (e1, e2)       -> And (normalise e1, normalise e2)
| Not (Or (e1, e2))  -> normalise (And (Not e1, Not e2))
| Not (And (e1, e2)) -> normalise (Or (Not e1, Not e2))
| Not (Not (e))      -> normalise (e)
| Not (True)         -> False
| Not (False)        -> True
| x                  -> x
\end{verbatim}

\end{document}
