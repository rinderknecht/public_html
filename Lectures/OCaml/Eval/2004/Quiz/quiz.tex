%%-*-latex-*-

\documentclass[a4paper]{article}

\usepackage[francais]{babel}
\usepackage[OT1]{fontenc}
\usepackage[latin1]{inputenc}
\usepackage{amsfonts}
\usepackage{xspace}
\usepackage{alltt}
\usepackage{amssymb,amsmath,stmaryrd}

\input{trace}
\input{commands}
\input{ocaml_syntax}

\title{Quiz de programmation fonctionnelle en Objective Caml}
\author{Christian Rinderknecht}
\date{Lundi 8 mars 2004}

\begin{document}

\maketitle


\noindent Pour chacun des programmes suivants,

\begin{itemize}

  \item construisez l'arbre correspondant,
 
  \item reliez les variables dans les expressions � leur lieur,

  \item identifiez les variables libres,

  \item si le programme est clos, d�crivez pr�cis�ment son ex�cution.

\end{itemize}

\bigskip

\noindent \textbf{Programme 1}

\begin{verbatim}
let x = 1 in ((let x = 2 in x) + x);;
\end{verbatim}

\noindent \textbf{Programme 2}

\begin{verbatim}
fun y -> x + (fun x -> x) y;;
\end{verbatim}

\noindent \textbf{Programme 3}

\begin{verbatim}
let x = 1 in
  let f = fun y -> x + y in
  let x = 2
in f(x);;
\end{verbatim}

\noindent \textbf{Programme 4}

\begin{verbatim}
let x = 0;;
let id = fun x -> x;;
let y = 2 in id (y);;
let x = (fun x -> fun y -> x + y) 1 2;;
x+1;;
\end{verbatim}

\noindent \textbf{Programme 5}

\begin{verbatim}
let compose = fun f -> fun g -> fun x -> f (g x) in
  let square = fun f -> compose f f in
  let double = fun x -> x + x in
  let quad = square double 
in square quad;;
\end{verbatim}

\end{document}
