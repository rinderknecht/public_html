%%-*-latex-*-

% ------------------------------------------------------------------------
%
\begin{frame}
\frametitle{Plan}

\begin{enumerate}

  \item Computer Networks and the Internet

    \begin{itemize}

      \item What is the Internet?

      \item The network edge

      \item \textbf{The network core}

      \item Network access and physical media

      \item ISPs and Internet backbones

      \item Delay and loss in packet-switched networks
 
      \item Protocol layers and their service models

    \end{itemize}

\end{enumerate}

\end{frame}

% ------------------------------------------------------------------------
%
\begin{frame}
\frametitle{The network core}

The network core is a mesh of interconnected routers. 

\bigskip

They are of two kinds:
\begin{itemize}

  \item \textbf{Circuit-switching} networks: the resources (buffers,
    link bandwidth) needed along a path (or \textbf{circuit}) to
    provide a communication between hosts are \emph{reserved} for the
    duration of the \textbf{session}.

  \item \textbf{Packet-switching} networks: the resources are
    \emph{not reserved} and thus the data may have to wait and are cut
    into packets.

\end{itemize}

\end{frame}

% ------------------------------------------------------------------------
%
\begin{frame}
\frametitle{The network core (cont)}

\begin{center}
  \includegraphics[scale=0.3]{01-04.eps}
\end{center}

\end{frame}

% ------------------------------------------------------------------------
%
\begin{frame}
\frametitle{Circuit-switched networks}

The \textbf{telephone network} is of this kind: even if no one talks
on the phone, the resources are still reserved until one hangs up.

\bigskip

Since bandwidth has been reserved, the data (voice) rate is
guaranteed.

\bigskip

Circuits are not shared and must be setup first: the routers (called
here \textbf{circuit switches}) along the path ``know'' that they are
part of a given circuit (i.e., keep a state in memory).

\end{frame}

% ------------------------------------------------------------------------
%
\begin{frame}
\frametitle{Circuit-switched networks (cont)}

\begin{center}
  \includegraphics[scale=0.32]{01-05.eps}
\end{center}

\end{frame}

% ------------------------------------------------------------------------
%
\begin{frame}
\frametitle{Circuit-switched networks/Multiplexing}
\label{fdm_tdm}

Each link divides its data rate equally between all the active
circuits it supports. This is called \textbf{multiplexing}. There are
two kinds of multiplexing.
\begin{itemize}

  \item \textbf{Frequency-division multiplexing (FDM)} divides and
  \emph{shares the frequency spectrum of the link} among all the
  connections established across it. In telephony, the bandwidth
  allocated to a circuit is 4~kHz. FM radio also use FDM.

  \item \textbf{Time-division multiplexing (TDM)} defines a period of
  time, called \textbf{frame}, divides it into equal \textbf{slots}
  which are reserved to a given circuit. In other words, each time
  slot is periodically repeated and, \emph{during each slot, the full
  bandwidth is available to the current circuit only.} Modern trend in
  telephony is to replace FDM with TDM.

\end{itemize}

\end{frame}

% ------------------------------------------------------------------------
%
\begin{frame}
\frametitle{Circuit-switched networks/Multiplexing/Figures}

\begin{center}
\includegraphics[scale=0.35]{01-06.eps}
\end{center}

\end{frame}

% ------------------------------------------------------------------------
%
\begin{frame}
\frametitle{Circuit-switched networks/Multiplexing/TDM examples}

For example, if a TDM link transmits 8,000 frames per second and each
slot allows sending 8~bits, then the transmission rate of a circuit on
this link is 64~kb/s (also noted 64~\textbf{Kbps}).

\bigskip

Now, suppose

\begin{itemize}

  \item we want to send a file of size 640,000 bits;

  \item all links in the network use TDM with 24 slots/frame and have
  a bit rate of 1.536 Mbps (mega-bits per second);

  \item it takes 500 ms (milliseconds, also noted \textbf{msec}) to
  establish an end-to-end circuit.

\end{itemize}

How long does it take to send the file?

\end{frame}

% ------------------------------------------------------------------------
%
\begin{frame}
\frametitle{Circuit-switched networks/Multiplexing/Solution}

\begin{enumerate}

  \item 24 slots means there are 24 circuits along each link.

  \item TDM implies that each circuit has the full bandwidth in
  turn. Therefore, each circuit has the full bandwidth 1/24 of the
  time.

  \item Thus the rate for each circuit is the link rate by its time
  using it: 
  
  1.536/24 = 64 kb/s = 64,000 b/s

  \item The file is transfered in 640,000/64,000 = 10 seconds.

  \item But we have to add the time to setup the circuit, so the total
  transfer time is 

  10 + 0.5 = 10.5 seconds.

\end{enumerate}

Note that \textbf{the transfer time is independent of the number of
links.}

\end{frame}

% ------------------------------------------------------------------------
%
\begin{frame}
\frametitle{Packet-switched networks}

The source breaks the messages into smaller pieces known as
\textbf{packets}. The packets are directed towards their destination
by \textbf{routers}, also called \textbf{packet switches}.

\bigskip

Packets are transmitted over each link at the \emph{full} transmission
rate: \emph{only one packet travels a link at a time}.

\bigskip

Most packet-switches use \textbf{store-and-forward transmission}: the
switch must receive the entire packet before it starts to retransmit
the first bit of it.

\end{frame}

% ------------------------------------------------------------------------
%
\begin{frame}
\frametitle{Packet-switched networks/Queuing}

Each router has multiple links attached to it. For each link, the
router has an \textbf{output buffer} (also called \textbf{output
queue}), which stores packets that are about to be transmitted.

\bigskip

If an incoming packet finds the next link busy, it is stored (queued)
in the corresponding output buffer until the link is available: this
delay is called \textbf{queuing delay}.

\bigskip

If the queue is already full (because of network congestion), the
incoming packet is discarded (dropped): this is a \textbf{packet
loss}.

\end{frame}

% ------------------------------------------------------------------------
%
\begin{frame}
\frametitle{Packet-switched networks/Statistical multiplexing}

The packets are retransmitted whenever they are available and the
output link is free. This leads to a random ordering, called
\textbf{statistical multiplexing}.

\bigskip

This very different from TDM in circuit switching, where each host
gets the same time slot in a revolving frame.

\end{frame}

% ------------------------------------------------------------------------
%
\begin{frame}
\frametitle{Packet-switched networks/Statistical multiplexing (cont)}

\begin{center}
  \includegraphics[scale=0.35]{01-07.eps}
\end{center}

\textbf{Warning!} In this figure, several packets for the same
connection are drawn on the same link, but, in fact, there is only one
at a time.

\end{frame}

% ------------------------------------------------------------------------
%
\begin{frame}
\frametitle{Packet-switched networks/Delay}

Let us consider on a very simple case how long it takes to send a
message of \(L\) bits on a series of \(Q\) links whose rate is \(R\)
bits/sec. 

\bigskip

Assume that there is no queuing delay and no connection establishment.

\bigskip

The packet must be transmitted on the first link: this takes \(L/R\)
seconds.

\bigskip

Then it must be retransmitted on the following \(Q-1\) links, that is
to say \(Q-1\) times.

\bigskip

Thus the total delay is simply \(QL/R\).

\end{frame}

% ------------------------------------------------------------------------
%
\begin{frame}
\frametitle{Packet switching versus circuit switching}

Let us compare packet switching and circuit switching.

\begin{itemize}

  \item Opponents to packet switching say that packet switching is
  not suitable for \textbf{real-time services} (e.g., telephone calls
  and video-conference) because of its variable and unpredictable
  delays (mainly queuing delays).

  \item Proponents of packet switching argue that
  
  \begin{itemize}

    \item it offers better sharing of the bandwidth than circuit
    switching,

    \item it is simpler, less costly to implement and more efficient.

  \end{itemize}

\end{itemize}

\end{frame}

% ------------------------------------------------------------------------
%
\begin{frame}
\frametitle{Packet switching versus circuit switching (cont)}

Why is packet switching more efficient?

\bigskip

Let us consider a simple example. Suppose
\begin{itemize}

  \item users share a 1 Mbps link;

  \item each user alternates periods of activity and of inactivity:
  
   \begin{itemize}

     \item when he is active, he generates data at the constant rate
     of 100~Kbps;

     \item when he is idle, he produces no data.

   \end{itemize}

  \item each user is active during 10\% of the time.

\end{itemize}
With circuit switching, 100~Kbps must be reserved for \emph{each}
user. Thus the link can only support 10 (1~Mbps/100~Kbps) users
simultaneously.

\end{frame}

% ------------------------------------------------------------------------
%
\begin{frame}
\frametitle{Packet switching versus circuit switching (cont)}

If there are 35 users, the probability that there are 11 or more
simultaneously active users is around 0.0004 (we don't show the
computation).
\begin{itemize}

  \item When there are 10 or less active users at the same time, the
  total used bandwidth is less or equal to 1 Mbps. This is similar to
  the circuit switching situation discussed before.

  \item When there are 11 or more active users, the output queue of
  the routers will start to grow (this is congestion).

\end{itemize}

Because the probability of having more than 11 active users is very
low, packet switching is almost as good as circuit switching,
\emph{while allowing more than three times the number of users} (35
users versus 10).

\end{frame}

% ------------------------------------------------------------------------
%
\begin{frame}
\frametitle{Message segmenting}

We have seen some advantages of packet-switched networks. Now, what
about the size of the packets?

\bigskip

If an application message is not cut into smaller pieces (packets),
i.e., the message itself if a big packet, this is called
\textbf{message switching}. Contrast this with small packets in the
following figures:
\begin{center}
\includegraphics[scale=0.35]{01-08.eps}\\
\includegraphics[scale=0.35]{01-09.eps}
\end{center}

\end{frame}

% ------------------------------------------------------------------------
%
\begin{frame}
\frametitle{Message segmenting (cont)}

When a message is segmented, the packet-switched network
\textbf{pipelines} the transmission, i.e., portions of the message are
transmitted in parallel by the source and the two packet switches.

\bigskip

The advantage of message segmentation with packet switching is that
the end-to-end delay is lower than with message switching. Let us see
why.

\bigskip

Consider a message of size \(7.5 \cdot 10^6\) bits, three links of
rate 1.5~Mbps connecting two hosts and assume that there is no
congestion.

\bigskip

How much time is required to send and receive the message with message
switching?

\end{frame}

% ------------------------------------------------------------------------
%
\begin{frame}
\frametitle{Message segmenting (cont)}

This is what happens:
\begin{center}
\includegraphics[scale=0.3]{01-10.eps}
\end{center}
It takes 15 seconds (\(3 \text{ links} \times 7.5 \cdot 10^6 \text{
bits}/1.5 \text{ Mbps}\)). Even a bit more, in fact, due to
store-and-forward delays.

\end{frame}

% ------------------------------------------------------------------------
%
\begin{frame}
\frametitle{Message segmenting (cont)}

Now suppose that the source breaks the message into 5,000 packets.

\bigskip

Then each packet is 1,500 (\(7.5 \cdot 10^6/5,000\)) bits long.

\bigskip

Assume that there is no congestion.

\bigskip

How long does it takes to move the 5,000 packets to destination?

\bigskip

The next figure shows what happens.

\end{frame}

% ------------------------------------------------------------------------
%
\begin{frame}
\frametitle{Message segmenting (cont)}

\begin{center}
\includegraphics[scale=0.35]{01-11.eps}
\end{center}

\end{frame}

% ------------------------------------------------------------------------
%
\begin{frame}
\frametitle{Message segmenting (cont)}

As shown in the previous figure, the first packet reached the first
switch after 1~ms. The second packet reaches it after 2~ms etc.

\bigskip

But, while the first packet is moving from the first switch to the
second, the second packet is \emph{at the same time} traveling from
the source to the first switch.

\bigskip

The last packet reaches then the first switch after 5,000 ms = 5
seconds. But it still has to cross two more links, therefore the
total delay is 5,002 ms (in fact, a little bit more due to
store-and-forward delays).

\bigskip

This is much less than with message switching! 

\bigskip

The reason is pipelining: once the first packet reaches the
destination, the source and the two switches are transmitting
simultaneously. 

\end{frame}

% ------------------------------------------------------------------------
%
\begin{frame}
\frametitle{Message segmenting (cont)}

There is another interest in message segmentation.

\bigskip

\textbf{Bit errors} can be introduced into the packets, when moving
along the networks.

\bigskip

When a switch detects a bit error, it discards the whole packet.
\begin{itemize}

  \item If the entire message is a packet (i.e., we do message
  segmenting), then the entire message is discarded.

  \item If the packets are small, only a small part of the message,
  corresponding to the erroneous packets, has to be retransmitted.

\end{itemize}

\end{frame}

% ------------------------------------------------------------------------
%
\begin{frame}
\frametitle{Message segmenting (cont)}

One drawback of segmentation is that the packets do not contain only
the application data but also some control information, called
\textbf{header}, that helps routing.

\bigskip

This situation is comparable to sending mail: the envelope contains
(written on itself) the control information for routing and the letter
contains the application data.

\bigskip

If we cut our message into pieces much smaller than the envelopes,
then the postal service has to carry a heavier load of envelopes than
of messages! 

\bigskip

Consequently, the price would not be the same.

\end{frame}

% ------------------------------------------------------------------------
%
\begin{frame}
\frametitle{Packet forwarding in computer networks}

Packet-switched networks can be divided in two kinds:
\begin{itemize}

  \item \textbf{Datagram networks} routers forward packets according
  to host destination \textbf{address}. Similar to a home address, a
  network address is a unique number on the network. The Internet is a
  datagram network.

  \item \textbf{Virtual circuit networks} routers forward packets
  according to virtual circuit numbers. X.25 and ATM (Asynchronous
  Transfer Mode) use such numbers.

\end{itemize}

\end{frame}

% ------------------------------------------------------------------------
%
\begin{frame}
\frametitle{Virtual circuit networks}

A \textbf{virtual circuit} (\textbf{VC}) consists of
\begin{itemize}

  \item a path, i.e., a series of links and switches;

  \item virtual circuit numbers (VC numbers, in short), one for each
  link along the path;

  \item a VC number translation table in each switch along the path.

\end{itemize}
Once a VC is established, the source can send packets into the VC with
the appropriate VC number.

\bigskip

Because a VC has a different number on each link, the switches must
replace the VC number of each incoming packet with a new one.

\bigskip

This new number is in the translation table of the switches.

\end{frame}

% ------------------------------------------------------------------------
%
\begin{frame}
\frametitle{Virtual circuit networks (cont)}

\textbf{Example.} The numbers in the figure at the next slide are
\textbf{interface numbers}.

\bigskip

They are local to each switch.

\bigskip

Assume a packet starts from host A and must arrive to B: it has to
cross three links. 

\bigskip

It begins with the VC number of the first link, say
12, and enters PS1 at the interface 1. 

\bigskip

There, its VC number has to be replaced by another one, say 22, and
the packet must go out through interface 2 (towards PS2).

\end{frame}


% ------------------------------------------------------------------------
%
\begin{frame}
\frametitle{Virtual circuit networks (cont)}

\begin{center}
  \includegraphics[scale=0.30]{01-12.eps}
\end{center}
 
\end{frame}

% ------------------------------------------------------------------------
%
\begin{frame}
\frametitle{Virtual circuit networks (cont)}

The VC number translation table at switch PS1 looks like

\bigskip

\begin{tabular}{cccc}
  In interface
& In VC\#
& Out interface
& Out VC\#\\
\hline
1 & 12 & 2 & 22\\
\ldots & \ldots & \ldots & \ldots
\end{tabular}

\bigskip

The network (actually, the switches) must maintain a \textbf{state
information} about the VC.

\bigskip

Even if the VC numbers are all the same, the switches still must
associate interfaces and VC numbers.

\bigskip

Each time a new circuit is created, a new entry must be added to the
table of the switches along the path. Dually, each time a connection
is released, the entries must be removed.

\end{frame}

% ------------------------------------------------------------------------
%
\begin{frame}
\frametitle{Datagram networks}

Datagram networks are similar to postal services.

\bigskip

Each host has a unique address in the network.

\bigskip

Elements in the address are hierarchical, in order to simplify the
routing. This is like putting on an envelope country, city, town,
district, street name, building number, apartment number.

\bigskip

We shall discuss later the packet structure.

\bigskip

For now on, remember that a datagram network do not maintain
connection-related information in the routers: these do not know which
connection they support.

\end{frame}

% ------------------------------------------------------------------------
%
\begin{frame}
\frametitle{Network taxonomy}

Here is the taxonomy of the networking concepts we reviewed:
\begin{center}
\includegraphics[scale=0.35]{01-13.eps}
\end{center}
\textbf{Warning!} \emph{This picture does not include the service view
of the networks.} A datagram network is neither a connectionless nor a
connection-oriented network. It can provide both services to different
applications, through TCP or UDP.

\end{frame}
