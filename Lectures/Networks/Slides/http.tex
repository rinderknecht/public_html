%%-*-latex-*-

% ------------------------------------------------------------------------
%
\begin{frame}
\frametitle{Plan}

\begin{itemize}

  \item Application layer

  \begin{itemize}

    \item Principles of application layer protocols

    \item \textbf{The Web and \textsc{http}}

    \item File transfer: \textsc{ftp}

    \item Electronic mail in the Internet

    \item DNS --- The Internet's directory service

    \item Socket programming with TCP and UDP

    \item Content distribution

  \end{itemize}

\end{itemize}

\end{frame}


% ------------------------------------------------------------------------
%
\begin{frame}
\frametitle{Application layer/The Web and \textsc{http}}

The \textbf{Hyper-Text Transfer Protocol} (\textbf{\textsc{http}}) is
the Web application-layer protocol.

\bigskip

\textsc{http} is implemented in two programs: the client and the
server, running on different and connected hosts, and exchanging
\textsc{http} messages.

\bigskip

A \textbf{Web page} (or \textbf{document}) consists of
\textbf{objects}. An object is simply a file (e.g. an \textsc{html}
file, a \textsc{jpeg} image, a \textsf{Java} applet, an audio file
etc.) which is addressable by a single \textbf{Universal Resource
Locator} (\textbf{URL}).

\bigskip

For instance, if a web page contains a \textbf{base \textsc{html}
file} and five \textsc{jpeg} images, then the Web page contains six
objects.

\end{frame}

% ------------------------------------------------------------------------
%
\begin{frame}
\frametitle{Application layer/The Web and \textsc{http} (cont)}

An URL is made of 
\begin{itemize}

  \item a protocol name,

  \item a Web server name,

  \item a file path on the server.

\end{itemize}

For instance \url{http://www.ietf.org/rfc/rfc2396.txt} specifies 
\begin{itemize}

  \item the use of the \textsc{http}, 

  \item a unique computer named \url{www.ietf.org},

  \item and the location of a text file or page to be accessed on that
  computer whose pathname is \url{/rfc/rfc2396.txt}.

\end{itemize}

\end{frame}

% ------------------------------------------------------------------------
%
\begin{frame}
\frametitle{Application layer/The Web and \textsc{http} (cont)}

A Web server houses web objects, each addressable by a URL. Popular
Web servers include Apache.]  When a user clicks on an
  \textbf{hyperlink}, its browser sends \textsc{http} request messages
  for the objects in the referred page to the server, which in turn
  responds with \textsc{http} response messages that contain the
  objects.

\bigskip

Because it relies on TCP, \textsc{http} does neither worry about data
loss nor ordering.

\end{frame}

% ------------------------------------------------------------------------
%
\begin{frame}
\frametitle{Application layer/The Web and \textsc{http} (cont)}

\begin{center}
  \includegraphics[scale=0.4]{02-06.eps}
\end{center}

\end{frame}

% ------------------------------------------------------------------------
%
\begin{frame}
\frametitle{Application layer/The Web and \textsc{http} (cont)}

It is important to understand that the server sends requested objects
to the client without storing any state information about the client.

\bigskip

For instance, if the same client asks for the same file two times in a
short lapse, the server never responds that it has just send it
\emph{because it does not remember}.

\bigskip

That is why \textsc{http} is said to be a \textbf{stateless
protocol}.

\bigskip

There are two kinds of TCP connections: \textbf{persistent} and
\textbf{non-persistent}. 

\end{frame}

% ------------------------------------------------------------------------
%
\begin{frame}
\frametitle{The Web and \textsc{http}/Non-persistent connections} 

Suppose a client uses a non-persistent connection to query a page made
of a base \textsc{html} file and ten \textsc{jpeg} images, all objects
being stored on the same server. The URL for the base \textsc{html}
file is \url{http://www.school.org/dep/index.html}
\begin{itemize}

  \item The \textsc{http} client initiates a TCP
    connection to the server \url{www.school.org} on port number 80,
    which is the default port number for \textsc{http}.\label{step1} 

  \item The \textsc{http} client send an \textsc{http} request message
    to the server via the socket associated with the TCP connection
    that was established in step~\ref{step1}. This message includes
    the path \url{/dep/index.html}.

\end{itemize}

\end{frame}

% ------------------------------------------------------------------------
%
\begin{frame}
\frametitle{The Web and \textsc{http}/Non-persistent connections} 

\begin{itemize}

  \item The \textsc{http} server receives the request via the socket
    associated with the connection, retrieves the object
    \url{/dep/index.html} from its storage (RAM or disk), encapsulates
    the object in an \textsc{http} response message and sends it to
    the client via the same socket.

  \item The \textsc{http} server tells TCP to close the connection
    (but the TCP server waits for the client acknowledgement).

  \item The \textsc{http} client receives the response message. The
    TCP connection terminates. The message indicates that the
    embedded object is an \textsc{html} file. The client extracts this
    file, parses the file and find references to the ten \textsc{jpeg}
    objects.

 \item The first four steps are repeated for the ten \textsc{jpeg}
   images.

\end{itemize}

\end{frame}

% ------------------------------------------------------------------------
%
\begin{frame}
\frametitle{The Web and \textsc{http}/Non-persistent connections (cont)} 

The steps described use \textbf{non-persistent connections} because
each TCP connection is closed after the server sends the object --- it
does not persist for other objects.

\bigskip

Thus, in our example, 11 TCP connections are generated.

\bigskip

However, it is possible that the browser actually opens several TCP
connections in parallel, but these connections still are
non-persistent.

\end{frame}

% ------------------------------------------------------------------------
%
\begin{frame}
\frametitle{The Web and \textsc{http}/Non-persistent connections (cont)} 

Let us estimate the amount of time that elapses between the moment the
client requests a base \textsc{html} file and the time it is received
entirely.

\bigskip

The \textbf{round-trip time} (\textbf{RTT}) is the time needed for a
small packet to travel from the client to the server and back to the
client. It is roughly two RTTs plus the transmission time.

\end{frame}

% ------------------------------------------------------------------------
%
\begin{frame}
\frametitle{The Web and \textsc{http}/Non-persistent connections (cont)} 

\begin{center}
  \includegraphics[scale=0.32]{02-07.eps}
\end{center}

\end{frame}

% ------------------------------------------------------------------------
%
\begin{frame}
\frametitle{The Web and \textsc{http}/Persistent connections}

Non-persistent connections allow only one object to be transmitted
over a given TCP connection, suffering two RTT delays.

\bigskip

With \textbf{persistent connections} the server let open the
connection after sending a response. For instance, the eleven previous
objects could have been sent during a single TCP persistent
connection.

\bigskip

A persistent connection is usually closed after a configurable time
interval. There are two kinds of persistent connections:
\begin{itemize}

  \item \textbf{without pipelining}: the response must be received
  before another request is made;

  \item \textbf{with pipelining}: multiple requests can be gathered
  into the same TCP segment, so the server can fulfill them in
  parallel (leading to less Internet traffic).

\end{itemize}

\end{frame}

% ------------------------------------------------------------------------
%
\begin{frame}[containsverbatim]
\frametitle{The Web and \textsc{http}/\textsc{http} requests}

As expected, there are two kinds of \textsc{http} messages: requests
and responses.

\bigskip

For example:
{\small
\begin{verbatim}
GET /dep/index.html HTTP/1.1
Host: www.school.org
Connection: close
User-agent: Mozilla/4.0
Accept-language:fr
\end{verbatim}
}
A request is written in ordinary \textsc{ascii} text (encoding
characters with 7 bits).

\end{frame}

% ------------------------------------------------------------------------
%
\begin{frame}[containsverbatim]
\frametitle{The Web and \textsc{http}/\textsc{http} requests (cont)}

This message is made of five lines, but can have less or more lines in
general.

\bigskip

The first line \verb+GET /dep/index.html HTTP/1.1+ is called the
\textbf{request line} and the following are the \textbf{header
  lines}.

\bigskip

The request line has three fields: 
\begin{itemize}

  \item the method field (\verb+GET+), 

  \item the URL field (\verb+/dep/index.html+),

  \item the version field (\verb+HTTP/1.1+).

\end{itemize}
The method field can take several values, \verb+GET+, \verb+POST+,
\verb+HEAD+ etc. The \verb+GET+ method is used to request an object.

\end{frame}

% ------------------------------------------------------------------------
%
\begin{frame}[containsverbatim]
\frametitle{The Web and \textsc{http}/\textsc{http} requests (cont)}

The header line \verb+Host: www.school.org+ specifies the host on
which the object resides.

\bigskip

By including the line \verb+Connection: close+ the browser tells the
server it does not want a persistent connection (thus it should be
closed after the object has been sent).

\bigskip

Line \verb+User-agent: Mozilla/4.0+ specifies the browser type and
version, a Mozilla user agent.

\bigskip

Finally, \verb+Accept-language:fr+ indicates that the user prefers to
receive a French version of the object, if available, otherwise a
default version.

\end{frame}

% ------------------------------------------------------------------------
%
\begin{frame}[containsverbatim]
\frametitle{The Web and \textsc{http}/\textsc{http} requests (cont))}

Let us take a look to the general shape of an
\textsc{http} request:
\begin{center}
\includegraphics[scale=0.35]{02-08.eps}
\end{center}
There is another field in the message called \textbf{body}. It is
empty when sending a \verb+GET+ request method but is filled when
using \verb+POST+. This method is used to request a page selected
using some specific information (in the body field), as with a form.

\end{frame}

% ------------------------------------------------------------------------
%
\begin{frame}[containsverbatim]
\frametitle{The Web and \textsc{http}/\textsc{http} requests (cont)}

In fact, not all forms use a \verb+POST+ method: sometimes the user's
information is put in the URL itself and a \verb+GET+ method is used
instead. 

For instance, if a form consists in two fields whose given values are
\texttt{monkeys} and \texttt{bananas}, the following URL is possible:
\begin{center}
\url{www.some-site.com/search?monkeys&bananas}
\end{center}

The \verb+HEAD+ method is similar to a \verb+GET+ one, except the
server does not return the requested object --- even if it sends back
a message. This method is often used by application developers for
debugging purposes.

Protocol \textsc{http} version 1.1 allows a \verb+PUT+ method to
upload objects to a web server and a \verb+DELETE+ method to delete an
object in a web server.

\end{frame}

% ------------------------------------------------------------------------
%
\begin{frame}[containsverbatim]
\frametitle{The Web and \textsc{http}/\textsc{http} responses}

This could be a response message to the request we presented:
{\small
\begin{verbatim}
HTTP/1.1 200 OK
Connection: close
Date: Thu, 06 Aug 1998 12:00:15 GMT
Server: Apache/1.3.0 (Unix)
Last-Modified: Mon, 22 Jun 1998 09:23:24 GMT
Content-Length: 6821
Content-Type: text/html

... data ... data ... data ...
\end{verbatim}
}

\end{frame}

% ------------------------------------------------------------------------
%
\begin{frame}[containsverbatim]
\frametitle{The Web and \textsc{http}/\textsc{http} responses (cont)}

The first line \verb+HTTP/1.1 200 OK+ is the \textbf{status line}. It
tells here that the object was found and is returned. The number
\verb+200+ is the status number (equivalent to the status name). 

\bigskip

The following six lines are the \textbf{header lines}. 

\bigskip

The first one, \verb+Connection: close+ acknowledges the
non-persistent underlying TCP connection. 

\bigskip

The \verb+Date+ line gives the time when the server retrieved the
object (not when the object was created).

\bigskip

The \verb+Server+ line states what kind of web server is running on
what kind of operating system (here, an Apache server on Unix). It is
analogous to the \verb+User-Agent+ line in the request.

\end{frame}

% ------------------------------------------------------------------------
%
\begin{frame}[containsverbatim]
\frametitle{The Web and \textsc{http}/\textsc{http} responses (cont)}

The \verb+Last-Modified+ line indicates the date when the object was
created or last modified. The \verb+Last-Modified+ line is very
important for object caching, both on the client side (browser cache)
and on the network side (proxy servers).

\bigskip

The \verb+Content-Length+ line gives the number of bytes the object is
made.

\bigskip

The \verb+Content-Type+ line records the kind of the object, in this
case a \textsc{html} file (the file extension, like \url{.html} or
\url{.htm} does not tell officially the content type).

\end{frame}

% ------------------------------------------------------------------------
%
\begin{frame}
\frametitle{The Web and \textsc{http}/\textsc{http}
  message format/Responses (cont)}

The general structure of an \textsc{http} response message is
\begin{center}
\includegraphics[scale=0.4]{02-09.eps}
\end{center}

\end{frame}

% ------------------------------------------------------------------------
%
\begin{frame}[containsverbatim]
\frametitle{The Web and \textsc{http}/\textsc{http} responses (cont)}

The status code and its associated phrase can be of several kinds:
\begin{itemize}

  \item \verb+200 OK+ means that the request succeeded;

  \item \verb+301 Moved Permanently+ means that the requested object
  has been permanently moved; the new URL is given in a
  \verb+Location+ header line in the response message and the client
  browser will automatically retrieve the new URL;

  \item \verb+400 Bad Request+ means that the request was not
  understood by the server;

  \item \verb+404 Not Found+ means that the object was not found on
  the server;

  \item \verb+505 HTTP Version Not Supported+ is self explanatory.

\end{itemize}

\end{frame}

% ------------------------------------------------------------------------
%
\begin{frame}[containsverbatim]
\frametitle{The Web and \textsc{http}/\textsc{http} responses (cont)}

It is recommended that you create some small \textsc{http} request
messages, send them to some web server and examine the response messages. 

\bigskip

If you get access to Unix machine (or Windows running
\textsf{Cygwin}\footnote{\url{http://www.cygwin.com}}), just type in a
terminal
{\small
\begin{verbatim}
$ telnet konkuk.ac.kr 80
Trying 194.254.134.22...
Connected to konkuk.ac.kr
Escape character is '^]'.
HEAD /~rinderkn/index.html HTTP/1.0
\end{verbatim}
}
(press the return key twice) and you will see the response from the
server.

\end{frame}

% ------------------------------------------------------------------------
%
\begin{frame}[containsverbatim]
\frametitle{The Web and \textsc{http}/\textsc{http} responses (cont)}

{\small
\begin{verbatim}
HTTP/1.1 200 OK
Date: Mon, 02 May 2005 05:25:56 GMT
Server: Apache/1.3.27 (Unix) mod_jk PHP/4.3.10
X-Powered-By: PHP/4.3.10
Connection: close
Content-Type: text/html

Connection closed by foreign host.
\end{verbatim}
}
Try to request a non-existent object, like \url{foo.html} and check
the response again.

\end{frame}

% ------------------------------------------------------------------------
%
\begin{frame}
\frametitle{The Web and \textsc{http}/Authorisation and cookies}

We mentioned before that an \textsc{http} server is stateless. This
keeps the design of the server simple.

\bigskip

But sometimes it is desirable to identify users, for instance in order
to restrict the access or to save personal informations or to provide
specific services to subscribers only.

\bigskip

\textbf{Authorisation}

Many web sites require users to provide a name and a password in order
to grant the access to the services and objects. Let us follow the
steps of such a process.

\end{frame}

% ------------------------------------------------------------------------
%
\begin{frame}[containsverbatim]
\frametitle{The Web and \textsc{http}/Authorisation and cookies (cont)}

First the user sends an ordinary request. The server responds with a
message with an empty body an with a \verb+401 Authorization Required+
status code. Also it includes a \verb+WWW-Authenticate+ header line
telling the client how to authenticate (here: username and password).

\bigskip

The browser receives that response message and prompts the user for a
name and a password. The client then resends the request with the name
and the password.

\bigskip

After getting the first object, the client has to resend the username
and password, but the user does not need to re-enter these data: the
browser keeps them in a cache memory.

\end{frame}

% ------------------------------------------------------------------------
%
\begin{frame}
\frametitle{The Web and \textsc{http}/Authorisation and cookies (cont)}

\textbf{Cookies}

Many commercial sites make use of \textbf{cookies}. These contain the
following components:
\begin{itemize}

  \item a cookie header line in the \textsc{http} response;

  \item a cookie header line in the \textsc{http} request;

  \item a cookie file kept on the user's system and managed by the
  browser;

  \item a database at the server site.

\end{itemize}

\end{frame}

% ------------------------------------------------------------------------
%
\begin{frame}[containsverbatim]
\frametitle{The Web and \textsc{http}/Authorisation and cookies (cont)}

Let us describe an example of session using cookies.
\begin{itemize}

  \item A user connects for the first time to an e-commerce site using
    cookies. 

  \item The web server creates a unique identification number for
    him and create an entry in its customer database indexed by this
    number (or \textbf{cookie}).

  \item The server responds to the client browser including a
    \verb+Set-cookie+ header line containing the cookie.

  \item The browser reads the \verb+Set-cookie+ line and appends at
    the end of a special file a line containing the web site name
    together with the cookie.

\end{itemize}

\end{frame}

% ------------------------------------------------------------------------
%
\begin{frame}[containsverbatim]
\frametitle{The Web and \textsc{http}/Authorisation and cookies (cont)}

\begin{itemize}

  \item Now, each time the user requests a page on the server, the
    browser sends the cookie along as a \verb+Cookie+ header
    line. This way the server knows exactly the visited pages and the
    order and time of visit.

    This allows the site to provide a ``shopping card'' service:
    during a single visit, the server keeps track of all the intended
    purchases, so that the user pays for all of them once, at the end
    of the session.

  \item The customer pays by giving his name, address and bank card
    number. The site now associates the cookie (and all the details of
    the visit) to the personal data about the customer.

\end{itemize}

\end{frame}

% ------------------------------------------------------------------------
%
\begin{frame}[containsverbatim]
\frametitle{The Web and \textsc{http}/Authorisation and cookies (cont)}

\begin{itemize}

  \item After shopping, if the customer returns to the site, his browser
    will put again the cookie in the requests. Thus the e-commerce
    site can propose new products based on the previous visit and he
    can even by with one click since the site recorded his personal
    banking information during last session.

\end{itemize}
Now we understand how can cookies enable a stateful session between
the client and the server on top of a stateless \textsc{http}
session.

\end{frame}

% ------------------------------------------------------------------------
%
\begin{frame}
\frametitle{The Web and \textsc{http}/Authorisation and cookies (cont)}

\textbf{Privacy}

As you also understand, cookies are a controversial feature because
they can provide a lot of private information to commercial web sites.

\bigskip

These sites can even record the behaviour of a specific user
\emph{across several web sites}. 

\bigskip

Many pages of commercial site request advertisement banners (GIFs or
JPEG images) from an advertising company. These requests may contain a
cookie, and if the user visits several sites which share the same
advertisement company, these company can track the user behaviour
across all the sites he visited.

\bigskip

You can visit \url{http://www.cookiecentral.com/} for news about the
cookie controversy.

\end{frame}

% ------------------------------------------------------------------------
%
\begin{frame}
\frametitle{The Web and \textsc{http}/Conditional \texttt{GET}}

By caching the previously retrieved objects, the traffic on the
internet is decreased: in case the client requests several times the
same object, the copy in the cache memory is delivered instead of
relaying the request again to the server.

\bigskip

This caching can take place either at the client side (i.e. in the
browser) or at a special node in the network.

\bigskip

But caching can create a new problem: the cached objects can be
\emph{stale}, i.e. the original object (at the server side) may have
changed since the last request.

\bigskip

The mechanism in \textsc{http} which ensures that all retrieved
objects are up-to-date despite caching is called \textbf{conditional
  \texttt{GET}}.

\end{frame}

% ------------------------------------------------------------------------
%
\begin{frame}[containsverbatim]
\frametitle{The Web and \textsc{http}/Conditional \texttt{GET} (cont)}

An \textsc{http} request message is a conditional \texttt{GET} if and
only if
\begin{itemize}

  \item the request message uses the \texttt{GET} method,

  \item the request message includes an \texttt{If-Modified-since}
  header line.

\end{itemize}

Let us consider an example.

\bigskip

First, a browser requests an \emph{uncached object} from some web
server:
{\small
\begin{verbatim}
GET /fruit/kiwi.gif HTTP/1.0
User-agent: Mozilla/4.0
\end{verbatim}
}

\end{frame}

% ------------------------------------------------------------------------
%
\begin{frame}[containsverbatim]
\frametitle{The Web and \textsc{http}/Conditional \texttt{GET} (cont)}

Second, the web server sends a response message with the object:
{\small 
\begin{verbatim}
HTTP/1.0 200 OK
Date: Wed, 12 Aug 1998 15:39:29
Server: Apache/1.3.0 (Unix)
Last-Modified: Mon, 22 Jun 1998 09:23:24
Content-Type: image/gif

... data ... data ... data ...
\end{verbatim}
}
The client displays the object and stores it in its local cache
\emph{along with the last modification date}.

\end{frame}

% ------------------------------------------------------------------------
%
\begin{frame}[containsverbatim]
\frametitle{The Web and \textsc{http}/Conditional \texttt{GET} (cont)}

One week later, the client requests the same object, but this may have
changed in the server. In order to get an up-to-date version the
user-agent issues a conditional \texttt{GET}:
{\small
\begin{verbatim}
GET /fruit/kiwi.gif HTTP/1.0
User-agent: Mozilla/4.0
If-modified-since: Mon, 22 Jun 1998 09:23:24
\end{verbatim}
}
Note that the value of the \texttt{If-modified-since} header line is
exactly the same as the one in the previous (and first, here)
request. This conditional \texttt{GET} instructs the server to send
back the requested object if and only if the object has been modified
since the specified date.

\end{frame}

% ------------------------------------------------------------------------
%
\begin{frame}[containsverbatim]
\frametitle{The Web and \textsc{http}/Conditional \texttt{GET} (cont)}

Suppose the object has not been modified in the meanwhile.

\bigskip

Then, fourth, the server sends back he following response message:
{\small
\begin{verbatim}
HTTP/1.0 304 Not Modified
Date: Wed, 19 Aug 1998 15:39:29
Server: Apache/1.3.0 (Unix)
\end{verbatim}
}
The body is empty: the object has not been sent (again). Note also the
status line \verb+Not Modified+.

\bigskip

As a final note on \textsc{http}, let us mention that this protocol
can be used without any browser or human user and also can carry
objects that are not part of web pages, such as XML files.

\end{frame}
