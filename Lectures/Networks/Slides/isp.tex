%%-*-latex-*-

% ------------------------------------------------------------------------
%
\begin{frame}
\frametitle{Plan}

\begin{enumerate}

  \item Computer Networks and the Internet

    \begin{itemize}

      \item What is the Internet?

      \item The network edge

      \item The network core

      \item Network access and physical media

      \item \textbf{ISPs and Internet backbones}

      \item Delay and loss in packet-switched networks
 
      \item Protocol layers and their service models

    \end{itemize}

\end{enumerate}

\end{frame}

% ------------------------------------------------------------------------
%
\begin{frame}
\frametitle{ISPs and Internet backbones}

You may recall that ISPs are interconnected in a hierarchy. The most
important level in this structure is the \textbf{backbone} or
\textbf{tier-1 ISP} (level one is the top of the hierarchy).

\bigskip

Internet backbones are usual ISPs but
\begin{itemize}

  \item its links speed is from 662 Mbps to 10 Gbps;

  \item it is directly connected to \emph{each} other backbones;

  \item it is connected to a large number of tier-2 ISPs and other
  customers; 

  \item it has an international coverage.

\end{itemize}
Hence their routers must be able to forward packets at a very high
rate.

\end{frame}

% ------------------------------------------------------------------------
%
\begin{frame}
\frametitle{ISPs and Internet backbones (cont)}

\begin{center}
\includegraphics[scale=0.30]{01-17.eps}
\end{center}

\end{frame}

% ------------------------------------------------------------------------
%
\begin{frame}
\frametitle{ISPs and Internet backbones (cont)}

A tier-2 ISP has typically a regional or national coverage (depends on
the size of the country, actually). 

\bigskip

Thus, in order to reach a large portion of the Internet, a tier-2 ISP
needs to route traffic through one of the backbones to which it si
connected.

\bigskip

In this case, the tier-2 ISP is said to be a \textbf{customer} of the
backbone and the backbone is said to be a \textbf{provider} to the
tier-2 ISP.

\bigskip

Tier-2 ISP can be directly connected to each other. In general, when
two ISPs are connected directly, they are said \textbf{to peer} each
other.

\bigskip

Within an ISP, the points at which it peers another ISP are called
\textbf{Points of Presence} (\textbf{POP}) and are a router or a group
of routers.

\end{frame}
