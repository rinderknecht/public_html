%%-*-latex-*-

% ------------------------------------------------------------------------
%
\begin{frame}
\frametitle{Plan}

\begin{itemize}

  \item Application layer

  \begin{itemize}

    \item Principles of application layer protocols

    \item The Web and \textsc{http}

    \item \textbf{File transfer: \textsc{ftp}}

    \item Electronic mail in the Internet

    \item DNS --- The Internet's directory service

    \item Socket programming with TCP and UDP

    \item Content distribution

  \end{itemize}

\end{itemize}

\end{frame}

% ------------------------------------------------------------------------
%
\begin{frame}
\frametitle{File Transfer Protocol (\textsc{ftp})}

The \textbf{File Transfer Protocol} (\textbf{\textsc{ftp}}) follows a
client/server model, as \textsc{http}.
\begin{center}
\includegraphics[scale=0.4]{02-10.eps}
\end{center}
The user wants to access a remote account in order to retrieve or send
some files. Therefore he must possess an identification and a password
on the remote machine. \textsc{ftp} runs on top of TCP.

\end{frame}

% ------------------------------------------------------------------------
%
\begin{frame}
\frametitle{File Transfer Protocol (\textsc{ftp}) (cont)}

First, he enters the name of the remote host to the \textsc{ftp}
client through the \textsc{ftp} user agent (under Unix, it is simply a
shell command-line but in Windows it is a graphical interface).

Once the server has authorised the user, he can copy files in
his local filesystem to the remote host or vice versa.

Both \textsc{http} and \textsc{ftp} are file transfer protocols but
there is an important technical difference: \textsc{ftp} uses two
different TCP connections to transfer a file: a \textbf{control
  connection} and a \textbf{data connection}.

\end{frame}

% ------------------------------------------------------------------------
%
\begin{frame}
\frametitle{File Transfer Protocol (\textsc{ftp}) (cont)}

The control connection is used for exchanging control information
between the two hosts, like identification, password, commands to
change remote directory, commands to \texttt{get} and \texttt{put}
files etc.

The data connection is used to actually exchange the files.

Because \textsc{ftp} uses a separate control connection, it is said to
send its control information \textbf{out-of-band}. Dually,
\textsc{http} sends its control information \textbf{in-band}.
\begin{center}
\includegraphics[scale=0.4]{02-11.eps}
\end{center}

\end{frame}

% ------------------------------------------------------------------------
%
\begin{frame}
\frametitle{File Transfer Protocol (\textsc{ftp}) (cont)}

First, the user starts an \textsc{ftp} session with the remote host by
initiating a control TCP connection on server port~21.

The \textsc{ftp} client sends user identification and password to the
server over this control connection.

After authentication, the client then sends commands to change remote
current directory. 

When the server receives a command for file transfer to the client
over the control connection, it opens a TCP data connection on client
port~20. 

\end{frame}

% ------------------------------------------------------------------------
%
\begin{frame}
\frametitle{File Transfer Protocol (\textsc{ftp}) (cont)}

After sending exactly one file over the data connection, the server
closes it.

So, during an \textsc{ftp} session, only the control connection
remains open, one data connection is open for one file at a time, then
closed, i.e. data connections are non-persistent.

Throughout a session, the \textsc{ftp} server must maintain a
\textbf{state} about the user:
\begin{itemize}

  \item the control connection must be associated with a specific
  user account;

  \item the server must keep track of a given user's current
  directory on its file-system.

\end{itemize}
By contrast, \textsc{http} is a stateless protocol --- it does not
keep any user state.

\end{frame}

% ------------------------------------------------------------------------
%
\begin{frame}[containsverbatim]
\frametitle{File Transfer Protocol (\textsc{ftp}) (cont)}

\textbf{Common commands and replies}

The commands from client to server, and replies, are sent over the
control connection in \textsc{ascii} (i.e. non-accentuated characters
and digits encoded with seven bits). Hence, \textsc{ftp} commands, as
\textsc{http} ones, are readable by people (plain text).

One command lies on one line and it consists of four characters and
some optional arguments. The more common are
\begin{itemize}

  \item \verb+USER username+ is used to send the user identification
  to the server;

  \item \verb+PASS password+ is used to send the user's password to the
  server;

  \item \verb+LIST+ is used to ask the server the list of all files
  in the current directory; this list is sent back on a new and
  non-persistent data connection;

\end{itemize}

\end{frame}

% ------------------------------------------------------------------------
%
\begin{frame}[containsverbatim]
\frametitle{File Transfer Protocol (\textsc{ftp}) (cont)}

\begin{itemize}

  \item \verb+RETR filename+ is used to retrieve (i.e. \texttt{get})
  a file from the current remote directory;

  \item \verb+STOR filename+ is used to store (i.e. \texttt{put}) a
  file into the current directory of the remote host.

\end{itemize}
There is typically a one-to-one correspondence between the commands
the user enters (by means of the user agent) and the commands the
\textsc{ftp} client sends, as exactly one client's \texttt{RETR} for
one user's \texttt{get}.

\end{frame}

% ------------------------------------------------------------------------
%
\begin{frame}[containsverbatim]
\frametitle{File Transfer Protocol (\textsc{ftp}) (cont)}

The replies consist in a three-digit number followed by an optional
message --- this is similar to the status code and phrase in the
status line of the \textsc{http} response messages.

Some typical replies are
\begin{itemize}

  \item \verb+331 Username OK, password required+

  \item \verb+125 Data connection already open; transfer starting+

  \item \verb+425 Can't open data connection+

  \item \verb+452 Error writing file+

\end{itemize}

\end{frame}
