%%-*-latex-*-

% ------------------------------------------------------------------------
%
\begin{frame}
\frametitle{Plan}

\begin{itemize}

  \item Application layer

  \begin{itemize}

    \item Principles of application layer protocols

    \item The Web and \textsc{http}

    \item File transfer: \textsc{ftp}

    \item Electronic mail in the Internet

    \item \textbf{DNS --- The Internet's directory service}

  \end{itemize}

\end{itemize}

\end{frame}

% ------------------------------------------------------------------------
%
\begin{frame}
\frametitle{Internet Directory Service (\textsc{dns})}

The internet protocols use \textbf{IP addresses} to identify
hosts. They are a series of numbers, between 0 and 255, separated by
periods like \url{202.30.38.143}.

But these numbers have are difficult to remember, contrary to
\textbf{host names}, which are made of a series of words separated by
periods, like \url{mail.konkuk.ac.kr}.

The internet provides an application-layer service that allows
applications to find the IP address of a host from its name: the
\textbf{Domain Name Service} (\textbf{\textsc{dns}}). 

Because of its functional similarity with the telephone directories,
which map subscri\-bers' names to their phone number, this service is
also called \textbf{Internet Directory Service}.

\end{frame}

% ------------------------------------------------------------------------
%
\begin{frame}
\frametitle{Internet Directory Service (\textsc{dns}) (cont)}

The \textsc{dns} is
\begin{itemize}

  \item a client/server protocol;

  \item a distributed database implemented in a hierarchy of
  \textbf{name servers};

  \item an application-layer protocol that allows hosts and name
  servers to communicate in order to provide the translation service.

\end{itemize}

Name servers are often machines running the UNIX operating system and
the Berkeley Internet Name Domain (BIND) software.

The \textsc{dns} protocol runs over \textsc{udp} and uses port 53. 

\textsc{dns} is commonly used by other application-layer protocols, like
\textsc{http}, \textsc{smtp} and \textsc{ftp}.

\end{frame}

% ------------------------------------------------------------------------
%
\begin{frame}
\frametitle{Internet Directory Service (\textsc{dns}) (cont)}

When a user wants to retrieve the web page whose URL is
\url{http://konkuk.ac.kr/index.html}, 
\begin{enumerate}

  \item its browser extracts the hostname \url{konkuk.ac.kr} from the URL,

  \item passes it to the client side of the \textsc{dns} application,

  \item this application sends to a \textsc{dns} server the hostname,

  \item after some delay, it receives back the IP address of the
    referred host,

  \item this IP address is returned to the browser,

  \item the browser opens a \textsc{tcp} connection to the host and to
    the \textsc{http} server process located there.

\end{enumerate}

\end{frame}

% ------------------------------------------------------------------------
%
\begin{frame}
\frametitle{Internet Directory Service (\textsc{dns}) (cont)}

DNS provides a few other important services in addition to translating
hostnames to IP addresses:

\begin{itemize}

  \item \textbf{Host aliasing.} A host with a complicated name can
    have one or more (simpler) names. For example
    \url{relay1.west-coast.enterprise.com} may have two alia\-ses,
    e.g., \url{enterprise.com} and \url{www.enterprise.com}. In this
    case, \url{relay1.west-coast.enterprise.com} is said to be the
    \textbf{canonical hostname}.

  \item \textbf{Mail server aliasing.} Mail server names can also be
  aliased in order to get a simple e-mail address, as
  \url{bob@hotmail.com} instead of
  \url{bob@relay1.west-coast.hotmail.com}.

\end{itemize}

\end{frame}

% ------------------------------------------------------------------------
%
\begin{frame}
\frametitle{Internet Directory Service (\textsc{dns}) (cont)}

\begin{itemize}

  \item \textbf{Load distribution.} Busy sites are replicated over
  multiple servers, hence an \emph{ordered list} of IP addresses is
  associated with one canonical hostname. Each time a client requires
  an IP address in the list, the first is returned and the list is
  rotated. This way, the load is balanced between all the servers.

\end{itemize}

\end{frame}

% ------------------------------------------------------------------------
%
\begin{frame}
\frametitle{Internet Directory Service (\textsc{dns}) (cont)}

What if there were only one name server in the internet?

\begin{itemize}

  \item If the name server crashes, so does the internet.

  \item A single name server would have to handle all the
  \textsc{dns} queries (from \textsc{http} requests and e-mails at
  hundreds of millions of hosts).

  \item A single name server cannot be geographically close to all
  the hosts, imposing unfair delays to the hosts.

  \item A single name server would need a huge database to store all
  the hostname on the internet.

\end{itemize}

Therefore \emph{a centralised database does not scale}.

\end{frame}

% ------------------------------------------------------------------------
%
\begin{frame}
\frametitle{Internet Directory Service (\textsc{dns}) (cont)}

That is why

\begin{itemize}

  \item there is a large number of name servers around the world

  \item which are organised in a hierarchy;

  \item no single name server has all the mappings for all the hosts;

  \item the mappings are distributed across the name servers;

  \item there are three kind of name servers:
  \begin{enumerate}

    \item \textbf{local name servers},

    \item \textbf{root name servers},

    \item \textbf{authoritative name servers}.

  \end{enumerate}

\end{itemize}

\end{frame}

% ------------------------------------------------------------------------
%
\begin{frame}
\frametitle{\textsc{dns}/Local name servers}

Each ISP has a \textbf{local name server} (also called a
\textbf{default name server}), so it is relatively close to the users
of this ISP:
\begin{itemize}

  \item in an institution, it can be on the same LAN as the user;

  \item in case of a residential ISP, it is separated from the users
  by no more than a few routers.

\end{itemize}

The IP address of the local name server has to be set by the user in
the network configuration of his host.

If the user requests a translation for a host which is part of the
same ISP domain, the request will be \emph{immediately} served by the
local name server. As host \url{surf.eurecom.fr} requesting the IP
address of \url{baie.eurecom.fr}.

\end{frame}

% ------------------------------------------------------------------------
%
\begin{frame}
\frametitle{\textsc{dns}/Root name servers}

When a local name server can not serve a request, it then acts as a
\textsc{dns} client and queries a \textbf{root name server}.

If the root name server has the required mapping, it returns the IP
address to the local name server which, in turn, delivers it to the
querying (initial) host.

If the root name server has not the required record, it knows the IP
address of an \emph{authoritative name server} which has it.

\end{frame}

% ------------------------------------------------------------------------
%
\begin{frame}
\frametitle{\textsc{dns}/Authoritative name
  servers}

Every host is registered with an authoritative name server.

This authoritative name server is a name server in the
(\emph{registered}) host's local ISP. 

In principle, each host should be registered in two authoritative name
servers, in case of failures.

By definition, a name server is authoritative for a given host if it
always translates the host name into its IP address.

Many name servers act both as local and authoritative name servers.

\end{frame}

% ------------------------------------------------------------------------
%
\begin{frame}
\frametitle{\textsc{dns}/Example}

Let us consider an example. Assume that host \url{surf.eurecom.fr}
queries the IP address of host \url{gaia.cs.umass.edu}.
\begin{enumerate}
  \item \url{surf.eurecom.fr} sends its query to its local name server
    \url{dns.eurecom.fr},

  \item which, in turn, forwards the query to a root name server,

  \item which, in turn, forwards the query to an authoritative name
    server for the host \url{gaia.cs.umass.edu}, named
    \url{dns.umass.edu}.
\end{enumerate}

\end{frame}

% ------------------------------------------------------------------------
%
\begin{frame}
\frametitle{\textsc{dns}/Example (cont)}

\begin{center}
  \includegraphics[scale=0.32]{02-16.eps}
\end{center}

\end{frame}

% ------------------------------------------------------------------------
%
\begin{frame}
\frametitle{\textsc{dns}/Example (cont)}

Finally, the IP address of host \url{gaia.cs.umass.edu} is forwarded
back to host \url{surf.eurocom.fr}, through all the \textsc{dns}
clients (steps 4, 5 and 6).

Until now we assumed that a root name server always knows an
authoritative name server for the requested host. In fact, this is not
always true: it usually knows an \textbf{intermediary name server},
which may know an authoritative name server or another intermediary
name server.

Let us reconsider the previous example and assume that the root name
server does not know an authoritative name server for
\url{gaia.cs.umass.edu} but, instead, a name server for the whole
domain (the university) \url{umass.edu}, called
\url{dns.umass.edu}. This intermediary name server knows all the name
servers for all the departments, in particular \url{cs.umass.edu}.

\end{frame}

% ------------------------------------------------------------------------
%
\begin{frame}
\frametitle{\textsc{dns}/Example (cont)}

\begin{center}
\includegraphics[scale=0.28]{02-17.eps}
\end{center}

\end{frame}

% ------------------------------------------------------------------------
%
\begin{frame}
\frametitle{\textsc{dns}/Recursive
  and iterative queries}

As in our example, when a name server \emph{A} becomes a \textsc{dns}
client of name server \emph{B} which, in turn, is client of \emph{C}
and that \emph{C} sends the result to \emph{B}, this is called a
\textbf{recursive query} because \emph{B} acts in behalf of \emph{A}.

It is also possible that \emph{B} responds to \emph{A} with the IP
address of name server \emph{C}, leaving to \emph{A} the task to
\emph{directly} query \emph{C}: this is an \textbf{iterative query}
--- \emph{A} makes all the queries.

It is possible to mix recursive and iterative queries for serving a
single user's request.

Iterative queries save bandwidth and processing resources in some name
servers (here, \emph{B}), that is why it is a good idea to use them to
ease the burden of root name servers (here, \emph{B}).

\end{frame}

% ------------------------------------------------------------------------
%
\begin{frame}
\frametitle{\textsc{dns}/Recursive
  and iterative queries (cont)}

\begin{center}
\includegraphics[scale=0.28]{02-18.eps}
\end{center}

\end{frame}

% ------------------------------------------------------------------------
%
\begin{frame}
\frametitle{\textsc{dns}/Caching}

In order to improve the performances, name servers make use of
\textbf{caches}.

When a name server receives a translation query and, later, the
corresponding IP address, it copies the pair (hostname, address) in
its memory or saves it on its local disk --- as well as serving the
result to its client.

Therefore, the next time the same query is received, the name server
can answer just but by searching its cache.

In order to cope with ephemeral hosts, mapping pairs are kept in the
caches only for a pre-configured period of time (often set to two
days).

\end{frame}

% ------------------------------------------------------------------------
%
\begin{frame}
\frametitle{\textsc{dns}/Records}

A (hostname, address) is embedded in a data structure called a
\textbf{resource record} (\textbf{RR}). In turn, resource records are
stored in databases in each name server.

Each \textsc{dns} message (either query or response) carries one or
more resource records.

More precisely, a resource record has the structure (\emph{name},
\emph{value}, \emph{type}, \emph{TTL}) where \emph{TTL} is the
\textbf{Time To Live} record; it determines the time at which a
resource should be removed from the cache.

We will ignore the \emph{TTL} field in the following description.

\end{frame}

% ------------------------------------------------------------------------
%
\begin{frame}
\frametitle{\textsc{dns}/Records (cont)}

The meaning of \emph{name} and \emph{value} depend on the field
\emph{type}:

\begin{itemize}

  \item if \emph{type} is \emph{A}, then \emph{name} is a hostname and
  \emph{value} is the corresponding IP address. This is the standard
  hostname-to-address mapping. For example (\url{konkuk.ac.kr},
  202.30.38.143, \emph{A}).

  \item if \emph{type} is \emph{NS}, then \emph{name} is a domain
  (such as \url{umass.edu}) and \emph{value} is the hostname of an
  authoritative name server for all the hosts in \emph{name}. This
  kind of record is used to route \textsc{dns} queries along in the
  query chain. For instance, (\url{umass.edu}, \url{dns.umass.edu},
  \emph{NS}).

\end{itemize}

\end{frame}

% ------------------------------------------------------------------------
%
\begin{frame}
\frametitle{\textsc{dns}/Records (cont)}

\begin{itemize}

  \item if \emph{type} is \emph{CNAME}, then \emph{value} is a
  canonical hostname for the alias hostname \emph{name}. This record
  provide querying hosts the canonical name of a host. As an example,
  (\url{google.com}, \url{www.google.com}, \emph{CNAME}).

  \item if \emph{type} is \emph{MX}, then \emph{value} is the
  canonical name of the mail server \emph{name}, like the record
  (\url{google.com}, \url{mail.google.com}, \emph{MX}). A company can
  both have an \emph{MX} record and a \emph{CNAME} record, allowing
  the same alias both for its mail server and its web server:
  depending on the querying application, the corresponding record will
  be requested.

\end{itemize}

\end{frame}

% ------------------------------------------------------------------------
%
\begin{frame}
\frametitle{\textsc{dns}/Records (cont)}

Now we can better understand what is being an authoritative name
server or not.

\begin{itemize}

  \item If a name server is authoritative for a given hostname, then
  it must contain a \emph{type A} record for this hostname. 

  \item If a name server is not authoritative for a given hostname,
  it may contain a \emph{type A} record in its cache. Otherwise it
  must contain a \emph{type NS} record for the domain including the
  hostname, as well as a \emph{type A} record giving the IP address of
  the domain name server given in the \emph{type NS} record.

  For example, assume that a root name server is not authoritative for
  the hostname \url{gaia.cs.umass.edu} and no \emph{type A} record is
  in cache for it. Therefore this name server must contain a record
  like (\url{umass.edu}, \url{dns.umass.edu}, \emph{NS}) and
  (\url{dns.umass.edu}, 128.119.40.111, \emph{A}).

\end{itemize}


\end{frame}

% ------------------------------------------------------------------------
%
\begin{frame}
\frametitle{\textsc{dns}/Messages}

\vspace*{-1pt}

There are only two kind of \textsc{dns} messages and both queries and
replies have the same format:
\begin{center}
\includegraphics[scale=0.37]{02-19.eps}
\end{center}

\end{frame}

% ------------------------------------------------------------------------
%
\begin{frame}
\frametitle{\textsc{dns}/Messages (cont)}

The first 12 bytes are the \textbf{header section}, which is made of
successive fields:
\begin{enumerate}

  \item the \textbf{identifier} is a 16 bits number that identifies
    a query and its answer;

  \item the \textbf{flags} are one bit flags:
  \begin{itemize}

    \item \textbf{query} is 0 and \textbf{reply} is 1;

    \item \textbf{authoritative} name server or not;

    \item \textbf{recursive query} required/not required (by client)
    or supported/not supported (by server);

  \end{itemize}

  \item the \textbf{numbers} are four numbers that count the number of
    occurrences of four kind of data in the next sections.
 
\end{enumerate}

\end{frame}

% ------------------------------------------------------------------------
%
\begin{frame}
\frametitle{\textsc{dns}/Messages (cont)}

There are now four \textbf{data sections}:

\begin{enumerate}

  \item the \textbf{question section} contains information about the
  current query:
  \begin{enumerate}

    \item the \textbf{hostname} being queried;

    \item the type of query, e.g., \emph{type A} or \emph{type MX};

  \end{enumerate}

  \item the \textbf{answer section} contains information about the
    current reply: possibly multiple records (because a host can be
    duplicated);

  \item the \textbf{authority section} contains records of other
    authoritative servers;

  \item the \textbf{additional section} complements other
    informations, e.g., if a reply is an \emph{MX} record, this section
    provides the IP address of the canonical mail server.

\end{enumerate}

\end{frame}
