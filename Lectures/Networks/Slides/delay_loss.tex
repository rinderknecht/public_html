%%-*-latex-*-

% ------------------------------------------------------------------------
%
\begin{frame}
\frametitle{Plan}

\begin{enumerate}

  \item Computer Networks and the Internet

    \begin{itemize}

      \item What is the Internet?

      \item The network edge

      \item The network core

      \item Network access and physical media

      \item ISPs and Internet backbones

      \item \textbf{Delay and loss in packet-switched networks}
 
      \item Protocol layers and their service models

    \end{itemize}

\end{enumerate}

\end{frame}

% ------------------------------------------------------------------------
%
\begin{frame}
\frametitle{Delay and loss in packet-switched networks}

Let us come back to the packets that travel a packet-switched network
like the Internet.

\bigskip

When a packet crosses a link and a router, it suffers different kinds
of delays: 
\begin{itemize}

  \item \textbf{nodal processing delay}, 

  \item \textbf{queuing delay},

  \item \textbf{transmission delay},
 
  \item \textbf{propagation delay}. 

\end{itemize}
All together, they give the \textbf{total nodal delay}.

\bigskip

Let us now look with more attention to each of these delays, because
it is necessary if we want to acquire a deep understanding of packet
switching.

\end{frame}

% ------------------------------------------------------------------------
%
\begin{frame}
\frametitle{Delay and loss in packet-switched networks/Types of delay}

\begin{itemize}

  \item \textbf{Processing delay} is the time needed to examine the
    packet's header and find where to direct the packet. Also it can
    include the time for checking for bit-level errors in the packet.

    \item \textbf{Queuing delay} happens when some earlier-arriving
      packet is already waiting in the queue, else it is 0.

    \item \textbf{Transmission delay} is the time needed to ``push''
      the packet out on the link (size of the packet/rate of the
      link).

    \item \textbf{Propagation delay} is the time needed to propagate
      a signal on the link (distance A-B/signal speed).

\end{itemize}

\end{frame}

% ------------------------------------------------------------------------
%
\begin{frame}
\frametitle{Delay and loss in packet-switched networks/Types of delay}

\begin{center}
  \includegraphics[scale=0.35]{01-18.eps}
\end{center}

\end{frame}

% ------------------------------------------------------------------------
%
\begin{frame}
\frametitle{Delay and loss in packet-switched networks/Types of delay}

What is the difference between transmission delay and propagation
delay?

\bigskip

Transmission delay depends on the packet's length (measured in bits)
and the bit rate of the link (measured in bits per seconds), but has
nothing to do with the distance between two routers.

\bigskip

Propagation delay is the time it takes a bit to propagate from one
router to the next; it is a function of the distance between to
adjacent routers, but has nothing to do with the packet's length or
the transmission rate of the link.

\end{frame}

% ------------------------------------------------------------------------
%
\begin{frame}
\frametitle{Queuing delay and packet loss}

The delay which can be the highest and is the most difficult to assess
is the queuing delay. 

\bigskip

The main reason is because it depends from
packet to packet, contrary to the other kinds of delays. For instance,
if ten packets arrive at very short intervals to an empty queue, the
first one will suffer no queuing delay but the tenth will have to wait
in the queue the other have been retransmitted.

\bigskip

That is why the usual tool to evaluate the queuing delay are
\textbf{statistical models}, including average queuing delay and the
probability that the queuing delay exceeds some specified value.

\end{frame}

% ------------------------------------------------------------------------
%
\begin{frame}
\frametitle{Queuing delay and packet loss (cont)}

The queuing delay depends on
\begin{itemize}

  \item the rate at which traffic arrives at the queue,

  \item the transmission rate of the (out-going) link,

  \item the nature of the incoming traffic:

  \begin{itemize}

    \item is it periodical?

    \item does it arrives in bursts?

  \end{itemize}

\end{itemize}

\end{frame}

% ------------------------------------------------------------------------
%
\begin{frame}
\frametitle{Queuing delay and packet loss (cont)}

Let us assume that the average rate at which packets reach the queue
is \(a\). It is measured in packets/sec.

\bigskip

Let \(R\) be the transmission rate, measured in bits/sec, which is the
rate at which bits are pushed in the out-going link.

\bigskip

Suppose that all packets are \(L\) bits long.

\bigskip

Then the average rate at which bits arrive at the queue is \(La\).

\bigskip

Assume the queue is unbounded.

\bigskip

The ratio \(La/R\), called the \textbf{traffic intensity}, plays an
important role in estimating the growth of the queue.

\end{frame}

% ------------------------------------------------------------------------
%
\begin{frame}
\frametitle{Queuing delay and packet loss (cont)}

If \(La/R > 1\) then the flow of incoming bits exceeds the capacity of
the out-going link: the queue will grow forever. So the most important
rule here is: \emph{design your network so that the traffic intensity
is no greater than 1}.

\bigskip

If \(La/R \leqslant 1\) then the nature of the arriving traffic
impacts the queuing delay. For example, if packets arrive
periodically, one each \(L/R\) second, then every packet will arrive
at an empty queue and there will be no queuing delay (this is the best
case).

\bigskip

If packets arrive in bursts, i.e., if the number of packets per second
varies, the delay can be significant (especially if the variation is
not linear).

\end{frame}

% ------------------------------------------------------------------------
%
\begin{frame}
\frametitle{Queuing delay and packet loss (cont)}

In practice, periodic arrival or linear bursts are improbable: the
arrival is \emph{random}. 

\bigskip

Thus the quantity \(La/R\) is not enough to fully characterise the
queuing delay, but it helps you to grasp the concept.

\bigskip

The curve in the facing column gives you more intuition on the average
queuing delay. The important thing to note is that a little variation
on \(La/R\), when this quantity is close to 1, may lead to a huge
augmentation of the average delay.

\end{frame}

% ------------------------------------------------------------------------
%
\begin{frame}
\frametitle{Queuing delay and packet loss (cont)}

\begin{center}
  \includegraphics[scale=0.40]{01-19.eps}
\end{center}

\end{frame}

% ------------------------------------------------------------------------
%
\begin{frame}
\frametitle{Queuing delay and packet loss (cont)}

In our previous presentation, we assumed an unbounded queue,
i.e., allowing an infinite number of packets to remain in the
queue. That is why the traffic intensity could approach~1 at any
distance.

\bigskip

But, in practice, routers have a finite memory, so a packet can arrive
and find a full queue. In this case the packet is dropped, i.e., it is
lost.

\bigskip

The packet losses increase as the traffic intensity increases.

\bigskip

Therefore, performance at a router is not only measured in terms of
traffic intensity but also of packet loss.

\end{frame}

% ------------------------------------------------------------------------
%
\begin{frame}
\frametitle{Queuing delay and packet loss (cont)}

Until now we considered the delay at one node (a router). What about
the \textbf{end-to-end delay}, that is the cumulated delay of each
node from the source to the destination? Assume there are \(N\)
routers along the path. Let us suppose that there is no congestion
(queuing delay is negligible).
\begin{itemize}

  \item Let \(d_{\text{proc}}\) be the processing delay at each node
  (including the source),

  \item \(d_{\text{prop}}\) the propagation delay of each link,

  \item the transmission rate of each node is \(R\) bits/sec.

\end{itemize}
The nodal delays accumulate and give an end-to-end delay
\[d_{\text{end-end}} = N (d_{\text{proc}} + d_{\text{trans}} +
d_{\text{prop}})\]
where \(d_{\text{trans}} = L/R\), where \(L\) is the packet size.

\end{frame}

% ------------------------------------------------------------------------
%
\begin{frame}
\frametitle{Delays and routes in the Internet}

\textsf{Traceroute} is a simple program that can run on any Internet
host. It is given a destination host and this program sends special
packets to this destination. Along their way, the packets pass through
several routers, which send back to the source a message containing
their name (if any) and their address, called \textbf{IP address} and
whose structure we will discuss later.

\bigskip

This allows \textsf{Traceroute} to reconstruct the path and to show it
to the end-user.

\bigskip

Imagine there are \(N\) routers along the path. Then the source will
send \(N\) special packets, all addressed to the destination. They are
numbered from 1 to \(N\) and sent by increasing numbers. The \(n\)-th
router receives the \(n\)-th packet, destroys it and sends an
identification message back to the source.

\end{frame}

% ------------------------------------------------------------------------
%
\begin{frame}
\frametitle{Delays and routes in the Internet (cont)}

The source records the elapsed time between the moment it sent a
packet and the moment the corresponding return message is received,
coming from a router. This delay is called \textbf{round-trip delay}.

\bigskip

\textsf{Traceroute} repeats the experiment three times, because the
round-trip delay can vary due to queuing delays, so the user gets an
idea of the delay variation.

\bigskip

The web site \url{http://www.traceroute.org} provides a list of links
to web sites, classified by countries, which provide an interface to
\textsf{Traceroute}. 

\bigskip

This way you can experiment by giving the address or name of a
destination and the interface will report the path from the host
running the web site to the destination host, as given by
\textsf{Traceroute}.

\end{frame}

% ------------------------------------------------------------------------
%
\begin{frame}[containsverbatim]
\frametitle{Delays and routes in the Internet (cont)}

Here is an example of output of \textsf{Traceroute} from a host in the
Faroe Islands (Denmark) to a host in France (at INRIA):
{\tiny
\begin{verbatim}
traceroute to pauillac.inria.fr (128.93.11.35), 30 hops max, 38 byte packets
 1  L4-0-0.bone.olivant.fo (212.55.32.1)  0.977 ms  0.612 ms  0.766 ms
 2  feth1-0-0.utland1.bone2.olivant.fo (212.55.32.98)  1.108 ms  1.234 ms  1.601 ms
 3  ser4-0.45M.ldn2nxi2.ip.tele.dk (195.215.170.85)  30.455 ms  30.676 ms  29.567 ms
 4  ge1-2-2.1000M.ldn2nxg1.ip.tele.dk (195.249.13.121)  30.745 ms  30.134 ms  31.431 ms
 5  pos6-0.2488M.asd9nxg1.ip.tele.dk (195.249.2.134)  37.281 ms  37.332 ms  36.983 ms
 6  Ge7-1.AMSBB1.Amsterdam.opentransit.net (193.251.254.9)  37.103 ms  37.334 ms  37.726 ms
 7  * * *
 8  * * *
 9  * * *
\end{verbatim}
}

\end{frame}

% ------------------------------------------------------------------------
%
\begin{frame}[containsverbatim]
\frametitle{Delays and routes in the Internet (cont)}

{\tiny
\begin{verbatim}
10  P14-0.PASCR2.Pastourelle.opentransit.net (193.251.128.105) 58.415 ms  57.089 ms  57.764 ms
11  193.51.185.2 (193.51.185.2)  57.802 ms  61.880 ms  58.485 ms
12  inria-g3-1.cssi.renater.fr (193.51.180.174)  60.999 ms  58.497 ms  57.900 ms
13  royal-inria.cssi.renater.fr (193.51.182.73)  58.860 ms  63.028 ms  58.961 ms
14  193.48.202.2 (193.48.202.2)  63.017 ms  58.502 ms  57.889 ms
15  rocq-gw-bb.inria.fr (192.93.1.100)  60.721 ms  61.437 ms  61.078 ms
16  pauillac.inria.fr (128.93.11.35)  60.759 ms  62.540 ms  61.703 ms
\end{verbatim}
}
The domain extensions (\texttt{.fo}, \texttt{.dk}, \texttt{.fr}) help
you to follow the packets along the corresponding countries. Note that
some routers have no name (only address). Routers 7, 8 and 9 do not
send back their identification messages, that is why
\textsf{Traceroute} prints an asterisk for each try.

\end{frame}
