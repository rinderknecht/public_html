%%-*-latex-*-

% ------------------------------------------------------------------------
%
\begin{frame}
\frametitle{Plan}

\begin{enumerate}

  \item Computer Networks and the Internet

    \begin{itemize}

      \item What is the Internet?

      \item \textbf{The network edge}

      \item The network core

      \item Network access and physical media

      \item ISPs and Internet backbones

      \item Delay and loss in packet-switched networks
 
      \item Protocol layers and their service models

    \end{itemize}

\end{enumerate}

\end{frame}

% ------------------------------------------------------------------------
%
\begin{frame}
\frametitle{The network edge}

End users may interact directly with the host (e.g., a Mac) or
indirectly (e.g., a web server).

\bigskip

A \textbf{client program} runs on a host that requests and receives
a service from a \textbf{server program} running on another
host. This is the \textbf{client/server model}.

\bigskip

In the \textbf{peer-to-peer model}, there is little or no use of
servers.

\end{frame}

% ------------------------------------------------------------------------
%
\begin{frame}
\frametitle{The network edge (cont)}

\begin{center}
  \includegraphics[scale=0.27]{01-03.eps}
\end{center}

\end{frame}

% ------------------------------------------------------------------------
%
\begin{frame}
\frametitle{The network edge/Connection-oriented service}

The goal is to transfer data between two end systems in the following
manner:

\begin{itemize}
 
  \item First, \textbf{handshaking} takes place: the two systems
  agree on the forthcoming exchange. This is like the `Hi/Hi (back)'
  in human protocols. Both hosts set their internal state in
  accordance, i.e., they record the fact that they are communicating
  with a known peer. Then data is transmitted.

  \item This is summarized in figure page~\pageref{protocols}: the
  two first messages consist in the handshaking and the two following
  (\texttt{GET} and the response containing the file) are the data
  communication itself.

  \item In the Internet the connection-oriented service is the
  \textbf{Transmission Control Protocol} (\textbf{TCP}), used by most
  of the applications (like telnet, SMTP, ftp, http).

\end{itemize}

\end{frame}

% ------------------------------------------------------------------------
%
\begin{frame}
\frametitle{The network edge/TCP added-services}

The TCP has been designed to carry \emph{more} than
connection-oriented service, but also
\begin{itemize}

  \item \textbf{reliability}: the (byte stream) data delivery, in
  order and in its entirety is guaranteed. As a coarse approximation,
  reliability is achieved by way of \textbf{acknowledgment} and
  \textbf{retransmission}: each time a packet is received, a special
  packet is sent back to acknowledge the receipt; when such
  acknowledgment is missing, the sender assumes the packet got lost
  and retransmits it.

  \item \textbf{flow control}: the sender slows down and avoids
  overwhelming the receiver by sending too many packets too fast;


  \item \textbf{congestion control}: the sender slows down when the
  routers start loosing packets because they are congested by a too
  heavy traffic.

\end{itemize}

\end{frame}

% ------------------------------------------------------------------------
%
\begin{frame}
\frametitle{The network edge/Connectionless service}

In connectionless services, the goal is still data transfer between
hosts but there is no handshaking.

In the Internet, the \textbf{User Datagram Protocol} (\textbf{UDP})
provides a connectionless service to the applications. This means:

\begin{itemize}

  \item no reliable transfer (the data can arrive too soon, i.e., when
  the receiver is not expecting it),

  \item no flow control,

  \item no congestion control.

\end{itemize}

The applications must handle themselves these aspects. 

Internet phone and video conferencing, streaming, DNS rely on UDP.

\end{frame}
