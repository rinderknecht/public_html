%%-*-latex-*-

% ------------------------------------------------------------------------
%
\begin{frame}
\frametitle{Plan}

\begin{itemize}

  \item \textbf{Application layer}

  \begin{itemize}

    \item Application layer protocols

    \item The Web and \textsc{http}

    \item File transfer: \textsc{ftp}

    \item Electronic mail in the Internet

    \item DNS --- The Internet's directory service

    \item Socket programming with TCP and UDP

    \item Content distribution

  \end{itemize}

\end{itemize}

\end{frame}

% ------------------------------------------------------------------------
%
\begin{frame}
\frametitle{Application layer}

The network applications are the reason why we use the Internet,
because they are the programs we interact with to connect our host
to another host on the network.

\bigskip

There are many popular network applications:
\begin{itemize}

  \item electronic mail, remote access to computers, file
  transfers, newsgroups, text chats (in the 1980's);
  
  \item Web (mid-1990's);

  \item instant messaging, peer-to-peer (end of 1990's).

\end{itemize}

\end{frame}

% ------------------------------------------------------------------------
%
\begin{frame}
\frametitle{Plan}

\begin{itemize}

  \item Application layer

  \begin{itemize}

    \item \textbf{Application layer protocols}

    \item The Web and \textsc{http}

    \item File transfer: \textsc{ftp}

    \item Electronic mail in the Internet

    \item DNS --- The Internet's directory service

    \item Socket programming with TCP and UDP

    \item Content distribution

  \end{itemize}

\end{itemize}

\end{frame}

% ------------------------------------------------------------------------
%
\begin{frame}
\frametitle{Application-layer protocols}

A network application is made of several distributed software
components, each of them running on different hosts.

\bigskip

Precisely, we do not say that programs communicate with each other,
because programs are basically texts. The execution of a program
gives birth to the notion of \textbf{process}. Processes are the
communicating entities.

\bigskip

Network applications use application-layer protocols which define the
format of the exchanged messages and the actions undertaken by the
hosts on the receipt and sending of given messages.

\end{frame}

% ------------------------------------------------------------------------
%
\begin{frame}
\frametitle{Application-layer protocols (cont)}

The figure in the next slide shows the \textbf{protocol stacks} at
each communicating host.

\bigskip

Applications are processes which interact, within the same layer, with
the \textbf{operating system} and the user, and use the transport
layer services for communicating across the network.

\bigskip

This behaviour is depicted using solid blue arrows.

\end{frame}

% ------------------------------------------------------------------------
%
\begin{frame}
\frametitle{Application-layer protocols (cont)}

\begin{center}
  \includegraphics[scale=0.27]{02-01.eps}
\end{center}

\end{frame}

% ------------------------------------------------------------------------
%
\begin{frame}
\frametitle{Application-layer protocols (cont)}

Keep in mind the difference between a \textbf{network application} and
an \textbf{application\hyp{}layer protocol}:
\begin{itemize}

  \item a network application is a process;

  \item an application-layer protocol is the definition of data
  formats and of the behaviour of communicating network applications.

\end{itemize}
In other words, a network application \textbf{implements} an
application-layer protocol, but also do more, like offering to the
user a graphical interface etc.

\bigskip

Some application-layer protocols are \textsc{pop} and \textsc{smtp}
for e-mail. The mail client, like \textsf{Outlook},
\textsf{Thunderbird} or \textsf{Evolution}, is a network application.

\end{frame}

% ------------------------------------------------------------------------
%
\begin{frame}
\frametitle{Application-layer protocols (cont)}

The figure in the next page shows that a network application is
made of a \textbf{client} and a \textbf{server}.

\bigskip

Usually a host implements both the client and the server, e.g.,  when
an \textsc{ftp} session exists between two hosts, either host can
transfer files to the other host.

\bigskip

However, the host that initiates the session is labeled the client.

\end{frame}


% ------------------------------------------------------------------------
%
\begin{frame}
\frametitle{Application-layer protocols (cont)}

\begin{center}
  \includegraphics[scale=0.27]{02-02.eps}
\end{center}

\end{frame}

% ------------------------------------------------------------------------
%
\begin{frame}
\frametitle{Application-layer protocols/Sockets}

A process sends and receives messages through its \textbf{socket}. By
analogy, a socket can be considered as an application's door to the
network, which is assumed (by the application) to provide a transport
service for the messages entering it. Formally, it is an \textbf{API}
(\textbf{Application Programmers' Interface}) between the application
and the network.

\hfill\includegraphics[scale=0.34]{02-03.eps}

\end{frame}

% ------------------------------------------------------------------------
%
\begin{frame}
\frametitle{Application-layer protocols/Addressing processes}

In order for one process on one host to send a message to another
process running on another host, it must identify the receiving
process.

\bigskip

For this, two elements are needed on the Internet:
\begin{itemize}

  \item the \textbf{IP address} of the receiving host, which is
  a 32 bits number globally unique;

  \item a \textbf{port number} within the receiving host, which
  allows addressing the message to the peer process.

\end{itemize}
Therefore, a network application must be assigned a port number,
e.g., 80 for Web server processes (using \textsc{http} protocol) or 25
for mail server processes (using \textsc{smtp} protocol).

\end{frame}

% ------------------------------------------------------------------------
%
\begin{frame}
\frametitle{Application-layer protocols/User agents}

A network application usually offers a graphical interface to the
user. It is usually called \textbf{user agent} because it relays the
wishes of the user to the core of the application network, which
interacts in turn with the transport layer through sockets.

\bigskip

Web browsers and mail readers provide such user agents, while
implementing specific application-layer protocols, respectively
\textsc{http} (web), \textsc{smtp} (sending e-mails), \textsc{pop3} or
\textsc{imap} (retrieving e-mails).

\end{frame}

% ------------------------------------------------------------------------
%
\begin{frame}
\frametitle{Application-layer protocols/Transport services}

The Internet offers several transport protocols to the network
applications:
\begin{itemize}

  \item \textbf{Reliable data transfer.} Some applications, such as
  e-mail, file transfer, remote host access, web document retrieving
  etc. require no data loss. Others, such as real-time or stored audio
  and video playing are \textbf{loss-tolerant applications}.

  \item \textbf{Bandwidth.} Some applications, such as Internet
  telephony, require a minimum bandwidth allocation: they are
  \textbf{bandwidth-sensitive applications}. Others, such as e-mail,
  can make use of any available bandwidth. They are \textbf{elastic
    applications}.

  \item \textbf{Timing.} Some applications, such as multi-player
  games, require timing constraints on the end-to-end delay: they are
  \textbf{time-sensitive applications}.

\end{itemize}

\end{frame}

% ------------------------------------------------------------------------
%
\begin{frame}
\frametitle{Application-layer protocols/Transport services (cont)}

\begin{center}
\includegraphics[scale=0.4]{02-04.eps}
\end{center}

\end{frame}

% ------------------------------------------------------------------------
%
\begin{frame}
\frametitle{Application-layer protocols/TCP services}

Let us revisit the TCP services:
\begin{itemize}

  \item \textbf{Connection-oriented service.} TCP has the client and
    the server exchange transport\hyp{}layer control information
    before they exchange application\hyp{}level messages: this is
    \textbf{handshaking}.

  After it, a \textbf{TCP connection} is said to exist between the
  sockets of the two processes. 

  The connection is \textbf{full-duplex}, i.e., the two processes can
  send information to the other at the same time. After, the
  connection must be torn down.

  \item \textbf{Reliable transport service.} The processes rely on
  TCP to deliver their messages correctly, entirely and orderly.

\end{itemize}

\end{frame}

% ------------------------------------------------------------------------
%
\begin{frame}
\frametitle{Application-layer protocols/TCP services (cont)}

TCP does \emph{not} guarantee
\begin{itemize}

  \item minimum transfer rate (due to flow and congestion control),

  \item end-to-end delay (due to routing protocols and queuing).

\end{itemize}
\begin{center}
\includegraphics[scale=0.42]{02-05.eps}
\end{center}

\end{frame}
