%%-*-latex-*-

% ------------------------------------------------------------------------
%
\begin{frame}
\frametitle{Plan}

\begin{itemize}

  \item Application layer

  \begin{itemize}

    \item Principles of application layer protocols

    \item The Web and \textsc{http}

    \item File transfer: \textsc{ftp}

    \item \textbf{Electronic mail in the Internet}

    \item DNS --- The Internet's directory service

    \item Socket programming with TCP and UDP

    \item Content distribution

  \end{itemize}

\end{itemize}

\end{frame}

% ------------------------------------------------------------------------
%
\begin{frame}
\frametitle{Electronic mail in the Internet}

Electronic mail (or \textbf{e-mail}) is one of the first application
of the Internet, and the most successful.

It is an \textbf{asynchronous} communication means, i.e. the two
correspondent do not need to be using the e-mail application at the
same moment.

The next figure provides a high-level view of the e-mail system. The
three major components are 
\begin{itemize}

  \item \textbf{user agents} (also called in this context
  \textbf{mail readers}),

  \item \textbf{mail servers},

  \item \textbf{Simple Mail Transfer Protocol} (\textbf{\textsc{smtp}}).

\end{itemize}


\end{frame}

% ------------------------------------------------------------------------
%
\begin{frame}
\frametitle{Electronic mail in the Internet (cont)}

\begin{center}
\includegraphics[scale=0.28]{02-12.eps}
\end{center}

\end{frame}

% ------------------------------------------------------------------------
%
\begin{frame}
\frametitle{Electronic mail in the Internet (cont)}

We will illustrate how the e-mail system works through the example of
a user Alice which sends an e-mail to its recipient Bob.

\begin{enumerate}

  \item When Alice is finished composing her message, her user agent
    sends it to its mail server, where it is put in the outgoing
    \textbf{message queue}.


  \item In turn, Alice's mail server will send it to Bob's mail
    server, where it will be stored in Bob's incoming message queue
    (also called \textbf{mailbox}).

  \item When Bob will check his e-mails, his user agent will download
    Alice's message from his mailbox in his mail server and display
    it.

\end{enumerate}

\end{frame}

% ------------------------------------------------------------------------
%
\begin{frame}
\frametitle{Electronic mail in the Internet (cont)}

\noindent \textbf{Mailboxes}

Each recipient, such as Bob, has a mailbox located in one of the mail
servers. Bob's mailbox manages the messages that have been sent to
him.

When Bob wants to read his new messages, he has to connect to his
mailbox and the hosting mail server will authenticate him, by
requesting a user name and a password.

\noindent \textbf{Delivery failures}

In case Alice's mail server fails to deliver her e-mail to Bob's mail
server, the message is kept in Alice's mail server outgoing queue and
the server will try to re-send it every 30~minutes or so. If there is
no success after several days (often 5 days), the server removes the
messages and informs Alice of the situation.

\end{frame}

% ------------------------------------------------------------------------
%
\begin{frame}
\frametitle{\textsc{smtp}}

The Simple Mail Transfer Protocol (\textsc{smtp}) is the main
application-layer protocol for Internet e-mail. It relies on
\textsc{tcp} to transfer the e-mails.

\textsc{smtp} is divided into a \textbf{client} and a \textbf{server}
entity: the client executes on the sender's mail server and the server
runs on the recipient's mail server. Both client and server run on
every mail server. A mail server becomes an \textsc{smtp} client when
he sends some e-mail, and when it receives some e-mail it behaves as
an \textsc{smtp} server.

Contrary to \textsc{http}, \textsc{smtp} does require the mail content
to be encoded in \textsc{ascii}. This implies that multimedia content
must be encoded by the sender's agent and decoded by the recipient's
agent.

\end{frame}

% ------------------------------------------------------------------------
%
\begin{frame}
\frametitle{\textsc{smtp} (cont)}

Let us illustrate a mail exchange between Alice and Bob in a more
detailed way.

\begin{itemize}

  \item Alice invokes her mail agent, provides Bob's \textbf{e-mail
    address}, composes a message and order the agent to send the
    message.

  \item Alice's agent sends the message to her mail server, where it
    is placed in a queue.

  \item The client side of \textsc{smtp}, running on Alice's mail
    server, detects the message in the queue. It opens a \textsc{tcp}
    connection to an \textsc{smtp} server running on Bob's mail
    server, on port 25.

  \item After some \textsc{smtp} handshaking, the \textsc{smtp} client
    sends Alice's message into the \textsc{tcp} connection.

  \item At Bob's mail server, the server side of \textsc{smtp}
    receives Alice's message and places it in Bob's mailbox.

\end{itemize}

\end{frame}

% ------------------------------------------------------------------------
%
\begin{frame}
\frametitle{\textsc{smtp} (cont)}

\begin{itemize}

  \item When Bob decides to read his messages he runs his mail agent,
    which connects to his mail server.

    Bob's mail server authenticates him and delivers to his mail
    agent Alice's message.

\end{itemize}
This situation is summarized in the following picture:
\begin{center}
\includegraphics[scale=0.4]{02-13.eps}
\end{center}
Note that there is a direct \textsc{tcp} connection between the two
mail servers.
\end{frame}

% ------------------------------------------------------------------------
%
\begin{frame}[containsverbatim]
\frametitle{\textsc{smtp} (cont)}

Let us detail the \textsc{smtp} session between Alice's client and
Bob's server.

In the following transcript, lines prefixed by \textbf{C:} are sent
verbatim by the client agent to its \textsc{tcp} socket, and lines
starting with \textbf{S:} are exactly the lines send to the
\textsc{tcp} connection by the server. Client hostname is
\url{crepes.fr} and server hostname \url{hamburger.edu}.
{\small
\begin{tabular}{rl}
\textbf{S:} & \verb+220 hamburger.edu+\\
\textbf{C:} & \verb+HELO crepes.fr+\\
\textbf{S:} & \verb+250 Hello crepes.fr, pleased to meet you+\\
\textbf{C:} & \verb+MAIL FROM: <alice@crepes.fr>+\\
\textbf{S:} & \verb+250 alice@crepes.fr ... Sender ok+\\
\textbf{C:} & \verb+RCPT TO: <bob@hamburger.edu>+\\
\textbf{S:} & \verb+250 bob@hamburger.edu ... Recipient ok+
\end{tabular}
}

\end{frame}

% ------------------------------------------------------------------------
%
\begin{frame}[containsverbatim]
\frametitle{\textsc{smtp} (cont)}

{\small
\begin{tabular}{rl}
\textbf{C:} & \verb+DATA+\\
\textbf{S:} & \verb+354 Enter mail, end with "." on a line by itself+\\
\textbf{C:} & \verb+Do you like ketchup?+\\
\textbf{C:} & \verb+How about pickles?+\\
\textbf{C:} & \verb+.+\\
\textbf{S:} & \verb+250 Message accepted for delivery+\\
\textbf{C:} & \verb+QUIT+\\
\textbf{S:} & \verb+221 hamburger.edu closing connection+
\end{tabular}
}

Note that \textsc{smtp} uses persistent \textsc{tcp} connections,
therefore if the sending mail server has several messages to send to
the same receiving mail server, it can send all the messages over the
same \textsc{tcp} connection.

\end{frame}

% ------------------------------------------------------------------------
%
\begin{frame}[containsverbatim]
\frametitle{\textsc{smtp} (cont)}

It is recommended to use \textsf{Telnet} to send some e-mail:
\begin{verbatim}
$ telnet mail.konkuk.ac.kr 25
\end{verbatim}
where \url{mail.konkuk.ac.kr} is the name of the \emph{remote}
server. This command establishes a \textsc{tcp} connection between the
local host and the remote mail server.

After invoking \textsf{Telnet} at the prompt, the client receives:
{\small
\begin{verbatim}
Trying 202.30.38.143...
Connected to mail.konkuk.ac.kr.
Escape character is '^]'.
220 konkuk.ac.kr ESMTP Postfix
\end{verbatim}
}
and can type now \textsc{smtp} commands.

\end{frame}

% ------------------------------------------------------------------------
%
\begin{frame}
\frametitle{Comparison with \textsc{http}}

Both \textsc{smtp} and \textsc{http} transfer files (objects
vs. mails) from one host to another use persistent \textsc{tcp}
connections (mandatory for \textsc{smtp}). The differences are:
\begin{enumerate}

  \item an \textsc{http} client pulls files (objects) from a server,
    whereas an \textsc{smtp} pushes them (mails) to a server: we say
    that \textsc{http} is a \textbf{pull protocol} whereas
    \textsc{smtp} is a \textbf{push protocol};

  \item \textsc{smtp} requires the body of each message to be encoded
    in \textsc{ascii}, contrary to \textsc{http}, thus accentuated
    characters or binary data must be encoded and decoded;

  \item when a file consists of several parts (like text and images),
    \textsc{smtp} places all of them in the same message, contrary to
    \textsc{http} which puts them in separate messages.

\end{enumerate}

\end{frame}

% ------------------------------------------------------------------------
%
\begin{frame}[containsverbatim]
\frametitle{Mail message formats and MIME}

The body of an e-mail can be preceded by a blank line and a
\textbf{header} made of header lines. Each of these lines is made of a
keyword followed by a value.

Header lines starting with keywords \verb+From:+ and \verb+To:+ are
mandatory. The header line starting with \verb+Subject:+ is
optional. For instance:
{\small
\begin{verbatim}
From: Alice
To: bob@hamburger.edu
Subject: Some topics
\end{verbatim}
}
A blank line follows the header and then the contents.

\textbf{Warning!} The header lines are \textbf{not} \textsc{smtp}
commands, but are part of the mail body. The header is interpreted by
the \emph{receiving mail agent} (not the \textsc{smtp} server), which
then displays it.

\end{frame}

% ------------------------------------------------------------------------
%
\begin{frame}[containsverbatim]
\frametitle{The MIME extension for non-\textsc{ascii} data}

The header can be used to tell the receiving mail agent how to handle
multimedia (i.e. non-\textsc{ascii}) and multi-part (i.e. attached
files) e-mails. These extra header lines and their interpretation
constitute the \textbf{Multipurpose Internet Mail Extensions}
(\textbf{MIME}).

The two header keywords enabling MIME are 
\begin{itemize}

  \item {\small \verb+Content-Transfer-Encoding:+}

  \item {\small \verb+Content-Type:+}

\end{itemize}
The first field instructs the receiving mail agent what kind of
encoding to \textsc{ascii} has been used, so it can reverse the
process. Then, the second field enables the receiving mail agent to
take an appropriate action on the message, as launching an image
viewer if the content is JPEG.

\end{frame}

% ------------------------------------------------------------------------
%
\begin{frame}[containsverbatim]
\frametitle{The MIME extension for non-\textsc{ascii} data (cont)}

Let us take an example. Imagine Alice wants to send a JPEG image to
Bob. Her mail agent generates a MIME message, which might look like
this:
{\small
\begin{verbatim}
From: Alice
To: bob@hamburger.edu
Subject: Picture
MIME-Version: 1.0
Content-Transfer-Encoding: base64
Content-Type: image/jpeg

... base64 encoded data ...
\end{verbatim}
}
Alice's mail agent encoded the JPEG file using base64, which is one
possible technique to produce \textsc{ascii} from binary.

\end{frame}

% ------------------------------------------------------------------------
%
\begin{frame}[containsverbatim]
\frametitle{The MIME extension for non-\textsc{ascii} data (cont)}

When Bob's mail agent reads this message, it reads the 
{\small 
\begin{verbatim}
Content-Transfer-Encoding: base64
\end{verbatim}
} 
header, so it proceeds to decode the message body with a
base64-decoder.

The header line 
{\small 
\begin{verbatim}
Content-Type: image/jpeg
\end{verbatim}
} 
tells Bob's mail agent that the message should be interpreted
(\emph{after decoding}) as a JPEG image. Thus, it may propose to Bob
to launch a JPEG viewer if he would like to.

The following header line tells the version of MIME used by Alice's
mail agent: {\small
\begin{verbatim}
MIME-Version: 1.0
\end{verbatim}
} 

\end{frame}

% ------------------------------------------------------------------------
%
\begin{frame}[containsverbatim]
\frametitle{The MIME extension for non-\textsc{ascii} data (cont)}

There are two formats of MIME types:
{\small
\begin{semiverbatim}
Content-Type: \emph{type}/\emph{subtype}
Content-Type: \emph{type}/\emph{subtype}; \emph{parameters}
\end{semiverbatim}
}
In the previous example, the type was \verb+image+ and the subtype was
\verb+jpeg+. The subtype restricts the meaning of the type, thus
allowing a more precise interpretation of the content. 

Most parameters are associated with a specific subtype and are similar
to a variable definition.

The set of types and subtypes, as well as parameter, is open and
increasing. New types should be registered at the Internet Assigned
Numbers Authority (IANA)\footnote{\url{http://www.iana.org/}}.

\end{frame}

% ------------------------------------------------------------------------
%
\begin{frame}[containsverbatim]
\frametitle{The MIME extension for non-\textsc{ascii} data (cont)}

Currently there are seven types defined. The following are common.
\begin{itemize}

  \item \textbf{text} tells the receiving mail agent that the body
  contains text. If the subtype is \verb+plain+, the mail agent does
  not need any extra software to display the body, except font
  support. The character set can be specified using a parameter, like
  \verb+charset=euc-kr+.

  Another popular pair is \verb+text/html+, which informs the
  receiving agent that it can display the content as a web page.

  \item \textbf{image} tells the receiving mail agent that the body
  is an image. Two pairs of type and subtype are popular:
  \verb+image/jpeg+ and \verb+image/gif+.

\end{itemize}

\end{frame}

% ------------------------------------------------------------------------
%
\begin{frame}[containsverbatim]
\frametitle{The MIME extension for non-\textsc{ascii} data (cont)}

\begin{itemize}

  \item \textbf{application} is used when the content does not fit
  any other type. It is often used for data that must be processed by
  an application before it can be viewed. For instance,
  \verb+application/msword+ instructs the receiving mail agent to
  launch Microsoft Word or OpenOffice to read the mail.

  \item \textbf{multipart} is used when the mail contains several
  parts because the sender wanted to send some other files together
  with his main message. If the attached files are not text, the pair
  \verb+multipart/mixed+ is often used.

\end{itemize}

\end{frame}

% ------------------------------------------------------------------------
%
\begin{frame}[containsverbatim]
\frametitle{The MIME extension for non-\textsc{ascii} data (cont)}

Let us say a little more about multi-part e-mails.

The mail agent needs to determine where each part starts and ends
inside the body, how each non-\textsc{ascii} part was encoded and what
kind of content is it. 

This is simply achieved by 
\begin{itemize}
  \item placing \emph{boundary characters} between each part,

  \item repeating \verb+Content-Transfer-Type:+ and 
   \verb+Content-Type:+ header line at the beginning of each part.

\end{itemize}

\end{frame}

% ------------------------------------------------------------------------
%
\begin{frame}[containsverbatim]
\frametitle{The MIME extension for non-\textsc{ascii} data (cont)}

Assume Alice wants to send a message to Bob consisting of some
\textsc{ascii} text and a JPEG image. After she types the text and
attach the image, the agent produces
{\small
\begin{verbatim}
From: alice@crepes.fr
To: bob@hamburger.edu
Subject: My picture
MIME-Version: 1.0
Content-Type: multipart/mixed; Boundary=StartOfNextPart

--StartOfNextPart
Dear Bob, look at my picture:
--StartOfNextPart
Content-Transfer-Encoding: base64
Content-Type: image/jpeg
... base64 encoded data ...
\end{verbatim}
}

\end{frame}

% ------------------------------------------------------------------------
%
\begin{frame}[containsverbatim]
\frametitle{Received Message header}

The \emph{receiving} \textsc{smtp} server can also insert a specific
line at the top of an e-mail: the \verb+Received:+ header line. This
line specifies the originating \textsc{smtp} server, the receiving
\textsc{smtp} server and the reception time. Like:
{\small
\begin{verbatim}
Received: from crepes.fr by hamburger.edu;
          12 Oct 98 15:27:39 GMT
From: alice@crepes.fr
To: bob@hamburger.edu
Subject: My picture
.....
\end{verbatim}
}

\end{frame}

% ------------------------------------------------------------------------
%
\begin{frame}[containsverbatim]
\frametitle{Received Message header (cont)}

A user can instruct his mail server to forward his e-mail to another
mail server. In this case each receiving mail server, i.e. the
forwarding one and the final one, will add a \verb+Received:+ line to
the original mail.

For instance, if Bob forwards his e-mails to the mail server
\verb+kimchi.kr+, the final messages will look like
{\small
\begin{verbatim}
Received: from hamburger.edu by kimchi.kr;
          3 Jul 01 15:30:01 GMT
Received: from crepes.fr by hamburger.edu;
          3 Jul 01 15:17:39 GMT
From: alice@crepes.fr
To: bob@hamburger.edu
Subject: My picture
.....
\end{verbatim}
}

\end{frame}

% ------------------------------------------------------------------------
%
\begin{frame}
\frametitle{Mail Access Protocols}

We implicitly assumed until now that Bob reads his e-mails by directly
logging in his mail server.

For a long time, this was the only way but since the mid 90's users
have a mail agent that connects to the user's mail server, downloads
the e-mails on the local disk and displays their contents to the user.

Using a mail agent allows to store locally the e-mails but also to
easily visualise their possible multimedia attachments (images,
videos, music etc.) --- the purpose of a mail server is to serve
mails, not to provide fancy features for display.

Note: the idea to run the recipient's mail server on the same machine
as the mail agent is not good. Indeed, the user may want to turn his
machine off and so the mail server would be, leading to the delivery
failure of subsequent incoming mails.

\end{frame}

% ------------------------------------------------------------------------
%
\begin{frame}
\frametitle{Mail Access Protocols (cont)}

Remember that a mail server is always on, it is usually shared by
several users and is maintained by the ISP.

So
\begin{itemize}

  \item Alice's mail agent has \emph{to push} her e-mail to her mail
  server,

  \item Bob's user agent has \emph{to pull} his e-mails to his local
  disk.

\end{itemize}

The first task can be done using \textsc{smtp} because it is a push
protocol by design. Why a two-step procedure (from Alice's mail agent
to her mail server, then from her mail server to Bob's mail server)?
Because otherwise, if Bob's mail server is down, Alice's mail agent
should keep trying again to send the e-mails for days, impeding Alice
to log out meanwhile.

\end{frame}

% ------------------------------------------------------------------------
%
\begin{frame}
\frametitle{Mail Access Protocols (cont)}

The second tasks cannot be achieved by \textsc{smtp}: we need a pull
protocol, or \textbf{Mail Access Protocol}.

There are several possibilities: \textbf{\textsc{pop3}} (\textbf{Post
  Office Protocol} version 3), \textbf{\textsc{imap}}
(\textbf{Internet Mail Access Protocol}) and \textsc{http}:
\begin{center}
\includegraphics[scale=0.4]{02-14.eps}
\end{center}

\end{frame}

% ------------------------------------------------------------------------
%
\begin{frame}
\frametitle{Mail Access Protocols/POP3}

\textsc{pop3} is an extremely simple mail access protocol. 

The session starts with the recipient's mail agent opens a
\textsc{tcp} connection to its mail server on port number 110.

Then three phases occur in turn:
\begin{enumerate}

  \item \textbf{Authorisation}: the user agent sends the user's name
  and password to the mail server, which authenticates him;

  \item \textbf{Transaction}: the user agent retrieves the messages
  and the user can: mark the mails to be deleted on the server,
  unmark them and get statistics about them;

  \item \textbf{Update}: the mails marked for deletion on the server
  are actually deleted.

\end{enumerate}

\end{frame}

% ------------------------------------------------------------------------
%
\begin{frame}
\frametitle{Mail Access Protocols/POP3
  (cont)}

In a \textsc{pop3} transaction, the user agent issue textual commands
and the server responds to each command with a textual reply.

There are two possible replies:
\begin{itemize}

  \item \texttt{+OK} (sometimes followed by server-to-client data)
  meaning the last client command was fine;

  \item \texttt{-ERR} meaning that something was wrong with the last
  command.

\end{itemize}

\end{frame}

% ------------------------------------------------------------------------
%
\begin{frame}[containsverbatim]
\frametitle{Mail Access Protocols/POP3 (cont)}

\textbf{Authorisation}

There are two principal commands: \texttt{user \emph{name}} and
\texttt{pass \emph{password}}.

Use \textsf{telnet} to try these commands:

{\small
\begin{verbatim}
$ telnet mail.konkuk.ac.kr 110
Trying 202.30.38.143...
Connected to mail.konkuk.ac.kr.
Escape character is '^]'.
+OK <11115.1117125523@konkuk.ac.kr>
user rinderkn
+OK 
\end{verbatim}
}

\end{frame}

% ------------------------------------------------------------------------
%
\begin{frame}[containsverbatim]
\frametitle{Mail Access Protocols/POP3 (cont)}

Example of failed authentication:
{\small 
\begin{verbatim}
$ telnet mail.konkuk.ac.kr 110
Trying 202.30.38.143...
Connected to mail.konkuk.ac.kr.
Escape character is '^]'.
+OK <15247.1117126049@konkuk.ac.kr>
user foo 
+OK 
pass bar
-ERR authorization failed
Connection closed by foreign host.
\end{verbatim}
}

\end{frame}

% ------------------------------------------------------------------------
%
\begin{frame}
\frametitle{Mail Access Protocols/POP3 (cont)}

\textbf{Transaction}

The users can configure his mail agent to \textbf{download and delete}
or \textbf{download and keep}, depending whether they wish to keep a
copy of their mails on the server or not.

According to this configuration, the user agent sends different
commands to the mail server, e.g. in the ``download and delete'' mode,
the user agent will issue the \texttt{list}, \texttt{retr} and
\texttt{dele} commands.

Imagine a user has two messages in his mailbox. In the following
transcript, lines prefixed by \textbf{C:} are sent verbatim by the
client agent to its \textsc{tcp} socket, and lines starting with
\textbf{S:} are exactly the lines sent to the \textsc{tcp} connection
by the server. 

\end{frame}


% ------------------------------------------------------------------------
%
\begin{frame}[containsverbatim]
\frametitle{Mail Access Protocols/POP3 (cont)}

{\small
\renewcommand\arraystretch{0.8}
\begin{tabular}{rl}
\textbf{C:} & \verb+list+\\
\textbf{S:} & \verb+1 498+\\
\textbf{S:} & \verb+2 912+\\
\textbf{S:} & \verb+.+\\
\textbf{C:} & \verb+retr 1+\\
\textbf{S:} & \verb+... blah blah ...+\\
\textbf{S:} & \verb+.+\\
\textbf{C:} & \verb+dele 1+\\
\textbf{C:} & \verb+retr 2+\\
\textbf{S:} & \verb+... blah blah ...+\\
\textbf{S:} & \verb+.+\\
\textbf{C:} & \verb+dele 2+\\
\textbf{C:} & \verb+quit+\\
\textbf{S:} & \verb+OK POP3 server signing off+
\end{tabular}
}

\end{frame}

% ------------------------------------------------------------------------
%
\begin{frame}
\frametitle{Mail Access Protocols/POP3 (cont)}

A problem with this ``download and delete'' approach is that the
recipient, Bob, may be nomadic and want to access his mail messages
from multiple machines, e.g. his office computer, his home computer
and his portable computer.

If Bob reads his messages from his office machine, he will not be able
to reread them from his home machine. 

In the ``download and keep'' mode, the user agent leaves the messages
on the server after downloading (a copy of) them.

During a \textsc{pop3} session, the server keeps a client state, but
not across sessions --- which greatly simplifies the server
implementation.

\end{frame}

% ------------------------------------------------------------------------
%
\begin{frame}
\frametitle{Mail Access Protocols/IMAP}

With \textsc{pop3}, once the recipient has downloaded his messages, he
can store them in a hierarchy of folders and later search information
in it (e.g. by sender's name or subject).

But a nomadic user would get a hierarchy with different messages \emph{for
each machine} from where he downloads the mails. One centralised
hierarchy would be preferable in this case. This is not possible with
\textsc{pop3}.

To solve this problem, \textbf{\textsc{imap}} (\textbf{Internet Mail
  Access Protocol}) provides more features than \textsc{pop3}, at the
cost of more complex implementations.

\end{frame}


% ------------------------------------------------------------------------
%
\begin{frame}
\frametitle{Mail Access Protocols/IMAP (cont)}

An \textsc{imap} server will associate each message with a folder
(initially, it is the ``Inbox'' folder).

The user can create and change folders on the server, as well as
searching the messages they contain according to specific criteria.

This means that \textsc{imap}, contrary to \textsc{pop3}, maintain
user state information \emph{across sessions}.

Another useful \textsc{imap} feature is the possibility for the client
to obtain only some components of a message. For instance, a user
agent can request just the body of a message or one part of a
multi-part MIME message. This is very useful in case of a
low-bandwidth connection.

\end{frame}

% ------------------------------------------------------------------------
%
\begin{frame}
\frametitle{Mail Access Protocols/Web mail}

A lot of users today access their e-mails through their Web
browsers. This technique was introduced in the mid-1990's by the
company Hotmail.

In this case, Alice's mail agent sends her messages using
\textsc{http} requests (like forms) to her web server which, in turn,
uses \textsc{smtp} to push the mail to the recipient's mail server.

Symmetrically, Bob retrieves his messages using \textsc{http} requests
to his web server which, in turn uses \textsc{imap} to get the messages
from the sender's mail server.

This web-based e-mail system is very useful for mobile users since it
only requires a web browser and an internet connection.

\end{frame}

