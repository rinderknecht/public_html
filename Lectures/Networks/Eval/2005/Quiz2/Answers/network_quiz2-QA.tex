%%-*-latex-*-

\documentclass[11pt,a4paper]{article}

\input{trace}

\title{Answers to the quiz \#2 of introduction to networking} 
\author{Christian Rinderknecht} 
\date{\today}

\usepackage{nath}

\begin{document}

\maketitle

\section{Review questions}

\begin{enumerate}

  \item \emph{Why is it that \textsc{ftp} sends control information
    out-of-band?}

  \textsc{ftp} uses two parallel \textsc{tcp} connections: one for
  sending control information (such as a request to transfer a file or
  change the current remote directory) and another connection for
  actually transferring the file. Because the control information is
  not sent over the same connection that the file is sent over,
  \textsc{ftp} sends control information out of band.

  \item \emph{Why \textsc{ftp} is said to be a stateful protocol?}

  Throughout a session, the \textsc{ftp} server must maintain a state
  about the user:
  \begin{itemize}

    \item the control connection must be associated with a specific
      user account;

    \item the server must keep track of a given user's current
      directory on its file system.

  \end{itemize}

  \item \emph{Are the objectives of flow control and congestion
    control the same?}

  No, their objectives are different. Flow control is about a host not
  overflowing the other host's receiving buffer, whilst congestion
  control is about not overflowing the router's queues.

  \item \emph{Give a very short description of how the
    connection-oriented service of the Internet provides reliable
    transport.}

  This service is TCP. It ensures reliable transport by means of
  acknowledgement and retransmission. If the sender does not receive
  the acknowledgement from the destination about a given packet, then
  this packet is transmitted again.

  \item \emph{Describe briefly Web caching at proxies and
    browsers. Why are they useful?}

  Proxies are special nodes in the internet which, among other
  features, allows several users (hosts), usually on the same Local
  Area Network (LAN), to share a same cache. This way, for instance,
  if several users retrieve the same Web objects, only the first one
  will be actually downloaded from the \textsc{http} server, the
  following being retrieved from the proxy cache (assuming they are
  up-to-date or no \texttt{GET} request is issued). 

  Browsers also have caches, implemented as files on the local file
  system. If the user requests the same object twice of more, the
  object is taken from the cache, not from the network.

  Web caching is useful because it reduces the traffic on the internet.

  \item \emph{Compare the common and different features of
    \textsc{smtp} and \textsc{http}.}

  The common feature is that both protocols can use persistent
  \textsc{tcp} connection (actually, this is mandatory in case of
  \textsc{smtp}).

  The differences are as follows:
  \begin{enumerate}

    \item An \textsc{http} client pulls files (objects) from a server,
    whereas an \textsc{smtp} pushes them (mails) to a server: we say
    that \textsc{http} is a \emph{pull protocol} whereas
    \textsc{smtp} is a \emph{push protocol}.

    \item \textsc{smtp} requires the body of each message to be encoded
    in \textsc{ascii}, contrary to \textsc{http}, thus accentuated
    characters or binary data must be encoded and decoded.

    \item When a file consists of several parts (like text and images),
    \textsc{smtp} places all of them in the same message, contrary to
    \textsc{http} which puts them in separate messages.

  \end{enumerate}

  \item \emph{What are the components needed to make (use of) cookies?}

  Cookies contain the following components:
  \begin{itemize}

    \item a cookie header line in the \textsc{http} response;

    \item a cookie header line in the \textsc{http} request;

    \item a cookie file kept on the user's system and managed by the
    browser;

    \item a database at the server site.

  \end{itemize}

  \item \emph{What is the conditional \texttt{GET} \textsc{http}
    request useful for?}

  Web caching can create a new problem: the cached objects can be
  \emph{stale}, i.e. the original object (at the server side) may have
  changed since the last request.

  The mechanism in \textsc{http} which ensures that all retrieved
  objects are up-to-date despite caching is called conditional
  \texttt{GET}.

  \item \emph{List five tasks that a protocol layer can perform. Is it
    possible that one (or more) of these tasks could be performed by
    two (or more) layers?}

  Five generic tasks are error control, flow control, segmentation and
  reassembly, multiplexing and connections set-up. These tasks can be
  duplicated at different levels. For example, error control is often
  provided at more than one layer.

  \item \emph{What information is used by a process running on one
    host to identify a process running on another host?}

  The IP address of the destination host and the port number of the
  destination socket.

\end{enumerate}

\section{Problems}

\begin{enumerate}

  \item \emph{Consider sending a file of \(M \times L\) bits over a
    path of \(Q\) links. Each link transmits at \(R\) bits per
    second. The network is lightly loaded so that there are no queuing
    delays. When a form of packet switching is used, the \(M \times
    L\) bits are broken up into \(M\) packets, each packet with \(L\)
    bits. Propagation delay is negligible.}
    \begin{enumerate}

      \item \emph{Suppose the network is a packet-switched virtual
        circuit network. Denote the VC set-up time by \(t_s\)
        seconds. Suppose the sending layers add a total of \(h\) bits
        of header to each packet. How long does it take to send the
        file from source to destination?}

        The time to transmit one packet onto a link is \(\frac{L +
          h}{R}\). The time to deliver the first of the \(M\) packets
        to the destination is \(Q{\frac{L+h}{R}}\). Every
        \(\frac{L+h}{R}\) seconds a new packet from the \(M-1\)
        remaining packets arrives at destination (this is due to
        pipelining). We must also add the initial setup time, because
        we use a VC network. Thus the total delay is
        \[
          t_{s} + (Q + M - 1){\frac{L+h}{R}}
        \]

      \item \label{datagram} \emph{Suppose the network is a
        packet-switched datagram network and a connectionless service
        is used. Now suppose each packet has \(2h\) bits of
        header. How long does it take to send the file?}

        Using a datagram network with a connectionless service implies
        there is no setup needed. Another difference with the previous
        situation is the header size. Therefore the total delay is
        simply
        \[
          (Q + M - 1){\frac{L + 2h}{R}}
        \]

      \item \emph{Repeat case~\ref{datagram} but assume message
        switching is used (that is, \(2h\) bits are added to the
        message, and the message is not segmented).}

        The time to transmit the entire message over one link is
        \(\frac{L M + 2h}{R}\). The time to transmit the entire
        message over \(Q\) links (i.e. to destination) is thus
        \[
          Q{\frac{L M + 2h}{R}}
        \]
 
      \item \emph{Finally, suppose that the network is a
        circuit-switched network. Further suppose that the
        transmission rate of the circuit between source and
        destination is \(R\) bit/s. Assuming \(t_s\) seconds of set-up
        and \(h\) bits of header appended to the entire file, how long
        does it take to send the file?}

        Because there is no store-and-forward delay at the switches,
        the total delay does not depend on the number of links. Also
        we must add the setup time. Therefore the end-to-end delay is
        \[
          t_{s} + \frac{M L + h}{R}
        \]

    \end{enumerate}

\end{enumerate}

\end{document}
