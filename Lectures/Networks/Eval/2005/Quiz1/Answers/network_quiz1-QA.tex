%%-*-latex-*-

\documentclass[11pt,a4paper]{article}

\input{trace}

\usepackage{graphicx}

\title{Answers to quiz \#1 of introduction to networking}
\author{Christian Rinderknecht}
\date{\today}

\begin{document}

\maketitle

\section{Review questions}

\begin{enumerate}

  \item \emph{What is the difference between a host and an end system?
  List different types of end systems. Is a mail server an end
  system?}

  There is no difference. Some hosts are workstations, PCs, mail
  servers, Web servers, etc.

  \item \emph{What is a client program? What is a server program?}

  A networking application is usually made of two parts of software,
  each running on a different host. The one which requests some
  services to the other is called the client. The other is the
  server.

  \item \emph{What are the two kinds of services that the Internet
  provides to the applications? Describe some typical features of
  each.}

  The Internet provides a connection-oriented service, called TCP, and
  a connectionless service, called UDP, to the applications. The
  principles of TCP are
  \begin{itemize}
    \item The two end systems perform a handshaking before starting a
    session;
    
    \item It provides reliable end-to-end transport, i.e. the data is
    guaranteed to be delivered completely and in order.

    \item It provides flow control, i.e. ensures that no end system
    will overwhelm the other by sending to fast too many data,

    \item It provides congestion control, i.e. makes sure that the end
    systems will not overflow the buffers of the routers in the
    network.
  \end{itemize}

   Connectionless services like UDP do not guarantee anything above.

  \item \emph{Are the objectives of flow control and congestion
  control the same?}

  No, their objectives are different. Flow control is about a host not
  overflowing the other host's receiving buffer, whilst congestion
  control is about not overflowing the router's queues.

  \item \emph{Give a very short description of how the
  connection-oriented service of the Internet provides reliable
  transport. What is the name of this service, by the way?}

  This service is TCP. It ensures reliable transport by means of
  acknowledgement and retransmission. If the sender does not receive
  the acknowledgement from the destination about a given packet, then
  this packet is transmitted again.

  \item \emph{Suppose there is exactly one packet switch between a
  sending host and a receiving host. The transmission rates between
  the sending host and the switch and between the switch and the
  receiving host are \(R_1\) and \(R_2\), respectively. Assuming that
  the switch uses store-and-forward packet switching, what is the
  total end-to-end delay to send a packet of length \(L\)? (Ignore
  queuing, propagation delay and processing delay.)}

  At time \(t_0\) the sending hosts starts to transmit the packet. At
  time \(t_1 = L/R_1\), the entire packet is transmitted and received
  by the router (no propagation delay). At time \(t_2 = t_1 + L/R_2\)
  the packet is transmitted by the router and received by the
  destination (again, no propagation delay). Thus the end-to-end delay
  is \(L(1/R_1 + 1/R_2)\).

\end{enumerate}

\section{Problems}

\begin{enumerate}

  \item \emph{Design and describe an application-level protocol to be
  used between an automatic teller machine (ATM) and a bank's
  centralised computer. Your protocol should allow a user's card and
  password to be verified, the account balance (which is maintained at
  the bank's centralised computer) to be queried, and an account
  withdrawal to be made (money is given to the customer).}

  \emph{Specify your protocol by listing the messages exchanged and
  the action taken by the ATM or the bank's centralised computer on
  transmission or receipt of each message.}

  \emph{Sketch the operation of your protocol for the case of a simple
  withdrawal with no errors, using a diagram similar to
  Figure~\ref{protocols}. Explicitly state the assumptions made by
  your protocol about the underlying end-to-end transport service.}

  \begin{figure}
  \begin{center}
  \includegraphics*[scale=0.35]{01-02.eps}
  \caption{Sketch of protocols}\label{protocols}
  \end{center}
  \end{figure}
 
  One example is as follows.

  Messages from ATM to server:

  {\small
  \begin{tabular}{l|l}
  Message name & Purpose\\
  \hline \hline
  HELO \(<\)userid\(>\)   & Let server know that there is a card\\
                          & in the ATM, then transmits user ID\\
                          & to server\\
  PASSWD \(<\)passwd\(>\) & User enters PIN, which is sent\\
                          & to server\\
  BALANCE                 & User requests balance\\
  WITHDRAWAL \(<\)amount\(>\) 
                          & User asks to withdraw money\\
  BYE                     & User all done
  \end{tabular}
  }

  \medskip

  Messages from server to ATM (display):

  \begin{tabular}{l|l}
  Message name & Purpose\\
  \hline \hline
  PASSWD  & Ask user for PIN (password)\\
  OK      & Last requested operation\\
          & (PASSWD, WITHDRAWAL) OK\\
  ERR     & Last requested operation\\
          & (PASSWD, WITHDRAWAL)\\
          & resulted in error\\
  AMOUNT \(<\)amount\(>\) & Sent in response to
                            BALANCE request\\
  BYE     & User done, display welcome\\
          & screen at ATM
  \end{tabular}   

  A correct withdrawal looks like this:
{\small
  \begin{verbatim}
HELO <userid>        -------------> (Check if valid userid)
                     <------------- PASSWD
PASSWD <passwd>      -------------> (Check password)
                     <------------- OK (password is OK)
BALANCE              ------------->
                     <------------- AMOUNT <amount>
WITHDRAWAL <amount>  -------------> (Check if enough money)
                     <------------- OK
ATM dispenses money
BYE                  ------------->
                     <------------- BYE
  \end{verbatim}
}

  \item \emph{Consider an application that transmits data at a steady
  rate: it generates an \(N\)-bit unit of data every \(k\) time units,
  where \(k\) is small and fixed. Also, when such an application
  starts, it will continue running for a long period of time. Answer
  the following questions, briefly justifying your answer.}

  \begin{enumerate}
 
    \item \emph{Would a packet-switched network or a circuit-switched
    network be more appropriate for this application? Why?}

    A circuit-switched network would be well suited to this
    application, because it involves long sessions with predictable
    bandwidth requirements. So bandwidth can be reserved for each
    session. The delay of setting up a circuit is low compared to the
    time the application is running.

    \item \emph{Suppose that a packet-switched network is used and the
    only traffic in this network comes from such applications as
    described above. Furthermore, assume that the sum of the
    application data rates is less than the capacities of each and
    every link. Is some form of congestion control needed? Why?}

    Given such generous link capacities, the network need no congestion
    control. In the worst case, all the applications are emitting on
    the same link, but the link offers enough bandwidth.

  \end{enumerate}

\end{enumerate}

\end{document}
