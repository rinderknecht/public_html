%%-*-latex-*-

\documentclass[12pt,a4paper]{article}

\input{trace}

\title{Answers to the final examination on Introduction to the Internet}
\author{Christian Rinderknecht}
\date{\today}

\usepackage{nath}

\begin{document}

\maketitle

\section{Problems}

\begin{enumerate}

  \item \emph{Consider the queuing delay in a router buffer. Suppose
    all the packets are \(L\) bits long, the transmission rate is
    \(R\) bit/sec and that \(N\) packets arrive simultaneously at the
    buffer every \(\frac{LN}{R}\) seconds.}

    \emph{Find the \textbf{average} queuing delay of a packet.}

    \emph{(\emph{Hint}: The queuing delay for the first packet is 0;
      for the second packet it is \(\frac{L}{R}\); for the third
      packet it is \(2{\frac{L}{R}}\) etc. The last packet (number
      \(N\)) has already been transmitted when the second batch
      (i.e. group) of packets arrives.)}

    It takes \(\frac{NL}{R}\) seconds to transmit the \(N\)
    packets. Thus, the buffer is empty when a batch of \(N\) packets
    arrive.

    The first of the \(N\) packets has no queuing delay. The second
    packet has a queuing delay of \(\frac{L}{R}\) seconds. The
    \(n\)-nth packet has a delay of \((n-1){\frac{L}{R}}\).

   Therefore the average delay is
   \setlength\mathindent{-1pc}
   \[
     {\frac{1}{N}}\sum_{n=1}^{N}(n-1){\frac{L}{R}} = \frac{L}{R}
     \frac{1}{N} \sum_{n=0}^{N-1}n = \frac{L}{R} \frac{1}{N}
     \frac{(N-1)N}{2} = \frac{1}{2} (N-1) \frac{L}{R} 
   \]

  \item \emph{Consider the queuing delay in a router buffer. Let \(I\)
    denote the traffic intensity, that is: \(I =
    \frac{La}{R}\). Suppose that the queuing delay takes the form
    \(\frac{IL}{R(1-I)}\) for \(I < 1\).}
    \begin{enumerate}

      \item \emph{Provide a formula for the total delay, that is, the
        queuing delay plus the transmission delay.}

      The total delay is
      \[
        \frac{IL}{R(1-I)} + \frac{L}{R} = \frac{L}{R} \frac{1}{1-I}
      \]

      \item \emph{Express the total delay as a function of
        \(\frac{L}{R}\).}

      Let us call \(d (\frac{L}{R})\) the total delay in function of
      \(\frac{L}{R}\). We have
      \[
       \frac{L}{R} \frac{1}{1-I} = \frac{L}{R} \frac{1}{1 -
         \frac{La}{R}} = \frac{x}{1 - ax} = d (x)
      \]
      where \(x = \frac{L}{R}\).

    \end{enumerate}

  \item \emph{We consider sending voice from host A to host B over a
    packet-switched network (for example, Internet phone). Host A
    converts analog voice to a digital \textbf{64~Kbps} bit stream on
    the fly. Host A then groups the bits into a \textbf{48-byte}
    packets. There is one link between host A and B; its transmission
    rate is \textbf{1~Mbps} and its propagation delay is
    \textbf{2~msec}.}

    \emph{As soon as host A gathers a packet, it sends it to host
      B. As soon as host B receives an \emph{entire} packet, it
      converts the packet's bits into an analog signal.}

    \emph{How much time elapses from the time a bit is created (from
      the original analog signal at host A) until the bit is decoded
      (as part of the analog signal at host B)?}

    Before any bit can be transmitted, all the bits in the same packet
    must be gathered first. This requires
    \[
      \frac{48 \times 8}{64 \times 10^3} \; \text{sec} = 6 \;
      \text{msec}
    \]
    The time required to transmit the packet is
    \[
      \frac{48 \times 8}{1 \times 10^{6}} \; \text{sec} = 0.384 \;
      \text{msec}
    \]
    The propagation delay is 2~msec.

    Therefore the delay between coding and decoding is
    \begin{center}
      6~msec + 0.384~msec + 2~msec = 8.384~msec
    \end{center}

\end{enumerate}


\section{Review questions}

\begin{enumerate}

  \item \emph{For a communication session between two hosts, which
    host is the client and which is the server?}

  The host that starts the session is, by definition, the client.

  \item \emph{List at least two user agents you personally use.}

  At least a web browser, like Firefox, and a mail agent, like
  Evolution.

  \item \emph{Why do \textsc{http}, \textsc{ftp}, \textsc{smtp},
    \textsc{pop3} and \textsc{imap} run on top of \textsc{tcp} rather
    than \textsc{udp}?}

  The applications which use these protocols require that the data they
  send is received in order and in its entirety. \textsc{tcp} provides
  this service whereas \textsc{udp} does not.

  \item \emph{Consider an e-commerce site that wants to keep a
    purchase record for each of its customers.}
  \begin{enumerate}

    \item \emph{Describe (briefly) how this can be done with
      \textsc{http} authentication.}

    \item \emph{Describe (briefly) how this can be done with cookies.}

  \end{enumerate}

  In both cases the site needs a database to record the customer
  information and the user first registers with a name and a password.

  With \textsc{http} authentication, during the next visits, the user
  provides again its name and password, allowing the site to
  authenticate him and update his record in the database.

  With cookies, the user does not provide a name and password during
  the next visits. Instead, the user is identified by his browser
  sending to the site server the cookie that it was given the first
  time.

  \item \emph{Is it possible that an organisation's web server and
    mail server have exactly the same alias for a hostname? What would
    be the type for the RR that contains the hostname of the mail
    server?}

  Yes, an organisation's mail server and web server can have the same
  alias as hostname. The MX record is used to map the mail server name
  to its IP address.

\end{enumerate}


\end{document}
