%%-*-latex-*-

\documentclass[12pt,a4paper]{article}

\input{trace}

\title{Final examination on Introduction to the Internet}
\author{Christian Rinderknecht}
\date{\today}

\usepackage{nath}

\begin{document}

\maketitle

\section{Problems}

\begin{enumerate}

  \item Consider the queuing delay in a router buffer. Suppose all the
    packets are \(L\) bits long, the transmission rate is \(R\)
    bit/sec and that \(N\) packets arrive simultaneously at the buffer
    every \(\frac{LN}{R}\) seconds.

    Find the \textbf{average} queuing delay of a packet.

    (\emph{Hint}: The queuing delay for the first packet is 0; for the
    second packet it is \(\frac{L}{R}\); for the third packet it is
    \(2{\frac{L}{R}}\) etc. The last packet (number \(N\)) has already
    been transmitted when the second batch (i.e. group) of packets
    arrives.)

  \item Consider the queuing delay in a router buffer. Let \(I\)
    denote the traffic intensity, that is: \(I =
    \frac{La}{R}\). Suppose that the queuing delay takes the form
    \(\frac{IL}{R(1-I)}\) for \(I < 1\).
    \begin{enumerate}

      \item Provide a formula for the total delay, that is, the
        queuing delay plus the transmission delay.

      \item Express the total delay as a function of \(\frac{L}{R}\).    

    \end{enumerate}

  \item We consider sending voice from host A to host B over a
    packet-switched network (for example, Internet phone). Host A
    converts analog voice to a digital \textbf{64~Kbps} bit stream on
    the fly. Host A then groups the bits into a \textbf{48-byte}
    packets. There is one link between host A and B; its transmission
    rate is \textbf{1~Mbps} and its propagation delay is
    \textbf{2~msec}. 

    As soon as host A gathers a packet, it sends it to host B. As soon
    as host B receives an \emph{entire} packet, it converts the
    packet's bits into an analog signal.

    How much time elapses from the time a bit is created (from the
    original analog signal at host A) until the bit is decoded (as
    part of the analog signal at host B)?

\end{enumerate}


\section{Review questions}

\begin{enumerate}

  \item For a communication session between two hosts, which host is
    the client and which is the server?

  \item List at least two user agents you personally use.

  \item Why do \textsc{http}, \textsc{ftp}, \textsc{smtp},
    \textsc{pop3} and \textsc{imap} run on top of \textsc{tcp} rather
    than \textsc{udp}?

  \item Consider an e-commerce site that wants to keep a purchase
    record for each of its customers. 
  \begin{enumerate}

    \item Describe (briefly) how this can be done with \textsc{http}
      authentication. 

    \item Describe (briefly) how this can be done with cookies.

  \end{enumerate}

  \item Is it possible that an organisation's web server and mail
    server have exactly the same alias for a hostname? What would be
    the type for the RR that contains the hostname of the mail server?

\end{enumerate}


\end{document}
