%%-*-latex-*-

\paragraph{Answer.}

We can generalise the answer to any header size. So let \(h =
40\). Then the size of each packet is \(L = h + S\). The number of
packets is \(\frac{F}{S}\), which is an integer by assumption. One
packet takes \(\frac{L}{R}\) seconds to cross over one link. So the
first packet takes \(2{\frac{L}{R}}\) seconds from source to
destination. Then, every \(\frac{L}{R}\) seconds another packet, among
the \(F/S-1\) remaining, arrives to destination, because of
pipelining. So the total time to receive all the packets is
\[
    2{\frac{L}{R}} + \biggl(\frac{F}{S}-1\biggr)\frac{L}{R} 
  = \biggl(1 + \frac{F}{S}\biggr)\frac{L}{R} 
  = \biggl(1 + \frac{F}{S}\biggr)\frac{h + S}{R}.
\]
In other words, the delay \({\cal D}\) in function of \(S\) is
expressed as
\[
  {\cal D}(S) = \frac{1}{R}\frac{(S + F)(h + S)}{S} =
  \frac{1}{R}\biggl(h + F + S + \frac{hF}{S}\biggr).
\]
The minumum of this function is reached at \(S_{m}\) such as
\[
  \frac{d}{dS} {\cal D} (S_m) = 0.
\]
Since
\[
  \frac{d}{dS} {\cal D} (S) = \frac{1}{R}\biggl(1 -
  \frac{hF}{S^2}\biggr).
\]
We deduce finally \(S_m = \lfloor\sqrt{hF}\rfloor\), where
\(\lfloor{x}\rfloor\) is the biggest integer which is smaller than
\(x\) (in this example, we could choose also the smallest integer
which is bigger than \(x\)).
