\paragraph{Answers.}

\begin{enumerate}

  \item The time to transmit one packet onto a link is \(\frac{L +
    h}{R}\). The time to deliver the first of the \(M\) packets to the
    destination is \(Q{\frac{L+h}{R}}\). Every \(\frac{L+h}{R}\)
    seconds a new packet from the \(M-1\) remaining packets arrives at
    destination (this is due to pipelining). We must also add the
    initial setup time, because we use a VC network. Thus the total
    delay is
    \[
      t_{s} + (Q + M - 1){\frac{L+h}{R}}
    \]

  \item Using a datagram network with a connectionless service implies
    there is no setup needed. Another difference with the previous
    situation is the header size. Therefore the total delay is simply
    \[
     (Q + M - 1){\frac{L + 2h}{R}}
    \]

  \item The time to transmit the entire message over one link is
    \(\frac{ML + 2h}{R}\). The time to transmit the entire message over
    \(Q\) links (i.e. to destination) is thus
    \[
      Q{\frac{ML + 2h}{R}}
    \]
 
  \item Because there is no store-and-forward delay at the switches,
    the total delay does not depend on the number of links. Also we
    must add the setup time. Therefore the end-to-end delay is
    \[
      t_{s} + \frac{ML + h}{R}
    \]

\end{enumerate}
