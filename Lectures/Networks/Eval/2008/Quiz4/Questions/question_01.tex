%%-*-latex-*-

\paragraph{Questions.}

\begin{enumerate}

  \item What advantage does a circuit-switched network have over a
    packet-switched network? What advantage does TDM have over FDM in
    a circuit-switched network?

  \item Why is it that packet switching is said to employ statistical
    multiplexing? Contrast statistical multiplexing with the
    multiplexing that takes place in TDM.

  \item What is meant by connection state information in a virtual
    circuit network?

  \item Suppose you are developing a standard for a new type of
    network. You need to decide whether your network will use VCs or
    datagram routing. What are the pros and cons for using VCs?

  \item What are the advantages of message segmentation in
    packet-switched networks? What are the disadvantages?

  \item Is HFC bandwidth dedicated or shared among users? Are
    collisions possible in a downstream HFC channel?

  \item Consider sending a series of packets from a sending host
    to a receiving host over a fixed route. List the delay components
    in the end-to-end delay for a single packet. Which of these delays
    are constant and which are variable?

  \item Consider an application that transmits data at a steady rate:
    it generates an \(N\)-bit unit of data every \(k\) time units,
    where \(k\) is small and fixed. Also, when such an application
    starts, it will continue running for a long period of time. Answer
    the following questions, briefly justifying your answer.

    \begin{enumerate}
 
      \item Would a packet-switched network or a circuit-switched
        network be more appropriate for this application? Why?

      \item Suppose that a packet-switched network is used and the
        only traffic in this network comes from such applications as
        described above. Furthermore, assume that the sum of the
        application data rates is less than the capacities of each and
        every link. Is some form of congestion control needed? Why?

    \end{enumerate}

\end{enumerate}

