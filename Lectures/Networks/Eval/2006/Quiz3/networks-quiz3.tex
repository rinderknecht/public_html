%%-*-latex-*-

\documentclass[11pt,a4paper]{article}

\input{trace}

\title{Exercise of introduction to networking} 
\author{Christian Rinderknecht} 
\date{Spring 2006}

\begin{document}

\maketitle


\begin{enumerate}

  \item Consider sending a file of \(M \times L\) bits over a path of
    \(Q\) links. Each link transmits at \(R\) bits per second. The
    network is lightly loaded so that there are no queuing
    delays. When a form of packet switching is used, the \(M \times
    L\) bits are broken up into \(M\) packets, each packet with \(L\)
    bits. Propagation delay is negligible.
    \begin{enumerate}

      \item Suppose the network is a packet-switched virtual circuit
        network. Denote the VC set-up time by \(t_s\) seconds. Suppose
        the sending layers add a total of \(h\) bits of header to each
        packet. How long does it take to send the file from source to
        destination?

      \item \label{datagram} Suppose the network is a packet-switched
        datagram network and a connectionless service is used. Now
        suppose each packet has \(2h\) bits of header. How long does
        it take to send the file?

      \item Repeat case~\ref{datagram} but assume message switching is
        used (that is, \(2h\) bits are added to the message, and the
        message is not segmented).

      \item Finally, suppose that the network is a circuit-switched
        network. Further suppose that the transmission rate of the
        circuit between source and destination is \(R\)
        bit/s. Assuming \(t_s\) seconds of set-up and \(h\) bits of
        header appended to the entire file, how long does it take to
        send the file?

    \end{enumerate}


\end{enumerate}

\end{document}
