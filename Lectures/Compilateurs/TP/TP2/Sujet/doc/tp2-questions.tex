%%-*-latex-*-

\begin{enumerate}

  \item \label{q1} Reconsid�rez la s�mantique op�rationnelle
  de la division (sans sp�cification des erreurs). L'ordre
  d'�valuation des op�randes n'y est pas sp�cifi�. R��crivez la 
  r�gle pour qu'elle force l'�valuation de son second op�rande
  avant le premier. Pour cela, inspirez-vous de la r�gle
  de la liaison locale.

  L'implantation fid�le de cette nouvelle s�mantique est
  maladroite. Proposez une variation simple du code pour la division
  qui r�pond au probl�me.

  \item \label{q2} Ajoutez les fonctions et les
  applications. Notez bien que le type des valeurs doit maintenant
  changer. Introduisez les modifications n�cessaires. N'oubliez pas
  d'enrichir le traitement d'erreurs dans les cas o� on attend un
  entier et on obtient en fait une fonction, ainsi que le cas
  contraire. Testez. (\emph{Gardez une trace de vos cas de test.}) Par
  exemple:
  {\small
  \begin{center}
  \texttt{1+2*(if true then let f = fun y -> y+1 in f(3+4) else 5)}
  \end{center}
  }

  \item \label{q3} Implantez les fonctions r�cursives
  natives. Testez par exemple
  {\small
   \begin{verbatim}
let rec f = fun n -> ifz n then 1 else n*f(n-1) in f 7
let rec x = x in x
   \end{verbatim}
  }
 
  \vspace*{-10pt}

  \item \label{q4} Soit $e$ l'arbre de syntaxe abstraite
  correspondant au programme \num{1}~\num{2}. Existe-t-il un
  environnement $\rho$ et une valeur $v$ tels que $\eval{\rho}{e}{v}$?
  Quelle diff�rence voyez-vous entre cet exemple et \lpar\kwd{fun}
  \ident{f} $\rightarrow$ \ident{f} \ident{f}\rpar{} \lpar\kwd{fun} \ident{f}
  $\rightarrow$ \ident{f} \ident{f}\rpar, donn� dans le cour?  V�rifiez
  votre raisonnement en soumettant ces expressions � votre calculette.

  \item \label{q5} V�rifiez que les constructions \kwd{let}
  \ident{x} \equal{} \ident{e}$_1$ \kwd{in} \ident{e}$_2$ et \lpar\kwd{fun}
  \ident{x} $\rightarrow$ \ident{e}$_2$\rpar{} \ident{e}$_1$ sont �quivalentes
  du point de vue de l'�valuation --- c'est-�-dire que l'une produit
  une valeur $v$ si et seulement si l'autre produit �galement $v$.

  \item \label{q6} Ajoutez au langage des liaisons globales de
  la forme $\kwd{let} \, x = e$.

  \item \label{q7} Implantez la s�mantique avec r�f�rences.

  \item \label{q8} V�rifiez que les constructions \kwd{let}
  \ident{x} \equal{} \ident{e}$_1$ \kwd{in} \ident{e}$_2$ et \lpar\kwd{fun}
  \ident{x} $\rightarrow$ \ident{e}$_2$\rpar{} \ident{e}$_1$ sont �quivalentes
  du point de vue du typage monomorphe, c'est-�-dire que l'une admet
  le type $\tau$ sous l'environnement $\Gamma$ si et seulement si
  l'autre admet aussi $\tau$ sous $\Gamma$.

  \item \label{q9} Peut-on typer les expressions ci-dessous de
  fa�on monomorphe et, si oui, avec quels types?

  \begin{center}
  \begin{tabular}{l}
    \Xlet \, \ident{f} \equal \, \Xfun \, \ident{x} $\rightarrow$ \ident{x}
    \, \Xin \, \ident{f} \ident{f}\\
    \Xfun \, \ident{f} $\rightarrow$ \ident{f} \ident{f}\\
    \num{1} \, \num{2}\\
    \Xlet \, \ident{f} \equal \, \Xfun \, \ident{x} $\rightarrow$ \ident{x}
    \Xin \, \Xlet \, \ident{x} \equal \, \ident{f} \num{1} \Xin \,
    \ident{f} \lpar\Xfun \, \ident{x} $\rightarrow$ \ident{x} \texttt{+}
    \num{1}\rpar 
  \end{tabular}
  \end{center}

  \item \label{q10} Implantez l'axiome \textsc{fun-opt} qui
  restreint l'environnement dans les fermetures aux variables libres
  de la fonction:
  \begin{mathpar}
  \inferrule*[right=\quad \textsc{fun-opt}]
    {\eval{\rho}
          {\cst{Fun} \, (x,e) \, \kwd{as} \, f}
          {\clos{x}{e}{\rho\arrowvert{\cal L} (f)}}}
    {}
   \end{mathpar}

 
\end{enumerate}
