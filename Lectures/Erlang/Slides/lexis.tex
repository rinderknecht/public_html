%% -*- latex -*-

% ------------------------------------------------------------------------
%
\begin{frame}[containsverbatim]
\frametitle{Terms/Atoms}

Let us define more precisely what kind of data \Erlang operates
on. These data are called \textbf{terms}.

\bigskip

Some terms are only identified by their name, called an
\textbf{atom}. An atom starts with a lower-case letter which can be
followed by a series of characters out of lower-case letters,
upper-case letters, digits and the underscore character (`\_').

\bigskip

Some atoms can be \textbf{quoted} by an opening single-quote and a
closing one.  In this case, it may contain blank characters. For
example
\begin{verbatim}
anna    apha_beta_proc  x_25    'This is a quoted atom'
x25     call_Java       x_25AB   x_  x____y
\end{verbatim}
\emph{Function names are atoms.}

\end{frame}

% ------------------------------------------------------------------------
%
\begin{frame}
\frametitle{Terms/Quoted atoms}

\textbf{Quoted atoms} look like some strings in some languages, like
\Bash, but they are very different they cannot be modified. Therefore,
they are like constant character strings.

\bigskip

In some programming languages, a character string can be modified in
place, but quoted atoms do not allow this.

\end{frame}

% ------------------------------------------------------------------------
%
\begin{frame}[containsverbatim]
\frametitle{Terms/Numbers}

\textbf{Numbers.} in \Erlang include integer numbers and floating-point
numbers. The syntax of integer is as expected, for example
\begin{verbatim}
1    1234    0    -97
\end{verbatim}
The lower and larger integers are limited by the actual \Prolog system
in use.

\bigskip

Floating-point numbers follow the usual syntax too, like
\begin{verbatim}
100    -7    3.14    -0.06    100.5    1.5e-3
\end{verbatim}
Atoms and numbers define the group of \textbf{constants}.

\end{frame}

% ------------------------------------------------------------------------
%
\begin{frame}[containsverbatim]
\frametitle{Terms/Numbers (cont)}

In \Erlang, there are no characters. Instead, they can be handled
throughout their \ASCII code.

\bigskip

There is a special operator \texttt{\$} which returns the \ASCII code
for a given character. For example \verb|$A| evaluates to \texttt{65}.

\bigskip

It is possible to input integers in a base which is not 10 by using a
special notation \texttt{\#}. For example \verb|16#ffff| represents
\texttt{65535} (in base 10). 

\bigskip

This operator works only if the base ranges from 2 to 16.

\end{frame}

% ------------------------------------------------------------------------
%
\begin{frame}[containsverbatim]
\frametitle{Terms/Variables}

A \textbf{variable} is a name for a term, but, contrary to atoms,
variables do not define any object. For example
\begin{verbatim}
X = 25
\end{verbatim}
does not define the constant term \texttt{25} but do give it the name
\texttt{X}. The term \texttt{25} is defined just by being written,
just like atoms.

\bigskip

One variable denotes a term in a set. For example, in
\begin{verbatim}
fact(0) -> 1;
fact(N) -> N * fact(N-1).
\end{verbatim}
the variable \texttt{N} denotes any number which is not the zero
integer.

\end{frame}

% ------------------------------------------------------------------------
%
\begin{frame}[containsverbatim]
\frametitle{Terms/Variables (cont)}

They must start with an upper-case letter and may be followed
by any number of letters, digit and underscores, in any order. For
example, the following are valid variables: 
\begin{verbatim}
X  Obj_List  Object2  Result  ObjList  X_  Obj__
\end{verbatim}
If a variable appears only once in the head of a clause and not in the
body, it is an \textbf{unknown variable} and can be replaced by an
underscore. First, consider
\ErlangIn{bool}

\end{frame}

% ------------------------------------------------------------------------
%
\begin{frame}
\frametitle{Terms/Variables (cont)}

\begin{columns}

  \column{0.5\textwidth} It is equivalent to
  \ErlangIn{bool2}

  \column{0.5\textwidth} and
  \ErlangIn{bool1}

\end{columns}

\bigskip

\textbf{Important:} When several underscores occur in the same head,
each one denotes a different term, in general.

\end{frame}

% ------------------------------------------------------------------------
%
\begin{frame}
\frametitle{Terms/ Variables/Lexical scoping}

Given an occurrence of a variable, the part of the program where
this variable is usable, or bound, is called the \textbf{scope}.

\bigskip

\Erlang uses \textbf{lexical scoping}, which means
\begin{itemize}
  
  \item the same variable always represents the same term inside a
    clause;

  \item the same variable in two different clauses represent different
    terms, in general.

\end{itemize}
More about the scope at page~\pageref{scope}.

\end{frame}

% ------------------------------------------------------------------------
%
\begin{frame}[containsverbatim]
\frametitle{Terms/Tuples}

Some terms can be linearly compounded into one term, called a
\textbf{tuple}.

\bigskip

The number of components of a tuple is called arity.
The syntax of tuples is different from mathematics in that it requires
curly braces instead of parentheses: 
\begin{verbatim}
{a, 12, 'hello'}
{1, 2, {3, 4}, {a, {b, {c}}}}
{}
\end{verbatim}
Note that tuples can
\begin{itemize}

  \item contain terms of different kinds,

  \item be embedded,

  \item be empty.

\end{itemize}

\end{frame}

% ------------------------------------------------------------------------
%
\begin{frame}[containsverbatim]
\frametitle{Terms/Lists}

A \textbf{list} is a term made of a series of other terms. The terms
are separated by commas and enclosed between an opening square bracket
and a closing one. For example
\begin{verbatim}
[1, abc, [12], 'foo bar']
[]
[a, b, c]
\end{verbatim}
are lists. Note that terms of different kinds can be mixed. Lists of
integers which denote \ASCII characters can be represented with a
shorthand, e.g. \verb|"abc"| is equivalent to \verb|[$a,$b,$c]|, that
is: \verb|[97,98,99]|.

\bigskip

Note that in \cpp{} also, there is no string type.

\end{frame}

% ------------------------------------------------------------------------
%
\begin{frame}[containsverbatim]
\frametitle{Terms/Lists (cont)}
\label{head_tail}

It is often useful to refer to the first element of a list. That is
why \Erlang provides a special notation for lists that allows to
distinguish the first elements, called the \textbf{head}, and to
collapse the remaining elements into a list called the \textbf{tail},
using the notation \texttt{[} \emph{Head} \texttt{|} \emph{Tail}
  \texttt{]}.

\bigskip

For example
\begin{verbatim}
[1, abc, {4,5}, "hello"]
[1, abc, {4,5}, [$h,$e,$l,$l,$o]]
[1, abc, {4,5} | ["hello"]]
[1, abc | [{4,5}, "hello"]]
[1 | [abc, {4,5}, "hello"]]
\end{verbatim}
are different ways to denote the same list. 

\end{frame}

% ------------------------------------------------------------------------
%
\begin{frame}[containsverbatim]
\frametitle{Terms/Lists (cont)}

Note that
\begin{verbatim}
[1, abc, {4,5} | "hello"]
\end{verbatim}
is \textbf{not} equivalent to the previous lists. It is in fact
equivalent to
\begin{verbatim}
[1, abc, {4,5}, $h, $e, $l, $l, $o]
\end{verbatim}
Note also that
\begin{verbatim}
[1, abc | {4,5}, "hello world"]
\end{verbatim}
is \textbf{invalid} because the tail must be \emph{one} single term.

\end{frame}

% ------------------------------------------------------------------------
%
\begin{frame}
\frametitle{Terms/Lists (cont)}

\begin{center}
\includegraphics{lexicon_tree}
\end{center}

\end{frame}

% ------------------------------------------------------------------------
%
\begin{frame}
\frametitle{Ground terms, values and expressions}

It is sometimes useful to distinguish between two kinds of terms:
\textbf{ground terms} and \textbf{non-ground terms}.

\bigskip

Ground terms are terms that do not contain any variable. 

\bigskip

An \textbf{expression} is a syntactic construct involving function
calls and terms. For example \texttt{\small 5 * fact(5-1)} is an
expression but not a term, whereas \texttt{\small X} and
\texttt{\small 7} are both and expression and a term. The evaluation
of an expression does not terminate or lead to a \textbf{value} or an
error.

\bigskip

A value is thus a ground term that cannot be further computed. For
example, \texttt{25} is a value but \texttt{2 + 5} is not (it is an
expression).

\end{frame}
