%% -*- latex -*-

% ------------------------------------------------------------------------
%
\begin{frame}[containsverbatim]
\frametitle{Function calls}

The order in which arguments of function calls are evaluated is
undefined (as in \cpp{} but contrary to \Java). For instance, in
\begin{verbatim}
f(g(),h())
\end{verbatim}
the order of evaluation of \texttt{g()} and \texttt{h()} is undefined,
i.e.,, the programmer has no guarantee whether \texttt{g()} will be
evaluated before or after \texttt{h()}.

\bigskip

The order of evaluation of elements in a list is also undefined. For
instance
\begin{verbatim}
[g(), h()]
\end{verbatim}

\end{frame}

% ------------------------------------------------------------------------
%
\begin{frame}[containsverbatim]
\frametitle{Functions/Modules}

Modules in \Erlang are a database of functions.

\bigskip

The module name must be the basename of the file containing it, and
the extension of the file must be \texttt{erl}. So, file
\texttt{foo.erl} contains the \Erlang module \texttt{foo}, which is
specified by the first line
\begin{verbatim}
-module(foo).
\end{verbatim}
The purpose of \Erlang modules is to restrict the visibility of the
function definitions, since many functions have a merely technical,
internal, purpose and are not directly related to the higher-level of
module functionalities or services. Exported functions need to be
listed, together with their arity. For example
\begin{verbatim}
-module(foo).
-export([fact/1]).
\end{verbatim}

\end{frame}

% ------------------------------------------------------------------------
%
\begin{frame}[containsverbatim]
\frametitle{Functions/Qualified calls}

Functions can be called in two ways, either by qualifying their name
or not. Qualification means writing the name of the module the called
function belongs to. For instance 
\begin{verbatim}
math:fact(4)
\end{verbatim}
is a qualified call, whereas
\begin{verbatim}
fact(N-1)
\end{verbatim}
is a non-qualified call.

\bigskip

When the caller and the called function are in different modules, the
qualified form is mandatory. When calling a function which is in the
same module as the caller, the two forms are allowed. There is
actually a difference we shall discuss later, but which is, at this
point, irrelevant.

\end{frame}

% ------------------------------------------------------------------------
%
\begin{frame}
\frametitle{Functions/Recursivity}

\textbf{Recursivity} consists in a function calling itself, just like
the factorial function did at page~\pageref{fact}. The function call
in the body of the second case is a recursive call.

\bigskip

This technique is the only way in \Erlang to implement the equivalent
of loops in other imperative programming languages, i.e., in
non-functional programming languages. 

\bigskip

Indeed, since there are no mutable variables in \Erlang,
i.e., variables whose value can be changed, a loop would be pointless.

\bigskip

The combination of matching and recursivity is the heart of \Erlang
programming.

\end{frame}
