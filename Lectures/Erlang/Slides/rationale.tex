%%-*-latex-*-

% ------------------------------------------------------------------------
%
\begin{frame}
\frametitle{Why \Erlang?}

This course should be devoted to the concepts that are widely
implemented in the programming languages available nowadays. So why
\Erlang only?

\bigskip

The reason is that you already must follow classes in \Java and \cpp,
so the \textbf{object-oriented programming} is something you are
learning anyway.

\bigskip

You also learn C, as a part of \cpp{} or by itself (for system
programming), so you are also familiar with \textbf{structured
  programming}.

\bigskip

You also have the possibility to choose my class of \emph{Artificial
  Intelligence}, which is mainly an introduction to the \Prolog
programming language, allowing you to gain some understanding of
\textbf{logic programming}.

\end{frame}

% ------------------------------------------------------------------------
%
\begin{frame}
\frametitle{Why \Erlang? (cont)}

What is missing is
\begin{itemize}

  \item concurrent programming,

  \item distributed programming,

  \item functional programming.

\end{itemize}
\textbf{Concurrent programming} consists in making a software run on
separate \emph{threads} or \emph{processes}, i.e. each part of the
software is logically running in parallel on the same run-time
environment (e.g.  a virtual machine or the operating system in case
of native-code compilation), therefore on the same operating system
(no networking involved).

\end{frame}

% ------------------------------------------------------------------------
%
\begin{frame}
\frametitle{Why \Erlang? (cont)}

Of course, if the hardware includes only one processor (CPU), the
concurrent execution will not be truly in parallel, but interleaved,
i.e. one thread is run for a while, then another is scheduled to use
the CPU etc. If several processors are present, true concurrency can
be achieved, depending on the compilers.

\bigskip

In the class of \emph{System Programming}, you learn how to program
concurrently in C, but the C language, despite it was designed for
this task, has no built-in features to handle the concurrency: it
relies instead on the operating system, which provides the support for
threading and processing.

\bigskip

Because of this dependence by design on low-level layers (i.e. the
operating system), which obscures the concepts at stakes, concurrency
in C is not an easy task.

\end{frame}

% ------------------------------------------------------------------------
%
\begin{frame}
\frametitle{Why \Erlang? (cont)}

\textbf{Distributed programming} is an extension of concurrent
programming. 

\bigskip

In the former case, the threads are running on different run-time
environments on different physical machines, which are interconnected
by means of a network (each machine may run several threads of the
whole application).

\bigskip

Again, using C for this task, for example, is possible but can be made
easier if the programming language includes features for distributing
the computations across a network.

\end{frame}

% ------------------------------------------------------------------------
%
\begin{frame}
\frametitle{Why \Erlang? (cont)}

\textbf{Functional programming} is a form of \emph{declarative
  programming} (which includes also logic programming), in which
disciple the programmer describes what he wants instead of how to make
it.

\bigskip

For example, the language \XSLT, which allows the specification of
transforms of \XML files, is declarative. The query language \SQL is
also declarative.

\bigskip

Declarative languages are recognisable by the purposeful lack of
variables whose values can be modified, in other words: the programmer
has no access to the memory whatsoever. In particular, there are no
looping constructs, no pointer arithmetics, no explicit dynamic memory
allocation or deallocation etc.

\bigskip

The declarative paradigm makes debugging of concurrent and distributed
application easier since there are no \textbf{side-effects}.

\end{frame}

% ------------------------------------------------------------------------
%
\begin{frame}
\frametitle{Why \Erlang? (cont)}

The sequential subset of \Erlang is functional because the lack of
side-effects eases the debugging.

\bigskip

Concurrency and distribution make the application much more difficult
to debug in presence of side-effects and shared memory.

\bigskip

Even in the case of sequential programming (i.e. one thread), if there
is no global state shared by several functions and if
every function call returns the same result with the same arguments
---~which is not the case always if side-effects are allowed~---, the
debugging is much easier.

\end{frame}
