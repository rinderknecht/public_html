%%-*-latex-*-

\documentclass[11pt,a4paper]{article}

\usepackage[british]{babel}
\usepackage[T1]{fontenc}
\usepackage[latin1]{inputenc}
\usepackage{amsmath,amssymb}
\usepackage{array}
\usepackage{multirow}
\usepackage{verbatim}

\input{trace}
\input{erlc}

\title{Answers to the mid-term exam on\\ Introduction to the Internet}

\author{Christian Rinderknecht}
\date{25 April 2008}

\begin{document}

\maketitle

\thispagestyle{empty}

\noindent A \emph{queue} is a like stack where items are pushed on one
end and popped on the other end. Adding an item in a queue is called
\emph{enqueuing}, whereas removing one is called
\emph{dequeuing}. Compare the following figures:
\[
\begin{array}{rr@{\;}cc|c|c|c|c|c|cc@{\;}l}
\cline{4-10}
\text{Queue}: & \textsc{Enqueue} & \rightarrow & & a & b & c & d & e &
& \rightarrow & \textsc{Dequeue}\\
\cline{4-10}\\
\cline{4-9}
\text{Stack}: & \textsc{Push}, \textsc{Pop} & \leftrightarrow & & a &
b & c & d & e &&&\\
\cline{4-9}
\end{array}
\]
\noindent Let \texttt{enqueue(E,Q)} be the queue where \texttt{E} is
the last item in and \texttt{Q} is the remaining queue. Let
\texttt{dequeue(Q)} be the pair \texttt{\{E,R\}} where \texttt{E} is
the first item out of queue \texttt{Q} (if there is none,
\texttt{dequeue(Q)} is undefined) and \texttt{R} is the remaining
queue.

\bigskip

\paragraph{Question 1.} Give the lower and upper bounds of a number
represented in balanced ternary with \(n\) trits.


\noindent\textbf{Question.} Define a relation
    \texttt{catenate(U,V,W)} to be true when list \texttt{W} is made
    of list \texttt{U} followed by list \texttt{V}. For example
{\small 
\begin{verbatim}
?- catenate(U,[3,4],[0,1,2,3,4]).
U = [0,1,2] ;
No
\end{verbatim}
}


\end{document}
