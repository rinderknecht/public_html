\paragraph{Question.} Define \texttt{enqueue/2} and
\texttt{dequeue/1} in the following two cases.
\begin{enumerate}

  \item \textbf{A one-stack implementation.} A simple idea to
    implement one queue is to use one stack. In this case,
    \textsc{Enqueue} is simply \textsc{Push} and \textsc{Dequeue}
    removes the item at the bottom of the stack.

  \item \textbf{A two-stack implementation.} We can implement one
    queue with two stacks instead of one: one for enqueuing, one for
    dequeuing.
    \[
    \begin{array}{r@{\;}cc|c|c|c|c|c|c|cc@{\;}l}
      \cline{3-6}\cline{8-10}
      \textsc{Enqueue} & \rightarrow & & a & b & c & & d & e & &
      \rightarrow & \textsc{Dequeue}\\
      \cline{3-6}\cline{8-10}
    \end{array}
    \]
    \noindent So \textsc{Enqueue} is \textsc{Push} on the first stack
    and \textsc{Dequeue} is \textsc{Pop} on the second. If the second
    stack is empty, we swap the stacks and reverse the (new) second:
    \[
    \begin{array}{lr@{\;}cc|c|c|c|c|cc@{\;}l}
      \cline{4-7}\cline{9-9}
      \text{If} & \textsc{Enqueue} & \rightarrow & & a & b & c & & &
      \rightarrow & \textsc{Dequeue} \, \textbf{???}\\
      \cline{4-7}\cline{9-9}\\
      \cline{4-4}\cline{6-9}
      \text{then} & \textsc{Enqueue} & \rightarrow & & & a & b & c & &
      \rightarrow & \textsc{Dequeue}\\
      \cline{4-4}\cline{6-9}
    \end{array}
    \]
    \noindent Let the pair \texttt{\{S,T\}} denote the queue where
    \texttt{S} is the stack for enqueuing and \texttt{T} the stack for
    dequeuing.

\end{enumerate}
