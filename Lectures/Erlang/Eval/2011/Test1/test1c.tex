%%-*-latex-*-

\documentclass[11pt,a4paper]{article}

\usepackage[british]{babel}
\usepackage[T1]{fontenc}
\usepackage[latin1]{inputenc}
\usepackage{amsmath,amssymb}

\title{Test 1c of \textsf{Erlang}}
\author{Christian Rinderknecht}
\date{7 October 2011}

\newcommand\fun[1]{\textsf{#1}}

\begin{document}

\maketitle

\thispagestyle{empty}

\begin{enumerate}

  \item Write a function \fun{inter/2} (\emph{set intersection}) such
    that, if \[\fun{inter}(P,Q) \xrightarrow{+} R,\] then \(R\)~is a
    list made of all the items in common between lists \(P\)
    and~\(Q\), without repetition. For example,
    \[\fun{inter}([\fun{a},\fun{b},\fun{a},\fun{c},\fun{e}],
                  [\fun{e},\fun{e},\fun{b},\fun{b},\fun{d}])
                  \xrightarrow{+} [\fun{b},\fun{e}].\] The order of
                  the items in \(R\)~is left unspecified.

  \item Write a function \fun{dist/2} (\emph{distribute}) such that,
    if \[\fun{dist}(P,Q) \xrightarrow{+} R,\] then \(R\)~is the list
    made of all the possible pairs between the items of \(P\)
    and~\(Q\). For example, we can compute a deck of card by
    distributing the colours over the figures as follows:
    \begin{align*}
      \fun{dist}(&[\fun{heart}, \fun{diamond}, \fun{club},
        \fun{spade}],\\
                 &[2, 3, 4, 5, 6, 7, 8, 9, 10, \fun{jack},
                    \fun{queen}, \fun{king},\fun{ace}])\\
      &\xrightarrow{+} 
    [\{\fun{heart}, 2\}, \{\fun{heart}, 3\}, \ldots, \{\fun{heart},
      \fun{ace}\},\\
     & \{\fun{diamond}, 2\}, \ldots, \{\fun{spade}, \fun{ace}\}].
    \end{align*}

\end{enumerate}

\end{document}
