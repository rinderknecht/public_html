%%-*-latex-*-

\documentclass[11pt,a4paper]{article}

\usepackage[british]{babel}
\usepackage[T1]{fontenc}
\usepackage[latin1]{inputenc}
\usepackage{amsmath,amssymb}

\usepackage{array}
\usepackage{multirow}

\title{Test 1g of \textsf{Erlang}}
\author{Christian Rinderknecht}
\date{7 October 2011}

\newcommand\fun[1]{\textsf{#1}}

\begin{document}

\maketitle

\thispagestyle{empty}


\begin{enumerate}

  \item Write a function \fun{flen} (\emph{flat length}) such that, if
    \[\fun{flen}(P) \xrightarrow{+} N,\]
    then \(N\) is the number of non-list items in list \(P\). For
    instance,
    \begin{align*}
    \fun{flen}([\fun{a},[\,],[0,[[\fun{b}]],[\,],1]])
    &\xrightarrow{+} 4,\\
    \fun{flen}([[\,],[[\,],[[\,]],[\,]]]) &\xrightarrow{+} 0,\\
    \fun{flen}([\,]) &\xrightarrow{+} 0.
    \end{align*}

  \item Write a function \fun{enc/1} (\emph{encode}) such that, if
    \[\fun{enc}(P) \xrightarrow{+} Q,\] then \(Q\) is the list of
    pairs made of each item of \(P\) and their number of consecutively
    repeated items. For instance,
    \begin{align*}
      \fun{enc}([\fun{a}, \fun{a}, \fun{b}, \fun{c}, \fun{c}, \fun{c},
        \fun{a}]) &\xrightarrow{+} [\{\fun{a},2\}, \{\fun{b},1\},
        \{\fun{c},3\}, \{\fun{a},1\}],\\
      \fun{enc}([\fun{a},\fun{b}]) &\xrightarrow{+} [\{\fun{a},1\},
        \{\fun{b},1\}].
    \end{align*}

  \item Improve upon \fun{enc/1} in such a way that item not repeated
    appear in the result as themselves. For instance,
    \begin{align*}
      \fun{enc}([\fun{a}, \fun{a}, \fun{b}, \fun{c}, \fun{c}, \fun{c},
        \fun{a}]) &\xrightarrow{+} [\{\fun{a},2\}, \fun{b},
        \{\fun{c},3\}, \fun{a}],\\
      \fun{enc}([\fun{a},\fun{b}]) &\xrightarrow{+} [\fun{a},
        \fun{b}].
    \end{align*}

\end{enumerate}

\end{document}
