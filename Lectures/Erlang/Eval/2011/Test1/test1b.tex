%%-*-latex-*-

\documentclass[11pt,a4paper]{article}

\usepackage[british]{babel}
\usepackage[T1]{fontenc}
\usepackage[latin1]{inputenc}
\usepackage{amsmath,amssymb}

\title{Test 1b of \textsf{Erlang}}
\author{Christian Rinderknecht}
\date{7 October 2011}

\newcommand\fun[1]{\textsf{#1}}

\begin{document}

\maketitle

\thispagestyle{empty}

\begin{enumerate}

  \item Write a function \fun{uniq/1} such that, if \[\fun{uniq}(P)
    \xrightarrow{+} Q,\] then \(Q\)~is the list of unique items of
    list \(P\), in the same original order. For example,
    \[\fun{uniq}([\fun{a},\fun{b},\fun{b},\fun{c},\fun{d},\fun{c}])
    \xrightarrow{+} [\fun{a},\fun{d}].\] If all items in \(P\) are
    unique or \(P\) is empty, then \(Q\) is~\(P\).

  \item Write a function \fun{dup/1} such that, if \[\fun{dup}(P)
    \xrightarrow{+} Q,\] then \(Q\)~is the list of duplicated items of
    list \(P\), in the same original order. For example,
    \[\fun{dup}([\fun{a},\fun{b},\fun{b},\fun{c},\fun{d},\fun{c}])
    \xrightarrow{+} [\fun{b},\fun{c}].\] If all items in \(P\) are
    unique or \(P\) is empty, then \(Q\) is~\([\,]\).

\end{enumerate}

\end{document}
