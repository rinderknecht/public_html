%%-*-latex-*-

\documentclass[11pt,a4paper]{article}

\usepackage[british]{babel}
\usepackage[T1]{fontenc}
\usepackage[latin1]{inputenc}
\usepackage{amsmath,amssymb}

\usepackage{array}
\usepackage{multirow}

\title{Test 1e of \textsf{Erlang}}
\author{Christian Rinderknecht}
\date{7 October 2011}

\newcommand\fun[1]{\textsf{#1}}

\begin{document}

\maketitle

\thispagestyle{empty}


\begin{enumerate}

  \item This is about shuffling three stacks in the following way (the
    top of the stacks is on the top of the figure):
\begin{center}
\begin{tabular}{>{$}c<{$}>{$}c<{$}>{$}c<{$}c>{$}c<{$}}
a_1 & a_2 & a_3 & \multirow{4}{*}{\(\rightarrow\dots\rightarrow\)} 
                  & a_3\\
b_1 & b_2 & b_3 & & a_1\\
c_1 & c_2 &     & & a_2\\
    & d_2 &     & & b_3\\
    &     &     & & b_1\\
    &     &     & & b_2\\
    &     &     & & c_1\\
    &     &     & & c_2\\
    &     &     & & d_2
\end{tabular}
\end{center}
If the three stacks are empty, then the result is empty too. Write a
function \fun{three/3} computing this kind of shuffle.

  \item Write a function \fun{pre/2} (\emph{prefix}) such that, if
    \[\fun{pref}(P,Q) \xrightarrow{+} R,\]
    and list \(P\) is the beginning of list~\(Q\) then \(R\)~is
    \fun{true}, otherwise \fun{false}. For example,
    \begin{align*}
      \fun{pre}([0,1,1,0,1],[0,1,1,0,1,1,1,0,0])
      &\xrightarrow{+} \fun{true},\\
      \fun{pre}([\fun{t},\fun{h},\fun{e}],[\fun{t},\fun{h},\fun{e},\fun{s},\fun{e}])
      & \xrightarrow{+} \fun{true},\\
      \fun{pre}([4,\fun{a},3],[4,\fun{a}])
      &\xrightarrow{+} \fun{false},\\
      \fun{pre}([\,],[1,2,3])
      &\xrightarrow{+} \fun{true}.
    \end{align*}
    Note that the empty list is the prefix of \emph{any} list.

\end{enumerate}

\end{document}

shuffle3([], [], [])      -> [].
shuffle3(L1, L2, [H3|T3]) -> [H3|shuffle(L2, T3, L1)];
shuffle3(L1, L2, [])      -> shuffle3(L2, [], L1);


