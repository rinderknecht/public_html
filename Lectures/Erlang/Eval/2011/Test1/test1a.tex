%%-*-latex-*-

\documentclass[11pt,a4paper]{article}

\usepackage[british]{babel}
\usepackage[T1]{fontenc}
\usepackage[latin1]{inputenc}
\usepackage{amsmath,amssymb}

\title{Test 1a of \textsf{Erlang}}
\author{Christian Rinderknecht}
\date{7 October 2011}

\newcommand\fun[1]{\textsf{#1}}

\begin{document}

\maketitle

\thispagestyle{empty}

\begin{enumerate}

  \item Write a function \fun{rmlst/1} (\emph{remove the last}) such
    that if \[\fun{rmlst}(P) \xrightarrow{+} Q,\] then \(Q\) is the
    list \(L\) without its last item. For example,
    \[\fun{rmlst}([1,2,3]) \xrightarrow{+} [1,2].\] If the list is
    empty, the result is the atom \fun{empty}: \(\fun{rmlst}([\,])
    \xrightarrow{+} \fun{empty}\).

  \item Write a function \fun{pair/2} such that if \[\fun{zip}(P,Q)
    \xrightarrow{+} R,\] then \(R\) is the list made of the pairs of
    items of lists \(P\) and~\(Q\). For example,
    \[\fun{zip}([1,2,3],[\fun{a},\fun{b},\fun{c}]) \xrightarrow{+}
          [\{1,\fun{a}\}, \{2,\fun{b}\}, \{3,\fun{c}\}].\] If one list
          is longer than the other, then the atom \fun{length} should
          be the result, e.g., \(\fun{zip}([1,2],[\fun{a}])
          \xrightarrow{+} \fun{length}\).

  \item Write a function \fun{twoway/1} such that if the input list
    is the same when read from right to left, the result is
    \fun{true}, otherwise it is \fun{false}. For instance,
    \begin{align*}
      \fun{twoway}([0,1,0,1,0]) &\xrightarrow{+} \fun{true},\\
      \fun{twoway}([0,1,1,0]) &\xrightarrow{+} \fun{true},\\
      \fun{twoway}([\fun{r},\fun{a},\fun{d},\fun{a},\fun{r}])
      &\xrightarrow{+} \fun{true},\\
      \fun{twoway}([0,1,1,0,0,1,0]) &\xrightarrow{+} \fun{false},\\
      \fun{twoway}([\,]) &\xrightarrow{+} \fun{false}.
    \end{align*}

\end{enumerate}

\end{document}
