%%-*-latex-*-

% ------------------------------------------------------------------------
%
\begin{frame}
\frametitle{A simple calculator}

Perhaps the simplest thing to start playing with is an integer
calculator, i.e., integers and the four basic arithmetic operators,
\(+\), \(-\), \(\times\) and \(/\) (integer division).

\bigskip

An \textbf{expression} may be built using these operations, integers
and parentheses. For example \((2+3)\times(4+5)\) is an expression.

\bigskip

A \textbf{value} is an integer, therefore it is a special case of
expression.

\bigskip

What a calculator does is to match an inputted expression with a
value, usually called \emph{the result}. For example, the above
expression evaluates in \(45\).

\bigskip

This process is called \textbf{evaluation}.

\end{frame}

% ------------------------------------------------------------------------
%
\begin{frame}
\frametitle{Evaluation as rewriting expressions/Reductions}

The evaluation of arithmetic expressions can be defined as a series of
transformations leading to a value (the result). These transformations
are called \textbf{rewrites} and a series of rewrites is a
\textbf{reduction} or, more generally, a \textbf{computation}.

\bigskip

As we learnt in school long time ago, the way to perform these
rewrites is to select a part of the current expression consisting of
an operator and its arguments \emph{as values}, then to replace this
sub-expression by the operation result.

\bigskip

In the following reduction, let us underline the sub-expression to be
rewritten and print in bold the result of each step:
\[
(2+3) \times (\underline{4+5}) \rightarrow (\underline{2+3}) \times
\textbf{9} \rightarrow \underline{\textbf{5} \times 9}
\rightarrow \textbf{45}
\]

\end{frame}

% ------------------------------------------------------------------------
%
\begin{frame}
\frametitle{Strategy of evaluation and rewrite systems}

Note that there are different ways to choose the sub-expression. For
example:
\[
(\underline{2+3}) \times (4+5) \rightarrow \textbf{5} \times
(\underline{4+5}) \rightarrow \underline{5 \times \textbf{9}}
\rightarrow \textbf{45}
\]
It is usual to write, as a summary: \((2+3)\times(4+5)
\xrightarrow{*} 45\) or \((2+3)\times(4+5)=45\).

\bigskip

So, there are two concepts involved here:
\begin{enumerate}

  \item the selection of a sub-expression made of an operator and its
    arguments as values,

  \item the rewriting of such sub-expression into a value.

\end{enumerate}

They are defined by a \textbf{rewrite system}.

\end{frame}

% ------------------------------------------------------------------------
%
\begin{frame}
\frametitle{Rewrite rules}

A rewrite system is a set of \textbf{rewrite rules}. 

\bigskip

Rewriting rules specify how to perform a rewrite.

\bigskip

For example, let us assume that we have an infinite set of rewriting
rules, i.e., an infinite rewrite system, like:
\begin{align*}
0 + 0 & \rightarrow 0\\
0 + 1 & \rightarrow 1\\
\dots & \rightarrow \dots\\
124 - 57 & \rightarrow 67\\
\dots & \rightarrow \dots
\end{align*}

\end{frame}

% ------------------------------------------------------------------------
%
\begin{frame}
\frametitle{Conditional rewrite rules and meta-variables}

This rewrite system is not enough for the kind of expressions we want
to evaluate, where it is possible to combine operations, like \((2+3)
\times (4+5)\).

\bigskip

So let us extend our system with rules that specify the
sub-expressions we can rewrite. 

\bigskip

Consider the multiplication:
\[
e_1 \times e_2 \rightarrow e'_1 \times e_2 \quad \text{if} \; e_1
\rightarrow e'_1
\]
where \(e_1\), \(e'_1\) and \(e_2\) denote any expression and are
called \textbf{meta-variables}.

\end{frame}

% ------------------------------------------------------------------------
%
\begin{frame}
\frametitle{Instance of a rewrite rule}

An \textbf{instance} of this rule is obtained when the meta-variables
are replaced by expressions.

\bigskip

For example, \( (2+3) \times (4+5) \rightarrow 5 \times (4+5) \) is an
instance since \(2 + 3 \rightarrow 5\). To understand why, replace
\(e_1\) by \(2+3\), \(e'_1\) by \(5\) and \(e_2\) by \(4+5\). 

\bigskip

Let us rewrite \(((2 + 3) \times 6) \times (4+5)\). First, we see that
the previous rule has the required shape, i.e., the left side of the
arrow is \(e_1 \times e_2\) and if \(e_1 = (2+3) \times 6\) and \(e_2
= 4+5\) then \(e_1 \times e_2\) is exactly the expression we wish to
rewrite. So, we replace \(e_1\) and \(e_2\) by their associated
expressions in the rule and we get an instance of it:
\[
((2 + 3) \times 6) \times (4+5) \rightarrow e'_1 \times (4+5) \quad
\text{if} \; (2+3) \times 6 \rightarrow e'_1
\]
Therefore, we now know that we have to rewrite \((2 + 3) \times 6\).

\end{frame}

% ------------------------------------------------------------------------
%
\begin{frame}
\frametitle{Instance of a rewrite rule (cont)}

We need another instance of the same rule. This time, let \(e_1 =
2+3\) and \(e_2 = 6\). We get
\[
(2+3) \times 6 \rightarrow e'_1 \times 6 \quad \text{if} \; 2 + 3
\rightarrow e'_1
\]
It is important to understand that this latter \(e'_1\) is not the
same as the former: they belong to two different instances of the same
rule. The latter \(e'_1\) is actually easy to find, since we have all
the rules for basic cases as \(2 + 3 \rightarrow 5\). Hence, \(e'_1 =
5\) and the second instance is now complete:
\[
(2+3) \times 6 \rightarrow 5 \times 6 \quad \text{if} \; 2 + 3
\rightarrow 5
\]
Going backwards, this implies that the former \(e'_1\) is
\(5 \times 6\). So, the first instance is
\[
((2+3) \times 6) \times (4+5) \rightarrow (5 \times 6) \times (4+5)
\quad \text{if} \; 2 + 3 \rightarrow 5
\]

\end{frame}

% ------------------------------------------------------------------------
%
\begin{frame}
\frametitle{Instance of a rewrite rule (cont)}

Our rule for multiplication, \(e_1 \times e_2 \rightarrow e'_1 \times
e_2\), actually depends on another, \(e_1 \rightarrow e'_1\). It is
usual to write them this way:
\begin{mathpar}
\inferrule
{e_1 \rightarrow e'_1}
{e_1 \times e_2 \rightarrow e'_1 \times e_2}
\;\rname{\TirName{Mult}_1}
\end{mathpar}
This is called an \textbf{inference rule}. Note that it has a name,
\(\rname{\TirName{Mult}_1}\).
The rewrite rule on the top, acting as a condition for the rule on the
bottom, is called a \textbf{premise}. The rule at the bottom is called
a \textbf{conclusion}.

\bigskip

A \textbf{substitution} is a mapping from meta-variables to
expressions which applies to an inference rule or rewrite rule. For
example, let \(\sigma\) be the following substitution: \(\{e_1 \mapsto
(2+3) \times 6; e_2 \mapsto 4+5\}\). This notation is equivalent to
\(\sigma(e_1) = (2+3) \times 6\) and \(\sigma(e_2) = 4+5\).

\end{frame}

% ------------------------------------------------------------------------
%
\begin{frame}
\frametitle{Instance of a rewrite rule (cont)}

Now we can make a clear definition of the notion of ``instance of a
rule'': 
\begin{quote}
\emph{An instance of a rule is a rule in which some meta-variables
  \(e_i\) are replaced by expressions \(\sigma(e_i)\).}
\end{quote}
For example, here is the instance of \(\rname{\TirName{Mult}_1}\)
using previously defined substitution \(\sigma\), and noted
\(\sigma\rname{\TirName{Mult}_1}\):
\begin{mathpar}
\inferrule
{(2 + 3) \times 6 \rightarrow e'_1}
{((2+3) \times 6) \times (4+5) \rightarrow e'_1 \times (4+5)}
\;\sigma\rname{\TirName{Mult}_1}
\end{mathpar}

\end{frame}

% ------------------------------------------------------------------------
%
\begin{frame}
\frametitle{Instance of a rewrite rule (cont)}

Now how do we find (a mapping for) \(e'_1\)?

\bigskip

Let us define another substitution \(\sigma'=\{e_1 \mapsto 2 + 3; e_2
\mapsto 6\}\) and apply it:
\begin{mathpar}
\inferrule
{2 + 3 \rightarrow e'_1}
{(2 + 3) \times 6 \rightarrow e'_1 \times 6}
\;\sigma'\rname{\TirName{Mult}_1}
\end{mathpar}
Note that the meta-variable \(e'_1\) in
\(\sigma'\rname{\TirName{Mult}_1}\) is not the same as in
\(\sigma\rname{\TirName{Mult}_1}\): they should have a different
mapping.

\bigskip

We have a basic rewrite rule \(2 + 3 \rightarrow 5\), thus we 
extend \(\sigma'\) with \(e'_1 \mapsto 5\) and get
\begin{mathpar}
\inferrule
{2 + 3 \rightarrow 5}
{(2 + 3) \times 6 \rightarrow 5 \times 6}
\;\sigma'\rname{\TirName{Mult}_1}
\end{mathpar}

\end{frame}

% ------------------------------------------------------------------------
%
\begin{frame}
\frametitle{Instance of a rewrite rule (cont)}

Therefore we can extend \(\sigma\) with \(e'_1 \mapsto 5 \times 6\)
and we get
\begin{mathpar}
\inferrule
{(2 + 3) \times 6 \rightarrow 5 \times 6}
{((2+3) \times 6) \times (4+5) \rightarrow (5 \times 6) \times (4+5)}
\;\sigma\rname{\TirName{Mult}_1}
\end{mathpar}
We can summarise this rewrite using the two instances of
\(\rname{\TirName{Mult}_1}\) as
\begin{mathpar}
\inferrule*[right=\(\;\sigma\rname{\TirName{Mult}_1}\)]
{
\inferrule*[right=\(\;\sigma'\rname{\TirName{Mult}_1}\)]
{2 + 3 \rightarrow 5}
{(2 + 3) \times 6 \rightarrow 5 \times 6}
}
{((2+3) \times 6) \times (4+5) \rightarrow (5 \times 6) \times (4+5)}
\end{mathpar}

\end{frame}

% ------------------------------------------------------------------------
%
\begin{frame}
\frametitle{Instance of a rewrite rule (cont)}

But what about the rewrite of an expression like \(7 \times (4+5)\)?

\bigskip

Let us define another substitution on \(\rname{\TirName{Mult}_1}\):
\(\sigma' = \{e_1 \mapsto 7; e_2 \mapsto 4+5\}\).

\bigskip

The corresponding instance is
\begin{mathpar}
\inferrule
{7 \rightarrow e'_1}
{7 \times (4 + 5) \rightarrow e'_1 \times (4 +5)}
\;\sigma'\rname{\TirName{Mult}_1}
\end{mathpar}
But there is no rule of shape ``\(v \rightarrow \dots\)'', where \(v\)
is \emph{a value because a value is, by definition, an expression
  completely reduced.}

\end{frame}

% ------------------------------------------------------------------------
%
\begin{frame}
\frametitle{Completing the multiplication rule}

This means that we need another rule for the evaluation of the
\emph{right} argument of \(\times\):
\begin{mathpar}
\inferrule
{e_2 \rightarrow e'_2}
{e_1 \times e_2 \rightarrow e_1 \times e'_2}
\;\rname{\TirName{Mult}_2}
\end{mathpar}
This inference rule says that we can rewrite a product by rewriting its
right argument.

\bigskip

For example, here is the instance \(\sigma'\rname{\TirName{Mult}_2}\):
\begin{mathpar}
\inferrule
{4 + 5 \rightarrow 9}
{7 \times (4+5) \rightarrow 7 \times 9}
\;\sigma'\rname{\TirName{Mult}_2}
\end{mathpar}

\end{frame}

% ------------------------------------------------------------------------
%
\begin{frame}
\frametitle{Strategy}

But now, something interesting happens. 

\bigskip

Consider again \((2 + 3) \times (4+5)\): both
\(\rname{\TirName{Mult}_1}\) and \(\rname{\TirName{Mult}_2}\) can be
instantiated by substitution \(\sigma'' = \{e_1 \mapsto 2+3; e_2
\mapsto 4+5\}\):
\begin{mathpar}
\inferrule
{2 + 3 \rightarrow 5}
{(2+3) \times (4+5) \rightarrow 5 \times (4+5)}
\;\sigma''\rname{\TirName{Mult}_1}
\and
\inferrule
{4 +5 \rightarrow 9}
{(2+3) \times (4+5) \rightarrow (2+3) \times 9}
\;\sigma''\rname{\TirName{Mult}_2}
\end{mathpar}

\end{frame}

% ------------------------------------------------------------------------
%
\begin{frame}
\frametitle{Strategy (cont)}

This means that \emph{our system does not specify the order of
  evaluation of \(\times\) arguments.} 

\bigskip

In other words, our rewrite system does not select only one
sub-expression to rewrite at each step.

\bigskip

We need a \textbf{strategy} to complete it.

\end{frame}

% ------------------------------------------------------------------------
%
\begin{frame}
\frametitle{Finiteness and soundness of reductions}

At this point, it is clear that these reductions satisfy the following
properties:
\begin{itemize}
 
  \item \textbf{Finiteness.} All the reductions are finite, i.e., all
    the computations terminate.

  \item \textbf{Soundness.} All the reductions of an expression end
    with the same value, if any.
  
\end{itemize}

About the second point, note indeed that \emph{some expressions have
  no value}, as
\[
(\underline{2+3})/(4-4) \rightarrow \textbf{5}/(\underline{4-4})
\rightarrow 5/\textbf{0} \nrightarrow
\]
The reduction is finite but does not end with a value: this situation
corresponds to an \textbf{error}, usually called \textbf{run-time
  error} in programming languages. Expression \(5/0\) is
\textbf{stuck}.

\end{frame}

% ------------------------------------------------------------------------
%
\begin{frame}
\frametitle{Another model of run-time errors}

We can modify the calculator so that, in case of error (here, the only
one is the division by zero), the result of a rewrite is a special
value, called \NaN (\emph{Not a Number}). Then we add the following
rules:
\begin{mathpar}
\inferrule
{}
{e/0 \rightarrow \NaN}
\quad\rname{\TirName{DivZero}}
\and
\inferrule
{}
{\NaN/e \rightarrow \NaN}
\quad\rname{\TirName{Div-Err}_1}
\and
\inferrule
{}
{e/\NaN \rightarrow \NaN}
\quad\rname{\TirName{Div-Err}_2}
\and
\inferrule
{}
{\NaN \times e \rightarrow \NaN}
\quad\rname{\TirName{Mult-Err}_1}
\and
\inferrule
{}
{e \times \NaN \rightarrow \NaN}
\quad\rname{\TirName{Mult-Err}_2}
\and
\inferrule
{}
{\NaN + e \rightarrow \NaN}
\quad\rname{\TirName{Add-Err}_1}
\and
\inferrule
{}
{e + \NaN \rightarrow \NaN}
\quad\rname{\TirName{Add-Err}_2}
\and
\inferrule
{}
{\NaN - e \rightarrow \NaN}
\quad\rname{\TirName{Sub-Err}_1}
\and
\inferrule
{}
{e - \NaN \rightarrow \NaN}
\quad\rname{\TirName{Sub-Err}_2}
\end{mathpar}

\end{frame}

% ------------------------------------------------------------------------
%
\begin{frame}
\frametitle{Another model of run-time errors (cont)}

The rationale of this system is that once an error occurs, no other
computations (leading to integer values) are required and the
reduction can end as soon as possible.

\bigskip

In this case:
\[
(2+3)/(4-4) \rightarrow 5/(4-4) \rightarrow 5/0 \rightarrow \NaN
\]
or, even the shortest possible:
\[
(2+3)/(4-4) \rightarrow (2+3)/0 \rightarrow \NaN
\]

\end{frame}

% ------------------------------------------------------------------------
%
\begin{frame}
\frametitle{Another model of run-time errors/Rephrasing soundness}

The interesting phenomenon here is that, now, any reduction ends in a
value. In the previous rewrite system, any reduction ends either in a
value or a \textbf{stuck expression}, i.e., an expression that cannot
be rewritten further.

\bigskip

With the new system, that explicitly specifies the reduction of the
\NaN error, the soundness property can be rephrased as 
\begin{quote}
\emph{All the reductions of an expression end with the same value.}
\end{quote}
Obviously, the drawback of the new system is that it requires a lot of
additional rules, and, as the system is augmented with new rules, new
rules for the errors have to be added... or are forgotten by mistake.

\end{frame}

% ------------------------------------------------------------------------
%
\begin{frame}
\frametitle{Specifying the strategy}

It is possible to force an order of evaluation on the arguments of
\(\times\). Assume we want to reduce the left argument first.

\bigskip

The idea is to change \(\rname{\TirName{Mult}_2}\) so that instead of
\(e_1\), which stands for any expression, we specify a value \(v\):
\begin{mathpar}
\inferrule
{e_2 \rightarrow e'_2}
{v \times e_2 \rightarrow v \times e'_2}
\;\rname{\TirName{New-Mult}_2}
\end{mathpar}
This way, it is not possible anymore to instantiate this rule with
expression \((2+3) \times (4+5)\) because \(2+3\) is not a value. 

\end{frame}

% ------------------------------------------------------------------------
%
\begin{frame}
\frametitle{Specifying the strategy (cont)}

The only possible reduction is:
\begin{mathpar}
\inferrule
{2 + 3 \rightarrow 5}
{(2+3) \times (4+5) \rightarrow 5 \times (4+5)}
\;\sigma_1\rname{\TirName{Mult}_1}
\and
\inferrule
{4 + 5 \rightarrow 9}
{5 \times (4+5) \rightarrow 5 \times 9}
\;\sigma_2\rname{\TirName{New-Mult}_2}
\end{mathpar}
where \(\sigma_1=\{e_1 \mapsto 2+3; e_2 \mapsto 4+5\}\) and
\(\sigma_2=\{v \mapsto 5; e_2 \mapsto 4+5\}\).

\end{frame}

% ------------------------------------------------------------------------
%
\begin{frame}
\frametitle{Determinism}

We have specified the strategy \textbf{in} the rewrite system.

\bigskip

Also, with \(\rname{\TirName{New-Mult}_2}\), given an expression,
there is only one rule that can be applied (or none): this is
sometimes called \textbf{determinism}, or \textbf{determinacy}:

\bigskip

\textbf{Determinism.} If \(e_1 \rightarrow e'_1\) and \(e_2
\rightarrow e'_2\) then \(e'_1 = e'_2\).

\bigskip

This is a stronger property than soundness, i.e., it implies
soundness. Indeed, determinism implies that there is only one
reduction from an expression to its value (if any).

\end{frame}

% ------------------------------------------------------------------------
%
\begin{frame}
\frametitle{Pragmatics of programming languages}

In many programming languages, the order of evaluation of operator
arguments is not specified, which usually means that there is a
sentence in the official document which defines the language like:
``The order of evaluation of the arguments of operators and functions
is not specified.''

\bigskip

\noindent Formally, this means that the underlying rewrite system is
\textbf{non-deterministic}.

\bigskip

One notable exception is \Java, which specifies that the arguments are
always evaluated from left to right, i.e., following the order of
writing.

\bigskip

Of course, if an order is specified, as in \Java, the compilers of
these languages must implement it. Otherwise one is chosen
arbitrarily and may change from one release to another.

\end{frame}

% ------------------------------------------------------------------------
%
\begin{frame}
\frametitle{Rewrite systems as models of languages}

By studying rewrite systems, we can learn the basic concepts involved
in programming languages, because rewrite systems are extremely
simple.

\bigskip

There are some languages, like \Maude, whose programs actually are
rewrite systems.

\bigskip

Some compilers transform the input program (called \textbf{source})
and output a program, called \textbf{byte-code}, which can be
reduced by a rewrite system.

\end{frame}

% ------------------------------------------------------------------------
%
\begin{frame}
\frametitle{Rewrite systems as models of languages (cont)}

This is the case of \Java. This is why, in general, we need a
\textbf{\Java virtual machine} to run the program: this JVM performs
the reductions according to a rewrite system.

\bigskip

Even if there is no byte-code, we always can interpret what the
program does by studying a rewrite system which models it. 

\end{frame}

% ------------------------------------------------------------------------
%
\begin{frame}
\frametitle{Sequentiality and parallelism}

Some programming languages, like \Ada, allow the programmer to specify
that different parts of its program can actually be executed
separately: this is called, in general, \textbf{parallelism}. This
parallelism is possible only if the underlying rewrite system is
non-deterministic.

\bigskip

Indeed, if all the strategies lead to the same value (soundness
property), it is possible to reduce different parts of an expression
in parallel if the rewrite system allows it.

\end{frame}

% ------------------------------------------------------------------------
%
\begin{frame}
\frametitle{Sequentiality and parallelism (cont)}

For example, the first argument of \(\times\) can be computed \emph{at
  the same time} as its second argument. In the case of \((2+3) \times
(4+5)\), this means that we can reduce in parallel \(2+3\), with
\(\rname{\TirName{Mult}_1}\), and \(4+5\), with
\(\rname{\TirName{Mult}_2}\), then \(5 \times 9\) is finally rewritten
in \(45\) (after \textbf{synchronisation} of the two reductions).

\end{frame}
