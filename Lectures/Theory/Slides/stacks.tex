%%-*-latex-*-

% ------------------------------------------------------------------------
%
\begin{frame}
\frametitle{Stacks}

Until now, we studied basic arithmetic and boolean expressions by
means of rewrite systems, which provided the formal definition of their
evaluation.

\bigskip

In order to check the expressive power of rewrite systems, let us show
how we can use them to define precisely a \textbf{data structure}
which is very useful in programming: the \textbf{stack}.

\bigskip

This approach allows us to define and understand the concept of stack
without referring to features like \textbf{pointers}, which depend on
specific \textbf{computer architectures} or \textbf{operating system
 libraries} etc.

\bigskip

For instance, in the case of \C, it is not possible to study stacks
without the notions of \textbf{memory}, \textbf{array} or
\textbf{memory allocation and deallocation} (\texttt{malloc} and
\texttt{free}), integer or \textbf{pointer arithmetic} etc.

\end{frame}

% ------------------------------------------------------------------------
%
\begin{frame}
\frametitle{Stacks/From the analogy to the formal notation}

A stack is similar to a box which is open on the top and that contains
a pile of paper sheets: we can only add a new sheet on its top (this
is called \textbf{to push}) and remove one on its top (this is called
\textbf{to pop}).

\bigskip

How do we model the fact that the stack has changed after a pop or a
push? The simplest is to imagine that we rewrite the operation on a
given stack into a (modified) stack.

\bigskip

Also, we do not want to specify actually the nature of the elements in
the stack, in order to be general.

\bigskip

Items in stacks can be of any kind, so we do not define them:
\emph{items are independent from the concept of stack.}

\end{frame}

% ------------------------------------------------------------------------
%
\begin{frame}
\frametitle{Stacks/Examples}

Let us define the ``stack values'':
\begin{itemize}

  \item let us note \proc{Empty} the empty stack;

  \item let us note \(\proc{Push} (i, s)\) the stack resulting of
    pushing item \(i\) onto the stack (value) \(s\).

\end{itemize}

\bigskip

For example:
\begin{itemize}

  \item \proc{Empty} is the empty stack;

  \item \(\proc{Push} (i, \proc{Empty})\) is the stack containing only
    item \(i\);

  \item \(\proc{Push} (i_1, \proc{Push} (i_2, \proc{Empty}))\) is the
    stack which only contains item \(i_1\) on the top and item \(i_2\)
    below (the bottom).

\end{itemize}

\end{frame}

% ------------------------------------------------------------------------
%
\begin{frame}[containsverbatim]
\frametitle{Stacks/Comparison with C}

In \C, we need the standard library. Then we need to define the stack
data structure using pointers. Since it is recursive, we need a trick
to declare it:
{\small
\begin{verbatim}
#include<stdlib.h>
typedef struct stack_ stack;
struct stack_ { int item; stack* next; };
\end{verbatim}
}
Then, we need to take care of deallocating a stack:
{\small
\begin{verbatim}
void free_stack (stack* s) {
  if (s != NULL) { stack* sub_stack = (*s).next;
                   free(s);
                   free_stack(sub_stack);
                 }
}
\end{verbatim}
}

\end{frame}

% ------------------------------------------------------------------------
%
\begin{frame}[containsverbatim]
\frametitle{Stacks/Comparison with C (cont)}

The definition of \proc{Push} in \C must take care of allocating
memory and checking whether this implies a memory overflow. If so, we
must exit but not forget to deallocate the previous stack, if not
\proc{Empty}. Note the type of \verb+push+: parameter \verb|s| must be
a pointer to a pointer because traditional \C does not include
reference passing. Also, the type of the items must be given
(\verb|int|).
{\small
\begin{verbatim}
void push (stack** s, int item) {
  stack* new_stack = malloc(sizeof(stack));
  if (new_stack == NULL) { if (s != NULL) free_stack(*s);
                           exit (2); }
  (*new_stack).item = item;
  (*new_stack).next = *s;
  *s = new_stack;
}
\end{verbatim}
}

\end{frame}

% ------------------------------------------------------------------------
%
\begin{frame}[containsverbatim]
\frametitle{Stacks/Comparison with C (cont)}

On this simple example, it is easy to see how the concept of stack is
buried in many important details in \C: memory management, address of
pointers as arguments, tricky recursive type definition, fixed type
for the items in the stack.

\bigskip

The approach with rewrite systems makes clear what the concept of
stack is and what it is not necessary. In particular, our formalism
does not make room for the concept of in-place modification: there is
not even he concept of memory. Contrast this to the \C implementation
of \verb|push|, which modifies the \emph{given} stack in memory.

\end{frame}

% ------------------------------------------------------------------------
%
\begin{frame}
\frametitle{Stacks/Comparison with C (cont)}

Also, in \C we must give the type of the elements in the stack, even
if the function does not use them: this is because the pointer
arithmetic needs to know the size of each cell.

\bigskip

Then, once we understand what it is, we can more safely
translate our understanding into programs because we already have in
mind what we want to implement, and we can focus on the specific
features of the language.

\end{frame}

% ------------------------------------------------------------------------
%
\begin{frame}
\frametitle{Stacks/Discussion}

Until now, there are only stacks (or ``stack values'') but no
\emph{computable} operation on them.

\bigskip

Operation \proc{Push} is not computable, i.e., there is no rewrite rule
for it: \proc{Push} is only used to define what a new stack is, given
another and an item to push on top of it.

\bigskip

This is different from the operator \(\times\) on integers, because
there were rewrite rules like \(2 \times 5 \rightarrow 10\). There are
no such rules for \proc{Push}.

\end{frame}

% ------------------------------------------------------------------------
%
\begin{frame}
\frametitle{Stacks/Discussion (cont)}

In other words, \proc{Empty} and \proc{Push} define the concept of
stack, they are \emph{values}.

\bigskip

But how about popping an item from the stack? This is should be part
of the concept too, because an item can only be popped from the top of
a stack. Should not this be part of the definition of the concept of
stack?

\end{frame}

% ------------------------------------------------------------------------
%
\begin{frame}
\frametitle{Stacks/Popping}

It is true that the way items are taken from a stack is part of the
definition of the concept of stack, but in an operational way, not
ontological. 

\bigskip

In other words, popping, while participating to the definition of what
a stack is, actually de-structures a stack. Therefore, popping can be
defined in terms of a sub-stack, i.e., it can be defined as a
computable operation.

\end{frame}

% ------------------------------------------------------------------------
%
\begin{frame}
\frametitle{Stacks/Popping (cont)}

Actually, this is very easy: let us take \proc{Pop} as the exact
inverse function of \proc{Push}:
\begin{mathpar}
\inferrule
{}
{\proc{Pop} (\proc{Push} (i, s)) \rightarrow s}
\quad\rname{\TirName{Pop}}
\end{mathpar}
Note that there is no rewrite rule for expression \(\proc{Pop}
(\proc{Empty})\) because there is no way to pop an item from an empty
stack.

\end{frame}

% ------------------------------------------------------------------------
%
\begin{frame}[containsverbatim]
\frametitle{Stacks/Popping in \C}

In contrast, in \C, the \proc{Pop} function would we implemented as
follows. Again, the programmer must take care of passing the address
of the pointer to the top of the stack, \verb|s|. In case of error, we
return an error code, so the popped element, if any, must be an
argument whose address is passed (\verb|top|). Some errors are
improbable, like \verb|s| being \verb|NULL|, but mandatory in
general.
{\small
\begin{verbatim}
int pop (stack** s, int* top) {
  if (s != NULL)
    if (*s != NULL) { *top = (**s).item;
                      *s = (**s).next;
                      return 0; }
    return 2;
  return 1;
}
\end{verbatim}
}

\end{frame}

% ------------------------------------------------------------------------
%
\begin{frame}[containsverbatim]
\frametitle{Stacks/Popping in \C (cont)}

Again, the concept of popping is buried in too many low-level details,
like handling three error cases, even if improbable. It is not obvious
at all, just by looking at the code, that \proc{Pop} is actually
exactly the inverse function of \proc{Push}.

\bigskip

Note also that the \C implementation \verb|pop| modifies the given
stack, because it would be expensive to make a copy of it. On the
other hand, modifying an argument, called \textbf{side-effect}, is
always risky because the caller must be aware of it.

\bigskip

There is no such notion of side-effect in our rewrite systems: it is
not necessary in order to understand what a stack actually is.

\end{frame}

% ------------------------------------------------------------------------
%
\begin{frame}[containsverbatim]
\frametitle{Stacks/Top}

Now let us consider a operator which is similar to popping. The
function \proc{Pop} returns the stack that remains when we ignore the
top element. The function \proc{Top} instead returns this top element
and ignores the remaining stack. This is fairly simple to define:
\begin{mathpar}
\inferrule
{}
{\proc{Top} (\proc{Push} (x, s)) \rightarrow x}
\quad\rname{\TirName{Top}}
\end{mathpar}
Just like \proc{Pop}, it does not apply to empty stacks. In \C, we do
not need a side-effect on the stack (only on the item):
{\small
\begin{verbatim}
int top (stack* s, int* item) {
  if (s == NULL) return 1;
  *item=(*s).item;
  return 0;}
\end{verbatim}
}

\end{frame}

% ------------------------------------------------------------------------
%
\begin{frame}
\frametitle{Stacks/Appending}

Let expression \(\proc{Append} (s_1, s_2)\) represent the
stack made of stack \(s_1\) on top of stack \(s_2\).

\bigskip

Let us express exactly what we mean using a rewrite system defined by
inference rules.
\begin{align*}
\proc{Append} (\proc{Empty}, \proc{Empty}) 
  & \rightarrow \proc{Empty}\\
\proc{Append} (\proc{Empty}, \proc{Push} (e, s)
  & \rightarrow \proc{Push} (e, s)\\
\proc{Append} (\proc{Push} (e, s), \proc{Empty}) 
  & \rightarrow \proc{Push} (e, s)\\
\proc{Append} (\proc{Push} (e_1, s_1), \proc{Push} (e_2, s_2))
  & \rightarrow
\end{align*}
\(\proc{Push}(e_1,\proc{Append}(s_1,\proc{Push}(e_2,s_2)))\)

Let us explain how we find theses rewrite rules.

\end{frame}

% ------------------------------------------------------------------------
%
\begin{frame}
\frametitle{Stacks/Appending (cont)}

The first step consists in understanding that we must find rules of
the shape
\[
\proc{Append} (s_1, s_2) \rightarrow \cdots
\]
Second, we understand that \(s_1\) and \(s_2\) are stacks. We already
know that stacks are built by \proc{Empty} or \proc{Push}. Thus, we
replace each stack by all of its possible basic values: \proc{Empty}
and \(\proc{Push} (\dots, \dots)\). This leads to four cases:
\begin{align*}
  \proc{Append} (\proc{Empty}, \proc{Empty})
& \rightarrow \cdots\\
  \proc{Append} (\proc{Empty}, \proc{Push} (e, s))
& \rightarrow \cdots\\
  \proc{Append} (\proc{Push} (e, s), \proc{Empty})
& \rightarrow \cdots\\
  \proc{Append} (\proc{Push} (e_1, s_1), \proc{Push} (e_2, s_2))
& \rightarrow \cdots
\end{align*}

\end{frame}

% ------------------------------------------------------------------------
%
\begin{frame}
\frametitle{Stacks/Appending (cont)}

Now we need to think about the interpretation of each case. An analogy
and a picture is helpful. In the first case, we want append two empty
stacks. It is obvious that the resulting stack will be empty too:
\[
  \proc{Append} (\proc{Empty}, \proc{Empty}) \rightarrow \proc{Empty}
\]
In the second and third case, we append an empty stack and a non-empty
one. The order in which we append them is irrelevant and the result is
always the non-empty one:
\begin{align*}
  \proc{Append} (\proc{Empty}, \proc{Push} (e, s))
& \rightarrow \proc{Push} (e, s)\\
  \proc{Append} (\proc{Push} (e, s), \proc{Empty})
& \rightarrow \proc{Push} (e, s)
\end{align*}

\end{frame}

% ------------------------------------------------------------------------
%
\begin{frame}
\frametitle{Stacks/Appending (cont)}

The last case is the more difficult one: we append two non-empty
stacks. The top item of the stack to be appended on top of the other is
\(e_1\). This item will also be on the top of the resulting
stack. That is why we expect something like
\[
  \proc{Append} (\proc{Push} (e_1, s_1), \proc{Push} (e_2, s_2))
 \rightarrow \proc{Push} (e_1, \dots)
\]
Now, the stack below \(e_1\) in the result can be computed by
appending \(s_1\) to the second stack, \(\proc{Push} (e_2, s_2)\):
\begin{multline*}
\proc{Append} (\proc{Push} (e_1, s_1), \proc{Push} (e_2, s_2))
\rightarrow\\ \proc{Push} 
(e_1, \overline{\proc{Append} (s_1, \proc{Push} (e_2, s_2))})
\end{multline*}

\end{frame}

% ------------------------------------------------------------------------
%
\begin{frame}
\frametitle{Stacks/Appending (cont)}

By looking carefully to the rewrite rules, we can devise a simpler
one.

\bigskip

Indeed, in the case of one of the stacks is empty, the result will
always be the other stack. Thus, we only need to check if the first or
the second stack is empty:
\begin{align*}
  \proc{Append} (\proc{Empty}, s) & \rightarrow s\\
  \proc{Append} (s, \proc{Empty}) & \rightarrow s
\end{align*}
How can we check that this is correct? 

\end{frame}

% ------------------------------------------------------------------------
%
\begin{frame}
\frametitle{Stacks/Appending (cont)}

Remember that \(s\) can be replaced by any stack, so let us try with
the two basic stacks:
\begin{align*}
  \proc{Append} (\proc{Empty}, \proc{Empty}) & \rightarrow s\\
  \proc{Append} (\proc{Empty}, \proc{Push} (e, s)) & \rightarrow s\\
  \proc{Append} (\proc{Empty}, \proc{Empty}) & \rightarrow s\\
  \proc{Append} (\proc{Push} (e, s), \proc{Empty}) & \rightarrow s
\end{align*}

\end{frame}

% ------------------------------------------------------------------------
%
\begin{frame}
\frametitle{Stacks/Appending (cont)}

The third rule is redundant. If we remove it, we find the original
system. Thus both systems are equivalent.

\bigskip

There is a different, though, but not with respect of the rewrites
themselves (both systems are sound) but about the \emph{strategy}: the
original system was deterministic while the second is
non-deterministic: in the case when the two stacks are empty, two
rules are possible.

\end{frame}

% ------------------------------------------------------------------------
%
\begin{frame}
\frametitle{Stacks/Appending (cont)}

We can simplify further. The last rule shows that during the rewrite,
the sub-expression \(\proc{Push} (e_2, s_2)\) is invariant, i.e., that
it is simply copied, without none of its sub-expressions (\(e_2\) and
\(s_2\)) being used some place else in the result.

This shows that the last rule does not care whether the second stack,
i.e., the one that will be below the first one, is empty or not: it is
just copied. Therefore we can simply write
\[
\proc{Append} (\proc{Push} (e_1, s_1), s) \rightarrow \proc{Push}
(e_1, \proc{Append} (s_1, s))
\]
We can rename \(s\) into \(s_2\), to make it more regular:
\[
\proc{Append} (\proc{Push} (e_1, s_1), s_2) \rightarrow \proc{Push}
(e_1, \proc{Append} (s_1, s_2))
\]

\end{frame}

% ------------------------------------------------------------------------
%
\begin{frame}
\frametitle{Stacks/Appending (cont)}

Finally:
\begin{align*}
  \proc{Append} (\proc{Empty}, s) 
& \rightarrow s 
& \rname{\TirName{Append}_1}\\
  \proc{Append} (s, \proc{Empty})
& \rightarrow s
& \rname{\TirName{Append}_2}\\
  \proc{Append} (\proc{Push} (e_1, s_1), s_2) 
& \rightarrow \proc{Push} (e_1, \proc{Append} (s_1, s_2))
& \rname{\TirName{Append}_3}
\end{align*}
The new last rule adds to the non-determinism. Let
\(\sigma_1(s) = \proc{Push} (e_1, s_1)\) and \(\sigma_2(s_2) =
\proc{Empty}\):
\begin{align*}
\proc{Append} (\underline{\proc{Push} (e_1, s_1)}, \proc{Empty}) 
& \rightarrow \overline{\proc{Push} (e_1, s_1)}
& \!\!\sigma_1\rname{\TirName{Append}_2}
\end{align*}
\begin{multline*}
\proc{Append} (\proc{Push} (e_1, s_1), \underline{\proc{Empty}}) 
 \rightarrow\\
\proc{Push} (e_1, \underline{\proc{Append} (s_1,\overline{\proc{Empty}})}) 
 \!\!\sigma_2\rname{\TirName{Append}_3}
\end{multline*}
%% & \rightarrow \proc{Push} (e_1, \overline{s_1})
%% & \sigma_3\rname{\TirName{Append}_3}

\end{frame}

% ------------------------------------------------------------------------
%
\begin{frame}
\frametitle{Stacks/Appending (cont)}

The only difference is the \textbf{complexity}, that is, in this
framework of rewriting systems, the number of steps needed to reach
the result. 

\bigskip

With rule \(\TirName{Append}_2\), if the second stack is empty,
we can conclude in one step. 

\bigskip

If rule \(\TirName{Append}_3\) is successively preferred, we have to
traverse all the elements of the first stack before terminating.

\bigskip

\textbf{Exercise.} Implement in \C the \proc{Append} function
\begin{enumerate}

  \item using loops;

  \item without loops.

\end{enumerate}

\end{frame}
