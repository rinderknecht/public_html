%%-*-latex-*-

% ------------------------------------------------------------------------
%
\begin{frame}
\frametitle{Introduction}

The purpose of this course is to give you some understanding on
different concepts that underlie many programming languages, some of
which you already are familiar with.

\bigskip

The point is that, despite you have a programming experience with some
languages, as \Java, \C, \Cpp, \Csharp etc., you do not really know
the \textbf{foundations of the features} of these languages, as
recursion, variable scoping, assignments, loops, classes and objects,
modules, typing, etc.

\bigskip

So the purpose of this course is to show you how some important
features of programming languages you know are built one on top of the
other, but also to introduce you to new concepts, that you can find in
programming languages that you do not know (yet), as
\textbf{functional programming languages}.

\end{frame}

% ------------------------------------------------------------------------
%
\begin{frame}
\frametitle{Introduction (cont)}

Since we wish to describe programming concepts, we need a language to
express ourselves. Two ways, at least, are possible:
\begin{itemize}

  \item to use a new, artificial, programming language which is
    extremely simple (i.e. with only a very few features) and step by
    step show how to enrich it by adding new features on top of the
    previous ones or by using external concepts;

  \item to use an existing programming language which has a very
    simple subset allowing the same building procedure.

\end{itemize}
The second approach is not possible with the languages you already
know: they are too specialised. But there is a family of programming
languages that are suitable: functional programming languages.

\end{frame}
