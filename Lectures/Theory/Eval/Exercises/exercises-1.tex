%%-*-latex-*-

\documentclass[11pt,a4paper]{article}

% Tracing the document for the HTML
%
\input{trace}

% Title
%
\title{Exercises on Concepts of Programming Languages}
\author{Christian Rinderknecht}
\date{9 February 2006}

% Maths
%
\usepackage{amsmath,amssymb}

% Inference rules
%
\usepackage{mathpartir}

% Code
%
\usepackage{clrscode}

% Miscellanea
%
\usepackage{xspace}

% New commands
%
\newcommand{\true}{\textsf{true}\xspace}
\newcommand{\false}{\textsf{false}\xspace}
\newcommand\rname[1]{\langle{#1}\rangle}
\newcommand{\NaN}{\textsf{NaN}\xspace}

% ------------------------------------------------------------------------
% Document

\begin{document}

\maketitle

\section{Calculator}

Assume the following system of inference rules:
\begin{mathpar}
\inferrule
{e_1 \rightarrow e'_1}
{e_1 \times e_2 \rightarrow e'_1 \times e_2}
\;\rname{\TirName{Mult}_1}
\and
\inferrule
{e_2 \rightarrow e'_2}
{e_1 \times e_2 \rightarrow e_1 \times e'_2}
\;\rname{\TirName{Mult}_2}
\and
\inferrule
{e_1 \rightarrow e'_1}
{e_1 + e_2 \rightarrow e'_1 + e_2}
\;\rname{\TirName{Add}_1}
\and
\inferrule
{e_2 \rightarrow e'_2}
{e_1 + e_2 \rightarrow e_1 + e'_2}
\;\rname{\TirName{Add}_2}
\and
\inferrule
{e_1 \rightarrow e'_1}
{e_1 - e_2 \rightarrow e'_1 - e_2}
\;\rname{\TirName{Sub}_1}
\and
\inferrule
{e_2 \rightarrow e'_2}
{e_1 - e_2 \rightarrow e_1 - e'_2}
\;\rname{\TirName{Sub}_2}
\and
\inferrule
{e_1 \rightarrow e'_1}
{e_1 / e_2 \rightarrow e'_1 / e_2}
\;\rname{\TirName{Div}_1}
\and
\inferrule
{e_2 \rightarrow e'_2}
{e_1 / e_2 \rightarrow e_1 / e'_2}
\;\rname{\TirName{Div}_2}
\end{mathpar}
We also assume that we have an infinity of rules for multiplying,
adding, subtracting and dividing (integer division) numbers.
\begin{enumerate}

  \item Is this system deterministic?

  \item Reduce the following expressions in all the possible ways and
    give at each rewrite step the corresponding substitution:
  \begin{enumerate}

    \item \((1 + 2) \times (3 + (4 \times 5))\)

    \item \(((1 + 2) / 0) \times (3 + 4)\)

    \item \((1 + (2 \times 3)) / ((4 + 5) - 9)\)

  \end{enumerate}

\end{enumerate}
Assume now the following variant and answer the previous questions again.
\begin{mathpar}
\inferrule
{e_1 \rightarrow e'_1}
{e_1 \times e_2 \rightarrow e'_1 \times e_2}
\;\rname{\TirName{Mult}_1}
\and
\inferrule
{e \rightarrow e'}
{v \times e \rightarrow v \times e'}
\;\rname{\TirName{Mult}_2}
\and
\inferrule
{e \rightarrow e'}
{e + v \rightarrow e' + v}
\;\rname{\TirName{Add}_1}
\and
\inferrule
{e_2 \rightarrow e'_2}
{e_1 + e_2 \rightarrow e_1 + e'_2}
\;\rname{\TirName{Add}_2}
\and
\inferrule
{e_1 \rightarrow e'_1}
{e_1 - e_2 \rightarrow e'_1 - e_2}
\;\rname{\TirName{Sub}_1}
\and
\inferrule
{e \rightarrow e'}
{v - e \rightarrow v - e'}
\;\rname{\TirName{Sub}_2}
\\
\inferrule
{e \rightarrow e'}
{e / v \rightarrow e' / v}
\;\rname{\TirName{Div}_1}
\and
\inferrule
{e_2 \rightarrow e'_2}
{e_1 / e_2 \rightarrow e_1 / e'_2}
\;\rname{\TirName{Div}_2}
\end{mathpar}
Assume now that we have a different rule \(\rname{\TirName{Div}_1}\)
and new rules for handling the division by zero:
\begin{mathpar}
\inferrule
{v \neq 0\\ e \rightarrow e'}
{e / v \rightarrow e' / v}
\;\rname{\TirName{Div}_1}
\and
\inferrule
{}
{e / 0 \rightarrow \NaN}
\quad\rname{\TirName{DivZero}}
\and
\inferrule
{}
{\NaN \times e \rightarrow \NaN}
\quad\rname{\TirName{Mult-Err}_1}
\and
\inferrule
{}
{e \times \NaN \rightarrow \NaN}
\quad\rname{\TirName{Mult-Err}_2}
\and
\inferrule
{}
{\NaN + e \rightarrow \NaN}
\quad\rname{\TirName{Add-Err}_1}
\and
\inferrule
{}
{e + \NaN \rightarrow \NaN}
\quad\rname{\TirName{Add-Err}_2}
\and
\inferrule
{}
{\NaN - e \rightarrow \NaN}
\quad\rname{\TirName{Sub-Err}_1}
\and
\inferrule
{}
{e - \NaN \rightarrow \NaN}
\quad\rname{\TirName{Sub-Err}_2}
\and
\inferrule
{}
{\NaN / e \rightarrow \NaN}
\quad\rname{\TirName{Div-Err}_1}
\and
\inferrule
{}
{e / \NaN \rightarrow \NaN}
\quad\rname{\TirName{Div-Err}_2}
\end{mathpar}
\noindent Answer the same questions again.

Is it better to have the following rule?
\begin{mathpar}
\inferrule
{e_1 \rightarrow e'_1}
{e_1 / e_2 \rightarrow e'_1 / e_2}
\;\rname{\TirName{Div}_1}
\end{mathpar}


\section{Boolean expressions}

Consider the system of inference rules
\begin{mathpar}
\inferrule
{}
{\true \wedge e \rightarrow e}
\quad\rname{\wedge_{\TirName{True}}}
\and
\inferrule
{}
{\false \wedge e \rightarrow \false}
\quad\rname{\wedge_{\TirName{False}}}\\
\inferrule
{e_1 \rightarrow e'_1}
{e_1 \wedge\, e_2 \rightarrow e'_1 \wedge\, e_2}
\;\rname{\wedge}
\and
\inferrule
{}
{\neg\false \rightarrow \true}
\quad\rname{\TirName{Not-False}}
\and
\inferrule
{}
{\neg\true \rightarrow \false}
\quad\rname{\TirName{Not-True}}
\and
\inferrule
{e \rightarrow e'}
{\neg e \rightarrow \neg e'}
\;\rname{\TirName{Not}}
\and
\inferrule
{}
{e_1 \vee\, e_2 \rightarrow \neg(\neg e_1 \wedge\, \neg e_2)}
\quad\rname{\TirName{Or}}
\end{mathpar}

\begin{enumerate}

  \item Is this system deterministic?

  \item Reduce the following expressions in several ways if possible,
    and give at each rewrite step the corresponding substitution:
  \begin{enumerate}
 
    \item \((\true \wedge (\false \vee \true)) \wedge \neg(\true \vee
      (\true \wedge \true))\)

    \item \(\neg((\true \wedge \true) \vee \neg(\false \vee \neg\true))\)

  \end{enumerate}

\end{enumerate}


\section{Arithmetic}

Let us model the integers. The number \(0\) is noted \proc{Zero}. If
an integer is noted \(n\), then \(\proc{Succ} (n)\) denotes the next
integer and \(\proc{Pred} (n)\) the previous. For example
\[
\begin{array}{c|c}
\text{New notation} & \text{Mathematical notation}\\
\hline
\proc{Zero} & 0\\
\proc{Succ} (\proc{Zero}) & 1\\
\proc{Succ} (\proc{Succ} (\proc{Zero})) & 2\\
\ldots & \ldots\\
\proc{Pred} (\proc{Zero}) & -1\\
\proc{Pred} (\proc{Pred} (\proc{Zero})) & -2\\
\ldots & \ldots
\end{array}
\]
Let us define now a function \proc{IsZero} which returns the boolean
\true if the argument is \proc{Zero} and \false if the argument is not
\proc{Zero}:
\begin{align*}
\proc{IsZero} (\proc{Zero}) & \rightarrow \true\\
\proc{IsZero} (\proc{Succ} (n)) & \rightarrow \false\\
\proc{IsZero} (\proc{Pred} (n)) & \rightarrow \false
\end{align*}
But this definition is broken, because it implies, for instance
\[
\proc{IsZero} (\proc{Pred} (\proc{Succ} (\proc{Zero}))) \rightarrow
\false
\]
What solution do you propose to fix it? Answer:
\begin{align*}
\proc{Succ} (\proc{Pred} (n)) & \rightarrow n\\
\proc{Pred} (\proc{Succ} (n)) & \rightarrow n
\end{align*}


\section{Stacks}

\begin{mathpar}
\inferrule
{}
{\proc{IsEmpty} (\proc{Push} (x, s)) \rightarrow \false}
\quad\rname{\TirName{IsNotEmpty}}
\and
\inferrule
{}
{\proc{IsEmpty} (\proc{Empty}) \rightarrow \true}
\quad\rname{\TirName{IsEmpty}}
\\
\inferrule
{}
{\proc{Length} (\proc{Empty}) \rightarrow 0}
\quad\rname{\TirName{ZeroLength}}
\and
\inferrule
{}
{\proc{Length} (\proc{Push} (x, s)) \rightarrow 1 + \proc{Length} (s)}
\quad\rname{\TirName{Length}}
\\
\inferrule
{}
{\proc{Rev} (\proc{Empty}) \rightarrow \proc{Empty}}
\quad\rname{\TirName{Rev-E}}
\and
\inferrule
{\proc{Rev} (\proc{Push} (x, s)) \rightarrow \proc{Append}
  (\proc{Rev} (s),\proc{Push} (x,\proc{Empty}))}
{}
\quad\rname{\TirName{Rev-P}}
\end{mathpar}

\end{document}
