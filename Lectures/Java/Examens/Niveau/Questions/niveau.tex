%%-*-latex-*-

\documentclass[a4paper]{article}

\usepackage[francais]{babel}
\usepackage[T1]{fontenc}
\usepackage[latin1]{inputenc}
\usepackage{amssymb}

\input{trace}

\newcommand{\T}{\textrm{T}}
\newtheorem{Def}{Definition}[section]

\title{Test de niveau en algorithmique}
\author{Christian Rinderknecht}
\date{Mardi 4 mars 2003}

\begin{document}

\maketitle

\section{Bin�me de Pascal}

\noindent Soit la fonction \verb+binome+ sp�cifi�e ainsi:

\begin{verbatim}
binome (n,p) {
  if p = 0 or p = n then return 1
  else return binome (n-1,p) + binome (n-1,p-1);
}
\end{verbatim}
\paragraph{Question 1.} Give the lower and upper bounds of a number
represented in balanced ternary with \(n\) trits.



\section{Complexit� et notations de Landau}

\noindent\textbf{Question.} Define a relation
    \texttt{catenate(U,V,W)} to be true when list \texttt{W} is made
    of list \texttt{U} followed by list \texttt{V}. For example
{\small 
\begin{verbatim}
?- catenate(U,[3,4],[0,1,2,3,4]).
U = [0,1,2] ;
No
\end{verbatim}
}


\section{Tris}

\paragraph{Question.} Define the meaning of the pointers
\(\upharpoonleft\), \(\upharpoonright\) and \(\Uparrow\) presented in
class and show how the input is analysed using the transition diagrams
of the previous questions.



\section{Arbres}

  \subsection{Arbres binaires}

    \begin{enumerate}

    \item D�finissez les arbres binaires.

    \item Qu'est-ce que le parcours en profondeur et en largeur?

    \item Parcours pr�fixe, postfixe et infixe?

    \item \textbf{D�nombrements sur les arbres binaires.} �tablir:
      \begin{enumerate}

      \item Un arbre binaire � \(n\) n{\oe}uds poss�de \(n+1\) branches
        vides.

      \item Dans un arbre binaire � \(n\) n{\oe}uds, le nombre de
        n{\oe}uds sans fils est inf�rieur ou �gal � \((n+1)/2\). Il y
        a �galit� si et seulement si tous les n{\oe}uds ont z�ro ou
        deux fils.

      \item La hauteur d'un arbre binaire non vide � \(n\) n{\oe}uds
        est comprise entre \(\lfloor\log_2{n}\rfloor\) et \(n-1\).

      \end{enumerate}

    \item Qu'est-ce qu'un arbre binaire �quilibr� en hauteur?

  \end{enumerate}

        
  \subsection{Arbres binaires de recherche}

  Let the word \texttt{abacabac} and the text
\texttt{babacacabacaab}. Show the different positions of the word with
respect to the text during a run of the Morris\hyp{}Pratt algorithm
(show the failure comparisons by underlining the pairs of different
letters).


\section{Tables de hachage}

  \begin{enumerate}
 
    \item Qu'est-ce qu'une table de hachage? Dans quelles
      circonstances sont-elles une structure de donn�e utile?

    \item Pourquoi recommande-t-on souvent de choisir un nombre
      premier pour la taille d'une table de hachage?

    \item Soient un ensemble \(E\) de \(n\) �l�ments et une fonction
      \(h: E \rightarrow [1..m]\) uniforme (c-�-d. \(\forall e \in
      E.\forall i \in [1..m].{\cal P} \{h(e)=i\}=1/m\)). Montrez que
      la probabilit� \(P\) que \(h\) soit injective vaut \(m!/(m-n)!
      m^{n}\). En particulier, si \(m=356\) et \(n=23\), alors \(P <
      1/2\). Parieriez-vous que deux �l�ves dans cette classe soient
      n�es le m�me jour du m�me mois? Qu'en conclure � propos des
      tables de hachage?
 
    \item Expliquez les m�thodes de r�solution des collisions par
      cha�nage interne et externe (\emph{hachage indirect}), et par
      calcul (\emph{hachage direct}).

  \end{enumerate}


\section{Indexation}

On d�sire �crire un programme qui, � la lecture d'un programme, rep�re
pour chaque identificateur les lignes o� il appara�t. Un
identificateur commence par une lettre et n'est compos� que de lettres
et de chiffres. Certains mots, dits r�serv�s, ne peuvent pas �tre
utilis�s comme identificateurs.

Apr�s la lecture, on demande d'imprimer le programme, puis la liste
(tri�e par ordre alphab�tique) des identificateurs suivis chacun de la
liste des num�ros de ligne o� il appara�t, ces num�ros de ligne �tant
rang�s en ordre croissant.

  \begin{enumerate}

    \item Faites une analyse compar�e des diff�rentes structures de
          donn�es possibles;

    \item programmez une m�thode utilisant des arbres binaires de
          recherche, et une m�thode utilisant un tableau de hachage;
          analysez les r�sultats.

  \end{enumerate}


\section{Graphes}

  \begin{enumerate}

    \item Qu'est-ce qu'un graphe non-orient�? Orient�?

    \item Qu'est-ce qu'un graphe connexe? Fortement connexe?

    \item Qu'est-ce qu'un parcours en profondeur d'abord? En largeur
          d'abord?

    \item Donnez un algorithme de calcul des composantes fortement
          connexes, et sa complexit� temporelle et spatiale dans le
          pire des cas.

    \item Citez des cas o� les graphes sont la structure de donn�e
          idoine.

  \end{enumerate}


\section{Automates et langages r�guliers}

  \begin{enumerate}

    \item Qu'est-ce qu'un automate fini?

    \item Qu'est-ce qu'un automate fini d�terministe et
          non-d�terministe? Quels sont les liens entre ces deux types
          d'automates?

    \item Qu'est-ce qu'un automate minimal? Y a-t-il unicit�? Donnez
          un algorithme de minimisation et discutez sa complexit�.

    \item Qu'est-ce qu'un langage r�gulier? Une expression r�guli�re?
 
    \item Quels liens y a-t-il entre les langages r�guliers et les
          automates finis?

    \item Citez des applications int�ressantes des automates finis et
          des expressions r�guli�res.

  \end{enumerate}



\end{document}
