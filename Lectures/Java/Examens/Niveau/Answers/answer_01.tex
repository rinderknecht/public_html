\paragraph{R�ponses.}

\begin{enumerate}

  \item Soit \(\T (n,p)\) le nombre cherch�. On a, en suivant la
    syntaxe de la d�finition de \texttt{binome}:
    \[\left\{
    \begin{array}{l}
      \T (n,0) = \T (n,n) = 1\\
      \T (n,p) = 1 + \T (n-1,p) + \T (n-1,p-1) \ \textrm{si} \
      0 < p < n
    \end{array}
    \right.\]
    Si l'on pose $\T' (n,p) = \T (n,p) + 1$, alors on obtient:
    \[\left\{
    \begin{array}{l}
      \T' (n,0) = \T' (n,n) = 2\\
      \T' (n,p) = \T' (n-1,p) + \T' (n-1,p-1) \ \textrm{si} \
      0 < p < n
    \end{array}
    \right.\] 
    Il en r�sulte par r�currence que \(\T'(n,p)=2 C_{n}^{p}\), et donc
    \(\T(n,p)=2 C_{n}^{p} - 1\). Ainsi, la fonction \texttt{binome}
    effectue environ deux fois plus de calculs que le r�sultat qu'elle
    fournit!

  \item Oui, il faut trouver une moyen de m�moriser les calculs
    ant�rieurs, un vecteur par exemple, ou alors une autre formule de
    r�currence telle que:
    \[C_{n}^{p} = \frac{n!}{p! (n-p)!} = \frac{n}{p} C_{n-1}^{p-1}\]
    Notez que cette derni�re peut provoquer un d�bordement de capacit�
    lorsque \(n C_{n-1}^{p-1}\) est sup�rieur au plus grand entier
    repr�sentable en machine. De plus, l'expression \texttt{ n *
      (binome(n-1,p-1) / p)} est incorrecte car il est possible que
    \(C_{n-1}^{p-1}\) ne soit pas divisible par \(p\).

\end{enumerate}
