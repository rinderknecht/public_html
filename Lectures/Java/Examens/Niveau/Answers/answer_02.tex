\paragraph{R�ponses.}

\begin{enumerate}

  \item Afin de comparer plusieurs algorithmes r�solvant un m�me
    probl�me, on introduit des mesures de ces algorithmes appel�es
    \emph{complexit�s}:
    \begin{itemize}

       \item la \emph{complexit� temporelle} est le nombre
         d'op�rations, par d�finition �l�mentaires, effectu�es par uen
         machine qui ex�cute l'algorithme;
 
       \item la \emph{complexit� spatiale} est le nombre d'unit�s de
         m�moire utilis�es par une machine qui ex�cute l'algorithme
         (l'unit� est caract�ristique du mod�le de machine).
 
    \end{itemize}
    Ces deux complexit�s d�pendent de la machine employ�e mais aussi
    de la taille des donn�es. Cette derni�re doit donc �tre mod�lis�e
    explicitement.

    \item Soient \(f\) et \(g\) deux fonctions de \(\mathbb{N}\) dans
      \(\mathbb{R}^{+}\).
      \begin{Def}[Majoration asymptotique]
      \begin{math}
         O(g) = \{f : \mathbb{N} \mapsto \mathbb{R}^{+} \mid
         \exists c \in \mathbb{R}^{+*}, \exists n_0 \in \mathbb{N},
         \forall n > n_0, f (n) \leqslant c \, g(n) \}
      \end{math}
      \end{Def}
      Abusivement, on notera \(f = O(g)\) ou \(f(x) = O (g(x))\) au
      lieu de \(f \in O(g)\). De m�me \(O(f) = O(g)\) au lieu de
      \(O(f) \subset O(g)\).
      \begin{Def}[Minoration asymptotique]
        \begin{math}
          f \in \Omega (g) \Leftrightarrow g \in O(f)
        \end{math}
      \end{Def}
      \begin{Def}[Encadrement asymptotique]
        \begin{math}
          \Theta (g) = O(g) \cap \Omega (g)
        \end{math}
      \end{Def}

\end{enumerate}
