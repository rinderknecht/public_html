%%-*-latex-*-

% ------------------------------------------------------------------------
%
\begin{frame}
\frametitle{XHTML}

The \textbf{eXtensible Hyper-Text Markup Language}
(\textbf{\textsf{XHTML}}) is a subset of \XML so that it becomes very
close to \textbf{\textsf{HTML 4.01}}, the language for specifying web
pages.

\bigskip

The user-agents, e.g., web browsers, are encouraged to accept
\XHTML. If this policy becomes widespread, it will enable the
following step of using full \XML.

\bigskip

See the W3C Recommendations for \XHTML and \HTML at
\begin{center}
\url{http://www.w3.org/TR/xhtml1/}\\
\url{http://www.w3.org/TR/html401/}
\end{center}

\end{frame}

% ------------------------------------------------------------------------
%
\begin{frame}[containsverbatim]
\frametitle{XHTML/Global Structure}

All \XHTML file containing English should have the following pattern:
{\small
\begin{semiverbatim}
<?xml version="1.0" encoding="\emph{encoding}"?>
<!DOCTYPE html
    PUBLIC "-//W3C//DTD XHTML 1.0 Strict//EN"
    "http://www.w3.org/TR/xhtml1/DTD/xhtml1-strict.dtd">
<html xmlns="http://www.w3.org/1999/xhtml"
      xml:lang="en" lang="en">
  <head>
    <title>\textit{the title of the window}</title>
  </head>
  <body>
     \emph{...contents and markup...}
  </body>
</html>
\end{semiverbatim}
}

\end{frame}

% ------------------------------------------------------------------------
%
\begin{frame}
\frametitle{XHTML/Headings}

Elements \texttt{h1}, \texttt{h2}, ..., \texttt{h6} allow six kinds of
headers, of decreasing font size. Consider the following document:
\smallXMLin{h.html}

\end{frame}

% ------------------------------------------------------------------------
%
\begin{frame}
\frametitle{XHTML/Text}

\begin{itemize}

  \item The empty element \texttt{<br/>} is interpreted by
    user-agents as a \textbf{line break};

  \item element \texttt{em} marks text to be \textbf{emphasized}
    (e.g., by using an italic font);

  \item element \texttt{strong} marks text to be emphasized stronger
    than with \texttt{em} (e.g., by using a bold font);

  \item element \texttt{p} delimits a \textbf{paragraph}.

\end{itemize}

\end{frame}

% ------------------------------------------------------------------------
%
\begin{frame}
\frametitle{XHTML/Lists}

There are three kinds of lists:
\begin{enumerate}
 
  \item unordered lists;

  \item ordered lists;

  \item lists of definitions.

\end{enumerate}

\end{frame}

% ------------------------------------------------------------------------
%
\begin{frame}
\frametitle{XHTML/Unordered lists}

Unordered lists are the well-known ``bullet lists'', where each
line is displayed after an indentation followed by a bullet, like the
following.
\begin{itemize}

  \item element \texttt{ul} contains an unordered list;

  \item element \texttt{li} (``list item'') contains an item in the
    list.

\end{itemize}

\end{frame}

% ------------------------------------------------------------------------
%
\begin{frame}[containsverbatim]
\frametitle{XHTML/Unordered lists/Example}

Try
\begin{verbatim}
<h3>The ingredients:</h3>
<ul>
  <li>100 g. flour,</li>
  <li>10 g. sugar,</li>
  <li>1 cup of water,</li>
  <li>2 eggs,</li>
  <li>salt and pepper.</li>
</ul>
\end{verbatim}

\end{frame}

% ------------------------------------------------------------------------
%
\begin{frame}
\frametitle{XHTML/Ordered lists}

Ordered lists are lists whose items are introduced by an indentation
followed by a number, in increasing order, like the following.
\begin{enumerate}

  \item element \texttt{ol} contains the ordered list,

  \item element \texttt{li} is the same as in unordered lists.

\end{enumerate}

\end{frame}

% ------------------------------------------------------------------------
%
\begin{frame}[containsverbatim]
\frametitle{XHTML/Ordered lists/Example}

Try
\begin{verbatim}
<h3>Procedure:</h3>
<ol>
  <li>Mix dry ingredients thoroughly;</li>
  <li>Pour in wet ingredients;</li>
  <li>Mix for 10 minutes;</li>
  <li>Bake for one hour at 300 degrees.</li>
</ol>
\end{verbatim}

\end{frame}

% ------------------------------------------------------------------------
%
\begin{frame}
\frametitle{XHTML/Lists of definitions}

A list of definition is a list whose items are introduced by a few
words in a bold font followed by the contents of the item
itself. Consider
\begin{description}

  \item[hacker]\ \\
    A clever programmer.

  \item[nerd]

  \item[geek]\ \\
    A technically bright but socially misfit person.

\end{description}


\end{frame}

% ------------------------------------------------------------------------
%
\begin{frame}
\frametitle{XHTML/Lists of definitions (cont)}

The elements involved are
\begin{itemize}

  \item \texttt{dl} (``definition list''), which contains the whole
    list of definitions;

  \item \texttt{dt} (``definition term''), which contains every term
    to be defined;

  \item \texttt{dd} (``definition description''), which contains every
    definitition of a term.

\end{itemize}

\end{frame}

% ------------------------------------------------------------------------
%
\begin{frame}[containsverbatim]
\frametitle{XHTML/Lists of definitions (cont)}

The example before corresponds to
{\small
\begin{verbatim}
<h3>Excerpt:</h3>
<dl>
  <dt><strong>hacker</strong></dt>
    <dd>A clever programmer.</dd>
  <dt><strong>nerd</strong></dt>
  <dt><strong>geek</strong></dt>
    <dd>A technically bright but socially misfit person.</dd>
</dl>
\end{verbatim}
}

\end{frame}

% ------------------------------------------------------------------------
%
\begin{frame}
\frametitle{XHTML/Tables}

A \textbf{table} is a rectangle divided into smaller rectangles,
called \textbf{cells}, which contain atomic pieces of data.

\bigskip

When read vertically, cells are said to belong to \textbf{columns},
whilst horizontally, they belong to \textbf{rows}.

\bigskip

A row or a column can have a \textbf{header}, i.e., a cell at their
beginning containing a name in bold face.

\bigskip

A table can have a \textbf{caption}, which is a short text describing
the contents of the table and displayed just above it, like a title.

\bigskip

Columns can be divided themselves into sub-columns, when needed.

\end{frame}

% ------------------------------------------------------------------------
%
\begin{frame}
\frametitle{XHTML/Tables (cont)}

\begin{center}
\emph{A test table with merged cells}\\

\begin{tabular}{|c|c|c|c|}
\hline
                 & \multicolumn{2}{|c|}{\textbf{Average}} & \textbf{Red}\\
\cline{2-3}
                 & \textbf{height}    &   \textbf{weight} & \textbf{eyes}\\
\hline
\textbf{Males}   & 1.9                & 0.003             & 40\%\\
\hline
\textbf{Females} & 1.7                & 0.002             & 43\%\\
\hline
\end{tabular}
\end{center}
\textbf{Males} and \textbf{Females} are row headers. The column headers
are \textbf{Average}, \textbf{Red eyes}, \textbf{height} and
\textbf{weight}. The column \textbf{Average} spans two columns; in
other words, it contains two sub-columns, \textbf{height} and
\textbf{weight}. 

The caption reads ``\emph{A test table with merged cells}''.

\end{frame}

% ------------------------------------------------------------------------
%
\begin{frame}
\frametitle{XHTML/Tables (cont)}

\begin{itemize}

  \item Element \texttt{table} contains the table; its attribute
    \texttt{border} specifies the width of the table borders, i.e., of
    the lines separating the cells from the rest.

  \item Element \texttt{caption} contains the caption.

  \item Element \texttt{th} (``table header'') contains a row or
    column header, i.e., the title of the row of column in bold type.

  \item Element \texttt{td} (``table data'') contains the data of a
    cell (if not a header).

  \item Element \texttt{tr} (``table row'') contains a row, i.e., a
    series of \texttt{td} elements, perhaps starting with a
    \texttt{th} element.

\end{itemize}

\end{frame}

% ------------------------------------------------------------------------
%
\begin{frame}[containsverbatim]
\frametitle{XHTML/Tables (cont)}

The corresponding \HTML code is
{\small
\begin{verbatim}
<table border="1">
  <caption><em>A test table with merged cells</em></caption>
  <tr><th rowspan="2"/>
      <th colspan="2">Average</th>
      <th rowspan="2">Red<br/>eyes</th>
  </tr>
  <tr><th>height</th><th>weight</th></tr>
  <tr><th>Males</th><td>1.9</td><td>0.003</td><td>40%</td></tr>
  <tr><th>Females</th>
      <td>1.7</td>
      <td>0.002</td>
      <td>43%</td>
  </tr>
</table>
\end{verbatim}
}

\end{frame}

% ------------------------------------------------------------------------
%
\begin{frame}[containsverbatim]
\frametitle{XHTML/Tables (cont)}

Notice the attributes \texttt{\textbf{rowspan}} and
\textbf{\texttt{colspan}} of the \texttt{th} element.

\bigskip

Attribute \texttt{rowspan} allows to specify how many rows the current
cell spans. For example, the first row, i.e., the one on the
top-left corner, is empty and covers two rows because
\verb|<th rowspan="2"/>|.

\bigskip

Attribute \texttt{colspan} allows to specify how many columns the
current cell spans. For example, the second cell (right next to the
first one) contains the text ``\textbf{Average}'' and covers two
columns because \verb|<th colspan="2">Average</th>|.

\bigskip

Notice the line break \verb|<br/>| in the third cell (first row, last
column) and how \textbf{height} and \textbf{weight} are automatically
at the right place.

\end{frame}

% ------------------------------------------------------------------------
%
\begin{frame}[containsverbatim]
\frametitle{XHTML/Hyperlinks}

Hyperlinks in \XHTML are specificied by the element ``\texttt{a}''
with its mandatory attribute \texttt{href}. 

\bigskip

For example, consider the following hyperlink to the author's web
page:
{\small
\begin{verbatim}
<a href="http://konkuk.ac.kr/~rinderkn/">See my web page.</a>
\end{verbatim}
}

\end{frame}

% ------------------------------------------------------------------------
%
\begin{frame}
\frametitle{XHTML/Validation}

Just like \XML documents, \XHTML documents can be and should be
validated before being published on the web or distributed.

\bigskip

The page \url{http://validator.w3.org/} allows you to do that.

\end{frame}
