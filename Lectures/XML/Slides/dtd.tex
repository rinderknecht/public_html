%%-*-latex-*-

% ------------------------------------------------------------------------
%
\begin{frame}
\frametitle{DTD}

We saw at page~\pageref{xml_intro:DOCTYPE} that the \textbf{Document
  Type Declaration} may contain markup which constrains the \XML
document it belongs to (elements, attributes etc.).

\bigskip

The content of a Document Type Declaration is made of a
\textbf{Document Type Definition}, abbreviated \DTD. So the
former is the container of the latter.

\bigskip

It is possible to have all or part of the \DTD in a separate file,
usually with the extension ``\texttt{.dtd}''.

\end{frame}

% ------------------------------------------------------------------------
%
\begin{frame}[containsverbatim]
\frametitle{DTD/Attribute lists}

We already saw \textbf{attribute lists} at
page~\pageref{xml_intro:ATTLIST} when setting up internal linking.

\bigskip

In general, the \texttt{ATTLIST} can be used to specify any kind of
attributes of an element, not just label and references. 

\bigskip

Consider the attribute declarations for element \texttt{memo}:
{\small
\begin{verbatim}
<!ATTLIST memo
          ident      CDATA          #REQUIRED
          security   (high | low)   "high"
          keyword    NMTOKEN        #IMPLIED>
\end{verbatim}
}
\texttt{CDATA} stands for \textbf{character data} and represents any
string. A \textbf{named token} (\texttt{NMTOKEN}) is a string starting
with a letter and which may contain letters, numbers and certain
punctuation.

\end{frame}

% ------------------------------------------------------------------------
%
\begin{frame}
\frametitle{DTD/Element declarations}

In order for a document to be validated, which requires more
constraints than to be merely well-formed, all the elements used
must be declared in the \DTD.

\bigskip

The name of each element must be associated a \textbf{content model},
i.e., a description of what it is allowed to contain, in terms of
textual data and sub-elements (mark-up).

\bigskip

This is achieved by means of the \DTD \texttt{ELEMENT}
declarations. There are five kinds of content models.

\end{frame}

% ------------------------------------------------------------------------
%
\begin{frame}[containsverbatim]
\frametitle{DTD/Element declarations (cont)}

The first kind is the \textbf{empty element}:
\begin{verbatim}
<!ELEMENT padding EMPTY>
\end{verbatim}
The second is elements with \textbf{no content restriction}:
\begin{verbatim}
<!ELEMENT open ALL>
\end{verbatim}
The third is elements containing \textbf{only text}:
\begin{verbatim}
<!ELEMENT emphasis (#PCDATA)>
\end{verbatim}
which means \textbf{parsed-character data}.

\end{frame}

% ------------------------------------------------------------------------
%
\begin{frame}[containsverbatim]
\frametitle{DTD/Element declarations (cont)}

The fourth is elements containing \textbf{only elements}:
{\small
\begin{verbatim}
<!ELEMENT section (title, para+)>
<!ELEMENT chapter (title, section+)>
<!ELEMENT report (title, subtitle?, (section+ | chapter+))>
\end{verbatim}
}
where \texttt{title}, \texttt{subtitle} and \texttt{para} are
elements.

\bigskip

The last is elements containing \textbf{both text and elements}:
{\small
\begin{verbatim}
<!ELEMENT para (#PCDATA | emphasis | ref)+>
\end{verbatim}
}
where \texttt{emphasis} and \texttt{ref} are element names.

\end{frame}

% ------------------------------------------------------------------------
%
\begin{frame}
\frametitle{DTD/Element declarations (cont)}

The technique to define the content model is similar to
\textbf{regular expressions}. Such an expression can be made up by
combining the following:
\begin{itemize}

  \item \texttt{(\(e_1\), \(e_2\), \dots , \(e_n\))} represents the
    elements represented by \(e_1\), followed by the elements
    represented by \(e_2\) etc. until \(e_n\);

  \item \texttt{\(e_1\) | \(e_2\)} represents the elements represented
    by \(e_1\) or \(e_2\);

  \item \texttt{(\(e\))} represents the elements represented by \(e\);

  \item \texttt{\(e\)?} represents the elements represented by \(e\)
    or none;

  \item \texttt{\(e\)+} represents the non-empty repetition of
    the elements represented by \(e\);

  \item \texttt{\(e\)*} represents the repetition of the elements
    represented by \(e\).

\end{itemize}

\end{frame}

% ------------------------------------------------------------------------
%
\begin{frame}[containsverbatim]
\frametitle{DTD/Element declarations (cont)}

\textbf{Warning.} When mixing text and elements, the only possible
regular expression is either
\begin{verbatim}
(#PCDATA)
\end{verbatim}
or
\begin{verbatim}
(#PCDATA | ...)*
\end{verbatim}

\end{frame}

% ------------------------------------------------------------------------
%
\begin{frame}[containsverbatim]
\frametitle{DTD/Internal and external subsets}

The part of a \DTD which is included in the same file as the \XML
document it applies to is called the \textbf{internal subset}. See
again the example page~\pageref{xml_intro:id_idref}.

\bigskip

The part of a \DTD which is in an independent file (\texttt{.dtd}) is
called the \textbf{external subset}.

\bigskip

If there is no internal subset and everything is in the external
subset we have a declaration like
\begin{verbatim}
<!DOCTYPE some_root_element SYSTEM "some.dtd">
\end{verbatim}

\end{frame}

% ------------------------------------------------------------------------
%
\begin{frame}[containsverbatim]
\frametitle{DTD/Validating parsers}

In order to validate an \XML document, its \DTD must completely
describe the elements and attributes used.

\bigskip

This is not mandatory when well-formedness is required.

\bigskip

Therefore, the example page~\pageref{xml_intro:id_idref} is
well-formed but not valid in the sense above, because the elements
\texttt{map} and \texttt{country} are not declared.

\bigskip

For example: 
\begin{verbatim}
<!ELEMENT map     (country+)>
<!ELEMENT country EMPTY>
\end{verbatim}
would be sufficient to validate the document.

\end{frame}
