% ------------------------------------------------------------------------
%
\begin{slide}
\titre{XSLT / Any node}
 
\raggedslides[0pt]

When there is no node test and predicate between two ``\verb|/|'', this
part of the location path evaluates to any node. Consider
\XSLThref{03list21.xsl}{the transform}
\smallXSLTin{03list21.xsl}
which matches all the \texttt{dish} nodes in the menu and selects all
their text nodes. 

\end{slide}


% ------------------------------------------------------------------------
%
\begin{slide}
\titre{XSLT / Using wildcards}

\raggedslides[0pt]

Wildcards can be used to select nodes when some element names do not
matter.  \XSLThref{03list14.xsl}{Consider} \smallXSLTin{03list14.xsl}
The result is the list of all the dishes of any course, in document
order.

\end{slide}


% ------------------------------------------------------------------------
%
\begin{slide}
\titre{XSLT / Location paths}

\raggedslides[0pt]


There are actually axis but they are not written because they are
implicit. The previous location path is actually a shorthand for
\begin{alltt}
/\textbf{child::}menu/\textbf{child::}*[3]/\textbf{child::}dish[2]/@*[2]
\end{alltt}
The axis are here \verb|child|, which means that we select, at each
step marked by \verb|/|, among the children of the nodes in the
current node set.

\end{slide}

% ------------------------------------------------------------------------
%
\begin{slide}
\titre{XSLT / Location paths (cont'd.)}

\raggedslides[0pt]

Actually, the previous location path is a shorthand for
\begin{alltt}
/child::menu/child::*[3]/child::dish[2]/\textbf{attribute::}*[2]
\end{alltt}
There is another axis which can be useful. 
\XSLThref{03list17_mod.xsl}{Consider}
\footXSLTin{03list17_mod.xsl}

\end{slide}


% ------------------------------------------------------------------------
%
\begin{slide}
\titre{XSLT / Location paths (cont'd.)}

\raggedslides[0pt]

The result of applying \XSLThref{03list17_mod.xsl}{the last transform}
to \XMLhref{menu.xml}{the \XML file}
page~\hyperlink{hyper:menu}{\pageref{menu}} is
\saxonCut{tmp.xml}{menu}{03list17_mod}
{\small\verbatiminput{tmp.xml}}

\end{slide}


% ------------------------------------------------------------------------
%
\begin{slide}
\titre{XSLT / Location paths (cont'd.)}

\raggedslides[0pt]

As you can see in the previous example, it is possible to use location
paths as value of the \texttt{match} attribute of a template. In order
for the template to match a node, this node must be in the set denoted
by the location path. For example, the dishes as main course are not
matched by \verb|/menu/entrees/dish|.

Note also that 
\begin{verbatim}
<xsl:value-of select="text()" />
\end{verbatim}
would have lead to the same result.

Also, the shorthand \verb|"../@title"| instead of
\verb|"parent::*/@title"| was possible.

\end{slide}

% ------------------------------------------------------------------------
%
\begin{slide}
\titre{XSLT / More \XPath axes}

\raggedslides[0pt]

Until now we have seen how to select nodes above the context node (the
\texttt{parent} axis) or just below it (the \texttt{child} and
\texttt{attribute} axes).

It is actually possible to use more kinds of axes to select nodes
which are located anywhere in the document tree.

Perhaps the simplest of these ``multi-level axes'' is the
\texttt{ancestor} axis. A node \(x\) is an ancestor of a node \(y\) is
either \(x\) is the parent of \(y\) or it is the ancestor of the
parent of \(y\). In particular, if \(y\) is the root, it has no
ancestors at all.

\end{slide}


% ------------------------------------------------------------------------
%
\begin{slide}
\titre{XSLT / The \texttt{ancestor} axis}
 
\raggedslides[0pt]

Consider
\XSLThref{03list19_mod.xsl}{the following transform}:
\smallXSLTin{03list19_mod.xsl}

\end{slide}


% ------------------------------------------------------------------------
%
\begin{slide}
\titre{XSLT / The \texttt{ancestor} axis (cont'd.)}
 
\raggedslides[0pt]

Applying \XSLThref{03list19_mod.xsl}{it}
to \XMLhref{menu.xml}{the menu} gives
\saxonCut{tmp.xml}{menu}{03list19_mod}
{\small\verbatiminput{tmp.xml}}

\end{slide}


% ------------------------------------------------------------------------
%
\begin{slide}
\titre{XSLT / All the axes}
 
\raggedslides[0pt]

\begin{center}
\begin{tabular}{ll}
\toprule
Axis & Description\\
\midrule
\texttt{self}       & The context node itself\\
\texttt{child}      & All the child nodes of the context node\\
\texttt{parent}     & The parent node of the context node\\
\texttt{ancestor}   & All ancestor nodes of the context node\\
\texttt{ancestor-or-self} 
                    & The context node followed by the ancestor nodes\\
\texttt{descendant} & All descendant nodes of the context node\\
\texttt{descendant-or-self} 
                    & The context node followed by the descendant nodes\\
\bottomrule
\end{tabular}
\end{center}


\end{slide}


% ------------------------------------------------------------------------
%
\begin{slide}
\titre{XSLT / All the axes (cont'd.)}
 
\raggedslides[0pt]

\begin{center}
\begin{tabular}{ll}
\toprule
Axis & Description\\
\midrule
\texttt{following-sibling} 
                    & The context node and all its following siblings\\ 
\texttt{following}  & All the following nodes\\
\texttt{preceding-sibling}
                    & All preceding siblings and the context node\\
\texttt{preceding}  & All the preceding nodes\\
\bottomrule
\end{tabular}
\end{center}

\end{slide}


% ------------------------------------------------------------------------
%
\begin{slide}
\titre{XSLT / Disjunctive patterns}
 
\raggedslides[0pt]

It is possible to match several tag names at once. Consider 
the transform
\smallXSLTin{04list11.xsl}

\end{slide}


% ------------------------------------------------------------------------
%
\begin{slide}
\titre{XSLT / Disjunctive patterns (cont'd.)}
 
\raggedslides[0pt]

applied to \XMLhref{people.xml}{the document}
\smallXMLin{people.xml}

\end{slide}


% ------------------------------------------------------------------------
%
\begin{slide}
\titre{XSLT / Disjunctive patterns (cont'd.)}
\label{cars1}
\hypertarget{hyper:cars1}{}

\raggedslides[0pt]

Consider the following document about some \XMLhref{cars1.xml}{cars}:
\smallXMLin{cars1.xml}
(It is badly designed but fits our didactic purpose here.)

We want a transform that extracts the attributes for each car and
print them.

\end{slide}


% ------------------------------------------------------------------------
%
\begin{slide}
\titre{XSLT / Disjunctive patterns (cont'd.)}
 
\raggedslides[0pt]

If we try \XSLThref{attr1.xsl}{the transform}
\smallXSLTin{attr1.xsl}

\end{slide}

% ------------------------------------------------------------------------
%
\begin{slide}
\titre{XSLT / Disjunctive patterns (cont'd.)}
 
\raggedslides[0pt]

the result is not what we expected:
\saxonCut{tmp.xml}{cars1}{attr1}
{\small\verbatiminput{tmp.xml}}


\end{slide}


% ------------------------------------------------------------------------
%
\begin{slide}
\titre{XSLT / Disjunctive patterns (cont'd.)}
 
\raggedslides[0pt]

The problem is that only the first attribute is selected. It is not
possible to match an attribute without first matching the
corresponding element name (tag). This is achieved by matching first
the element and then applying recursively the templates to the
attributes of this element. Consider 
\XSLThref{attr2.xsl}{the transform}
\footXSLTin{attr2.xsl}

\end{slide}


% ------------------------------------------------------------------------
%
\begin{slide}
\titre{XSLT / Disjunctive patterns (cont'd.)}
 
\raggedslides[0pt]

Applying \XSLThref{attr2.xsl}{it} to the
\XMLhref{cars1.xml}{car list}, we get what we wanted:
\saxonCut{tmp.xml}{cars1}{attr2}
{\small\verbatiminput{tmp.xml}}

\end{slide}


% ------------------------------------------------------------------------
%
\begin{slide}
\titre{XSLT / Disjunctive patterns (cont'd.)}
\label{cars2}
\hypertarget{hyper:cars2}{}

\raggedslides[0pt]

Consider now \XMLhref{cars2.xml}{another version} of the car list,
without attributes:
\smallXMLin{cars2.xml}

\end{slide}


% ------------------------------------------------------------------------
%
\begin{slide}
\titre{XSLT / Disjunctive patterns (cont'd.)}
 
\raggedslides[0pt]

We want the same output as with the first car list
(page~\hyperlink{hyper:cars1}{\pageref{cars1}}). This achieved by the 
\XSLThref{elem.xsl}{transform}:
\footXSLTin{elem.xsl}

\end{slide}


% ------------------------------------------------------------------------
%
\begin{slide}
\titre{XSLT / Disjunctive patterns (cont'd.)}
 
\raggedslides[0pt]

But what if we want one transform that can produce the same output on
both kinds of car lists? The solution lies in using disjunctive
patterns. Consider \XSLThref{attr_or_elem.xsl}{the transform}:
\footXSLTin{attr_or_elem.xsl}

\end{slide}


% ------------------------------------------------------------------------
%
\begin{slide}
\titre{XSLT / Inserting elements}

\raggedslides[0pt]

Until now, the output of the \XSLT was plain text. Let us produce some
other format instead, like \HTML. The technique is pretty simple:
inside a template, any pair of opening and closing tags will be copied
to the output.  \smallXSLTin{05list05.xsl}

\end{slide}


% ------------------------------------------------------------------------
%
\begin{slide}
\titre{XSLT / Inserting elements (cont'd.)}

\raggedslides[0pt]

Applying \XSLThref{05list05.xsl}{this transform} to the
\XMLhref{cars1.xml}{first car list}
(page~\hyperlink{hyper:cars1}{\pageref{cars1}}) yields
\saxonCut{tmp.html}{cars1}{05list05}
{\small\verbatiminput{tmp.html}}
One can open this \XML file in a web browser because it is also a
valid \HTML document. This allows to display nicely the list of cars.

\end{slide}


% ------------------------------------------------------------------------
%
\begin{slide}
\titre{XSLT / Inserting elements (cont'd.)}

\raggedslides[0pt]

It is possible to use a built-in \XSLT element for achieving the same
result:
{\small
\begin{verbatim}
<xsl:element name="td">Matiz</xsl:element>
\end{verbatim}
}
\noindent instead of
{\small
\begin{verbatim}
<td>Matiz</td>
\end{verbatim}
}

\end{slide}

