\paragraph{Answer.}
\begin{align*}
\proc{Dequeue}(\proc{Queue}(S,\proc{Push}(e,T)))
&\stackrel{1}{\rightarrow} (e,\proc{Queue}(S,T))\\
\proc{Dequeue}(\proc{Queue}(S,\proc{Empty}))
&\stackrel{2}{\rightarrow}
\proc{Dequeue}(\proc{Queue}(\proc{Empty}, \proc{Rev}(S)))\\
\proc{Enqueue}(e,\proc{Queue}(S,T)) 
&\stackrel{3}{\rightarrow}
  \proc{Queue}(\proc{Push}(e,S),T)
\end{align*}
\noindent But there is a problem with \proc{Dequeue} if the queue is
empty. Indeed, in this case we have
\begin{multline*}
\proc{Dequeue}(\proc{Queue}(\proc{Empty}, \proc{Empty}))\\
\stackrel{2}{\rightarrow} \proc{Dequeue}(\proc{Queue}(\proc{Empty},
\proc{Rev}(\proc{Empty})))\\
\rightarrow \proc{Dequeue}(\proc{Queue}(\proc{Empty}, \proc{Empty}))
\end{multline*}
\noindent which is a never-ending loop! The fix consists in forbidding
the use of rule \(2\) when the queue is empty. One easy way is simply
to say
\begin{gather*}
\proc{Dequeue}(\proc{Queue}(S,\proc{Empty})) \stackrel{2}{\rightarrow}
\proc{Dequeue}(\proc{Queue}(\proc{Empty}, \proc{Rev}(S)))\\
\qquad \text{if} \, S \neq \proc{Empty}
\end{gather*}
\noindent Another way is to add an error-case rule like
\[
\proc{Dequeue}(\proc{Queue}(\proc{Empty},\proc{Empty})) 
\stackrel{4}{\rightarrow} \proc{Error}
\]
\noindent \textbf{and} prefer it over rule \(3\).
