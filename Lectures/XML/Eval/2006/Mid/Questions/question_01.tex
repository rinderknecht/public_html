A \emph{queue} is a stack where the items are popped at the
\textbf{bottom}, not at the top. Pushing an item in a queue is called
\emph{enqueuing}, whereas the special pop is called \emph{dequeuing}.
\[
\begin{array}{r@{\;}cc|c|c|c|c|c|cc@{\;}l}
\cline{3-9}
\proc{Enqueue} & \rightarrow & & a & b & c & d & e & & \rightarrow &
\proc{Dequeue}\\
\cline{3-9}\\
\cline{3-8}
\proc{Push}, \proc{Pop} & \leftrightarrow & & a & b & c & d & e &&&\\ 
\cline{3-8}
\end{array}
\]
\noindent We want to implement queues with two stacks:
\[
\begin{array}{r@{\;}cc|c|c|c|c|c|c|cc@{\;}l}
\cline{3-6}\cline{8-10}
\proc{Enqueue} & \rightarrow & & a & b & c & & d & e & & \rightarrow &
\proc{Dequeue}\\
\cline{3-6}\cline{8-10}
\end{array}
\]
\noindent So \proc{Enqueue} is \proc{Push} on the first stack and
\proc{Dequeue} is \proc{Pop} on the second. If the second stack is
empty, we swap the stacks and reverse the (new) second:
\[
\begin{array}{r@{\;}cc|c|c|c|c|cc@{\;}l}
\cline{3-6}\cline{8-8}
\proc{Enqueue} & \rightarrow & & a & b & c & & & \rightarrow &
\proc{Dequeue} \, \textbf{???}\\
\cline{3-6}\cline{8-8}\\
\cline{3-3}\cline{5-8}
\proc{Enqueue} & \rightarrow & & & a & b & c & & \rightarrow &
\proc{Dequeue}\\
\cline{3-3}\cline{5-8}\\
\end{array}
\]
\paragraph{Data.} \noindent Let \(\proc{Queue}(S,T)\) be the
queue made of the stacks \(S\) and \(T\); let \(\proc{Push}(e,S)\) be
the stack \(S\) with item \(e\) on top; let \proc{Empty} be the empty
stack.

\paragraph{Functions.} Let \(\proc{Enqueue}(e,Q)\) rewrite to
the queue \(Q\) where item \(e\) is added; let \(\proc{Dequeue}(Q)\)
rewrite to the pair \((e, Q')\) where \(e\) is the item on the head of
queue \(Q\) and \(Q'\) is the remaining queue.

\paragraph{Question.} Define \(\proc{Enqueue}(e,Q)\) and
\(\proc{Dequeue}(Q)\), where \(Q\) is a queue.
