%%-*-latex-*-

\documentclass[11pt,a4paper]{article}

\usepackage[british]{babel}
\usepackage[T1]{fontenc}
\usepackage[utf8]{inputenc}
\usepackage{graphicx}
\usepackage{pstricks,pst-tree}
\usepackage{multicol}

\input{trace}

\title{Final examination on Logic Circuit Design}

\author{Christian Rinderknecht}
\date{24 October 2013}

\begin{document}

\maketitle

\section{Binary Search Trees}

\noindent Consider the following classes implementing persistent
binary search trees:
{\small
\begin{verbatim}
public class Pair<Fst,Snd> {
  protected final Fst fst;
  protected final Snd snd;
  public Pair (final Fst f, final Snd s) { fst = f; snd = s; }
  public Fst fst () { return fst; }
  public Snd snd () { return snd; }
}

public abstract class BST<Key extends Comparable<Key>> {
  public    abstract boolean isEmpty ();
  protected abstract Pair<Key,BST<Key>> min_aux (final Int<Key> p);
}

public final class Ext<Key extends Comparable<Key>> extends BST<Key>{
  public boolean isEmpty () { return true; }
  protected final Pair<Key,BST<Key>> min_aux (final Int<Key> p) {
    return new Pair<Key,BST<Key>>(p.root,p.right); }
}

public final class Int<Key extends Comparable<Key>>
       extends BST<Key> {
  protected final Key root;
  protected final BST<Key> left, right;
  public Int (final Key i, final BST<Key> l, final BST<Key> r) {
    root = i; left = l; right = r; }
  public boolean isEmpty () { return fakse; }
  public Pair<Key,BST<Key>> min_aux (final Int<Key> p) {
    Pair<Key,BST<Key>> m = left.min_aux(this);
    return new Pair<Key,BST<Key>>(m.fst(),
                          new Int<Key>(p.root,m.snd(),p.right));
  }
  public Pair<Key,BST<Key>> min () { return left.min_aux(this); }
}
\end{verbatim}
}
Extend the classes \texttt{BST}, \texttt{Ext} and \texttt{Int} with a method \texttt{rm} (\emph{remove}) which returns the same tree without a given key and whose signature is
\begin{verbatim}
public BST<Key> rm (final Key k);
\end{verbatim}
Note: you must use the method \texttt{min\_aux}.

\section{Leaf trees}

Using a functional style, design a binary tree whose leaves alone
contain comparable keys, for instance numbers, while the other
internal nodes do not. Such kind of tree is called a \emph{leaf tree}
and an example is shown in Figure~\ref{fig:leaf_tree}. The nodes
without information are simply called \emph{nodes}, whereas the nodes
carrying a comparable key are \emph{leaves}.

Write a method \texttt{sort} which operates on a leaf tree and
produces the nondecreasingly ordered stack of its leaves. The example
in the figure yields the stack \texttt{(1,2,3,4)}, where the
leftmost number is at the top of the stack. The signature is
\begin{verbatim}
public Stack<Key> sort ();
\end{verbatim}
Give the best and worst costs with the corresponding configurations.

Note: you can reuse any of the methods of the class \texttt{Stack} we
have defined during the course and you do not need to define insertion
into the tree.
\begin{figure}[h]
\centering
\includegraphics{tree}
\caption{A leaf tree\label{fig:leaf_tree}}
\end{figure}

\section{Mirroring}

\begin{multicols}{2}
Add a method \texttt{mirror} to the class \texttt{BST}, which takes a
binary search tree and returns the same tree as in a mirror. Consider
the facing example. What does the inorder traverval of the mirror
yields?
\par\vfill\columnbreak
\begin{center}
\includegraphics[bb=71 656 233 721]{mirrors}
\end{center}
\end{multicols}


\end{document}
