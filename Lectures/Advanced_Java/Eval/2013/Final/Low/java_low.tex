%%-*-latex-*-

\documentclass[11pt,a4paper]{article}

\usepackage[british]{babel}
\usepackage[T1]{fontenc}
\usepackage[utf8]{inputenc}

\input{trace}

\title{Answers to the mid-term exam on\\ Introduction to the Internet}

\author{Christian Rinderknecht}
\date{24 October 2013}

\begin{document}

\maketitle

\noindent Consider the following classes implementing ephemeral
stacks:
\begin{verbatim}
public class Stack<Key extends Comparable<Key>> {
    protected Cell<Key> top;
    public Stack () { top = null; }
    public void push (final Key k) {
      top = new Cell<Key>(k,top); }
}

public class Cell<Key extends Comparable<Key>> {
  protected Key value;
  protected Cell<Key> next;
  public Cell (final Key key, final Cell<Key> cell) {
    value = key; next = cell; }
}
\end{verbatim}
Extend the class \texttt{Stack<Key>} with two methods defined as follows.
\begin{enumerate}

  \item The method \texttt{find\_front} whose signature is
\begin{verbatim}
public int find_front (final Key k) throws NotFound;
\end{verbatim}
    performs a linear search in \texttt{this}: if the key~\texttt{k}
    is absent, then the exception \texttt{NotFound} (to be defined) is
    thrown; otherwise, the index of the first cell
    containing~\texttt{k} is returned (by definition, the top of the
    stack has index~\(0\)) and is moved to the top. This heuristics
    improving the efficiency of linear search when it is used often is
    called \emph{move to front}.

  \item The method \texttt{find\_swap} whose signature is
\begin{verbatim}
public int find_swap (final Key k) throws NotFound;
\end{verbatim}
    performs the same operations as \texttt{find\_front}, except that
    the first cell containing~\texttt{k} (if any) is swapped with the
    previous one (if any).
\end{enumerate}

\newpage

\noindent The main class for testing is
\begin{verbatim}
public class EphObj {
  public static void main (String[] args) {
    Stack<Integer> s = new Stack<Integer>();
    s.push(5); s.push(3); s.push(1); s.push(0);
    s.print(); // 0 1 3 5
    try { System.out.println(s.find_front(5)); } // 3
    catch (NotFound x) {}
    s.print(); // 5 0 1 3
    try { System.out.println(s.find_front(5)); } // 0
    catch (NotFound x) {}
    s.print(); // 5 0 1 3
    try { System.out.println(s.find_front(1)); } // 2
    catch (NotFound x) {}
    s.print(); // 1 5 0 3
    try { System.out.println(s.find_front(7)); }
    catch (NotFound x) { System.out.println("Not found."); }

    try { System.out.println(s.find_swap(3)); } // 3
    catch (NotFound x) {}
    s.print(); // 1 5 3 0
    try { System.out.println(s.find_swap(3)); } // 2
    catch (NotFound x) {}
    s.print(); // 1 3 5 0
    try { System.out.println(s.find_swap(1)); } // 0
    catch (NotFound x) {}
    s.print(); // 1 3 5 0
    try { System.out.println(s.find_swap(7)); }
    catch (NotFound x) { System.out.println("Not found."); }

    try { System.out.println(s.find_swap(3)); }
    catch (NotFound x) {}
    s.print(); // 3 1 5 0
  }
}
\end{verbatim}

\end{document}
