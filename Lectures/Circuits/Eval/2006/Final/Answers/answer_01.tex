\paragraph{Answers.}
\begin{enumerate}

  \item The general shape of an \(n\)-bit binary number, written
    \(b_{n-1}b_{n-2}\dots b_{0}\), is, by definition
    \[
    b_{n-1} \times 2^{n-1} + b_{n-2} \times 2^{n-2} + \dots + b_0
    \]
    where each \(b_i\) can be either \(0\) or \(1\).

    Therefore, there is two choices for \(b_{n-1}\), two for
    \(b_{n-2}\) etc. so the total number of choices for \(\{b_{n-1},
    b_{n-2}, \dots, b_0\}\) is \(\underbrace{2 \times 2 \times \dots
      \times 2}_{n \; \text{times}} = 2^n\), which is the number of
    numbers that can be represented by an \(n\)-bit binary number.

    The highest number it can represent, called max\(_n\), is achieved
    when \(b_i = 1\), for all the \(i\), that is:
    \begin{equation}
      \text{max}_{n} = 2^{n-1} + 2^{n-2} + \dots + 2 + 1 \label{max}
    \end{equation}
    If we multiply it by \(2\) each sides, we get 
    \begin{equation}
      2 \times \text{max}_{n} = 2^{n} + 2^{n-1} + \dots + 2^{2} + 2
      \label{double_max}
    \end{equation}
    Forming \((\ref{double_max}) - (\ref{max})\) we get
    \begin{align*}
      2 \times \text{max}_n - \text{max}_n &= 2^{n} - 1\\
      \text{max}_n &= 2^{n} - 1
    \end{align*}
    The smallest number that can be represented is achieved when the
    \(b_i = 0\), for all \(i\), i.e. \(\text{min}_{n} = 0\).

    By the way, we can check that \(\text{max}_n - \text{min}_n + 1 =
    2^n\), as expected.

  \item \label{two_complement} The general shape of an \(n\)-bit,
    \(2\)-complement binary number written \(
    b_{n-1}b_{n-2}\dots{b_0}\) is
    \[
     -b_{n-1} \times 2^{n-1} + b_{n-2} \times 2^{n-2} + \dots + b_0
    \]
    Therefore, the lowest number that can be represented is achieved
    when \(b_{n-1}=1\) and \(b_{n-2}, b_{n-3}, \dots, b_0\) are \(0\):
    \(\text{min}_n=-2^{n-1}\).

    The highest number is achieved when \(b_{n-1}=0\) and \(b_{n-2},
    b_{n-3}, \dots, b_0\) are \(1\): \(\text{max}_n = 2^{n-2} +
    2^{n-3} + \dots + 1 = 2^{n-1} - 1\).

    The total number of numbers that can be represented is thus
    \[
     \text{max}_n - \text{min}_n + 1 = 2^{n-1} - 1 + 2^{n-1} + 1 = 2^{n}
    \]
    This is, of course, the same number as with the (unsigned) binary
    numbers.

  \item When \(0 \leqslant d_i \leqslant 9\), for all \(i\).

  \item Consider the \(n\)-bit binary number general form:
    \[
     b_{n-1} 2^{n-1} + b_{n-2} 2^{n-2} + \dots + b_1 2 + b_0
    \]
    We can group the bits by groups of three, from right to left:
    \[
     \dots + (b_8 2^8 + b_7 2^7 + b_6 2^6) + (b_5 2^5 + b_4 2^4 + b_3 2^3)
     + (b_2 2^2 + b_1 2 + b_0) 
    \]
    We can factorise \(2^3\), \(2^6\) etc. and get
    \[
     \dots + (b_8 2^2 + b_7 2 + b_6) \times 2^6 + (b_5 2^2 + b_4 2 + b_3)
     \times 2^3 + (b_2 2^2 + b_1 2 + b_0)
    \]
    That is, since \(8 = 2^3\) and \(x^{pq} = (x^p)^q\), then \(2^{3q}
    = (2^3)^q = 8^q\) and
    \[
     \dots + (b_8 2^2 + b_7 2 + b_6) \times 8^2 + (b_5 2^2 + b_4 2 + b_3)
     \times 8^1 + (b_2 2^2 + b_1 2 + b_0) \times 8^0
    \]
    Therefore, in order to convert an octal number to its equivalent
    binary representation, we convert separately its digits into
    binary and simply catenate them.

  \item Because there are no carries produced by the \(1\)-bit
    multiplications. When multiplying a number \(N\) by \(M\), from
    right to left, every time a bit of \(M\) is \(1\), then \(N\) is
    the partial result; else it is \(0\). The final addition can imply
    carries, of course, but the first stage is very easy (copy \(N\)
    or write \(0\)).

  \item The general shape of an \(n\)-bit, \(2\)-complement binary
    number written \( b_{n-1}b_{n-2}\dots{b_0}\) is
    \[
     -b_{n-1} \times 2^{n-1} + b_{n-2} \times 2^{n-2} + \dots + b_0
    \]
    If the leftmost bit is \(1\) then the number is negative. Why?
    Assume that in an \(n\)-bit \(2\)-complement number, the leftmost
    bit is \(1\). Then the highest positive value that can be formed
    with the remaining \(n-1\) bits is having only \(1\)s. In other
    words, the \(n\) bits are all \(1\)s. So this number is
    \[
     N = -2^{n-1} + 2^{n-2} + 2^{n-3} + \dots + 2 + 1
    \]
    We proved earlier that
    \[
    2^{n-1} + 2^{n-2} + \dots + 2 + 1 = 2^n - 1
    \]
    So
    \[
    N = -2^{n-1} + (2^{n-1} - 1) = -1 < 0
    \]
    If the leftmost bit is \(0\) then the number is positive. Why?
    Because it is simply of the form
    \[
    0 \times 2^{n-1} + b_{n-2} 2^{n-2} + b_{n-3} 2^{n-3} + \dots + b_1 2 +
    b_0 \geqslant 0
    \]

  \item Yes, it can fail. Consider again
    question~\ref{two_complement}: the numbers that can be coded as an
    \(n\)-bit, \(2\)-complement binary number range from \(-2^{n-1}\)
    to \(2^{n-1}-1\). Therefore the negation of \(-2^{n-1}\) is out of
    range (overflow).

  \item
From the truth table, we build a Karnaugh map. The unspecified values
of \(F\) are represented by \(a\), \(b\), \(c\) and \(d\). The ranges
where a given variable is \(1\) are given around the map.
\begin{center}
\kvnoindex
\karnaughmap{4}{{$F$}:}{{$C$}{$A$}{$D$}{$B$}}{1{$a$}1{$b$}10{$c$}01101{$d$}000}{}
\end{center}
In order to find the simplest sum of products, we can choose values
for \(a\), \(b\), \(c\) and \(d\) that suit best the groupings of
\(1\)'s in the map. There are several possibilities. An optimum
covering is achieved if \(a=b=c=d=1\). The corresponding min-terms
lead to \(F_1\):
\[
F_1 = \overline{A}\,B + \overline{B}\,\overline{C} +
\overline{B}\,\overline{D}
\]
\begin{center}
\karnaughmap{4}{{$F_1$}:}{{$C$}{$A$}{$D$}{$B$}}{1{\textbf{1}}1{\textbf{1}}10{\textbf{1}}01101{\textbf{1}}000}{%
\put(2.5,2){\oval(.8,3.7)}
\put(0,1.5){\oval(1.8,0.8)[r]}
\put(4,1.5){\oval(1.8,0.8)[l]}
}
\end{center}
Other variable assignments with optimum covering are possible:
\begin{align*}
F_2 &= \overline{A}\,B + \overline{A}\,\overline{C} +
\overline{B}\,\overline{D}\\
F_3 &= \overline{A}\,B + \overline{A}\,\overline{D} +
\overline{B}\,\overline{C}
\end{align*}
\begin{center}
\karnaughmap{4}{{$F_2$}:}{{$C$}{$A$}{$D$}{$B$}}{1{\textbf{1}}1{\textbf{1}}10{\textbf{0}}01101{\textbf{1}}000}{%
\put(1,3){\circle{2}}
\put(0,0){\oval(1.9,1.9)[rt]}
\put(0,4){\oval(1.9,1.9)[rb]}
\put(4,0){\oval(1.9,1.9)[lt]}
\put(4,4){\oval(1.9,1.9)[lb]}
\put(1.5,2){\oval(.8,3.7)}
}
\karnaughmap{4}{{$F_3$}:}{{$C$}{$A$}{$D$}{$B$}}{1{\textbf{1}}1{\textbf{1}}10{\textbf{1}}01101{\textbf{0}}000}{%
\put(1.5,2){\oval(.8,3.7)}
\put(0,3){\oval(1.9,1.9)[r]}
\put(1,0){\oval(1.9,1.9)[t]}
\put(1,4){\oval(1.9,1.9)[b]}
\put(4,3){\oval(1.9,1.9)[l]}
}
\end{center}
In order to find the simplest product of sums, we can choose
\emph{different} values for \(a\), \(b\), \(c\) and \(d\), since the
only constraint is given by the truth table and any choice would agree
with it. A different value assignment would yield a different function
\(F\), but this does not matter, only the part of \(F\) which is
constrained by the table.

An optimum covering of \(0\)'s is achieved if we take \(F_1\) again:
\[
F_1 = (\overline{A} + \overline{B}) (B + \overline{C} + \overline{D})
\]
\begin{center}
\karnaughmap{4}{{$F_1$}:}{{$C$}{$A$}{$D$}{$B$}}{1{\textbf{1}}1{\textbf{1}}10{\textbf{1}}01101{\textbf{1}}000}{%
\put(0,3){\oval(1.9,1.9)[r]}
\put(4,3){\oval(1.9,1.9)[l]}
\put(0,0){\oval(1.9,1.9)[rt]}
\put(0,4){\oval(1.9,1.9)[rb]}
\put(4,0){\oval(1.9,1.9)[lt]}
\put(4,4){\oval(1.9,1.9)[lb]}
\put(1.5,2){\oval(.8,3.7)}
}
\end{center}

\end{enumerate}

