\paragraph{Question.} The \emph{negabinary} representation of a number
is similar to the binary representation, except that the base is not
\(2\) but \(-2\). Therefore, the shape of the negabinary
representation with \(n\) bits \(b_{n-1}, b_{n-2}, \dots, b_0\) is
\[
b_{n-1}(-2)^{n-1} + b_{n-2}(-2)^{n-2} + \dots + b_1(-2)^1 + b_0
\]
Devise a \textsf{C} function converting a decimal number, possibly
negative, into its negabinary form: 
{\small
\begin{verbatim}
char* from10toNeg2 (int dec);
\end{verbatim}
}

\paragraph{Hint.} In \textsf{C}, if \texttt{a => 0} and \texttt{b < 0},
then \texttt{a/b <= 0} and \texttt{a\%b >= 0}. Also, if \texttt{a <=
0} and \texttt{b < 0}, then \texttt{a/b => 0} and \texttt{a\%b <=
0}. For instance, \(17 = (-8)(-2) + 1\) and \(-17 = 8(-2) - 1\).

