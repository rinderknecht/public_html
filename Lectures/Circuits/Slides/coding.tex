%%-*-latex-*-

% ------------------------------------------------------------------------
%
\begin{frame}
\frametitle{Decimal numbers}

Writing \(123\) in base \(10\) means \(1 \times 10^2 + 2 \times 10^1 +
3 \times 10^0\). This encoding is called \textbf{positional number in
  base \(10\)} and we write \textbf{\(123_{10}\)} to mean that \(123\)
is a positional number in base \(10\).

\bigskip

More generally, \(d_{n-1}d_{n-2}\dots{d_0}\) represents
\[
d_{n-1} \times 10^{n-1} + d_{n-2} \times 10^{n-2} + \dots + d_{0}
\times 10^{0}
\]
To way to encode \textbf{fractional numbers} is done by extending this
system to negative exponents:
\(d_{n-1}\dots{d_0}.d_{-1}d_{-2}\dots{d_{-m}}\) represents
\[
d_{n-1} \times 10^{n-1} + \dots + d_{0} \times
10^{0} + d_{-1} \times 10^{-1} + d_{-2} \times 10^{-2} + \dots +
d_{-m} \times 10^{-m}
\]

\end{frame}

% ------------------------------------------------------------------------
% 
\begin{frame}
\frametitle{Binary numbers}

Most electronic devices are usually placed in two distinct electric
states, either charged or discharged, low voltage or high voltage
etc. 

\bigskip

This leads to the interest in \textbf{binary numbers}, i.e. positional
numbers in base \(2\). In this case, the two digits are usually \(0\)
and \(1\), and are usually called \textbf{bits}.

\bigskip

For example, \(10110101_2\) is an \(8\)-bit number. A way to get the
encoding of it in base \(10\) consists in using the general formula
for positional numbers and compute using the operations in base
\(10\):
\begin{gather*}
1 \times 2^7 + 0 \times 2^6 + 1 \times 2^5 + 1 \times 2^4 + 0 \times
2^3 + 1 \times 2^2 + 0 \times 2^1 + 1 \times 2^0\\
128 + 32 + 16 + 4 + 1\\
181
\end{gather*}

\end{frame}

% ------------------------------------------------------------------------
%
\begin{frame}
\frametitle{Binary numbers (cont)}

A more convenient way to do the computation is to use the array
\[
\begin{array}{rrrrrrrr}
128 & 64 & 32 & 16 & 8 & 4 & 2 & 1\\
\hline
  1 &  0 &  1 &  1 & 0 & 1 & 0 & 1
\end{array}
\]
and then sum the powers of \(2\) indexed by \(1\)s:
\[
\begin{array}{rrrrrrrr}
\mathbf{128} & 64 & \mathbf{32} & \mathbf{16} & 8 & \mathbf{4} & 2 &
\mathbf{1}\\
\hline
  1 &  0 &  1 &  1 & 0 & 1 & 0 & 1
\end{array}
\]
\[
128 + 32 + 16 + 4 + 1 = 181
\]

\end{frame}

% ------------------------------------------------------------------------
% 
\begin{frame}
\frametitle{Binary numbers (cont)}

The greatest \(8\)-bit binary number is made of all bits \(1\):
\[
\begin{array}{rrrrrrrr}
128 & 64 & 32 & 16 & 8 & 4 & 2 & 1\\
\hline
  1 &  1 &  1 &  1 & 1 & 1 & 1 & 1
\end{array}
\]
That is to say: \(128 + 64 + 32 + 16 + 8 + 4 + 2 + 1 = 255\).

\bigskip

The greatest binary number of \(n\) bits equals
\begin{equation}
\text{max}_{n} = 2^{n-1} + 2^{n-2} + \dots + 2 + 1
\label{max}
\end{equation}
If we multiply it by \(2\), we get 
\begin{equation}
2 \times \text{max}_{n} = 2^{n} + 2^{n-1} + \dots + 2^{2} + 2
\label{double_max}
\end{equation}

\end{frame}

% ------------------------------------------------------------------------
% 
\begin{frame}
\frametitle{Binary numbers (cont)}
\label{max_bin}

Forming \((\ref{double_max}) - (\ref{max})\) we get
\begin{align*}
2 \times \text{max}_n - \text{max}_n &= 2^{n} - 1\\
\text{max}_n &= 2^{n} - 1
\end{align*}
Therefore,
\(
\text{max}_8 = 2^{8} - 1 = 255
\).

\bigskip

This means: we can count \(2^n\) different values with an \(8\)-bit
number, including \(0\).

\bigskip

Given an \(n\)-bit binary number, the bit corresponding to \(2^{n-1}\)
is called \textbf{most significant bit (MSB)} and the bit associated
to \(2^0\) is the \textbf{least significant bit (LSB)}.

\bigskip

For example: the MSB of \(181\) is \(1\) and its LSB is \(1\) also.

\end{frame}

% ------------------------------------------------------------------------
% 
\begin{frame}
\frametitle{From decimal to binary numbers}

Let us consider again the encoding of an \(n\)-bit number
\(d_{n-1}d_{n-2}\dots{d_0}\) in base \(10\):
\begin{align*}
N &= d_{n-1} \times 10^{n-1} + d_{n-2} \times 10^{n-2} + \dots + d_1 \times
10 + d_0\\
  &= 10 \times (d_{n-1} \times 10^{n-2} + d_{n-2} \times 10^{n-3} +
\dots + d_1) + d_0
\end{align*}
This means that \(d_0\) is the remainder of the integer division
\(N/10\), because, by definition, all digits are smaller than the
base (here \(d_i < 10\), for all \(i\)). We can proceed in the same
fashion:
\begin{align*}
N &= 10 \times (10 \times (d_{n-1} \times 10^{n-3} + \dots + d_2) +
d_1) + d_0\\
(N-d_0) / 10 &= 10 \times (d_{n-1} \times 10^{n-3} + \dots + d_2) +
d_1
\end{align*}
which means that \(d_1\) is the remainder of the integer division
\((N - d_0)/10^{2}\).

\end{frame}

% ------------------------------------------------------------------------
% 
\begin{frame}
\frametitle{From decimal to binary numbers (cont)}

\begin{columns}
  \column{0.5\textwidth}
    This leads to a simple procedure to convert a number from its
    decimal representation to its binary one:

    \bigskip

    \begin{enumerate}

      \item divide it by \(2\);

      \item collect the remainder, which is a bit;

      \item start again with the \textbf{quotient} if it is not \(0\),

      \item otherwise end.

    \end{enumerate}

  \column{0.5\textwidth}
    \[
    \renewcommand\arraystretch{0.8}
    \begin{array}{r|ll}
      213 & 1 & \text{LSB}\\
      106 & 0 &\\
      53 & 1 &\\
      26 & 0 &\\
      13 & 1 &\\
      6 & 0 &\\
      3 & 1 &\\
      1 & 1 & \text{MSB}\\
      0 &   &
    \end{array}
    \]
    \[
    213_{10} = 11010101_2
    \]
\end{columns}

\end{frame}

% ------------------------------------------------------------------------
% 
\begin{frame}
\frametitle{From decimal to binary numbers (cont)}

In order to convert a fractional decimal number into a binary number,
we convert the integer part as just described and convert separately
the fractional part as follows.

\bigskip

The fractional part of a decimal number is of the shape:
\begin{align*}
F &= d_{-1} \times 10^{-1} + d_{-2} \times 10^{-2} + \dots + d_{-m}
\times 10^{-m}\\
10 \times F &= d_{-1} + d_{-2} \times 10^{-1} + \dots + d_{-m}
\times 10^{-m+1}\\
10 (10 F - d_{-1}) &= d_{-2} + \dots + d_{-m} \times 10^{-m+2}
\end{align*}
So \(d_{-1}\) is the integer part of the multiplication by \(10\),
i.e., \(d_{-1}=\lfloor 10F \rfloor\), \(d_{-2}\) is the integer part
of the fractional part of \(10 F\) multiplied by \(10\) etc.

\end{frame}

% ------------------------------------------------------------------------
% 
\begin{frame}
\frametitle{From decimal to binary numbers (cont)}

Therefore, let us multiply the fractional part of the decimal number
successively by \(2\) and retain the integer part as the bit, until
the fractional part becomes \(0\).

\bigskip

For example
\[
\renewcommand\arraystretch{0.85}
\setlength\extrarowheight{4pt}
\begin{array}{r}
0.625\\
\times 2\\
\hline
\mathbf{1}.250\\
\times 2\\
\hline
\mathbf{0}.500\\
\times 2\\
\hline
\mathbf{1}.000
\end{array}
\]
\[
0.625_{10} = 0.101_2
\]

\end{frame}

% ------------------------------------------------------------------------
% 
\begin{frame}
\frametitle{Octal and hexadecimal numbers}

Another common base for numbers used in digital systems is \(8\). The
corresponding encoding of numbers are called \textbf{octal
  numbers}. The digits used in octal are \(0, 1, 2, 3, 4, 5, 6, 7\).

\bigskip

Another common base is \(16\); the corresponding representation of
numbers is named \textbf{hexadecimal numbers}. The digits are from
\(0\) to \(9\), followed by A to F because we need more digits than
the base \(10\) allows, i.e., the Arabic numbers. This means that
\(\text{A}_{16} = 10_{10}\) , \(\text{B}_{16} = 11_{10}\) etc. until
\(\text{F}_{16} = 15_{10}\).

\bigskip

For example
\begin{align*}
174_8    &= 1 \times 8^2 + 7 \times 8^1 + 4 \times 8^0 = 124_{10}\\
\text{B}78_{16} &= 11 \times 16^2 + 7 \times 16^1 + 8 \times 16^0 =
2936_{10}
\end{align*}

\end{frame}

% ------------------------------------------------------------------------
% 
\begin{frame}
\frametitle{Octal numbers and binary numbers}

Consider the \(n\)-bit binary number general form:
\[
b_{n-1} 2^{n-1} + b_{n-2} 2^{n-2} + \dots + b_1 2 + b_0
\]
We can group the bits by groups of three, from right to left:
\[
\dots + (b_8 2^8 + b_7 2^7 + b_6 2^6) + (b_5 2^5 + b_4 2^4 + b_3 2^3)
+ (b_2 2^2 + b_1 2 + b_0) 
\]
We can factorise \(2^0\), \(2^3\), \(2^6\) etc. and get
\[
\dots + (b_8 2^2 + b_7 2 + b_6) \times 2^6 + (b_5 2^2 + b_4 2 + b_3)
\times 2^3 + (b_2 2^2 + b_1 2 + b_0)
\]
That is, since \(8 = 2^3\) and \(x^{pq} = (x^p)^q\), then
\(2^{3q} = (2^3)^q = 8^q\) and 
\[
\dots + (b_8 2^2 + b_7 2 + b_6) \times 8^2 + (b_5 2^2 + b_4 2 + b_3)
\times 8^1 + (b_2 2^2 + b_1 2 + b_0) \times 8^0
\]

\end{frame}

% ------------------------------------------------------------------------
% 
\begin{frame}
\frametitle{Octal numbers and binary numbers (cont)}

Therefore, in order to convert an octal number to its equivalent
binary representation, we convert separately its digits into binary
and simply catenate them.

\bigskip

If we want to convert from binary representation to octal
\begin{enumerate}

  \item group the bits three-by-three from right to left, 

  \item convert each group to octal digits,

  \item catenate the partial results.

\end{enumerate}
For example
\begin{align*}
7351_8 &= (7_8)(3_8)(5_8)(1_8) = (111_2)(011_2)(101_2)(001_2)\\
       &= 111011101001_2
\end{align*}

\end{frame}

% ------------------------------------------------------------------------
% 
\begin{frame}
\frametitle{Octal numbers and decimal numbers}

There are two way to convert from decimal to octal:
\begin{enumerate}

  \item the \emph{direct way} consists in dividing successively the
    quotients by \(8\) and saving the remainders;

  \item the \emph{indirect way} consists in converting from decimal to
    binary (by successive divisions by \(2\)) and then from binary to
    octal (easy).

\end{enumerate}
The indirect way is usually easier because less error-prone.

\end{frame}

% ------------------------------------------------------------------------
% 
\begin{frame}
\frametitle{Hexadecimal numbers}

All the previous discussion about octal numbers is meaningful in a
similar way for hexadecimal numbers, except that the bit groupings
have to be four bits long.

\bigskip

For example
\begin{align*}
\text{B}396_{16} &= (\text{B}_{16})(3_{16})(9_{16})(6_{16})\\
                 &= (1011_2)(0011_2)(1001_2)(0110_2)\\
                 &= 1011001110010110_2
\end{align*}

\end{frame}

% ------------------------------------------------------------------------
% 
\begin{frame}
\frametitle{Fractional octal and hexadecimal numbers}

The same rules apply for the fractional part of octal and hexadecimal
numbers.

\bigskip

Consider
\begin{align*}
515.6_8 &= 110\;\; 001\;\, 101.110_2\\
14\text{D}.\text{C}_{16} &= 1\;\, 0100\;\, 1101.1100_2
\end{align*}

\end{frame}

% ------------------------------------------------------------------------
% 
\begin{frame}
\frametitle{Addition}

Consider two decimal numbers
\begin{align*}
N_1 &= \dots + a_2 10^2 + a_1 10 + a_0\\
N_2 &= \dots + b_2 10^2 + b_1 10 + b_0
\end{align*}
and their addition in this way:
\[
N_1 + N_2 = \dots + (a_2 + b_2) 10^2 + (a_1 + b_1) 10 + (a_0 + b_0)
\]
The problem is that it is possible that \(10 \leqslant a_i + b_i\) for
some \(i\). For instance
\begin{align*}
23 + 8 &= (2 \times 10 + 3) + 8 = 2 \times 10 + (3 + 8) = 2
\times 10 + (\mathbf{1} \times 10 + 1)\\
 &= (2 + \mathbf{1}) \times 10 + 1 = 31
\end{align*}

\end{frame}

% ------------------------------------------------------------------------
% 
\begin{frame}
\frametitle{Addition}

In this case, a \textbf{carry}, i.e., an overflowing digit \(1\), has
to be added to the digits of the next power. That is why addition
works from right to left.

\bigskip

\begin{columns}
  \column{0.5\textwidth}
    Another example:
    \[
    \renewcommand\arraystretch{0.9}
    \setlength\extrarowheight{4pt}
    \begin{array}{r@{\;}c@{\;}c@{\;}c@{\;}c}
      &   &   & \mathbf{1} &\\
      & 2 & 3 & 6 & 7\\
      + & 1 & 3 & 2 & 6\\
      \hline
      & 3 & 6 & 9 & 3
    \end{array}
    \]
  \column{0.5\textwidth}
    The same phenomenon happens with binary, octal or hexadecimal
    numbers. With binary:
    \[
    \renewcommand\arraystretch{0.9}
    \begin{array}{c|c|r}
      x & y & x + y\\
      \hline
      0 & 0 & 0\\
      0 & 1 & 1\\
      1 & 0 & 1\\
      1 & 1 & \mathbf{1}0
    \end{array}
    \]
\end{columns}

\end{frame}

% ------------------------------------------------------------------------
% 
\begin{frame}
\frametitle{Subtraction}

During addition, the addition of two digits may cause an
\textbf{overflow}, i.e., the result is greater or equal than the base.

\bigskip

The risk of subtraction is to have an \textbf{underflow} while
subtracting two digits, i.e., the result may be lower than \(0\). In
this case, we ``borrow'' a carry, i.e., a digit of \(1\) to the next
power (subtraction works from right to left, as addition).

\bigskip

For example, consider
\[
\renewcommand\arraystretch{0.85}
\setlength\extrarowheight{4pt}
\begin{array}{r@{\;}c@{\;}c@{\;}c}
  & \mathbf{3} &   &\\
  & 4 & 7 & 2\\
- & 3 & 9 & 1\\
\hline
  &   & 8 & 1
\end{array}
\]

\end{frame}

% ------------------------------------------------------------------------
% 
\begin{frame}
\frametitle{Subtraction on binary numbers}

The principle is the same for binary numbers. The bit subtractions are
\[
\renewcommand\arraystretch{0.85}
\setlength\extrarowheight{4pt}
\begin{array}{r@{\,}rrr@{}rrr@{\,}rrr@{\,}r}
  & 0 & &   & \mathbf{1}0 & &   & 1 & &   & 1\\
- & 0 & & - & 1 & & - & 0 & & - & 1\\
\cline{1-2}\cline{4-5}\cline{7-8}\cline{10-11}
  & 0 & &    & 1 & &   & 1 & &   & 0
\end{array}
\]
where \(\mathbf{1}\) is the \textbf{borrow}.
\[
\renewcommand\arraystretch{0.85}
\setlength\extrarowheight{4pt}
\begin{array}{r@{\;}c@{\;}c@{\;}c@{\;}c@{\;}c@{\;}c@{\;}c@{\;}c}
  &   &   &   &   & \mathbf{0} & \mathbf{1} &   &\\
  & 1 & 0 & 1 & 1 & 1 & 0 & 0 & 1\\
- & 0 & 0 & 1 & 1 & 0 & 0 & 1 & 1\\
\hline
  & 1 & 0 & 0 & 0 & 0 & 1 & 1 & 0
\end{array}
\]
Here we had to borrow two times, until a \(1\) is found.

\end{frame}

% ------------------------------------------------------------------------
% 
\begin{frame}
\frametitle{Multiplication}

Consider the familiar decimal multiplication on an example:
\[
\renewcommand\arraystretch{0.85}
\setlength\extrarowheight{4pt}
\begin{array}{c@{\;}c@{\;}c@{\;}c@{\;}c@{\;}c}
  &  &        & 1 & 4 & 7\\
  &  & \times & 2 & 6 & 5\\
\cline{3-6}
  &  &        & 7 & 3 & 5\\
+ &  &      8 & 8 & 2 &  \\
+ & 2 &      9 & 4 &   &  \\
\hline
& 3 &      8 & 9 & 5 & 5
\end{array}
\]
Notice that when two digits are multiplied, a carry can be produced,
which can be \(8\) at most (from \(9 \times 9 = \mathbf{8} \times 10 +
1\)).

\end{frame}

% ------------------------------------------------------------------------
% 
\begin{frame}
\frametitle{Multiplication (cont)}

Instead, the multiplication of bits never generates a carry. The rules
are simply:
\[
\renewcommand\arraystretch{0.85}
\setlength\extrarowheight{4pt}
\begin{array}{r@{\,}rrr@{}rrr@{\,}rrr@{\,}r}
       & 0 & &        & 0 & &        & 1 & &        & 1\\
\times & 0 & & \times & 1 & & \times & 0 & & \times & 1\\
\cline{1-2}\cline{4-5}\cline{7-8}\cline{10-11}
       & 0 & &        & 0 & &        & 0 & &        & 1
\end{array}
\]
For example
\[
\renewcommand\arraystretch{0.85}
\setlength\extrarowheight{4pt}
\begin{array}{c@{\;}c@{\;}c@{\;}c@{\;}c@{\;}c@{\;}c}
  &   &        & 1 & 0 & 1 & 1\\
  &   & \times & 1 & 0 & 0 & 1\\
\cline{3-7}
  &   &        & 1 & 0 & 1 & 1\\
  &   &      0 & 0 & 0 & 0 &  \\
  & 0 &      0 & 0 & 0 &   &  \\
1 & 0 &      1 & 1 &   &   &  \\
\hline
1 & 1 &      0 & 0 & 0 & 1 & 1
\end{array}
\]

\end{frame}

% ------------------------------------------------------------------------
% 
\begin{frame}
\frametitle{\(2\)-complement binary}
\label{two_complement}

Until now, we said nothing about what happens if we subtract a number
from a number which is strictly smaller. In practice, we found ourselves
with a borrow from nowhere. Or what happens if we add two numbers and
we get a carry we cannot use.

\bigskip

What do we do if we have to compute, for example \(3 - 7\) in decimal?
In this case, we use an unary negative operator and write \(-4\), but
we cannot do the same with binary numbers because the purpose of
binary numbers is to be handled at the hardware level, where there is
no minus sign.

\bigskip

What we need is an encoding of binary numbers that copes both with
positive and negative numbers in a uniform way, i.e., the negativity
or positivity is coded in the bits of the number
\emph{themselves}. One of these encodings is called
\textbf{\(2\)-complement binary numbers}.

\end{frame}

% ------------------------------------------------------------------------
% 
\begin{frame}
\frametitle{\(2\)-complement binary (cont)}

The idea consists in interpreting the leftmost bit as a negative
positional value, that is to say
\[
N = b_{n-1}b_{n-2}\dots{b_0}
\]
means
\[
N = -b_{n-1} \times 2^{n-1} + b_{n-2} \times 2^{n-2} + \dots + b_0
\]
Thus, in an \(8\)-bit \(2\)-complement binary number, the leftmost bit
has a positional value of \(-128\) rather than \(+128\) (it is
\textbf{not} the MSB). For example, \(N = 01110101\) is interpreted as
\[
\begin{array}{rrrrrrrr}
-128 & \mathbf{64} & \mathbf{32} & \mathbf{16} & 8 & \mathbf{4} & 2 &
\mathbf{1}\\
\hline
  0 &  1 &  1 &  1 & 0 & 1 & 0 & 1
\end{array}
\]
\[
64 + 32 + 16 + 4 + 1 = 117
\]

\end{frame}

% ------------------------------------------------------------------------
% 
\begin{frame}
\frametitle{\(2\)-complement binary (cont)}

Another example:
\[
\begin{array}{rrrrrrrr}
\mathbf{-128} & 64 & \mathbf{32} & \mathbf{16} & 8 & \mathbf{4} & 2 & 1\\
\hline
  1 &  0 &  1 &  1 & 0 & 1 & 0 & 0
\end{array}
\]
\[
-128 + 32 + 16 + 4 = -76
\]
In other words, the interpretation of \(2\)-complement numbers is the
usual one, except that the leftmost bit has a negative positional
value.

\bigskip

\textbf{If the leftmost bit is \(\mathbf{1}\) then the number is
  negative.} Why?

\bigskip

Assume that in an \(n\)-bit \(2\)-complement
number, the leftmost bit is \(1\). Then the highest positive value
that can be formed with the remaining \(n-1\) bits is having only
\(1\)s. In other words, the \(n\) bits are all \(1\)s.

\end{frame}

% ------------------------------------------------------------------------
% 
\begin{frame}
\frametitle{\(2\)-complement binary (cont)}

So this number is
\[
N = -2^{n-1} + 2^{n-2} + 2^{n-3} + \dots + 2 + 1
\]
We showed page~\pageref{max_bin} that
\[
2^{n-1} + 2^{n-2} + \dots + 2 + 1 = 2^n - 1
\]
So
\[
N = -2^{n-1} + (2^{n-1} - 1) = -1 < 0
\]
\textbf{If the leftmost bit is \(\mathbf{0}\) then the number is
  positive.}  Why? Because it is simply of the form
\[
0 \times 2^{n-1} + b_{n-2} 2^{n-2} + b_{n-3} 2^{n-3} + \dots + b_1 2 +
b_0 \geqslant 0
\]

\end{frame}

% ------------------------------------------------------------------------
% 
\begin{frame}
\frametitle{\(2\)-complement binary (cont)}

Because the leftmost bit of a \(2\)-complement number tells the sign,
it is sometimes called the \emph{sign bit}, but this is incorrect
because, in \(n\)-bit numbers, the leftmost bit \(b_{n-1}\) is not used
to encode the sign itself, but \(-b_{n-1} \times 2^{n-1}\).

\bigskip

When interpreting a \(2\)-complement number, it is necessary to say
how many bits are involved in order to determine the bit with negative
positional value.

\bigskip

With an \(8\)-bit \(2\)-complement binary number, the lowest value
that can be represented is \(10000000_2\), i.e., \(-128_{10}\), and
the highest is \(01111111_{2}\), i.e., \(+127_{10}\).

\bigskip

With \(n\)-bits, it ranges from \(-2^{n-1}\) to \(2^{n-1} - 1\):
\textbf{the interval is asymmetric}, since \(2^{n-1}\) is out of
range.

\end{frame}

% ------------------------------------------------------------------------
% 
\begin{frame}
\frametitle{\(2\)-complement binary/Addition}

The addition of two \(2\)-complement numbers is carried out as usual,
without regard for the sign of each one.

\bigskip

Of course, there can be an overflow (i.e., the result is too big to
fit the bit length) or an underflow (i.e., the result is too low to
fit the bit length). For example
\[
\renewcommand\arraystretch{0.85}
\setlength\extrarowheight{4pt}
\begin{array}{r@{\;}c@{\;}c@{\;}c@{\;}c@{\;}c@{\;}c@{\;}c@{\;}c}
  & \mathbf{1} &   &   &   &   &   &   &\\
  & 0 & 1 & 0 & 1 & 1 & 0 & 1 & 1\\
+ & 0 & 1 & 0 & 0 & 0 & 1 & 0 & 0\\
\hline
  & 1 & 0 & 0 & 1 & 1 & 1 & 1 & 1
\end{array}
\]
is wrong because we added two positive numbers and the result is
negative.

\end{frame}

% ------------------------------------------------------------------------
% 
\begin{frame}
\frametitle{\(2\)-complement binary/Addition (cont)}

An overflow can occur if the two numbers are positive and an underflow
can occur if the two numbers are negative. But why and when are we
sure that a problem occurred during addition? Let us consider all the
cases with \(n\) bits. In the following, X, Y and Z are bits at
position \(n-2\).
\[
\renewcommand\arraystretch{0.85}
\setlength\extrarowheight{4pt}
\begin{array}{r@{\;}c@{\;}c@{\;}ccr@{\;}c@{\;}c@{\;}c}
  &   & ?        &       & &   & \mathbf{1} & ?        &\\
  & 0 & \text{X} & \dots & &   & 0          & \text{X} & \ldots\\
+ & 0 & \text{Y} & \dots & & + & 0          & \text{Y} & \ldots\\
\cline{1-4}\cline{6-9}
  & 0 & \text{Z} & \dots & &   & 1          & \text{Z} & \ldots
\end{array}
\]
In the first case, we add two positive numbers with no carry out of
the position \(n-2\): there is no overflow. It is \textbf{correct}.

In the second case, there is a carry out of the position \(n-2\):
there is an overflow because the result is interpreted as a negative
number. It is \textbf{incorrect}.

\end{frame}

% ------------------------------------------------------------------------
% 
\begin{frame}
\frametitle{\(2\)-complement binary/Addition (cont)}

\[
\renewcommand\arraystretch{0.85}
\setlength\extrarowheight{4pt}
\begin{array}{r@{\;}c@{\;}c@{\;}ccc@{\;}r@{\;}r@{\;}c@{\;}c}
  &   & ?        &       & &   & \mathbf{1} & \mathbf{1} & ?        &\\
  & 1 & \text{X} & \dots & &   &            & 1          & \text{X} & \ldots\\
+ & 0 & \text{Y} & \dots & & + &            & 0          & \text{Y} & \ldots\\
\cline{1-4}\cline{6-10}
  & 1 & \text{Z} & \dots & &   &            & 0          & \text{Z} & \ldots
\end{array}
\]
In the first case, we add a negative number to an absolutely smaller
positive one with no carry out of the position \(n-2\): it is
\textbf{correct} and the result is negative.

\bigskip

In the second case, there is a carry out of the position \(n-2\),
which represents the addition of \(2^{n-1}\). This is why it cancels
out the bit \(1\) with negative positional value, \(-2^{n-1} + 2^{n-1}
= 0\), so it is \textbf{correct} (and \emph{we ignore the leftmost
  carry}).

\end{frame}

% ------------------------------------------------------------------------
% 
\begin{frame}
\frametitle{\(2\)-complement binary/Addition (cont)}

\[
\renewcommand\arraystretch{0.85}
\setlength\extrarowheight{4pt}
\begin{array}{r@{\;}r@{\;}c@{\;}c@{\;}ccc@{\;}r@{\;}r@{\;}c@{\;}c}
  & \mathbf{1} &   & ?        &       & &   & \mathbf{1} & \mathbf{1} & ?        &\\
  &            & 1 & \text{X} & \dots & &   &            & 1          & \text{X} & \ldots\\
+ &            & 1 & \text{Y} & \dots & & + &            & 1          & \text{Y} & \ldots\\
\cline{1-5}\cline{7-11}
  &            & 0 & \text{Z} & \dots & &   &            & 1          & \text{Z} & \ldots
\end{array}
\]
In the first case, there is no carry out of the position \(n-2\) and
the result is positive, \((-2^{n-1})+(-2^{n-1}) = -2^n\), instead of
negative, so it is \textbf{incorrect} (underflow with \(n\) bits).

\bigskip

In the second case, there are two carries and the result is
\textbf{correct}, i.e.,
\[(+2^{n-1})+(-2^{n-1})+(-2^{n-1}) = -2^{n-1}\]
(\emph{We ignore the leftmost carry}.)

\end{frame}

% ------------------------------------------------------------------------
% 
\begin{frame}
\frametitle{\(2\)-complement binary/Addition (cont)}

\textbf{Summary}

\bigskip

In order to perform the 2-complement addition, consider the two
numbers as normal binary numbers and then perform the usual addition.

\bigskip

Considering the possible carries out of the positions \(n-1\) and
\(n-2\):
\begin{itemize}

  \item if there is only \textbf{one carry in total}, then
  the sum is \textbf{incorrect}:
  \begin{itemize}
   
    \item overflow if the result is negative, 

    \item underflow if the result is positive;

  \end{itemize}

  \item otherwise the sum is correct and the carries, if any, are
    discarded.

\end{itemize}

\end{frame}

% ------------------------------------------------------------------------
% 
\begin{frame}
\frametitle{\(2\)-complement binary/Complement and negation}

The \textbf{complement} of a normal binary number consists simply in
turning all its bits \(1\) into \(0\) and all its bits \(0\) into
\(1\). For example, the complement of \(10010_2\) is \(01101_2\), or
simply \(1101_2\). The complement of the complement is the number
itself.

\bigskip

This operation is sometimes called the \textbf{\(1\)-complement}.

\bigskip

How do we find the \textbf{negation} of a \(2\)-complement number?
In other words, given the \(n\)-bit, \(2\)-complement number \(N =
b_{n-1}b_{n-2}\dots{b_0}\), what are the bits of \(-N\)?

\bigskip

What we want is to solve the following puzzle:
\[
\renewcommand\arraystretch{0.85}
\setlength\extrarowheight{4pt}
\begin{array}{rc@{\;}c@{\;}c@{\;}c@{\;}c}
  & 0           & 0          &      0      & \dots & 0\\
- & b_{n-1}     & b_{n-2}    & b_{n-3}     & \dots & b_0\\
\hline
  & \textbf{?} & \textbf{?} & \textbf{?} & \dots & \textbf{?}
\end{array}
\]

\end{frame}

% ------------------------------------------------------------------------
% 
\begin{frame}
\frametitle{\(2\)-complement binary/Complement and negation (cont)}

Notice that, with \(n\)-bit, \(2\)-complement binary numbers, the
following holds:
\[
0 = -1 + 1 \Longleftrightarrow
\underbrace{000\dots00}_{n \, \text{bits}} = 
\underbrace{111\dots11}_{n \, \text{bits}}
+ 
\underbrace{000\ldots01}_{n \, \text{bits}}
\]
So our initial question is equivalent to
\[
\renewcommand\arraystretch{0.85}
\setlength\extrarowheight{4pt}
\begin{array}{rc@{\;}c@{\;}c@{\;}c@{\;}ccc}
  & 1          & 1          &      1     & \dots & 1   & + & 00\dots{}01\\
- & b_{n-1}    & b_{n-2}    & b_{n-3}    & \dots & b_0 &   &\\
\hline
  & \textbf{?} & \textbf{?} & \textbf{?} & \dots & \textbf{?} & &
\end{array}
\]

\end{frame}

% ------------------------------------------------------------------------
% 
\begin{frame}
\frametitle{\(2\)-complement binary/Complement and negation (cont)}

Notice also that subtracting from \(1\) is the same as complementing:
\[
1 - 1 = 0 \Longleftrightarrow 1 - 0 = 1
\]
If we write \(\overline{0} = 1\) and \(\overline{1} = 0\) for the
complement, we simply have
\(
1 - b = \overline{b}
\)
and
\[
\setlength\extrarowheight{4pt}
\begin{array}{rc@{\;}c@{\;}c@{\;}c@{\;}ccc}
  & 1          & 1          &      1     & \dots & 1   & + & 00\dots{}01\\
- & b_{n-1}    & b_{n-2}    & b_{n-3}    & \dots & b_0 &   &\\
\hline
  & \overline{b_{n-1}} & \overline{b_{n-2}} & \overline{b_{n-3}} &
  \dots & \overline{b_0} & &\\
+ &          0 &          0 &          0 & \dots &          1 & &\\
\cline{1-6}
  & \textbf{?} & \textbf{?} & \textbf{?} & \dots & \textbf{?} & &
\end{array}
\]

\end{frame}

% ------------------------------------------------------------------------
% 
\begin{frame}
\frametitle{\(2\)-complement binary/Complement and negation (cont)}

\label{one_complement}

The last step is the addition of \(1\), which is easy. In summary:
\[
A - B = A + (-B) = A + (\overline{B} + 1) = (A + \overline{B}) + 1
\]
where \(\overline{B}\) is the bitwise 1-complement of \(B\) (i.e., bit
by bit 1-complement).

\bigskip

\noindent For example, let us negate the \(8\)-bit, \(2\)-complement
binary number \(10110100\):
\[
\renewcommand\arraystretch{0.85}
\setlength\extrarowheight{4pt}
\begin{array}{rc@{\;}c@{\;}c@{\;}c@{\;}c@{\;}c@{\;}c@{\;}c}
  & 1 & 0 & 1 & 1 & 0 & 1 & 0 & 0\\
\hline
  & 0 & 1 & 0 & 0 & 1 & 0 & 1 & 1\\
+ & 0 & 0 & 0 & 0 & 0 & 0 & 0 & 1\\
\hline
  & 0 & 1 & 0 & 0 & 1 & 1 & 0 & 0
\end{array}
\]
\textbf{Exercise.} Negate the \(8\)-bit, \(2\)-complement binary
number \(10000000\).

\end{frame}

% ------------------------------------------------------------------------
% 
\begin{frame}
\frametitle{\(2\)-complement binary/Complement and negation (cont)}

Negation gives a simple way to convert from \(2\)-complement binary to
decimal when the number is negative: first, negate it and then convert
to decimal as usual, and finally add a minus sign to the result.

\bigskip

For example,
\[
10001001 = -01110111 = -119_{10}
\]
Conversely, in order to convert a negative decimal number to
\(2\)-complement binary notation, we convert the negated decimal
(i.e., we forget the minus sign) and then negate the result.

\end{frame}

% ------------------------------------------------------------------------
% 
\begin{frame}
\frametitle{Gray codes}

There are several ways to code numbers using a binary alphabet, i.e.,
\(0\) and \(1\). One is the usual binary code and another is the
\(2\)-complement binary code.

\bigskip

There is another one, quite useful, called \textbf{unit distance
  code}, or \textbf{Gray code}. It is designed so that only one bit
changes when a number is incremented or decremented. For example, a
Gray code for numbers from \(0\) to \(15\) is
\[
\begin{array}{rr|rr|rr|rr}
\text{Number} & \text{Code} & \text{Number} & \text{Code} &
\text{Number} & \text{Code} & \text{Number} & \text{Code}\\
\hline
0 & 0000 &  4 & 0110 &  8 & 1100 & 12 & 1010\\
1 & 0001 &  5 & 0111 &  9 & 1101 & 13 & 1011\\
2 & 0011 &  6 & 0101 & 10 & 1111 & 14 & 1001\\
3 & 0010 &  7 & 0100 & 11 & 1110 & 15 & 1000
\end{array}
\]

\end{frame}

% ------------------------------------------------------------------------
% 
\begin{frame}
\frametitle{Binary-Coded Decimal}

There is another interesting binary coding for numbers, where each
decimal digit is separately encoded in binary: \textbf{Binary-Coded
  Decimal}, or, in short, \textbf{BCD}.

\bigskip

For example, to code the number \(37_{10}\) in BCD, we do
\[
37_{10} = (3_{10})(7_{10}) = (0011_2)(0111_2) = 00110111_2
\]
Every decimal digit can be encoded with four bits.

\bigskip

Conversely, to convert a BCD number into decimal, we separate the bits
in groups of four and convert them separately with the usual manner.

\end{frame}

% ------------------------------------------------------------------------
% 
\begin{frame}
\frametitle{Binary-Coded Decimal (cont)}

Note that, since four bits are used for coding each decimal digit,
some combinations of bits are useless because there is no digit
corresponding to numbers \(10\) to \(15\), hence the coding is not as
compact as for the usual binary coding.

\bigskip

So, for instance, \(110001010111_2\) is not a BCD.

\end{frame}

% ------------------------------------------------------------------------
% 
\begin{frame}
\frametitle{Binary-Coded Decimal/Addition}

If we add BCD numbers as basic binary numbers we do not always get the
correct result. For example
\[
\renewcommand\arraystretch{0.85}
\setlength\extrarowheight{4pt}
\begin{array}{cc@{\;}c@{\;}c@{\;}c@{\;}cc@{\;}c@{\;}c@{\;}c}
  & 0 & 0 & 1 & 1 & & 1 & 0 & 0 & 0\\
+ & 0 & 1 & 0 & 0 & & 0 & 0 & 0 & 1\\
\hline
  & 0 & 1 & 1 & 1 & & 1 & 0 & 0 & 1
\end{array}
\]
is correct, but
\[
\renewcommand\arraystretch{0.85}
\setlength\extrarowheight{4pt}
\begin{array}{cc@{\;}c@{\;}c@{\;}c@{\;}cc@{\;}c@{\;}c@{\;}c}
  & 1 & 0 & 0 & 0 & & 0 & 1 & 1 & 1\\
+ & 1 & 0 & 0 & 1 & & 0 & 1 & 0 & 0\\
\hline
1 & 0 & 0 & 0 & 1 & & 1 & 0 & 1 & 1
\end{array}
\]
is not, because \(1011_2 = 11_{10}\) is not a decimal digit.

\end{frame}

% ------------------------------------------------------------------------
% 
\begin{frame}
\frametitle{Binary-Coded Decimal/Addition (cont)}

Actually, the problem happens wherever the number corresponding to a
digit ranges from \(9+1=10\) to \(9+9=18\).

\bigskip

To be more precise, there are two sub-cases:
\begin{enumerate}

  \item from \(10_{10}\) to \(15_{10}\): the four bits range from
    \(1010\) to \(1111\) and there is no carry;

  \item from \(16_{10}\) to \(18_{10}\): the four bits range from
    \(0000\) to \(0010\) and there is a carry.

\end{enumerate}

\end{frame}

% ------------------------------------------------------------------------
% 
\begin{frame}
\frametitle{Binary-Coded Decimal/Addition (cont)}

In the first case, we would like to map \(1010_2 = 10_{10}\) to
\(0001\;0000_2 = 10_\text{BCD}\), i.e., to \(0000\) plus a carry,
\(1011_2 = 11_{10}\) to \(0001\;0001_2 = 11_\text{BCD}\) etc. until
\(1111_2 = 15_{10}\) to \(0001\;0101_2 = 15_\text{BCD}\). This can be
simply achieved by adding \(6_{10} = 0110_2\) to the digit, e.g.
\[10_{10} + 6_{10} = 1010_2 + 0110_2 = 0001\;0000_2\]

\bigskip

\noindent In the second case, we would like to map
\(0001\;0000_2=16_{10}\) to \(0001\;0110_2 = 16_\text{BCD}\)
etc. until \(0001\;0010_2=18_{10}\) to \(0001\;1000_2 =
18_\text{BCD}\). This can be also achieved by simply adding \(6 =
0110_2\) to the rightmost four bits, e.g.
\[0000 + 0110 = 0110\]

\end{frame}

% ------------------------------------------------------------------------
% 
\begin{frame}
\frametitle{Binary-Coded Decimal/Addition (cont)}

Therefore the rule for addition is:
\begin{enumerate}

  \item add the two numbers as usual;

  \item to each block of 4 bits in the result, add \(0110\) if a carry
    was issued from it or if it is strictly greater than \(1001\),
    else add \(0000\).\label{coding:add0000}

\end{enumerate}
For instance
\[
\renewcommand\arraystretch{0.85}
\setlength\extrarowheight{4pt}
\begin{array}{c@{}cc@{\;}c@{\;}c@{\;}c@{\;}cc@{\;}c@{\;}c@{\;}c@{\;}cc@{\;}c@{\;}c@{\;}c}
  &   &   &   &   & 1 & &   &   &   &   & &   &   &   &\\  
  &   & 0 & 1 & 1 & 0 & & 1 & 0 & 0 & 0 & & 0 & 1 & 1 & 0\\
+ &   & 0 & 0 & 1 & 1 & & 1 & 0 & 0 & 0 & & 0 & 0 & 0 & 1\\
\cline{2-16}
  & 1 & 1 & 1 &   &   & &   &   &   &   & &   &   &   &\\
  &   & 1 & 0 & 1 & 0 & & 0 & 0 & 0 & 0 & & 0 & 1 & 1 & 1\\
+ &   & 0 & 1 & 1 & 0 & & 0 & 1 & 1 & 0 & & 0 & 0 & 0 & 0\\
\cline{2-16}
  & 1 & 0 & 0 & 0 & 0 & & 0 & 1 & 1 & 0 & & 0 & 1 & 1 & 1
\end{array}
\]

\end{frame}

% ------------------------------------------------------------------------
% 
\begin{frame}
\frametitle{Binary-Coded Decimal/Addition (cont)}

\begin{columns}
  \column{0.5\textwidth}
    Actually, the additions of \(0110\) must be carried over again if
    any block is strictly greater than \(1001\), due to the carries of
    step~\eqref{coding:add0000}. 

    \bigskip

    For instance \(36 + 65 = 101\) in BCD is given in the facing
    column.
  \column{0.5\textwidth}
    \[
    \renewcommand\arraystretch{0.85}
    \setlength\extrarowheight{4pt}
    \begin{array}{r@{\;}c@{\;}cc@{\;}c@{\;}c@{\;}c@{\;}cc@{\;}c@{\;}c@{\;}c@{}}
      &   & & 1&1& &  & & 1& & &\\
      &   & & 0&0&1&1 & & 0&1&1&0\\
      + &   & & 0&1&1&0 & & 0&1&0&1\\
      \cline{2-12}
      &   & &  & &1&1 & & 1&1& &\\
      &   & & 1&0&0&1 & & 1&0&1&1\\
      + &   & &  & & &  & & 0&1&1&0\\
      \cline{2-12} 
      & 1 & & 1&1& &  & &  & & &\\
      &   & & 1&0&1&0 & & 0&0&0&1\\
      + &   & & 0&1&1&0 & &  & & &\\
      \cline{2-12}
      & 1 & & 0&0&0&0 & & 0&0&0&1
    \end{array}
    \]
\end{columns}

\end{frame}

% ------------------------------------------------------------------------
% 
\begin{frame}
\frametitle{Binary-Coded Decimal/Subtraction}

BCD subtraction consists in
\begin{enumerate}

  \item a normal binary subtraction;

  \item to each block of four bits that borrowed to the next group,
    subtract \(0110\), otherwise \(0000\).

\end{enumerate}
The reason of the latter is that when we borrow, we actually borrow
\(16_{10}\) instead of \(10_\text{BCD}\).
\[
\renewcommand\arraystretch{0.85}
\setlength\extrarowheight{4pt}
\begin{array}{rc@{\;}c@{\;}c@{\;}c@{\;}cc@{\;}c@{\;}c@{\;}c@{\;}cc@{\;}c@{\;}c@{\;}c}
  &   &   &   & 0 & & 1 & 0 &   &   & & 0 & 1 &   &\\  
  & 0 & 0 & 1 & 1 & & 0 & 1 & 0 & 1 & & 1 & 0 & 0 & 0\\
- & 0 & 0 & 1 & 0 & & 0 & 1 & 1 & 0 & & 0 & 0 & 1 & 0\\
\cline{2-15}
  & 0 & 0 & 0 & 0 & & 1 & 1 & 1 & 1 & & 0 & 1 & 1 & 0\\
- & 0 & 0 & 0 & 0 & & 0 & 1 & 1 & 0 & & 0 & 0 & 0 & 0\\
\cline{2-15}
  & 0 & 0 & 0 & 0 & & 1 & 0 & 0 & 1 & & 0 & 1 & 1 & 0
\end{array}
\]

\end{frame}
