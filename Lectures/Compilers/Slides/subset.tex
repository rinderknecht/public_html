%%-*-latex-*-

% ------------------------------------------------------------------------
% 
\begin{frame}
\frametitle{Equivalence of DFAs and NFAs}

\bigskip

NFA are easier to build than DFA because one does not have to worry,
for any state, of having out-going edges carrying a unique label.

\bigskip

The surprising thing is that NFA and DFA actually have the same
expressiveness, i.e. all that can be defined by means of a NFA can
also be defined with a DFA (the converse is trivial since a DFA is
already a NFA).

\bigskip

More precisely, there is a procedure, called \textbf{the subset
construction}, which converts any NFA to a DFA.

\end{frame}

% ------------------------------------------------------------------------
% 
\begin{frame}
\frametitle{Subset construction}

Consider that, in a NFA, from a state \(q\) with several out-going
edges with the same label \(a\), the transition function \(\delta_N\)
leads, in general, to \emph{several} states.

\bigskip

The idea of the subset construction is to create a new automaton where
these edges are merged. 

\bigskip

So we create a state \(p\) which corresponds to the set of states
\(\delta_N (q,a)\) in the NFA. Accordingly, we create a state \(r\)
which corresponds to the set \(\{q\}\) in the NFA. We create an edge
labeled \(a\) between \(r\) and \(p\). The important point is that
\emph{this edge is unique}.

\bigskip

This is the first step to create a DFA from a NFA.

\end{frame}

% ------------------------------------------------------------------------
% 
\begin{frame}
\frametitle{Subset construction (cont)}

Graphically, instead of the non-determinism
\begin{center}
\includegraphics[bb=62 660 222 730]{nfa_edges}
\end{center}
where \(\delta_N (q, a) = \{p_0, p_1, \dots, p_n\}\), we get the
determinism
\begin{center}
\includegraphics[bb=60 712 144 730]{nfa_subset}
\end{center}

\end{frame}

% ------------------------------------------------------------------------
% 
\begin{frame}
\frametitle{Subset construction (cont)}

Now, let us present the complete algorithm for the subset
construction. Let us start from a NFA \({\cal N} = (Q_N, \Sigma,
\delta_N, q_0, F_N)\). The goal is to construct a DFA \({\cal D} =
(Q_D, \Sigma, \delta_D, \{q_0\}, F_D)\) such that \(L({\cal D}) =
L({\cal N})\).

\bigskip

Notice that the input alphabet of the two automata are the same and
the initial state of \({\cal D}\) if the set containing only the
initial state of \({\cal N}\).

\bigskip

The other components of \({\cal D}\) are constructed as follows.
\begin{itemize}

  \item \(Q_D\) is the set of subsets of \(Q_N\); i.e. \(Q_D\) is the
  \textbf{power set} of \(Q_N\). Thus, if \(Q_D\) has \(n\) states,
  \(Q_D\) has \(2^n\) states. Fortunately, often not all these states
  are \textbf{accessible} from the initial state of \(Q_D\), so these
  inaccessible states can be discarded.

\end{itemize}

\end{frame}

% ------------------------------------------------------------------------
% 
\begin{frame}
\frametitle{Subset construction (cont)}

\label{state_explosion}

Why is \(2^n\) the number of subsets of a finite set of cardinal
\(n\)?

\bigskip

Let us order the \(n\) elements and represent each subset by an
\(n\)-bit string where bit \(i\) corresponds to the \(i\)-th element:
it is \(1\) if the \(i\)-th element is present in the subset and \(0\)
if not.

\bigskip

This way, we counted all the subsets and not more (a bit cannot always
be \(0\) since all elements are used to form subsets and cannot always
be \(1\) if there is more than one element).

There are \(2\) possibilities, \(0\) or \(1\), for the first bit;
\(2\) possibilities for the second bit etc. Since the choices are
independent, we multiply all: \(\underbrace{2 \times 2 \times
  \dots \times 2}_{n \; \text{times}} = 2^n\).

\bigskip

Hence the number of subsets of an \(n\)-element set is also \(2^n\).

\end{frame}

% ------------------------------------------------------------------------
% 
\begin{frame}
\frametitle{Subset construction (cont)}

Resuming the definition of DFA \({\cal D}\), the other components are
defined as follows.
\begin{itemize}

  \item \(F_D\) is the set of subsets \(S\) of \(Q_N\) such as \(S
  \cap F_N \neq \varnothing\). That is, \(F_D\) is all sets of \(N\)'s
  states that include at least one final state of \({\cal N}\).

  \item for each set \(S \subseteq Q_N\) and for each input \(a \in
  \Sigma\),
  \[
    \delta_D(S, a) = \bigcup_{q \in S}{\delta_N (q, a)}
  \]
  In other words, to compute \(\delta_D (S, a)\) we look at all the
  states \(q\) in \(S\), see what states of \({\cal N}\) are reached
  from \(q\) on input \(a\) and take the union of all those states to
  make the next state of \({\cal D}\).

\end{itemize}

\end{frame}

% ------------------------------------------------------------------------
% 
\begin{frame}
\frametitle{Subset construction/Example/Transition table}

\begin{columns}

  \column{0.5\textwidth} Let us consider the NFA given by its
  transition table page~\pageref{nfa_01_suffix_table}:
  \[
  \begin{array}{r@{}l||c|c}
    \multicolumn{2}{c||}{\text{NFA} \; {\cal N}} & 0 & 1\\
    \hhline{==::==}
    \rightarrow & q_0 & \{q_0, q_1\} & \{q_0\}\\
    & q_1 & \varnothing  & \{q_2\}\\
    \#          & q_2 & \varnothing  & \varnothing
  \end{array}
  \]
  and let us create an equivalent DFA. 

  \column{0.5\textwidth} First, we form all the subsets of the sets of
  the NFA and put them in the first column:
  \[
  \begin{array}{r@{}l||c|c}
    \multicolumn{2}{c||}{\text{DFA} \; {\cal D}} & 0 & 1\\
    \hhline{==::==}
    & \varnothing & &\\
    & \{q_0\}           & &\\
    & \{q_1\}           & &\\
    & \{q_2\}           & &\\
    & \{q_0, q_1\}      & &\\
    & \{q_0, q_2\}      & &\\
    & \{q_1, q_2\}      & &\\
    & \{q_0, q_1, q_2\} & &
  \end{array}
  \]
\end{columns}

\end{frame}

% ------------------------------------------------------------------------
% 
\begin{frame}
\frametitle{Subset construction/Example/Transition table (cont)}

\begin{columns}

  \column{0.5\textwidth} Then we annotate in this first column the
  states with \(\rightarrow\) if and only if they contain the initial
  state of the NFA, here \(q_0\), and we add a \(\#\) if and only if
  the states contain at least a final state of the NFA, here \(q_2\).

  \column{0.5\textwidth}
  \[
  \begin{array}{r@{}l||c|c}
    \multicolumn{2}{c||}{\text{DFA} \; {\cal D}} & 0 & 1\\
    \hhline{==::==}
    & \varnothing & &\\
    \rightarrow & \{q_0\}  & &\\
       & \{q_1\}           & &\\
    \# & \{q_2\}           & &\\
       & \{q_0, q_1\}      & &\\
    \# & \{q_0, q_2\}      & &\\
    \# & \{q_1, q_2\}      & &\\
    \# & \{q_0, q_1, q_2\} & &
  \end{array}
  \]
\end{columns}

\end{frame}

% ------------------------------------------------------------------------
% 
\begin{frame}
\frametitle{Subset construction/Example/Transition table (cont)}
\[
\scriptsize
\begin{array}{r@{}l||c|c}
\multicolumn{2}{c||}{\text{DFA} \; {\cal D}} & 0 & 1\\
\hhline{==::==}
            & \varnothing & \varnothing & \varnothing\\
\rightarrow & \{q_0\} & \delta_N(q_0, 0) & \delta_N(q_0, 1)\\
            & \{q_1\} & \delta_N(q_1, 0) & \delta_N(q_1, 1)\\
         \# & \{q_2\} & \delta_N(q_2, 0) & \delta_N(q_2, 1)\\
            & \{q_0, q_1\} & \delta_N(q_0,0) \cup \delta_N(q_1,0) 
                           & \delta_N(q_0,1) \cup \delta_N(q_1,1)\\
         \# & \{q_0, q_2\} & \delta_N(q_0,0) \cup \delta_N(q_2,0) 
                           & \delta_N(q_0,1) \cup \delta_N(q_2,1)\\
         \# & \{q_1, q_2\} & \delta_N(q_1,0) \cup \delta_N(q_2,0) 
                           & \delta_N(q_1,1) \cup \delta_N(q_2,1)\\
         \# & \{q_0, q_1, q_2\} 
            & \delta_N(q_0,0) \cup \delta_N(q_1,0) \cup \delta_N(q_2,0)
            & \delta_N(q_0,1) \cup \delta_N(q_1,1) \cup \delta_N(q_2,1)
\end{array}
\]

\end{frame}

% ------------------------------------------------------------------------
% 
\begin{frame}
\frametitle{Subset construction/Example/Transition table (cont)}
\[
\begin{array}{r@{}l||c|c}
\multicolumn{2}{c||}{\text{DFA} \; {\cal D}} & 0 & 1\\
\hhline{==::==}
            & \varnothing & \varnothing & \varnothing\\
\rightarrow & \{q_0\} & \{q_0, q_1\} & \{q_0\}\\
            & \{q_1\} & \varnothing & \{q_2\}\\
         \# & \{q_2\} & \varnothing & \varnothing\\
            & \{q_0, q_1\} & \{q_0,q_1\} & \{q_0,q_2\}\\
         \# & \{q_0, q_2\} & \{q_0,q_1\} & \{q_0\}\\
         \# & \{q_1, q_2\} & \varnothing & \{q_2\}\\
         \# & \{q_0, q_1, q_2\} & \{q_0, q_1\} & \{q_0,q_2\}
\end{array}
\]

\end{frame}

% ------------------------------------------------------------------------
% 
\begin{frame}
\frametitle{Subset construction/Example/Transition diagram}

The transition diagram of the DFA \({\cal D}\) is then
\begin{center}
\includegraphics[bb=42 628 291 758]{dfa_from_nfa}
\end{center}
where states with out-going edges which have no end are final states.

\end{frame}

% ------------------------------------------------------------------------
% 
\begin{frame}
\frametitle{Subset construction/Example/Transition diagram (cont)}

\begin{columns}

  \column{0.5\textwidth} If we look carefully at the transition
  diagram, we see that the DFA is actually made of two parts which are
  disconnected. i.e. not joined by and edge.

  \bigskip

  In particular, since we have only one initial state, this means that
  one part is not accessible, i.e. some states are never used to
  recognise or reject an input word, and we can remove this part.

  \column{0.5\textwidth}
  \begin{center}
    \includegraphics[bb=48 642 150 755]{dfa_from_nfa_simple1}
  \end{center}
\end{columns}

\end{frame}

% ------------------------------------------------------------------------
% 
\begin{frame}
\frametitle{Subset construction/Example/Transition diagram (cont)}

\begin{columns}

  \column{0.5\textwidth} It is important to understand that the states
  of the DFA are subsets of the NFA states.

  \bigskip

  This is due to the construction and, when finished, it is possible
  to hide this by \textbf{renaming the states}. For example, we can
  rename the states of the previous DFA in the following manner:
  \(\{q_0\}\) into \(A\), \(\{q_0, q_1\}\) in \(B\) and \(\{q_0,
  q_2\}\) in \(C\).

  \bigskip

  So the transition table changes:

  \column{0.5\textwidth}
  \[
  \begin{array}{r@{}l||c|c}
    \multicolumn{2}{c||}{\text{DFA} \; {\cal D}} & 0 & 1\\
    \hhline{==::==}
    \rightarrow & \{q_0\}     & \{q_0, q_1\} & \{q_0\}\\
                & \{q_0,q_1\} & \{q_0,q_1\}  & \{q_0,q_2\}\\
             \# & \{q_0,q_2\} & \{q_0,q_1\}  & \{q_0\}
  \end{array}
  \]
  \[
  \begin{array}{r@{}l||c|c}
    \multicolumn{2}{c||}{\text{DFA} \; {\cal D}} & 0 & 1\\
    \hhline{==::==}
    \rightarrow & A & B & A\\
                & B & B & C\\
             \# & C & B & A
  \end{array}
  \]
\end{columns}

\end{frame}

% ------------------------------------------------------------------------
% 
\begin{frame}
\frametitle{Subset construction/Example/Transition diagram (cont)}
So, finally, the DFA is simply
\begin{center}
\includegraphics[bb=48 642 150 755]{dfa_from_nfa_simple}
\end{center}

\end{frame}

% ------------------------------------------------------------------------
% 
\begin{frame}
\frametitle{Subset construction/Optimisation}

Even if in the worst case the resulting DFA has an exponential number
of states of the corresponding NFA, it is in practice often possible
to avoid the construction of inaccessible states.
\begin{itemize}

  \item The singleton containing the initial state (in our example,
  \(\{q_0\}\)) is accessible.

  \item Assume we have a set \(S\) of accessible states; then for
  each input symbol \(a\), we compute \(\delta_D(S,a)\): this new set
  is also accessible.

  \item Repeat the last step, starting with \(\{q_0\}\), until no
  new (accessible) sets are found.

\end{itemize}

\end{frame}

% ------------------------------------------------------------------------
% 
\begin{frame}
\frametitle{Subset construction/Optimisation/Example}

\begin{columns}

  \column{0.5\textwidth} Let us consider the NFA given by its
  transition table page~\pageref{nfa_01_suffix_table}:
  \[
  \begin{array}{r@{}l||c|c}
    \multicolumn{2}{c||}{\text{NFA} \; {\cal N}} & 0 & 1\\
    \hhline{==::==}
    \rightarrow & q_0 & \{q_0, q_1\} & \{q_0\}\\
                & q_1 & \varnothing  & \{q_2\}\\
    \#          & q_2 & \varnothing  & \varnothing
  \end{array}
  \]
  
  \column{0.5\textwidth} Initially, the sole subset of accessible
  states is \(\{q_0\}\):
  \[
  \begin{array}{r@{}l||c|c}
    \multicolumn{2}{c||}{\text{DFA} \; {\cal D}} & 0 & 1\\
    \hhline{==::==}
    \rightarrow & \{q_0\} & \delta_N(q_0,0) & \delta_N(q_0,1)
  \end{array}
  \]
  that is
  \[
  \begin{array}{r@{}l||c|c}
    \multicolumn{2}{c||}{\text{DFA} \; {\cal D}} & 0 & 1\\
    \hhline{==::==}
    \rightarrow & \{q_0\} & \{q_0,q_1\} & \{q_0\}
  \end{array}
  \]
\end{columns}

\end{frame}

% ------------------------------------------------------------------------
% 
\begin{frame}
\frametitle{Subset construction/Optimisation/Example (cont)}

\begin{columns}

  \column{0.5\textwidth} Therefore \(\{q_0,q_1\}\) and \(\{q_0\}\) are
  accessible sets. But \(\{q_0\}\) is not a new set, so we only add to
  the table entries \(\{q_0, q_1\}\) and compute the transitions from
  it:
  \[
  \begin{array}{r@{}l||c|c}
    \multicolumn{2}{c||}{\text{DFA} \; {\cal D}} & 0 & 1\\
    \hhline{==::==}
    \rightarrow & \{q_0\}      & \{q_0, q_1\} & \{q_0\}\\
                & \{q_0, q_1\} & \{q_0, q_1\} & \{q_0, q_2\}
  \end{array}
  \]

  \column{0.5\textwidth}
  This step uncovered a new set of accessible states, \(\{q_0,
  q_2\}\), which we add to the table and repeat the procedure, and
  mark it as final state since \(q_2 \in \{q_0, q_2\}\):
  \[
  \begin{array}{r@{}l||c|c}
    \multicolumn{2}{c||}{\text{DFA} \; {\cal D}} & 0 & 1\\
    \hhline{==::==}
    \rightarrow & \{q_0\}      & \{q_0, q_1\} & \{q_0\}\\
                & \{q_0, q_1\} & \{q_0, q_1\} & \{q_0, q_2\}\\
             \# & \{q_0, q_2\} & \{q_0, q_1\} & \{q_0\}
  \end{array}
  \]
  We are done since there is no more new accessible sets.
\end{columns}

\end{frame}

% ------------------------------------------------------------------------
% 
\begin{frame}
\frametitle{Subset construction/Tries}

Lexical analysis tries to recognise a prefix of the input character
stream (in other words, the first lexeme of the given
program). Consider the C keywords \term{const} and \term{continue}:
\begin{center}
\includegraphics[bb=48 675 353 730]{nfa_kwd}
\end{center}
This example shows that a NFA is much more comfortable than a DFA for
specifying tokens for lexical analysis: we design \emph{separately}
the automata for each token and then merge their initial states into
one, leading to one (possibly big) NFA.

\bigskip

It is possible to apply the subset construction to this NFA.

\end{frame}

% ------------------------------------------------------------------------
% 
\begin{frame}
\frametitle{Subset construction/Tries (cont)}

\label{trie_kwd}

After forming the corresponding NFA as in the previous example, it is
actually easy to construct an equivalent DFA by \textbf{sharing their
  prefixes}, hence obtaining a tree-like automaton called
\textbf{trie} (pronounced as the word `try'):
\begin{center}
\includegraphics[bb=48 675 353 730]{trie_kwd}
\end{center}
Note that this construction only works for a list of constant words,
like keywords.

\end{frame}

% ------------------------------------------------------------------------
% 
\begin{frame}
\frametitle{Subset construction/Text searching}

This technique can easily be generalized for searching constant
strings (like keywords) in a text, i.e. not only as a prefix of a
text, but \emph{at any position}.

\bigskip

It suffices to add a loop on the initial state for each possible input
symbol. If we note \(\Sigma\) the language alphabet, we get
\begin{center}
\includegraphics[bb=48 675 355 738]{trie_kwd_search}
\end{center}

\end{frame}

% ------------------------------------------------------------------------
% 
\begin{frame}
\frametitle{Subset construction/Text searching (cont)}

It is possible to apply the subset construction to this NFA or to use
it directly for searching the two keywords at any position in a text.

\bigskip

In case of direct use, the difference between this NFA and the trie
page~\pageref{trie_kwd} is that there is no need here to ``restart''
by hand the recognition process once a keyword has been recognised: we
just continue.

\bigskip

This works because of the loop on the initial state, which always
allows a new start.

\bigskip

Try for instance the input \texttt{constantcontinue}.

\end{frame}

% ------------------------------------------------------------------------
% 
\begin{frame}
\frametitle{Subset construction/Bad case}

The subset construction can lead, in the worst case, to a number of
states which is the total number of state subsets of the NFA.

\bigskip

In other words, if the NFA has \(n\) states, the equivalent DFA by
subset construction can have \(2^n\) states (see
page~\pageref{state_explosion} for the count of all the subsets of a
finite set).

\end{frame}

% ------------------------------------------------------------------------
% 
\begin{frame}
\frametitle{Subset construction/Bad case (cont)}

Consider the following NFA, which recognises all binary strings which
have \(1\) at the \(n\)-th position from the end:
\begin{center}
\includegraphics[bb=48 711 305 758]{subset_bad_case}
\end{center}
The language recognised by this NFA is \(\Sigma^{*} 1 \Sigma^{n-1}\),
where \(\Sigma=\{0,1\}\), that is: all words of length greater or
equal to \(n\) are accepted as long as the \(n\)-th bit from the
\textbf{end} is \(1\).

\bigskip

Therefore, in any equivalent DFA, all the prefixes of length \(n\)
should not lead to a stuck state, because the automaton must wait
until the \textbf{end} of the word to accept or reject it.

\end{frame}

% ------------------------------------------------------------------------
% 
\begin{frame}
\frametitle{Subset construction/Bad case (cont)}

If the states reached by these prefixes are all different, then there
are at least \(2^n\) states in the DFA.

\bigskip

Equivalently (by contraposition), if there are less than \(2^n\)
states, then some states can be reached by several strings of length
\(n\):
\begin{center}
\includegraphics[bb=48 643 235 730]{dfa_bad_case}
\end{center}
where words \(x1w\) and \(x'0w\) have length \(n\).

\end{frame}

% ------------------------------------------------------------------------
% 
\begin{frame}
\frametitle{Subset construction/Bad case (cont)}

Let us call the DFA \({\cal D} = (Q_D, \Sigma, \delta_D, q_D, F_D)\),
where \(q_D=\{q_0\}\). 

\bigskip

The extended transition function is noted \(\hat{\delta}_D\) as
usual. The situation of the previous picture can be formally expressed
as 
\begin{gather}
\hat{\delta}_D (q_D, x1) = \hat{\delta}_D (q_D, x'0) = q \label{ext:1}\\
\lvert{x1w}\lvert = \lvert{x'0w}\lvert = n
\end{gather}
where \(\lvert{u}\lvert\) is the length of \(u\).

\end{frame}

% ------------------------------------------------------------------------
% 
\begin{frame}
\frametitle{Subset construction/Bad case (cont)}

Let \(y\) be a any string of 0 and 1 such as \(\lvert{wy}\lvert = n -
1\).

\bigskip

Then \(\hat{\delta}_D(q_D, x1wy) \in F_D\) since there is a 1 at the
\(n\)-th position from the end:
\begin{center}
\includegraphics[bb=48 643 287 730]{dfa_bad_case_complete}
\end{center}
Also, \(\hat{\delta}_D(q_D,x'0wy) \not\in F_D\) because there is a 0
at the \(n\)-th position from the end.

\end{frame}

% ------------------------------------------------------------------------
% 
\begin{frame}
\frametitle{Subset construction/Bad case (cont)}

On the other hand, equation~(\ref{ext:1}) implies
\[
\hat{\delta}_D (q_D, x1wy) = \hat{\delta}_D (q_D, x'0wy) = p
\]
So there is contradiction because a state (here, \(p\)) must be either
final or not final, it cannot be both...

\bigskip

As a consequence, we must reject our initial assumption: there are at
least \(2^n\) states in the equivalent DFA.

\bigskip

This is a very bad case, even if it is not the worst case (\(2^{n+1}\)
states).

\end{frame}
