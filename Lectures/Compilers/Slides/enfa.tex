%%-*-latex-*-

% ------------------------------------------------------------------------
% 
\begin{frame}
\frametitle{NFA with \(\epsilon\)-transitions (\eNFA)}

We shall now introduce another extension to NFA, called \eNFA, which
is a NFA whose labels can be the empty string, noted \(\epsilon\). 

\bigskip

The interpretation of this new kind of transition, called
\(\epsilon\)-transition, is that the current state changes by
following this transition \emph{without reading any input}. This is
sometimes referred as a \textbf{spontaneous transition}.

\bigskip

The rationale, i.e., the intuition behind that, is that \(\epsilon a =
a \epsilon = a\), so recognising \(\epsilon a\) or \(a\epsilon\) is
the same as recognising \(a\). In other words, we do not need to read
something more than \(a\) as input.

\end{frame}

% ------------------------------------------------------------------------
% 
\begin{frame}
\frametitle{\eNFA{}/Example}

\label{enfa_num}

For example, we can specify signed natural and decimal numbers by
means of the \eNFA
\begin{center}
\includegraphics[bb=49 660 288 758]{enfa_num}
\end{center}
This is not the simplest \eNFA we can imagine for these numbers, but
note the utility of the \(\epsilon\)-transition between \(q_0\) and
\(q_1\).

\end{frame}

% ------------------------------------------------------------------------
% 
\begin{frame}
\frametitle{\eNFA (cont)}

In case of lexical analysers, \eNFA allow to design separately a NFA
for each token, then create an initial (respectively, final) state
connected to all their initial (respectively, final) states with an
\(\epsilon\)-transition. For instance, for keywords \term{fun} and
\term{function} and identifiers:
\begin{center}
\includegraphics[bb=48 630 421 732,scale=0.88]{enfa_kwd_id}
\end{center}

\end{frame}

% ------------------------------------------------------------------------
% 
\begin{frame}
\frametitle{\eNFA (cont)}

In lexical analysis, once we have a single \eNFA, we can
\begin{enumerate}

  \item either remove all the \(\epsilon\)-transitions and 
  \begin{enumerate}
    \item either create a NFA and then maybe a DFA;

    \item or create directly a DFA,

  \end{enumerate}

  \item or use a formal definition of \eNFA that directly leads to a
  recognition algorithm, just as we did for DFA and NFA.

\end{enumerate}
Both methods assume that it is always possible to create an equivalent
NFA, hence a DFA, from a given \eNFA. 

\bigskip

In other words, \textbf{DFA, NFA and \eNFA have the same expressive
  power.}

\end{frame}

% ------------------------------------------------------------------------
% 
\begin{frame}
\frametitle{\eNFA (cont)}

The first method constructs explicitly the NFA and maybe the DFA,
while the second does not, at the possible cost of more computations
at run-time.

\bigskip

Before entering into the details, we need to define formally an
\eNFA, as suggested by the second method.

\bigskip

The only difference between an NFA and an \eNFA is that the
transition function \(\delta_E\) takes as second argument an element
in \(\Sigma \cup \{\epsilon\}\), with \(\epsilon \not\in \Sigma\),
instead of \(\Sigma\) --- but the alphabet still remains \(\Sigma\).

\end{frame}

% ------------------------------------------------------------------------
% 
\begin{frame}
\frametitle{\eNFA{}/\(\epsilon\)-closure}

We need now a function called \textbf{\(\eClose\)}, which takes an
\eNFA \(\cal E\), a state \(q\) of \(\cal E\) and returns all the
states which are accessible in \(\cal E\) from \(q\) with label
\(\epsilon\). The idea is to achieve a \textbf{depth-first traversal}
of \(\cal E\), starting from \(q\) and following only
\(\epsilon\)-transitions.

\bigskip

Let us call \(\eDFS\) (``\(\epsilon\)-Depth-First-Search'') the
function such as \(\eDFS(q, Q)\) is the set of states reachable from
\(q\) following \(\epsilon\)-transitions and which is not included in
\(Q\), \emph{\(Q\) being interpreted as the set of states already
  visited in the traversal}. The set \(Q\) ensures the termination of
the algorithm even in presence of cycles in the automaton. Therefore,
let
\[
\eClose (q) = \eDFS (q, \varnothing) \quad \text{if} \; q \in Q_E
\]
where the \eNFA is \({\cal E} = (Q_E, \Sigma, \delta_E, q_0, F_E)\).

\end{frame}

% ------------------------------------------------------------------------
% 
\begin{frame}
\frametitle{\eNFA{}/\(\epsilon\)-closure (cont)}

Now we define \(\eDFS\) as follows:
\begin{align}
   \eDFS (q, Q) 
&= \varnothing 
& \text{if} \; q \in Q \label{eDFS_1}\\
  \eDFS (q, Q) 
&= \{q\} \quad \cup \bigcup_{p \in \delta_E(q,\epsilon)}{\eDFS (p, Q
    \cup \{q\})}
& \text{if} \; q \not\in Q \label{eDFS_2}
\end{align}
The \eNFA page~\pageref{enfa_num} leads to the
following \(\epsilon\)-closures:
\begin{columns}
  \column{0.5\textwidth}
  \begin{align*}
    \eClose (q_1) &= \{q_1\}\\
    \eClose (q_0) &= \{q_0, q_1\}\\
    \eClose (q_5) &= \{q_5\}
  \end{align*}

  \column{0.5\textwidth}
  \begin{align*}
    \eClose (q_2) &= \{q_2\}\\
    \eClose (q_3) &= \{q_3, q_5\}\\
    \eClose (q_4) &= \{q_4, q_3, q_5\}
  \end{align*}
\end{columns}

\end{frame}

% ------------------------------------------------------------------------
% 
\begin{frame}
\frametitle{\eNFA{}/\(\epsilon\)-closure (cont)}

Consider, as a more difficult example, the following \eNFA \(\cal E\):
\begin{center}
\includegraphics[bb=48 677 236 735]{enfa_eg}
\end{center}
\(\eClose (q_0)\)
\begin{align*}
  &= \eDFS (q_0, \varnothing) 
  & \text{since} \; q_0 \in Q_E\\
  & = \{q_0\} \cup \eDFS (q_1, \{q_0\}) 
              \cup \eDFS (q_4, \{q_0\})
  & \text{by eq.~\ref{eDFS_2}}\\
  &= \{q_0\} \cup \biggl( \{q_1\} \cup \bigcup_{p \in
                \delta_E(q_1,\epsilon)}{\eDFS (p, \{q_0,
                q_1\})}\biggr)
  & \text{by eq.~\ref{eDFS_2}}\\
  &\mathrel{\phantom{=}} \phantom{\{q_0\}}{} 
   \cup \biggl( \{q_4\} \cup \bigcup_{p \in
     \delta_E(q_4,\epsilon)}{\eDFS (p, \{q_0, q_4\})}
   \biggr)
  & \text{by eq.~\ref{eDFS_2}}
\end{align*}

\end{frame}

% ------------------------------------------------------------------------
% 
\begin{frame}
\frametitle{\eNFA{}/\(\epsilon\)-closure (cont)}

\begin{align*}
\eClose (q_0)
  &= \{q_0\} \cup \biggl( \{q_1\} \cup \bigcup_{p \in
                \{q_2\}}{\eDFS (p, \{q_0, q_1\})}\biggr)\\
  &\mathrel{\phantom{=}} \phantom{\{q_0\}}{} 
   \cup \biggl( \{q_4\} \cup \bigcup_{p \in \varnothing}{\eDFS (p,
     \{q_0, q_4\})} \biggr)\\
  &= \{q_0\} \cup (\{q_1\} \cup \eDFS (q_2, \{q_0, q_1\}))
     \cup (\{q_4\} \cup \varnothing)\\
  &= \{q_0, q_1, q_4\} \cup \eDFS (q_2, \{q_0, q_1\})\\
  &= \{q_0, q_1, q_4\} \cup \biggl(\{q_2\} \cup \bigcup_{p \in
       \delta_E(q_2,\epsilon)}{\eDFS (p, \{q_0, q_1,
       q_2\})}\biggr)
\end{align*}

\end{frame}

% ------------------------------------------------------------------------
% 
\begin{frame}
\frametitle{\eNFA{}/\(\epsilon\)-closure (cont)}

\begin{align*}
\eClose (q_0)
  &= \{q_0, q_1, q_4\} \cup \biggl(\{q_2\} \cup \bigcup_{p \in
     \{q_1, q_3\}}{\eDFS (p, \{q_0, q_1, q_2\})}\biggr)\\
  &= \{q_0, q_1, q_2, q_4\} \cup \eDFS (q_1, \{q_0, q_1,
     q_2\})\\
  &\mathrel{\phantom{=}} \phantom{\{q_0,q_2,q_2,q_4\}} \cup \eDFS
     (q_3, \{q_0, q_1, q_2\})\\
  &= \{q_0, q_1, q_2, q_4\} \cup \varnothing \qquad\quad
     \text{by eq.~\ref{eDFS_1}, since} \; q_1 \in \{q_0, q_1, q_2\}\\
  &\mathrel{\phantom{=}} \;\,
   \cup \biggl(\{q_3\} \cup \bigcup_{p \in \delta_E(q_3,
     \epsilon)}{\eDFS (p, \{q_0, q_1, q_2, q_3\})}\biggr)
   \quad \text{by eq.~\ref{eDFS_2}}\\
  &= \{q_0, q_1, q_2, q_3, q_4\} \cup \bigcup_{p \in
     \varnothing}{\eDFS (p, \{q_0, q_1, q_2, q_3\})}\\
  &= \{q_0, q_1, q_2, q_3, q_4\}
\end{align*}

\end{frame}

% ------------------------------------------------------------------------
% 
\begin{frame}
\frametitle{\eNFA{}/\(\epsilon\)-closure (cont)}

It is useful to extend \(\eClose\) to sets of states, not just
states. 

\bigskip

Let us note \(\eCloseSet\) this extension, which we can easily
define as
\[
\eCloseSet (Q) = \bigcup_{q \in Q}{\eClose (q)}
\]
for any subset \(Q \subseteq Q_E\) where the \eNFA is \({\cal E} =
(Q_E, \Sigma, \delta_E, q_E, F_E)\).

\end{frame}

% ------------------------------------------------------------------------
% 
\begin{frame}
\frametitle{\eNFA{}/\(\epsilon\)-closure /
  Optimisation}

\label{enfa_closure}

Compute the \(\epsilon\)-closure of \(q_0\) in the following \eNFA
\(\cal E\):
\begin{center}
\includegraphics[bb=65 618 315 721]{enfa_closure}
\end{center}
where the sub-\eNFA \(\cal E'\) contains only \(\epsilon\)-transitions
and all its \(Q'\) states are accessible from \(q_3\).

\end{frame}

% ------------------------------------------------------------------------
% 
\begin{frame}
\frametitle{\eNFA{}/\(\epsilon\)-closure/Optimisation (cont)}

\begin{align*}
   \eClose (q_0) 
&= \eDFS (q_0, \varnothing)\\
&= \{q_0\} \cup \eDFS (q_1, \{q_0\}) \cup \eDFS (q_2, \{q_0\})\\
&= \{q_0\} \cup (\{q_1\} \cup \eDFS (q3, \{q_0, q_1\}))\\
&\mathrel{\phantom{=}} \phantom{\{q_0\}}{} \cup (\{q_2\} \cup \eDFS
   (q_3, \{q_0, q_2\}))\\
&= \{q_0, q_1, q_2\} \cup \eDFS (q3, \{q_0, q_1\}) \cup \eDFS
   (q_3, \{q_0, q_2\})\\
&= \{q_0, q_1, q_2, q_3,\} \cup (\{q_3\} \cup Q') \cup (\{q_3\} \cup Q')\\
&= \{q_0, q_1, q_2, q_3,\} \cup Q'
\end{align*}
We compute \(\{q_3\} \cup Q'\) two times, that is, we traverse two
times \(q_3\) and all the states of \(\cal E'\), which can be
inefficient if \(Q'\) is big.

\end{frame}

% ------------------------------------------------------------------------
% 
\begin{frame}
\frametitle{\eNFA{}/\(\epsilon\)-closure/Optimisation (cont)}

The way to avoid duplicating traversals is to change the definitions
of \(\eClose\) and \(\eCloseSet\). 

Dually, we need a new definition of \(\eDFS\) and create function
\(\eDFSset\) which is similar to \(\eDFS\) except that it applies to
set of states instead of one state:
\begin{align*}
   \eClose (q) 
&= \eDFS (q, \varnothing)
&& \text{if} \; q \in Q_E\\
   \eCloseSet (Q)
&= \eDFSset (Q, \varnothing)
&& \text{if} \; Q \subseteq Q_E
\end{align*}
We interpret \(Q'\) in \(\eDFS (q, Q')\) and \(\eDFSset (Q, Q')\) as
the set of states that have already been visited in the depth-first
search.

\bigskip

Variables \(q\) and \(Q\) denote, respectively, a state and a set of
states that have to be explored.

\end{frame}

% ------------------------------------------------------------------------
% 
\begin{frame}
\frametitle{\eNFA{}/\(\epsilon\)-closure/Optimisation (cont)}

In the first definition we computed the \emph{new reachable states},
but in the new one we compute the \emph{currently reached
  states}. Then let us redefine \(\eDFS\) this way:
\begin{align*}
  \eDFS (q, Q') 
&= Q'
&& \text{if} \; q \in Q' \tag{1'}\\
  \eDFS (q, Q')
&= \eDFSset (\delta_E (q, \epsilon), Q' \cup \{q\})
&& \text{if} \; q \not\in Q' \tag{2'}
\end{align*}
Contrast with the first definition
\begin{align*}
   \eDFS (q, Q') 
&= \varnothing 
& \text{if} \; q \in Q' \tag{1} \label{empty}\\
  \eDFS (q, Q') 
&= \{q\} \quad \cup \bigcup_{p \in \delta_E(q,\epsilon)}{\eDFS (p, Q'
    \cup \{q\})}
& \text{if} \; q \not\in Q' \tag{2} \label{parallel}
\end{align*}
Hence, in~\eqref{empty} we return \(\varnothing\) because there is no
new state, i.e., none not already in \(Q'\), whereas in (1') we return
\(Q'\) itself.

\end{frame}

% ------------------------------------------------------------------------
% 
\begin{frame}
\frametitle{\eNFA{}/\(\epsilon\)-closure/Optimisation (cont)}

The new definition of \(\eDFSset\) is not more difficult than the
first one:
\begin{align}
   \eDFSset (\varnothing, Q')
&= Q'\label{eDFSset:1}\\
   \eDFSset (\{q\} \cup Q, Q')
&= \eDFSset (Q, \eDFS (q, Q')) 
& \text{if} \; q \not\in Q \label{eDFSset:2}
\end{align}
Notice that the definitions of \(\eDFS\) and \(\eDFSset\) are
\textbf{mutually recursive}, i.e., they depend on each other.

\bigskip

In~\eqref{parallel} we traverse states in \emph{parallel} (consider
the union operator), starting from each element in \(\delta_E (q,
\epsilon)\), whereas in (2') and (\ref{eDFSset:2}), we traverse them
\emph{sequentially} so we can use the information collected (currently
reached states) in the previous searches.

\end{frame}

% ------------------------------------------------------------------------
% 
\begin{frame}
\frametitle{\eNFA{}/\(\epsilon\)-closure/Optimisation (cont)}

Coming back to our example page~\ref{enfa_closure}, we find
\begin{align*}
   \eClose (q_0) 
&= \eDFS (q_0, \varnothing)
&& q_0 \in Q_E\\
&= \eDFSset (\{q_1, q_2\}, \{q_0\})
&& \text{by eq.~(2')}\\
&= \eDFSset (\{q_2\}, \eDFS (q_1, \{q_0\}))
&& \text{by eq.~(4)}\\
&= \eDFSset (\{q_2\}, \eDFSset (\{q_3\}, \{q_0,
   q_1\}))
&& \text{by eq.~(2')}\\
&= \eDFSset (\{q_2\}, \eDFSset (\varnothing, \eDFS (q_3, \{q_0, q_1\})))
&& \text{by eq.~(4)}\\
&= \eDFSset (\{q_2\}, \eDFS (q_3, \{q_0, q_1\}))
&& \text{by eq.~(3)}\\
&= \eDFSset (\{q_2\}, \{q_0, q_1, q_3\} \cup Q')\\
&= \eDFSset (\varnothing, \eDFS (q_2, \{q_0, q_1, q_3\} \cup Q'))
&& \text{by eq.~(4)}
\end{align*}

\end{frame}

% ------------------------------------------------------------------------
% 
\begin{frame}
\frametitle{\eNFA{}/\(\epsilon\)-closure/Optimisation (cont)}

\begin{align*}
   \eClose (q_0) 
&= \eDFS (q_2, \{q_0, q_1, q_3\} \cup Q')
&& \text{by eq.~(3)}\\
&= \eDFSset (\{q_3\}, \{q_0, q_1, q_2, q_3\} \cup Q')
&& \text{by eq.~(2')}\\
&= \eDFSset (\varnothing, \eDFS (q_3, \{q_0, q_1, q_2, q_3\} \cup Q'))
&& \text{by eq.~(4)}\\
&= \eDFS (q_3, \{q_0, q_1, q_2, q_3\} \cup Q')
&& \text{by eq.~(3)}\\
&= \{q_0, q_1, q_2, q_3\} \cup Q'
&& \text{by eq.~(1')}\\
\end{align*}
The important thing here is that we did not compute (traverse) several
times \(Q'\). Note that some equations can be used in a different
order and \(q\) can be chosen arbitrarily in equation (4), but the
result is always the same.

\end{frame}

% ------------------------------------------------------------------------
% 
\begin{frame}
\frametitle{Extended transition functions for {\eNFA}s}

The \(\epsilon\)-closure allows to explain how a \eNFA recognises or
rejects a given input. Let \({\cal E} = (Q_E, \Sigma, \delta_E, q_0,
F_E)\).

\bigskip

We want \(\hat{\delta}_E (q, w)\) be the set of states reachable from
\(q\) along a path whose labels, when concatenated, for the string
\(w\). The difference here with NFA's is that several \(\epsilon\) can
be present along this path, despite not contributing to \(w\).
For all state \(q \in Q_E\), let
\begin{align*}
   \hat{\delta}_E (q, \epsilon)
&= \eClose (q)\\
   \hat{\delta}_E (q, wa)
&= \eCloseSet \Biggl(\bigcup_{p \in \hat{\delta}_E (q, w)}{\delta_N (p,
     a)}\Biggr)
&& \text{for all} \; a \in \Sigma, w \in \Sigma^{*}
\end{align*}
This definition is based on the regular identity \(wa =
((w\epsilon^{*})a)\epsilon^{*}\).

\end{frame}

% ------------------------------------------------------------------------
% 
\begin{frame}[containsverbatim]
\frametitle{Extended transition functions for {\eNFA}s/Example}

Let us consider again the \eNFA recognising natural and decimal
numbers, at page~\pageref{enfa_num}, and compute the states reached on
the input \verb+5.6+:
\begin{align*}
   \hat{\delta}_E (q_0, \epsilon)
&= \eClose (q_0) = \{q_0, q_1\}\\
   \hat{\delta}_E (q_0, \mathtt{5})
&= \eCloseSet \Biggl( \bigcup_{p \in \hat{\delta}_E (q_0,
   \epsilon)}{\delta_N (p, \mathtt{5})} \Biggr)\\
&= \eCloseSet (\delta_N (q_0, \mathtt{5}) \cup \delta_N (q_1,
   \mathtt{5})) = \eCloseSet (\varnothing \cup \{q_1, q_4\})\\
&= \{q_1, q_3, q_4, q_5\}\\
   \hat{\delta}_E (q_0, \mathtt{5.})
&= \eCloseSet \Biggl( \bigcup_{p \in \hat{\delta}_N (q_0,
   \mathtt{5})}{\delta_N (p, \mathtt{.})} \Biggr)\\
&= \eCloseSet (\delta_N (q_1, \mathtt{.}) \cup \delta_N (q_3,
   \mathtt{.}) \cup \delta_N (q_4, \mathtt{.}) \cup \delta_N (q_5,
   \mathtt{.}))
\end{align*}

\end{frame}

% ------------------------------------------------------------------------
% 
\begin{frame}[containsverbatim]
\frametitle{Extended transition functions for {\eNFA}s/Example (cont)}

\begin{align*}
   \hat{\delta}_E (q_0, \mathtt{5.})
&= \eCloseSet (\{q_2\} \cup \varnothing \cup \varnothing \cup
   \varnothing)
 = \{q_2\}\\
   \hat{\delta}_N (q_0, \mathtt{5.6})
&= \eCloseSet \Biggl( \bigcup_{p \in \hat{\delta}_E (q_0,
     \mathtt{5.})}{\delta_N (p, \mathtt{6})} \Biggr)\\
&= \eCloseSet (\delta_N (q_2, \mathtt{6}))\\
&= \eCloseSet (\{q_3\})\\
&= \{q_3, q_5\} \ni q_5
\end{align*}
Since \(q_5\) is a final state, the string \verb+5.6+ is recognised as
a number.

\end{frame}

% ------------------------------------------------------------------------
% 
\begin{frame}
\frametitle{Subset construction for {\eNFA}s}

Let us present now how to construct a DFA from a \eNFA such as both
recognise the same language.

\bigskip

The method is a variation of the subset construction we presented for
NFA: we must take into account the states reachable through
\(\epsilon\)-transitions, with help of \(\epsilon\)-closures.

\end{frame}

% ------------------------------------------------------------------------
% 
\begin{frame}
\frametitle{Subset construction for {\eNFA}s (cont)}

Assume that \({\cal E} = (Q, \Sigma, \delta, q_0, F)\) is an
\eNFA. Let us define as follows the equivalent DFA \({\cal D} = (Q_D,
\Sigma, \delta_D, q_D, F_D)\).
\begin{enumerate}

  \item \(Q_D\) is the set of subsets of \(Q_E\). More precisely, all
  accessible states of \(\cal D\) are \(\epsilon\)-closed subsets of
  \(Q_E\), i.e., sets \(Q \subseteq Q_E\) such as \(Q = \eClose(Q)\).

  \item \(q_D = \eClose (q_0)\), i.e., we get the start state
  of \(\cal D\) by \(\epsilon\)-closing the set made of only the start
  state of \(\cal E\).

  \item \(F_D\) is those sets of states that contain at least one
  final state of \(\cal E\), that is to say \(F_D = \{Q \, \lvert
  \, Q \in Q_D \; \text{and} \; Q \cap F_E \neq \varnothing\}\).

  \item For all \(a \in \Sigma\) and \(Q \in Q_D\), let
  \(
  \delta_D (Q, a) = \eCloseSet \biggl(\bigcup_{q \in Q}{\delta_E (q,
    a)}\biggr)
  \)

 \end{enumerate}

\end{frame}

% ------------------------------------------------------------------------
% 
\begin{frame}
\frametitle{Subset construction for {\eNFA}s/Example}

Let us consider again the \eNFA page~\pageref{enfa_num}. Its transition
table is
\[
\begin{array}{r@{}l||c|c|c|c|c}
\multicolumn{2}{c||}{\cal E} & + & - & 0, \ldots, 9 & \mathtt{.} &
\epsilon\\
\hhline{==::=====}
\rightarrow & q_0 & \{q_1\} & \{q_1\} & \varnothing & \varnothing 
                  & \{q_1\}\\
            & q_1 & \varnothing  & \varnothing & \{q_1, q_4\} 
                  & \{q_2\} & \varnothing\\
            & q_2 & \varnothing  & \varnothing & \{q_3\} & \varnothing
                  & \varnothing\\
            & q_3 & \varnothing & \varnothing & \{q_3\} & \varnothing
                  & \{q_5\}\\
            & q_4 & \varnothing & \varnothing & \varnothing  
                  & \varnothing & \{q_3\}\\
         \# & q_5 & \varnothing & \varnothing & \varnothing 
                  & \varnothing & \varnothing 
\end{array}
\]

\end{frame}

% ------------------------------------------------------------------------
% 
\begin{frame}
\frametitle{Subset construction for {\eNFA}s/Example (cont)}

By applying the subset construction to this \eNFA, we get the table
\[
\begin{array}{r@{}l||c|c|c|c}
\multicolumn{2}{c||}{\cal D} & + & - & 0, \ldots, 9 & \mathtt{.}\\
\hhline{==::====}
\rightarrow 
   & \{q_0,q_1\} 
             & \{q_1\} & \{q_1\} & \{q_1, q_3, q_4, q_5\} & \{q_2\}\\
   & \{q_1\} & \varnothing  & \varnothing & \{q_1, q_3, q_4, q_5\} 
             & \{q_2\}\\
\# & \{q_1, q_3, q_4, q_5\} 
             & \varnothing  & \varnothing & \{q_1, q_3, q_4, q_5\} 
             & \{q_2\}\\
   & \{q_2\} & \varnothing & \varnothing & \{q_3, q_5\} & \varnothing\\
\# & \{q_3, q_5\}
             & \varnothing & \varnothing & \{q_3, q_5\} & \varnothing
\end{array}
\]

\end{frame}

% ------------------------------------------------------------------------
% 
\begin{frame}
\frametitle{Subset construction for {\eNFA}s/Example (cont)}

Let us rename the states of \(\cal D\) and get rid of the empty sets:
\[
\begin{array}{r@{}l||c|c|c|c}
\multicolumn{2}{c||}{\cal D} & + & - & 0, \ldots, 9 & \mathtt{.}\\
\hhline{==::====}
\rightarrow 
   & A & B & B & C & D\\
   & B &   &   & C & D\\
\# & C &   &   & C & D\\
   & D &   &   & E &\\
\# & E &   &   & E &
\end{array}
\]

\end{frame}

% ------------------------------------------------------------------------
% 
\begin{frame}
\frametitle{Subset construction for {\eNFA}s/Example (cont)}

The transition diagram of \(\cal D\) is therefore
\begin{center}
\includegraphics[bb=49 610 208 758]{dfa_from_enfa_num}
\end{center}

\end{frame}
