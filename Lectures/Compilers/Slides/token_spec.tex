%%-*-latex-*-

% ------------------------------------------------------------------------
% 
\begin{frame}
\frametitle{Lexer}

The lexical analyser is the first phase of a compiler. Its main task
is to read the input characters and produce a sequence of tokens that
the syntax analyser uses.
\begin{center}
\includegraphics[bb=71 640 368 721]{lexer_parser}
\end{center}
Upon receiving a request for a token (\emph{get token}) from the
parser, the lexical analyser reads input characters until a lexeme is
identified and returned to the parser together with the corresponding
token.

\end{frame}

% ------------------------------------------------------------------------
% 
\begin{frame}
\frametitle{Lexer (cont)}

Usually, a lexical analyser is in charge of
\begin{itemize}

  \item stripping out from the source program \textbf{comments} and
  \textbf{white spaces}, in the form of blank, tabulation and newline
  characters;

  \item keeping trace of the position of the lexemes in the source
  program, so the error handler can refer to exact positions in
  error messages.

\end{itemize}

\end{frame}

% ------------------------------------------------------------------------
% 
\begin{frame}[containsverbatim]
\frametitle{Lexer/Tokens, lexemes, patterns}

A \textbf{token} is a set of strings which are interpreted in the same
way, for a given source language. For instance, \tokenName{id} is a
token denoting the set of all possible identifiers.

\bigskip

A \textbf{lexeme} is a string belonging to a token. For example,
\verb+5.4+ is a lexeme of the token \tokenName{num}.

\bigskip

The tokens are defined by means of \textbf{patterns}. A pattern is a
kind of compact rule describing all the lexemes of a token. A pattern
is said to \emph{match} each lexeme in the token.

\bigskip

For example, in the \Pascal statement
\begin{verbatim}
const pi = 3.14159;
\end{verbatim}
the substring \texttt{pi} is a lexeme for the token \tokenName{id}
(\emph{identifier}).

\end{frame}

% ------------------------------------------------------------------------
% 
\begin{frame}
\frametitle{Lexer/Tokens, lexemes, patterns (cont)}

\label{tokens_lexemes}

{\small
\begin{center}
\begin{tabular}{>{\bfseries\sffamily}l|>{\tt}l|l}
\hline\hline
  \multicolumn{1}{c|}{\textsc{Token}}
& \multicolumn{1}{c|}{\textsc{Sample lexemes}} 
& \multicolumn{1}{c}{\textsc{Informal pattern}}\\
\hline
id       & pi count D2 \dots  & letter followed by letters and digits\\
relop    & < <= = >= >   & \texttt{<} or \texttt{<=} or \texttt{<} or \texttt{=} or \texttt{>=} or \texttt{>}\\
const    & const         & \texttt{const}\\
if       & if            & \texttt{if}\\
num      & 3.14 4 .2E2 \ldots  & any numeric constant\\
literal  & "message" ""  \ldots & any characters between \texttt{"} and \texttt{"} except \texttt{"}\\
\hline
\end{tabular}
\end{center}
}

\end{frame}

% ------------------------------------------------------------------------
% 
\begin{frame}
\frametitle{Lexer/Tokens, lexemes, patterns (cont)}

Most recent programming languages distinguish a finite set of strings
that match the identifiers but are not part of the identifier token:
the \textbf{keywords}.

\bigskip

For example, in \Ada, \texttt{function} is a keyword and, as such, is
not a valid identifier.

\bigskip

In C, \texttt{int} is a keyword and, as such, cannot be used an
identifier (e.g. to declare a variable).

\bigskip

Nevertheless, it is common \textbf{not} to create explicitly
a \tokenName{keyword} token and let each keyword lexeme be the only
one of its own token, as displayed in the table
page~\pageref{tokens_lexemes}.

\end{frame}

% ------------------------------------------------------------------------
% 
\begin{frame}
\frametitle{Specification of tokens}

\textbf{Regular expressions} are an important notation for specifying
patterns. Each pattern matches a set of strings, so regular
expressions will serve as names for sets of strings.

\bigskip

\textbf{Strings and formal languages}

\bigskip

The term \textbf{alphabet} denotes any finite set of symbols. Typical
examples of symbols are letters and digits. The set \(\{0, 1\}\) is
the \emph{binary alphabet}. \textsc{ascii} is an example of computer
alphabet.

\end{frame}

% ------------------------------------------------------------------------
% 
\begin{frame}
\frametitle{Specification of tokens (cont)}

A \textbf{string} over some alphabet is a finite sequence of symbols
drawn from that alphabet. The terms \textbf{sentence}
and \textbf{word} are often used as synonyms. 

\bigskip

The length of string \(s\), usually noted \(\strlen{s}\), is the
number of occurrences of symbols in \(s\). For
example, \texttt{banana} is a string of length six. The \textbf{empty
string}, denoted \(\varepsilon\), is a special string of length zero.

\end{frame}

% ------------------------------------------------------------------------
% 
\begin{frame}
\frametitle{Specification of tokens/Strings and formal languages (cont)}

\label{table1}

\setlength\tymin{70pt}

\begin{center}
\begin{tabulary}{\linewidth}{C|L}
\hline\hline
  \textsc{Term}
& \multicolumn{1}{c}{\textsc{Informal definition}}\\
\hline
  \emph{prefix} of \(s\)
& A string obtained by removing zero or more trailing symbols of
  string \(s\); e.g. \texttt{ban} is a prefix of \texttt{banana}.\\
\hline
  \emph{suffix} of \(s\)
& A string formed by deleting zero or more of the leading symbols
  of \(s\); e.g. \texttt{nana} is a suffix of \texttt{banana}.\\
\hline
  \emph{substring} of \(s\)
& A string obtained by deleting a prefix and a suffix from \(s\);
e.g. \texttt{nan} is a substring a \texttt{banana}. Every prefix and
every suffix of \(s\) is a substring \(s\), but not every substring
of \(s\) is a prefix or a suffix of \(s\). For every string \(s\),
both \(s\) and \(\varepsilon\) are prefixes, suffixes and substrings
of \(s\).\\
\hline
\end{tabulary}
\end{center}

\end{frame}

% ------------------------------------------------------------------------
% 
\begin{frame}
\frametitle{Specification of tokens/Strings and formal languages (cont)}

\label{table2}

\setlength\tymin{70pt}

\begin{center}
\begin{tabulary}{\linewidth}{C|L}
\hline\hline
  \textsc{Term}
& \multicolumn{1}{c}{\textsc{Informal definition}}\\
\hline
  \emph{proper} prefix, suffix or substring of \(s\)
& Any non-empty string \(x\) that is, respectively, a prefix, suffix,
  substring of \(s\) such that \(s \neq x\); e.g. \(\varepsilon\)
  and \texttt{banana} are \textbf{not} proper prefixes
  of \texttt{banana}.\\
\hline
  \emph{subsequence} of \(s\)
& Any string formed by deleting zero or more not necessarily
  contiguous symbols from \(s\); e.g. \texttt{baaa} is a subsequence
  of \texttt{banana}.\\
\hline
\end{tabulary}
\end{center}

\end{frame}

% ------------------------------------------------------------------------
% 
\begin{frame}
\frametitle{Specification of tokens/Strings and formal languages (cont)}

The term \textbf{language} denotes any set of strings over some fixed
alphabet. 

\bigskip

The \textbf{empty set}, noted \(\varnothing\), or \(\{\varepsilon\}\),
the set containing only the empty word are languages. The set of valid
C programs is an infinite language.

\bigskip

If \(x\) and \(y\) are strings, then the \textbf{concatenation}
of \(x\) and \(y\), written \(xy\) or \(x \cdot y\), is the string
formed by appending \(y\) to \(x\). 

\bigskip

For example, if \(x = \texttt{dog}\) and \(y = \texttt{house}\),
then \(xy = \texttt{doghouse}\).

\bigskip

The empty string is the identity element under
concatenation: \(s \varepsilon = \varepsilon s = s\).

\end{frame}

% ------------------------------------------------------------------------
% 
\begin{frame}
\frametitle{Specification of tokens/Strings and formal languages (cont)}

If we think of concatenation as a product, we can define string
exponentiation as follows.
\begin{itemize}

  \item \(s^0 = \varepsilon\)

  \item \(s^{n} = s^{n-1} s\), if \(n > 0\).

\end{itemize}
Since \(\varepsilon s = s\), \(s^1 = s\), then \(s^2 = ss\), \(s^3 =
sss\) etc.

\end{frame}

% ------------------------------------------------------------------------
% 
\begin{frame}
\frametitle{Specification of tokens/Strings and formal languages (cont)}

We can now revisit the definitions we gave in table
page~\pageref{table1} and~\pageref{table2} using a formal
notation. Let \(L\) be the language under consideration.
\begin{center}
\begin{tabular}{c|>{$}c<{$}}
\hline\hline
  \textsc{Term}
& \multicolumn{1}{c}{\textsc{Formal definition}}\\
\hline
  \(x\) is a \emph{prefix} of \(s\)
& \exists y \in L. s = x y\\
\hline
  \(x\) is a \emph{suffix} of \(s\)
& \exists y \in L.s = y x\\
\hline
  \(x\) is a \emph{substring} of \(s\)
& \exists u, v \in L. s = u x v\\
\hline
  \(x\) is a \emph{proper prefix} of \(s\)
& \exists y \in L. y \neq \varepsilon \; \text{and} \; s = x y\\
\hline
  \(x\) is a \emph{proper suffix} of \(s\)
& \exists y \in L. y \neq \varepsilon \; \text{and} \; s = y x\\
\hline
  \(x\) is a \emph{proper substring} of \(s\)
& \exists u, v \in L. u v \neq \varepsilon \; \text{and} \; s = u x
v\\
\hline
\end{tabular}
\end{center}

\end{frame}

% ------------------------------------------------------------------------
% 
\begin{frame}
\frametitle{Specification of tokens/Operations on languages}

It is possible to define operation in languages. For lexical analysis,
we are interested mainly in \textbf{union}, \textbf{concatenation}
and \textbf{closure}. Let \(L\) and \(M\) be two languages.
\begin{center}
\begin{tabular}{c|>{$}c<{$}}
\hline\hline
  \textsc{Operation}
& \multicolumn{1}{c}{\textsc{Formal definition}}\\
\hline
  \emph{union} of \(L\) and \(M\)
& L \cup M = \{ s \mid s \in L \; \text{or} \; s \in M\}\\
  \emph{concatenation} of \(L\) and \(M\)
& L M = \{s t \mid s \in L \; \text{and} \; t \in M\}\\
  \emph{Kleene closure} of \(L\)
& L^{*} = \bigcup_{i=0}^{\infty}{L^{i}} \; \text{where} \; L^0
  = \{\varepsilon\}\\
  \emph{positive closure} of \(L\)
& L^{+} = \bigcup_{i=1}^{\infty}{L^{i}}\\
\hline
\end{tabular}
\end{center}
In other words, \(L^{*}\) means ``zero or more concatenations
of \(L\)'', and \(L^{+}\) means ``one or more concatenations
of \(L\).'' 

%Note that \(L^{*} = L^0 \cup \bigcup_{i=1}^{\infty}{L^{i}} = \{\varepsilon\} \cup L^{+}\)

\end{frame}

% ------------------------------------------------------------------------
% 
\begin{frame}
\frametitle{Specification of tokens/Operations on languages/Examples}

Let \(L
= \{\texttt{A}, \texttt{B}, \ldots, \texttt{Z}, \texttt{a}, \texttt{b}, \ldots, \texttt{z}\}\)
and \(D = \{\texttt{0}, \texttt{1}, \ldots, \texttt{9}\}\). 
\begin{enumerate}

  \item \(L\) is the alphabet consisting of the set of upper and lower
  case letters and \(D\) is the alphabet of the decimal digits.

  \item Since a symbol is a string of length one, the sets \(L\)
  and \(D\) are finite languages too.

\end{enumerate}
These two ways of considering \(L\) and \(D\) and the operations on
languages allow us to create new languages from other languages
defined by their alphabet.

\bigskip

Here are some examples of new languages created from \(L\) and \(D\):
\begin{itemize}

  \item \(L \cup D\) is the language of letters and digits.

  \item \(L D\) is the language whose words consist of a letter followed
  by a digit.

\end{itemize}

\end{frame}

% ------------------------------------------------------------------------
% 
\begin{frame}
\frametitle{Specification of tokens/Operations on languages/Examples
  (cont)}

\begin{itemize}

  \item \(L^{4}\) is the language whose words are four-letter strings.

  \item \(L^{*}\) is the language made up on the alphabet \(L\),
  i.e. the set of all strings of letters, including the empty
  string \(\varepsilon\).

  \item \(L(L \cup D)^{*}\) is the language whose words consist of
  letters and digits and beginning with a letter.

  \item \(D^{+}\) is the language whose words are made of one or more
  digits, i.e. the set of all decimal integers.

\end{itemize}

\end{frame}
