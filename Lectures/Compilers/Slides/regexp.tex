%%-*-latex-*-

% ------------------------------------------------------------------------
% 
\begin{frame}
\frametitle{Regular expressions}

In Pascal, an identifier is a letter followed by zero or more letters
or digits, that is, and identifier is a member of the set defined
by \(L(L \cup D)^{*}\).

\bigskip

The notation we introduced so far is comfortable for mathematics but
not for computers. Let us introduce another notation,
called \textbf{regular expressions}, for describing the same languages
and define its meaning in terms of the mathematical notation.

\bigskip

With this notation, we might define \Pascal identifiers as
\begin{center}
\term{letter} \lparen\term{letter} \disj \term{digit}\rparen\kleene
\end{center}
where the vertical bar means ``or'', the parentheses group
subexpressions, the star means ``zero or more instances of'' the
previous expression and juxtaposition means concatenation.

\end{frame}

% ------------------------------------------------------------------------
% 
\begin{frame}
\frametitle{Regular expressions (continued)}

A regular expression \(r\) is built up out of simpler regular
expressions using a set of rules, as follows. Let \(\Sigma\)
be an alphabet and \(L(r)\) the language denoted by \(r\).
\begin{enumerate}

   \item \(\epsilon\) is a regular expression that denotes
     \(\{\varepsilon\}\). \label{regexp_empty}

   \item If \(a \in \Sigma\), then \(a\) is a regular expression that
     denotes \(\{a\}\). This is ambiguous: \(a\) can denote a
     language, a word or a letter --- it depends on the
     context. \label{regexp_sym}

   \item Assume \(r\) and \(s\) denote the languages \(L(r)\) and
     \(L(s)\); \(a\) denotes a letter. Then \label{regexp_rec}
   \begin{enumerate}
    
     \item \(r\) \disj \(s\) is a regular expression
     denoting \(L(r) \cup L(s)\).

     \item \(r s\) is a regular expression denoting \(L(r) L(s)\).

     \item \(r\)\kleene{} is a regular expression
     denoting \((L(r))^{*}\).

%%     \item \lparen\(r\)\rparen{} is a regular expression
%%     denoting \(L(r)\).

     \item \(\overline{a}\) is a regular expression denoting
       \(\Sigma\backslash \{a\}\).

   \end{enumerate}

\end{enumerate}

\end{frame}

% ------------------------------------------------------------------------
% 
\begin{frame}
\frametitle{Regular expressions (continued)}

A language described by a regular expression is a \textbf{regular
language}.

\bigskip

Rules~\ref{regexp_empty} and~\ref{regexp_sym} form the base of the
definition. Rule~\ref{regexp_rec} provides the inductive step.

\bigskip

Unnecessary parentheses can be avoided in regular expressions if
\begin{itemize}

  \item the unary operator \kleene{} has the highest precedence and
  is left associative,

  \item concatenation has the second highest precedence and is left
  associative,

  \item \disj{} has the lowest precedence and is left associative.

\end{itemize}
Under those conventions, \lparen\(a\)\rparen{} \disj
\lparen\lparen\(b\)\rparen\kleene\lparen\(c\)\rparen\rparen{} is
equivalent to \(a\) \disj \(b\)\kleene\(c\).

\bigskip

Both expressions denote the language containing either the string
\(a\) or zero or more \(b\)'s followed by one \(c\): \(\{a, c, bc,
bbc, bbbc, \dots\}\).

\end{frame}

% ------------------------------------------------------------------------
% 
\begin{frame}
\frametitle{Regular expressions/Examples}

\begin{itemize}

  \item The regular expression \(a\) \disj \(b\) denotes the set
    \(\{a, b\}\).

  \item The regular expression \lparen\(a\) \disj
    \(b\)\rparen\lparen\(a\) \disj \(b\)\rparen{} denotes \(\{aa, ab,
    ba, bb\}\), the set of all strings of \(a\)'s and \(b\)'s of
    length two. Another regular expression for the set is \(aa\) \disj
    \(ab\) \disj \(ba\) \disj \(bb\).

  \item The regular expression \(a\)\kleene{} denotes the set of all
    strings of zero or more \(a\)'s, i.e. \(\{\varepsilon, a, aa, aaa,
    \dots\}\).

  \item The regular expression \lparen\(a\) \disj
    \(b\)\rparen\kleene{} denotes the set of all strings containing
    zero of more instances of an \(a\) or \(b\), that is the language
    of all words made of \(a\)'s and \(b\)'s. Another expression is
    \lparen\(a\)\kleene\(b\)\kleene\rparen\kleene.

\end{itemize}

\end{frame}

% ------------------------------------------------------------------------
% 
\begin{frame}
\frametitle{Regular expressions/Algebraic laws}

If two regular expressions \(r\) and \(s\) denote the same language,
we say \(r\) and \(s\) are \textbf{equivalent} and write \(r =
s\).
\begin{center}
\begin{tabular}{c|l}
\hline\hline
  \textsc{Law}
& \multicolumn{1}{c}{\textsc{Description}}\\
\hline
  \(r\) \disj \(s\) = \(s\) \disj \(r\)
& \disj is commutative\\
\hline
  \(r\) \disj \lparen\(s\) \disj \(t\)\rparen{}
  = \lparen\(r\) \disj \(s\)\rparen{} \disj \(t\)
& \disj is associative\\
\hline
  \lparen\(rs\)\rparen \(t\) = \(r\)\lparen\(st\)\rparen
& concatenation is associative\\
\hline
  \(r\)\lparen\(s\) \disj \(t\)\rparen{} = \(rs\) \disj \(rt\)
& concatenation distributes over \disj\\
  \lparen\(s\) \disj \(t\)\rparen \(r\) = \(sr\) \disj \(tr\)
&\\
\hline
  \(\epsilon r = r\) 
& \(\epsilon\) is the identity element\\
  \(r \epsilon = r\)
& for the concatenation\\
\hline
\end{tabular}
\end{center}

\end{frame}

% ------------------------------------------------------------------------
% 
\begin{frame}
\frametitle{Regular expressions/Algebraic laws (cont)}

\begin{center}
\begin{tabular}{c|l}
\hline\hline
  \textsc{Law}
& \multicolumn{1}{c}{\textsc{Description}}\\
\hline
  \(r\)\kleene\kleene = \(r\)\kleene
& Kleene closure is idempotent\\
\hline
  \(r\)\kleene = \(r\)\plus \disj \(\epsilon\)
& Kleene closure and positive closure\\
  \(r\)\plus = \(r r\)\kleene
& are closely linked\\
\hline
\end{tabular}
\end{center}

\end{frame}

% ------------------------------------------------------------------------
% 
\begin{frame}
\frametitle{Regular definitions}

It is convenient to give names to regular expressions and define new
regular expressions using these names as if they were symbols.

\bigskip

\begin{columns}

  \column{0.5\textwidth} If \(\Sigma\) is an alphabet, then a
  \textbf{regular definition} is a series of definitions of the form
  \begin{align*}
    d_1 &\rightarrow r_1\\
    d_2 &\rightarrow r_2\\
    &\cdots\\
    d_n &\rightarrow r_n
  \end{align*}

  \column{0.5\textwidth} where each \(d_i\) is a distinct name and
  each \(r_i\) is a regular expression over the alphabet \(\Sigma \cup
  \{d_1, d_2, \dots, d_{i-1}\}\), i.e. the basic symbols and the
  previously defined names. The restriction to \(d_j\) such \(j < i\)
  allows to construct a regular expression over \(\Sigma\) only by
  repeatedly replacing all the names in it.

\end{columns}

\end{frame}

% ------------------------------------------------------------------------
% 
\begin{frame}
\frametitle{Regular definitions/Examples}

As we have stated, the set of \Pascal identifiers can be defined by
the regular definitions
\begin{align*}
\term{letter} & \rightarrow \text{\exc{A} \disj \exc{B} \disj
  \ldots \disj \exc{Z} \disj \exc{a} \disj \exc{b} \disj \ldots \disj
  \exc{z}} \\
\term{digit} & \rightarrow \text{\exc{0} \disj \exc{1} \disj \exc{2}
  \disj \exc{3} \disj \exc{4} \disj \exc{5} \disj \exc{6} \disj
  \exc{7} \disj \exc{8} \disj \exc{9}}\\
\term{id} & \rightarrow \text{\term{letter} \lparen\term{letter}
  \disj \term{digit}\rparen\kleene}
\intertext{Unsigned numbers in \Pascal are strings like
\texttt{5280}, \texttt{39.37}, \texttt{6.336E4}
or \texttt{1.894E-4}.}
\term{digit} & \rightarrow \text{ \exc{0} \disj \exc{1} \disj \exc{2}
  \disj \exc{3} \disj \exc{4} \disj \exc{5} \disj \exc{6} \disj
  \exc{7} \disj \exc{8} \disj \exc{9}}\\
\term{digits} & \rightarrow \text{\term{digit} \term{digit}\kleene}\\
\term{optional\_fraction} & \rightarrow \text{\exc{.} \term{digits}
  \disj} \, \epsilon\\
\term{optional\_exponent} & \rightarrow \text{\lparen\exc{E} \lparen
  \exc{+} \disj \exc{-} \disj} \, \epsilon \, \text{\rparen{}
  \term{digits}\rparen{} \disj} \, \epsilon\\
\term{num} & \rightarrow \text{\term{digits} \term{optional\_fraction}
  \term{optional\_exponent}}
\end{align*}

\end{frame}

% ------------------------------------------------------------------------
% 
\begin{frame}
\frametitle{Regular definitions/Shorthands}

Certain constructs occur so frequently in regular expressions that it
is convenient to introduce notational shorthands for them.

\bigskip

\textbf{Zero or one instance.} The unary operator \opt{} means ``zero
or one instance of.'' Formally, by definition, if \(r\) is a regular
expression then \(r\)\opt = \(r\) \disj \(\epsilon\). In other
words, \lparen\(r\)\rparen\opt{} denotes the
language \(L(r) \cup \{\varepsilon\}\).
\begin{align*}
\term{digit} & \rightarrow \text{\exc{0} \disj \exc{1} \disj \exc{2}
  \disj \exc{3} \disj \exc{4} \disj \exc{5} \disj \exc{6} \disj
  \exc{7} \disj \exc{8} \disj \exc{9}}\\
\term{digits} & \rightarrow \text{\term{digit}\plus}\\
\term{optional\_fraction} & \rightarrow \text{\lparen\exc{.}
  \term{digits}\rparen\opt}\\
\term{optional\_exponent} & \rightarrow \text{\lparen\exc{E} \lparen
  \exc{+} \disj \exc{-}\rparen\opt{} \term{digits}\rparen\opt}\\
\term{num} & \rightarrow \text{\term{digits} \term{optional\_fraction}
  \term{optional\_exponent}}
\end{align*}

\end{frame}

% ------------------------------------------------------------------------
% 
\begin{frame}
\frametitle{Regular definitions/Shorthands (cont)}

It is also possible to write:
\begin{align*}
\term{digit} & \rightarrow \text{\exc{0} \disj \exc{1} \disj \exc{2}
  \disj \exc{3} \disj \exc{4} \disj \exc{5} \disj \exc{6} \disj
  \exc{7} \disj \exc{8} \disj \exc{9}}\\
\term{digits} & \rightarrow \text{\term{digit}\plus}\\
\term{fraction} & \rightarrow \text{\exc{.} \term{digits}}\\
\term{exponent} & \rightarrow \text{\exc{E} \lparen \exc{+} \disj
  \exc{-}\rparen\opt{} \term{digits}}\\
\term{num} & \rightarrow \text{\term{digits} \term{fraction}\opt{}
  \term{exponent}\opt}
\end{align*}

\end{frame}

% ------------------------------------------------------------------------
% 
\begin{frame}[containsverbatim]
\frametitle{Regular definitions/Shorthands (cont)}

If we want to specify the characters \verb+?+, \verb+*+, \verb|+|,
\verb+|+, we write them with a preceding backslash, e.g. \verb|\?|,
or between double-quotes, e.g. \verb+"?"+. Then, of course, the
character double-quote must have a backslash: \verb+\"+

\bigskip

It is also sometimes useful to match against end of lines and end of
files: \verb+\n+ stands for the control character ``end of line'' and
\term{\$} is for ``end of file''.

\end{frame}

% ------------------------------------------------------------------------
% 
\begin{frame}[containsverbatim]
\frametitle{Non-regular languages}

Some languages cannot be described by any regular expression.

\bigskip

For example, the language of balanced parentheses cannot be recognised
by any regular expression: \lparen\rparen,
\lparen\lparen\rparen\rparen, \lparen\rparen\lparen\rparen,
\lparen\lparen\lparen\rparen\rparen\lparen\rparen\rparen{} etc.

\bigskip

Another example is the C programming language: it is not a regular
language because it contains embedded blocs between \verb+{+ and
\verb+}+. Therefore, a lexer cannot recognise valid C programs: we
need a parser.

\end{frame}
