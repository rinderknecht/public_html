%%-*-latex-*-

% ------------------------------------------------------------------------
%
\begin{frame}
\frametitle{From regular expressions to \eNFA{s}}

We let behind the regular expressions when we introduced informally
the transition diagrams for the token recognition.

\bigskip

Let us show now that regular expressions, used in lexers to specify
tokens, can be converted to \eNFA{s}, so to DFA. This proves that
\emph{regular languages are recognisable languages}.

\bigskip

Actually, it is possible to prove that any \eNFA can be converted to a
regular expression denoting the same language, but we will not do so.

\bigskip

Therefore, keep in mind that \textbf{regular languages are
  recognisable languages}. In other words, using a regular expression
or a finite automaton is only a matter of convenience.

\end{frame}

% ------------------------------------------------------------------------
%
\begin{frame}
\frametitle{From regular expressions to \eNFA{s} (cont)}

The construction we present here to build an \eNFA from a regular
expression is called \textbf{Thompson's construction}.

\bigskip

Let us first associate an \eNFA to the basic regular expressions.
\begin{itemize}

  \item For the expression \(\epsilon\), construct the
   following NFA, where \(i\) and \(f\) are \textbf{new} states
  \begin{center}
    \includegraphics[bb=48 710 135 730]{thompson_epsilon}
  \end{center}

  \item For \(a \in \Sigma\), construct the following NFA, where \(i\)
    and \(f\) are \textbf{new} states
  \begin{center}
    \includegraphics[bb=48 710 135 730]{thompson_symbol}
  \end{center}

\end{itemize}

\end{frame}

% ------------------------------------------------------------------------
%
\begin{frame}
\frametitle{From regular expressions to \eNFA{s} (cont)}

Now let us associate NFAs to complex regular expressions.

\bigskip

Assume \(N(s)\) and \(N(t)\) are the NFAs for regular expressions
\(s\) and \(t\).
\begin{itemize}

  \item For the regular expression \(st\), construct the following NFA
    \(N(st)\), where \textbf{no new state} is created:

\end{itemize}

\begin{center}
\includegraphics[bb=65 660 295 714]{thompson_conc}
\end{center}
The final state of \(N(s)\) becomes a normal state, as well as the
initial state of \(N(t)\).

\bigskip

This way only remains a unique initial state \(i\) and a unique final
state \(f\).

\end{frame}

% ------------------------------------------------------------------------
%
\begin{frame}
\frametitle{From regular expressions to \eNFA{s} (cont)}

\begin{itemize}

  \item For the regular expression \(s\) \disj \(t\), construct the
  following NFA \(N(s \, \text{\disj} \, t)\)

\end{itemize}
\begin{center} 
\includegraphics[bb=65 590 272 715,scale=0.9]{thompson_disj}
\end{center}
where \(i\) and \(f\) are \textbf{new} states. Initial and final
states of \(N(s)\) and \(N(t)\) become normal.

\end{frame}

% ------------------------------------------------------------------------
%
\begin{frame}
\frametitle{From regular expressions to \eNFA{s} (cont)}

\begin{itemize}

  \item For the regular expression \(s\)\kleene, construct the following
    NFA \(N(s\text{\kleene})\), where \(i\) and \(f\) are \textbf{new}
    states:

\end{itemize}
\begin{center}
\includegraphics[bb=50 620 255 718]{thompson_kleene}
\end{center}
Note that we added two \(\epsilon\) transitions and that the initial
and final states of \(N(s)\) become normal states.

\end{frame}

% ------------------------------------------------------------------------
%
\begin{frame}
\frametitle{From regular expressions to \eNFA{s} (cont)}

But how do we apply these simple rules when we have a complex regular
expression, having many level of nested parentheses etc?

\bigskip

Actually, the \textbf{abstract syntax tree} of the regular expression
direct, i.e., orders, the application of the rules.

\bigskip

\begin{columns}

  \column{0.5\textwidth} If the syntax tree has the shape
  \begin{center}
    \includegraphics{re_ast_conc}
  \end{center}
  then we construct first \(N(s)\), \(N(t)\) and finally \(N (st)\).

  \column{0.5\textwidth} If the syntax tree has the shape
  \begin{center}
    \includegraphics{re_ast_disj}
  \end{center}
  then we construct first \(N(s)\), \(N(t)\) and finally \(N (s \,
  \text{\disj} \, t)\).

\end{columns}

\end{frame}

% ------------------------------------------------------------------------
%
\begin{frame}
\frametitle{From regular expressions to \eNFA{s} (cont)}

If the syntax tree has the shape
\begin{center}
\includegraphics{re_ast_kleene}
\end{center}
then we construct first \(N(s)\) and finally \(N(s\text{\kleene})\).

\bigskip

This pattern-matchings are applied first at the \textbf{root} of the
abstract syntax tree of the regular expression.

\end{frame}

% ------------------------------------------------------------------------
%
\begin{frame}
\frametitle{From regular expressions to \eNFA{s}/Exercise}

Consider the regular expression
\lparen\nt{a} \disj\nt{b}\rparen\kleene\nt{a}\nt{b}\nt{b} and its
abstract syntax tree
\begin{center}
\includegraphics{re_ast}
\end{center}
Apply the previous rules to build the corresponding \eNFA.
\end{frame}
