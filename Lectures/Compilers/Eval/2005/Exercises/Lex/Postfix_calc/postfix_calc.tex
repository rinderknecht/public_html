\documentclass[a4paper,11pt]{article}

\input{trace}

% Language
%
\usepackage[english]{babel}

% Alternate font encoding
%
\usepackage{times}
\usepackage{mathptmx}
\renewcommand{\ttdefault}{cmtt}
\renewcommand{\sfdefault}{cmss}

% Miscellanea
%
\usepackage{xspace,url}

% New environments and commands
%
\newcommand{\Lex}{\textsf{Lex}\xspace}
\newcommand{\Pascal}{\textsf{Pascal}\xspace}
\newcommand{\Clang}{\textsf{C}\xspace}

%

\author{Christian Rinderknecht}
\date{25 October 2005}
\title{Homework on Lex}


\begin{document}

\maketitle
\thispagestyle{empty}

\begin{center}
\textbf{Due date: 15 November 2005}
\end{center}

Write an integer postfix calculator in \Lex. 

For example, expressions such as \verb|1 2 +| and \verb|1 2 3 4 /*-|
should be evaluated respectively to \verb+3+, i.e. \(1 + 2\), and
\verb|-0.5|, i.e. \(1 - (2 \times 3/4)\). The rule is that once
an operator is found, its arguments must have been entered
before. Another correct expression is \verb|1 2 + 3 *|, which means,
in infix notation: \((1 + 2) \times 3\).

As a shorthand, it is possible to write \verb|1-| instead of
\verb|0 1 -|.

White spaces only serve to separate numbers, but are otherwise
optional. The end of line denotes the end of an expression.

The input file may have several lines (each of one must be a postfix
arithmetic expression), therefore the calculation for each line must
be printed in the same order as the input lines. For example, if the
input file contains
\begin{verbatim}
1  2+
1 2 3 4 / *-
1 2 + 3 *
\end{verbatim}
the calculator should print
\begin{verbatim}
3
-.5
9
\end{verbatim}

%% Comments in the style of the \Pascal language are allowed\footnote{A
%%   \Pascal comment is placed between \verb+(*+ and \verb+*)+, and
%%   anything }. Thus
%% \begin{verbatim}
%% (* Some calculations
%%    for my project *)
%% 1  2+         (* Note the spacing here *)
%% 1 2 3 4 /* -
%% \end{verbatim}
%% is a valid input.

\noindent \textbf{Notes}
\begin{itemize}

  \item You probably need the \Clang function
    \verb+double atof(char*)+, which converts character strings to
    double precision floating-point numbers.

  \item Error messages must be printed and the position in the input
    file must be given (line number and column number) to help fixing
    the problem.

  \item Modern implementations of \Lex are called \textsf{Flex}. You
    can find its source code or binary versions for Windows and Linux
    on the Web, as well as documentation (manuals or tutorials).

\end{itemize}

\end{document}
