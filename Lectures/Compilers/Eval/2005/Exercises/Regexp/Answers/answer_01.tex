\paragraph{Answer 1.} The token may be different for different
languages. Even if we do not know exactly their names for the C
language, we still can find meaningful names based on what they
denote. Also we may decide or not to make a token for each
keyword. This is a creative work!  Therefore several solutions may be
acceptable as long as they enable a meaningful
implementation. However, it is usual to call the identifiers
``identifiers''. One answer to the question is therefore

\begin{multicols}{2}
\begin{center}
\begin{tabular}{l|>{\tt}l}
\hline\hline
  \multicolumn{1}{c|}{\textsc{Token}}
& \multicolumn{1}{c}{\textsc{Lexeme}}\\
\hline
  keyword & int\\
  identifier & max\\
  symbol & (\\
  keyword & int\\
  identifier & i\\
  symbol & ,\\
  keyword & int\\
  identifier & j\\
  symbol & )\\
  symbol & \{\\
  \hline
\end{tabular}
\end{center}
\par \vfill\columnbreak
\begin{center}
\begin{tabular}{l|>{\tt}l}
\hline\hline
  \multicolumn{1}{c|}{\textsc{Token}}
& \multicolumn{1}{c}{\textsc{Lexeme}}\\
\hline
  keyword & return\\
  identifier & i\\
  relation & >\\
  identifier & j\\
  symbol & ?\\
  identifier & i\\
  symbol & :\\
  identifier & j\\
  terminator & ;\\
  symbol & \}\\
  \hline
\end{tabular}
\end{center}
\end{multicols}

Note that the comments are recognised by the lexer and then discarded,
i.e., they are not transmitted to the parser. The same happens to
spaces, i.e., blanks and tabulations, which are used for separating
lexemes in the source but are not meaningful by themselves.
