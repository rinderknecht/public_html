\paragraph{Answer 4.}

\begin{enumerate}

  \item The first regular expression can be simplified in the
    following way:
   \begin{align*}
     \lparen \epsilon \, \disjM{} \, a\kleeneM{} \, \disjM{} \,
     b\kleeneM{} \, \disjM{} \, a \, \disjM{} \, b \rparen\kleeneM
     &= \lparen \epsilon \, \disjM{} \, a\kleeneM{} \, \disjM{} \,
     b\kleeneM{} \, \disjM{} \, b\rparen\kleeneM & \text{\emph{since}}
     \, L(a) \subset L(a\kleeneM)\\
     &= \lparen \epsilon \, \disjM{} \, a\kleeneM{} \, \disjM{} \,
     b\kleeneM\rparen\kleeneM & \text{\emph{since}} \,
     L(b) \subset L(b\kleeneM)\\
     &= \lparen \epsilon \, \disjM{} \, a\plusM{} \, \disjM{} \,
     b\plusM\rparen\kleeneM & \text{\emph{since}} \, \{\epsilon\}
     \subset L(x\kleeneM)\\
     &= \lparen a\plusM{} \, \disjM{} \, b\plusM\rparen\kleeneM &
     \text{\emph{since}} \, \lparen\epsilon \, \disjM{} \,
     x\rparen\kleeneM = x\kleeneM
   \end{align*}
   Words in \(L(\lparen a\plusM{} \, \disjM{} \,
   b\plusM\rparen\kleeneM)\) are of the form \(\epsilon\) or
   \((a\ldots a)(b \ldots b)(a\ldots a)(b\ldots b)\ldots\) where
     the dots stand for ``repetition any number of time, including
     zero." So we recognise \(\lparen a \, \disjM{} \,
   b\rparen\kleeneM\). Therefore \(\lparen \epsilon \, \disjM{}
   \, a\kleeneM{} \, \disjM{} \, b\kleeneM{} \, \disjM{} \, a \,
   \disjM{} \, b \rparen\kleeneM = \lparen a \, \disjM{} \,
   b\rparen\kleeneM\).

  \item The second regular expression can be simplified in the
    following way. We note first that the expression is made of the
    disjunction of three regular sub-expressions (i.e., it is a union
    of three sub-languages). The simplest idea is then to check
    whether one of these sub-languages is redundant, i.e., if one is
    included in another. If so, we can simply remove it from the
    expression.
  \begin{align*}
     a \lparen a \, \disjM{} \, b \rparen\kleeneM{} b 
     \, \disjM{} \, \lparen a b \rparen\kleeneM{} 
     \, \disjM{} \, \lparen b a \rparen\kleeneM
     &= a \lparen a \, \disjM{} \, b \rparen\kleeneM{} b 
     \, \disjM{} \, \epsilon
     \, \disjM{} \, \lparen a b \rparen\plusM{} 
     \, \disjM{} \, \lparen b a \rparen\kleeneM
     & \text{\emph{since}} \, \lparen ab \rparen\kleeneM =
     \epsilon \, \disjM{}\, \lparen ab \rparen\plusM\\
     &= a \lparen a \, \disjM{} \, b \rparen\kleeneM{} b 
     \, \disjM{} \, \lparen a b \rparen\plusM{} 
     \, \disjM{} \, \lparen b a \rparen\kleeneM
     & \text{\emph{since}} \, \{\epsilon\} \subset L(\lparen ba
     \rparen\kleeneM)
  \end{align*}
  We have:
  \begin{align*}
    \lparen ab\rparen\plusM
   &= \lparen ab\rparen \lparen ab\rparen \ldots \lparen ab \rparen\\
   &= a\lparen ba\rparen\lparen ba\rparen\ldots\lparen  ba\rparen b \;
    \disjM{} \; ab\\
   &= a\lparen ba \rparen\kleeneM b
   \end{align*}
   Also \(L(\lparen ba \rparen) \subset L(\lparen a \, \disjM{}
   \, b\rparen\kleeneM)\) and then \(L(\lparen ba
   \rparen\kleeneM) \subset L(\lparen a \, \disjM{} \,
   b\rparen\kleeneM)\), because \(\lparen a \, \disjM{} \,
   b\rparen\kleeneM\) denotes all the words. Therefore
   \begin{align*}
      L(a\lparen ba \rparen\kleeneM b) 
    &\subset L(a \lparen a \, \disjM{} \, b\rparen\kleeneM b)\\
      L(\lparen ab\rparen\plusM) 
    &\subset L(a \lparen a \, \disjM{} \, b\rparen\kleeneM b)
   \end{align*}

   As a consequence, one possible answer is
   \[  
     a \lparen a \, \disjM{} \, b \rparen\kleeneM{} b 
     \, \disjM{} \, \lparen a b \rparen\kleeneM{} 
     \, \disjM{} \, \lparen b a \rparen\kleeneM
     = a \lparen a \, \disjM{} \, b\rparen\kleeneM b 
     \, \disjM{} \, \lparen ba \rparen\kleeneM
   \]
   The intersection between \(L(a \lparen a \, \disjM{} \,
   b\rparen\kleeneM b)\) and \(L(\lparen ba \rparen\kleeneM)\)
   is empty because all the words of the former start with
     \(a\), while all the words of the other start with \(b\) (or is
     \(\epsilon\)).  Therefore we cannot simply further this way.

\end{enumerate}
