\paragraph{Answer.}

\begin{itemize}

  \item The pointer \(\upharpoonleft\) points to the last character in
    the input buffer which was the current character when the current
    state was initial.

  \item The pointer \(\upharpoonright\) points to the current
    available character in the input buffer.

  \item The pointer \(\Uparrow\) points to the last character in the
    input buffer which was the current character when the current
    state was final.

\end{itemize}

The transition diagram of figure~\ref{num_diagram} is tried first.

The first available character is \texttt{r}, which cannot be read by
\texttt{num}. Then the next regular expression, \texttt{id}, is
tried. The processing is summarised in the following array, where the
changes made at each step are shown in bold font.
   \[
   \begin{array}{|c||c@{\,}|@{\,}c@{\,}|@{\,}c@{\,}|@{\,}c@{\,}|@{\,}c@{\,}|@{\,}c@{\,}|@{\,}c@{\,}|@{\,}c@{\,}|@{\,}c@{\,}|@{\,}c@{\,}|@{\,}c@{\,}|@{\,}c@{\,}|@{\,}c@{\,}|@{\,}c@{\,}|@{\,}c@{\,}|@{\,}c@{\,}|@{\,}c@{\,}|@{\,}c@{\,}|@{\,}c@{\,}|}
   \hhline{~-------------------}
     \multicolumn{1}{c||}{\text{States}}
   & \texttt{r} 
   & \texttt{e} 
   & \texttt{t} 
   & \texttt{u}
   & \texttt{r}
   & \texttt{n}
   & \textvisiblespace
   & \textvisiblespace
   & \texttt{x}
   & \texttt{+}
   & \textvisiblespace
   & \texttt{.}
   & \texttt{5}
   & \texttt{E}
   & \texttt{2}
   & \texttt{+}
   & \texttt{y}
   & \texttt{\huge \_}
   & \texttt{0}\\
   \hhline{=::===================}
   1 & \bm{\upharpoonright} &&&&&&&&&&&&&&&&&&\\
   initial & \bm{\upharpoonleft}\upharpoonright
   &&&&&&&&&&&&&&&&&&\\
   \hhline{-||-------------------}
   2 & \upharpoonleft & \bm{\upharpoonright} &&&&&&&&&&&&&&&&&\\
   \text{final} & \upharpoonleft & \bm{\Uparrow}\upharpoonright
   &&&&&&&&&&&&&&&&&\\ 
   \hhline{-||-------------------}
   2 & \upharpoonleft & \Uparrow & \bm{\upharpoonright}
   &&&&&&&&&&&&&&&&\\
   \text{final} & \upharpoonleft && \bm{\Uparrow}\upharpoonright
   &&&&&&&&&&&&&&&&\\
   \hhline{-||-------------------}
   2 & \upharpoonleft && \Uparrow & \bm{\upharpoonright}
   &&&&&&&&&&&&&&&\\
   \text{final} & \upharpoonleft &&& \bm{\Uparrow}\upharpoonright
   &&&&&&&&&&&&&&&\\
   \hhline{-||-------------------}
   2 & \upharpoonleft &&& \Uparrow & \bm{\upharpoonright}
   &&&&&&&&&&&&&&\\
   \text{final} & \upharpoonleft &&&& \bm{\Uparrow}\upharpoonright
   &&&&&&&&&&&&&&\\
   \hhline{-||-------------------}
   2 & \upharpoonleft &&&& \Uparrow & \bm{\upharpoonright}
   &&&&&&&&&&&&&\\
   \text{final} & \upharpoonleft &&&&& \bm{\Uparrow}\upharpoonright
   &&&&&&&&&&&&&\\
   \hhline{-||-------------------}
   2 & \upharpoonleft &&&&& \Uparrow & \bm{\upharpoonright}
   &&&&&&&&&&&&\\
   \text{final} & \upharpoonleft &&&&&& \bm{\Uparrow}\upharpoonright
   &&&&&&&&&&&&\\
   \hhline{-||-------------------}
   \end{array}
   \]
We are stuck in state \(2\), so we have recognised the lexeme
\texttt{return} with the regular expression \texttt{id}. In order to
determine the corresponding token, we need to check the keyword list
\texttt{keyword}: \texttt{return} is in the list, so the token is
\texttt{kwd}. We reset the \(\upharpoonleft\) to the value of
\(\upharpoonright\), i.e. it points to the first \textvisiblespace{},
and start again the analysis by trying to match the regular expression
\texttt{num} against the input. This fails. Therefore, the second
regular expression, \texttt{id}, is tried. It fails to match the
current character. Then the third regular expression, \texttt{ws}, is
tried. One transition diagram for this regular expression, i.e. one
recognising the same words, is
\begin{center}
\includegraphics{dfa_ws_opt}
\end{center}
Therefore the analysis goes on in the following manner (states are
taken in the transition diagram for regular expression \texttt{ws}).
\[
\begin{array}{|c||c@{\,}|@{\,}c@{\,}|@{\,}c@{\,}|@{\,}c@{\,}|@{\,}c@{\,}|@{\,}c@{\,}|@{\,}c@{\,}|@{\,}c@{\,}|@{\,}c@{\,}|@{\,}c@{\,}|@{\,}c@{\,}|@{\,}c@{\,}|@{\,}c@{\,}|@{\,}c@{\,}|@{\,}c@{\,}|@{\,}c@{\,}|@{\,}c@{\,}|@{\,}c@{\,}|@{\,}c@{\,}|}
\hhline{~-------------------}
\multicolumn{1}{c||}{\text{States}}
& \texttt{r} 
& \texttt{e} 
& \texttt{t} 
& \texttt{u}
& \texttt{r}
& \texttt{n}
& \textvisiblespace
& \textvisiblespace
& \texttt{x}
& \texttt{+}
& \textvisiblespace
& \texttt{.}
& \texttt{5}
& \texttt{E}
& \texttt{2}
& \texttt{+}
& \texttt{y}
& \texttt{\huge \_}
& \texttt{0}\\
\hhline{=::===================}
1 &&&&&&& \upharpoonleft\upharpoonright &&&&&&&&&&&&\\
\hhline{-||-------------------}
2 &&&&&&& \upharpoonleft & \bm{\upharpoonright} &&&&&&&&&&&\\
\text{final} &&&&&&& \upharpoonleft & \bm{\Uparrow}\upharpoonright &&&&&&&&&&&\\
\hhline{-||-------------------}
2 &&&&&&& \upharpoonleft & \Uparrow & \bm{\upharpoonright} &&&&&&&&&&\\
\text{final} &&&&&&& \upharpoonleft && \bm{\Uparrow}\upharpoonright &&&&&&&&&&\\
\hhline{-||-------------------}
\end{array}
\]
At that point, the analysis get stuck. There is no action associated
to state \(2\) in the transition diagram of regular expression
\texttt{ws}, therefore we restart the analysis without emitting any
lexeme. The pointer \(\upharpoonleft\) is set to the value of pointer
\(\upharpoonright\), i.e. it points now to the character \texttt{x},
and the regular expression \texttt{num} is tried first. It fails to
recognise the current character, therefore the analysis restarts with
the next regular expression, \texttt{id}. The analysis then proceeds
as follows (states refers to the transition diagram of \texttt{id}).
\[
\begin{array}{|c||c@{\,}|@{\,}c@{\,}|@{\,}c@{\,}|@{\,}c@{\,}|@{\,}c@{\,}|@{\,}c@{\,}|@{\,}c@{\,}|@{\,}c@{\,}|@{\,}c@{\,}|@{\,}c@{\,}|@{\,}c@{\,}|@{\,}c@{\,}|@{\,}c@{\,}|@{\,}c@{\,}|@{\,}c@{\,}|@{\,}c@{\,}|@{\,}c@{\,}|@{\,}c@{\,}|@{\,}c@{\,}|}
\hhline{~-------------------}
\multicolumn{1}{c||}{\text{States}}
& \texttt{r} 
& \texttt{e} 
& \texttt{t} 
& \texttt{u}
& \texttt{r}
& \texttt{n}
& \textvisiblespace
& \textvisiblespace
& \texttt{x}
& \texttt{+}
& \textvisiblespace
& \texttt{.}
& \texttt{5}
& \texttt{E}
& \texttt{2}
& \texttt{+}
& \texttt{y}
& \texttt{\huge \_}
& \texttt{0}\\
\hhline{=::===================}
1 &&&&&&&&& \upharpoonleft\upharpoonright &&&&&&&&&&\\
\hhline{-||-------------------}
2 &&&&&&&&& \upharpoonleft & \bm{\upharpoonright} &&&&&&&&&\\
\text{final} &&&&&&&&& \upharpoonleft & \bm{\Uparrow} \upharpoonright &&&&&&&&&\\
\hhline{-||-------------------}
\end{array}
\]
Then the analysis gets stuck. Hence we emit the lexeme \texttt{x} and
check for it in the keyword table. It is not present in the table, so
it is to interpreted as an identifier, i.e. the token and lexemes are
written \token{id}{x}. The analysis restarts again with resetting
\(\upharpoonleft\) to the same value as \(\upharpoonright\) and tries
the first regular expression, \texttt{num}. This fails because
\textvisiblespace{} (the current character) is not recognised by
\texttt{num}. Therefore, the nex regular expression, \texttt{id}, is
tried. Same problem. 

The third expression, \texttt{ws}, succeeds: \textbf{TO BE FINISHED}
