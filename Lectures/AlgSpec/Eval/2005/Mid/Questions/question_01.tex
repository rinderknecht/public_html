\paragraph{Question 1.} We want an algebraic specification of sets
made of elements of the same type \type{element}. Let us call it 
\(\proc{Set}(\type{element})\). Here is the signature:
   \begin{itemize}

     \item \textbf{Defined types}

     \begin{itemize}

       \item The type of the sets is noted \type{t}.

       \item The type of the set elements is noted \type{element}.
  
     \end{itemize}

     \item \textbf{Constructors}
     \begin{itemize}

       \item \proc{Empty} : \type{t}\\
       The term \proc{Empty} represents the set (noted
       \(\varnothing\) in mathematics) which contains no element.

       \item \(\proc{Add} : \type{element} \times \type{t} \rightarrow
         \type{t}\)\\ The term \(\proc{Add} (e, s)\) denotes
         the set \(s\) augmented by adding element \(e\). If \(e\)
         was already in \(s\), then the result is \(s\). The order
         in which elements are added is not meaningful, so all
         function on sets should not depend on any element order.

     \end{itemize}

     \item \textbf{Functions}
     \begin{itemize}

       \item \(\proc{Union} : \type{t} \times \type{t} \rightarrow
         \type{t}\)\\
       The term \(\proc{Union} (s_1, s_2)\) denotes the set
       made of the elements of \(s_1\) and \(s_2\) (noted \(s_1
       \cup s_2\) in mathematics). It is commutative and
       associative.

       \item \(\proc{Inter} : \type{t} \times \type{t} \rightarrow
         \type{t}\)\\
       The term \(\proc{Inter} (s_1, s_2)\) denotes the set
       made of elements of \(s_1\) which also belong to \(s_2\)
       (noted \(s_1 \cap s_2\) in mathematics). The function
       name stands for ``intersection''. It is commutative and
       associative. It is distributive over the set union.

       \item \(\proc{Mem} : \type{element} \times \type{t} \rightarrow
         \type{boolean}\)\\
       The term \(\proc{Mem} (e, s)\) is \kw{true} if and
       only if \(e\) belongs to s, otherwise it is
       \kw{false}. The function name is a shorthand for ``membership''.

     \end{itemize}

   \end{itemize}
   Complete this signature with equations defining the constructors
   and functions.

