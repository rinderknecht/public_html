\paragraph{Question 2.} Consider the
\(\proc{Stack}(\type{item})\) signature as follows:
  \begin{itemize}

    \item \textbf{Defined types}

    \begin{itemize}

      \item The type of the stacks is called \type{t}.

      \item The type of the items is called \type{item}.
 
    \end{itemize}

    \item \textbf{Constructors}

    \begin{itemize}
 
      \item \(\proc{Empty} : \type{t}\)\\
       Term \proc{Empty} denotes the empty stack.

      \item \(\proc{Push} : \type{item} \times \type{t} \rightarrow
        \type{t}\)\\
      Term \(\proc{Push} (e, s)\) denotes stack \(s\) with
      item \(e\) on the top.

    \end{itemize}

    \item \textbf{Functions}

    \begin{itemize}

      \item \(\proc{Pop} : \type{t} \rightarrow \type{item} \times
        \type{t}\)\\
      This is the projection of \proc{Push}.

      \item \(\proc{Append} : \type{t} \times \type{t} \rightarrow
        \type{t}\)\\
      Term \(\proc{Append} (s_1, s_2)\) denotes the stack
      made of \(s_2\) on the bottom and \(s_1\) on the top.

      \item \(\proc{Nth} : \type{t} \times \type{int} \rightarrow
        \type{item}\)\\
      Term \(\proc{Nth} (s, n)\) denotes the \(n^{th}\) item
      in stack \(s\). The item must exist.

      \item \(\proc{Flatten} :
        \proc{Stack}(\proc{Stack}(\type{item}).\type{t}).\type{t}
        \rightarrow \proc{Stack}(\type{item})\)\\ Term
        \(\proc{Flatten} (s)\), where \(s\) denotes a stack of
        stack of items, represents the stack made by appending all the
        stacks in \(s\), in the same order.

     \item \(\proc{Exists} : (\type{item} \rightarrow \type{boolean})
       \times \proc{Stack}(\type{item}).\type{t} \rightarrow
       \type{boolean}\)\\ Term \(\proc{Exists} (f, s)\) is
       \kw{true} if and only if there exists an item \(e\) in \(s\)
       such that \(f(e)\) is \kw{true}. Otherwise it is
       \kw{false}.

    \end{itemize}
    
  \end{itemize}

  Give equations defining \proc{Nth}, \proc{Flatten} and \proc{Exists}.
