%%-*-latex-*-

\documentclass[11pt,a4paper]{article}

\input{trace}

\title{Answers to the mid-term exam on\\ Introduction to the Internet}

\author{Christian Rinderknecht}
\date{17 June 2005}

\usepackage[english]{babel}
\usepackage{vaucanson-g}
\usepackage{amsmath,amssymb}
\usepackage{clrscode}
\usepackage{graphicx}

\newcommand\type[1]{\textsf{#1}}

\begin{document}

\maketitle

\section{Binary tree specification}

Let us recall an algebraic specification of binary trees over
nodes. Let us call it \(\proc{Bin-Tree}(\type{node})\). Here is the
signature:
\begin{itemize}

  \item \textbf{Defined types}

  \begin{itemize}

    \item The type of the binary trees is noted \type{t}.

    \item The type of the nodes is noted \type{node}.
  
  \end{itemize}

  \item \textbf{Constructors}
  \begin{itemize}

    \item \proc{Empty} : \type{t}\\
    The term \proc{Empty} represents the empty tree, i.e. the tree
    that contains no node.

    \item \(\proc{Make} : \type{node} \times \type{t} \times
    \type{t} \rightarrow \type{t}\)\\ The term \(\proc{Make}
    (r, t_1, t_2)\) denotes the tree whose root is \(r\), left
    subtree is \(t_1\) and right subtree is \(t_2\). Graphically:
    \begin{center}
      \includegraphics{non-empty_tree}
    \end{center}

  \end{itemize}

  \item \textbf{Functions}
  \begin{enumerate}

     \item \(\proc{Mem}: \type{t} \times \type{node} \rightarrow
     \type{bool}\)\\
     The term \(\proc{Mem} (t, e)\) is \proc{True} if node \(e\)
     occurs in tree \(t\), otherwise it is equal to
     \proc{False}. (We assume we have equality on nodes.)

     \item \(\proc{Min-Depth}: \type{t} \times \type{node} 
     \rightarrow \type{int}\)\\ 
     The term \(\proc{Min-Depth} (t, e)\) denotes the minimum depth at
     which node \(e\) occurs in tree \(t\). (The root is at depth 0.)
     If \(e\) is not in \(t\), the value is unspecified. In case the
     node \(e\) appears several times in \(t\), the value is the
     smallest depth of occurrence. (We assume we can compare nodes for
     equality and that we have the function \(\proc{Min}: \type{int}
     \times \type{int} \rightarrow \type{int}\) which returns the
     smallest integer argument.) For example, if \(t\) is
     \begin{center}
       \includegraphics{int_tree_example}
     \end{center}
     then \(\left\{ 
     \begin{aligned}
       \proc{Min-Depth} (t, 7) & \quad \text{is unspecified
         (i.e. undefined)}\\
       \proc{Min-Depth} (t, 1) &= 0\\
       \proc{Min-Depth} (t, 3) &= 2\\
       \proc{Min-Depth} (t, 5) &= \proc{Min} (1, 2) = 1
     \end{aligned}
     \right.\)

     \item \(\proc{Inv}: \type{t} \rightarrow \type{t}\)\\ The term
     \(\proc{Inv} (t)\) denotes the tree \(t\) in a mirror, i.e. the
     left subtrees of \(t\) are the right subtrees of \(\proc{Inv}
     (t)\) and the right subtrees become the left
     subtrees. Therefore \(\proc{Inv} (\proc{Inv} (t)) = t\). For
     example, the following trees are mirrors of each other:
     \begin{center}
       \includegraphics[bb=71 656 233 721]{mirrors}
     \end{center}

     \item \(\proc{Sum}: \proc{Bin-Tree}(\type{int}).\type{t}
     \rightarrow \type{int}\)\\
     The term \(\proc{sum} (t)\) denotes the sum of all (integer) nodes
     in tree \(t\). If \(t\) is empty, the sum is unspecified. For
     example, if \(t\) is
     \begin{center}
       \includegraphics[bb=71 656 141 721]{int_tree_example}
     \end{center}
     then \(\proc{Sum} (t) = 1 + 5 + 3 + 4 + 6 + 5 = 24\).

  \end{enumerate}

\end{itemize}


\paragraph{Question 1.} Give the lower and upper bounds of a number
represented in balanced ternary with \(n\) trits.


\noindent\textbf{Question.} Define a relation
    \texttt{catenate(U,V,W)} to be true when list \texttt{W} is made
    of list \texttt{U} followed by list \texttt{V}. For example
{\small 
\begin{verbatim}
?- catenate(U,[3,4],[0,1,2,3,4]).
U = [0,1,2] ;
No
\end{verbatim}
}


\end{document}
