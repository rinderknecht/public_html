\paragraph{Answers.}

\begin{itemize}

  \item \(\proc{Lower} : \type{t} \rightarrow \type{int}\)\\
  First, we know that we have to left unspecified the case when the
  array is empty, i.e. there is no equation whose side is
  \(\proc{Lower} (\proc{Empty})\).

  We still have to consider the two other constructors, \proc{Create}
  and \proc{Set}:
  \begin{align}
       \proc{Lower} (\proc{Create} (x, y)) 
    &= \textbf{?} \label{Lower_1}\\
       \proc{Lower} (\proc{Set} (a, n, e)) 
    &= \textbf{?} \label{Lower_2}
  \end{align}
  In case of equation~\ref{Lower_1}, we must distinguish two cases: if
  \(x < y\) or \(x \geqslant y\):
  \begin{align*}
       \proc{Lower} (\proc{Create} (x, y)) 
    &= \textbf{?} 
    &  \text{if} \;x < y\\
       \proc{Lower} (\proc{Create} (x, y)) 
    &= \textbf{?} 
    &  \text{if} \;x \geqslant y\\
       \proc{Lower} (\proc{Set} (a, n, e)) 
    &= \textbf{?}
  \end{align*}
  These cases are easy to complete because we know explicitly the
  lower bound:
  \begin{align*}
       \proc{Lower} (\proc{Create} (x, y)) 
    &= x
    &  \text{if} \;x < y\\
       \proc{Lower} (\proc{Create} (x, y)) 
    &= y
    &  \text{if} \;x \geqslant y\\
       \proc{Lower} (\proc{Set} (a, n, e)) 
    &= \textbf{?}
  \end{align*}
  The last case (equation~\ref{Lower_2}) is also easy to solve because
  the argument is simply the array \(a\) with one element modified:
  the call to \proc{Set} does not change the lower bound. Therefore:
  \begin{align*}
       \proc{Lower} (\proc{Create} (x, y)) 
    &= x
    &  \text{if} \;x < y\\
       \proc{Lower} (\proc{Create} (x, y)) 
    &= y
    &  \text{if} \;x \geqslant y\\
       \proc{Lower} (\proc{Set} (a, n, e)) 
    &= \proc{Lower} (a)
  \end{align*}

  \item \(\proc{Upper} : \type{t} \rightarrow \type{int}\)\\
  This case is the dual of \proc{Lower}:
  \begin{align}
       \proc{Upper} (\proc{Create} (x, y)) 
    &= y
    &  \text{if} \;x < y \label{Upper_1}\\
       \proc{Upper} (\proc{Create} (x, y)) 
    &= x
    &  \text{if} \;x \geqslant y \label{Upper_2}\\
       \proc{Upper} (\proc{Set} (a, n, e)) 
    &= \proc{Upper} (a) \label{Upper_3}
  \end{align}

  \item \(\proc{Get} : \type{t} \times \type{int} \rightarrow
    \proc{Item}.\type{t}\)\\
  First, we know that the case of the empty array is not specified for
  \proc{Get} (there is no element to get), so there is no equation
  whose side is \(\proc{Get} (\proc{Empty}, n)\).

  The remaining constructors \proc{Create} and \proc{Set} have to be
  considered:
  \begin{align}
     \proc{Get} (\proc{Create} (x, y), n) 
  &= \textbf{?} \label{Get_1}\\
     \proc{Get} (\proc{Set} (a, p, e), n)
  &= \textbf{?} \label{Get_2}
  \end{align}
  We know from the signature that if the index \(n\) is out of
  bounds, the value of the call to \proc{Get} is unspecified. 

  This means that equation~\ref{Get_1} should be restricted to the
  case
  \[
    \proc{Lower} (\proc{Create} (x, y)) \leqslant n 
    \leqslant \proc{Upper} (\proc{Create} (x, y))
  \]
  which can be split into two cases, depending on \(x\) and \(y\):
  \begin{align*}
     \proc{Get} (\proc{Create} (x, y), n) 
  &= \textbf{?}
  & \text{if} \; x \leqslant n \leqslant y\\
     \proc{Get} (\proc{Create} (x, y), n) 
  &= \textbf{?} 
  & \text{if} \; y \leqslant n \leqslant x\\
     \proc{Get} (\proc{Set} (a, p, e), n)
  &= \textbf{?}
  \end{align*}
  We have to restrict also equation~\ref{Get_2} to the case
  \[
    \proc{Lower} (\proc{Set} (a, p, e)) \leqslant n 
    \leqslant \proc{Upper} (\proc{Set} (a, p, e))
  \]
  which, using the equations~\ref{Lower_2} and~\ref{Upper_3} of
  \proc{Lower} and \proc{Upper}, is equivalent to
  \(
    \proc{Lower} (a) \leqslant n \leqslant \proc{Upper} (a)
  \).
  Therefore we have:
  \begin{align*}
     \proc{Get} (\proc{Create} (x, y), n) 
  &= \textbf{?}
  & \text{if} \; x \leqslant n \leqslant y\\
     \proc{Get} (\proc{Create} (x, y), n) 
  &= \textbf{?} 
  & \text{if} \; y \leqslant n \leqslant x\\
     \proc{Get} (\proc{Set} (a, p, e), n)
  &= \textbf{?}
  & \text{if} \;\proc{Lower} (a) \leqslant n \leqslant \proc{Upper} (a)
  \end{align*}
  Also \(p\) must be in-between bounds:
  \(\proc{Lower} (a) \leqslant p \leqslant \proc{Upper} (a)\).
  So:
  \begin{align*}
     \proc{Get} (\proc{Create} (x, y), n) 
  &= \textbf{?}
  & \text{if} \; x \leqslant n \leqslant y\\
     \proc{Get} (\proc{Create} (x, y), n) 
  &= \textbf{?} 
  & \text{if} \; y \leqslant n \leqslant x\\
     \proc{Get} (\proc{Set} (a, p, e), n)
  &= \textbf{?}
  & \text{if} \;\proc{Lower} (a) \leqslant p, n \leqslant \proc{Upper} (a)
  \end{align*}

  The signature tells us that a newly created array is filled with the
  special element \proc{Item}.\proc{Default}. So if we pick any
  element between the bounds, it is always equal to
  \proc{Item}.\proc{Default}:
  \begin{align*}
     \proc{Get} (\proc{Create} (x, y), n) 
  &= \proc{Default}
  & \text{if} \; x \leqslant n \leqslant y\\
     \proc{Get} (\proc{Create} (x, y), n) 
  &= \proc{Default}
  & \text{if} \; y \leqslant n \leqslant x\\
     \proc{Get} (\proc{Set} (a, p, e), n)
  &= \textbf{?}
  & \text{if} \;\proc{Lower} (a) \leqslant p, n \leqslant \proc{Upper} (a)
  \end{align*}
  The last equation has to be split in two cases, because we can
  express differently the result if \(p = n\) or not:
  \begin{align*}
     \proc{Get} (\proc{Create} (x, y), n) 
  =& \;\proc{Default}
  & \text{if} \; x \leqslant n \leqslant y\\
     \proc{Get} (\proc{Create} (x, y), n) 
  =& \;\proc{Default}
  & \text{if} \; y \leqslant n \leqslant x\\
     \proc{Get} (\proc{Set} (a, n, e), n)
  =& \;\textbf{?}
  & \text{if} \;\proc{Lower} (a) \leqslant n \leqslant \proc{Upper}
     (a)\\
     \proc{Get} (\proc{Set} (a, p, e), n)
  =& \;\textbf{?} &  \text{if} \; p \neq n\\
  && \text{and} \;\proc{Lower} (a) \leqslant p, n \leqslant \proc{Upper}
     (a)
  \end{align*}
  If \(p = n\) the result is \(e\) because, by construction, it is the
  element at position \(n\) in \(\proc{Set} (a, n, e)\). It \(p \neq
  n\), the result cannot be \(e\), by construction. Hence we must look
  for it in \(a\). As a conclusion:
  \begin{align*}
     \proc{Get} (\proc{Create} (x, y), n) 
  =& \;\proc{Default}
  & \text{if} \; x \leqslant n \leqslant y\\
     \proc{Get} (\proc{Create} (x, y), n) 
  =& \;\proc{Default}
  & \text{if} \; y \leqslant n \leqslant x\\
     \proc{Get} (\proc{Set} (a, n, e), n)
  =& \;e
  & \;\text{if} \;\proc{Lower} (a) \leqslant n \leqslant \proc{Upper}
     (a)\\
     \proc{Get} (\proc{Set} (a, p, e), n)
  =& \;\proc{Get} (a, n)
  &  \text{if} \; p \neq n\\
  && \;\text{and} \;\proc{Lower} (a) \leqslant p \leqslant \proc{Upper}
    (a)
  \end{align*}
Notice we removed the condition \(\proc{Lower} (a) \leqslant n
\leqslant \proc{Upper} (a)\) from the last equation. This is a
simplication, because, in this case, we can delay the condition check
on \(n\) to the recursive call \(\proc{Get} (a, n)\). Then, if \(a\)
is an unmodified array (\proc{Create}), \(n\) will be checked against
the bounds (see the first two equations). Otherwise, \(n\) may equal
an index in bounds (third equation), which settles the problem too.

\end{itemize}
