%%-*-latex-*-

\documentclass[11pt,a4paper]{article}

\input{trace}

\title{Answers to the mid-term exam on\\ Introduction to the Internet}

\author{Christian Rinderknecht}
\date{17 May 2005}

\usepackage[english]{babel}
\usepackage{amsmath,amssymb}
\usepackage{clrscode}

\newcommand\type[1]{\textsf{#1}}

\begin{document}

\maketitle

\section{Arrays}

We want an algebraic specification of arrays. An array is a list whose
size is fixed at the construction time and whose elements are items of
the same type. There is a special item which is used as a default
element when creating a new array (i.e., a freshly created array
contains these default elements). Then the user can change an element
to another value by specifying the integer index (i.e., position) in
the array and the new value.

\noindent Let us call \proc{Item} the specification of some items whose
signature is as follows.
\begin{itemize}

   \item \textbf{Defined types}
   The type of the items is noted \type{t}

   \item \textbf{Constructors}\\ \proc{Default} : \type{t}\\ The term
     \proc{Default} is a distinguished item, whose interpretation is
     left to the user of this specification.

\end{itemize}

\noindent Let us call now \(\proc{Array}(\proc{Item})\) the
specification of the arrays over items of type
\proc{Item}.\type{t}. The signature is as follows.
\begin{itemize}

  \item \textbf{Defined types}
  The type of the arrays is noted \type{t}.

  \item \textbf{Constructors}
  \begin{itemize}

    \item \(\proc{Empty} : \type{t}\)\\
    The term \proc{Empty} represents the array which contains no
    element.

    \item \(\proc{Create} : \type{int} \times \type{int} \rightarrow
    \type{t}\)\\ 
    The term \(\proc{Create} (x, y)\) denotes an array whose
    elements are indexed from \(x\) to \(y\), if \(x < y\),
    or from \(y\) to \(x\) otherwise. The type \type{int} denotes
    the set of positive integers. Integers \(x\) and \(y\) are
    called \emph{bounds}. If \(x < y\) then \(x\) is the
    \emph{lower bound} and \(y\) is the \emph{upper bound}.

    For example, \(\proc{Create}(3, 5)\) is an array whose elements
    are indexed by integers \(3\), \(4\) and \(5\). It contains three
    elements. These elements are all equal to
    \proc{Item}.\proc{Default}. Also \(\proc{Create} (5, 3)\) is the
    same as \(\proc{Create} (3, 5)\). Therefore, in both cases, the
    lower bound is \(3\) and the upper bound is \(5\).

    \item \(\proc{Set} : \type{t} \times \type{int} \times
    \proc{Item}.\type{t} \rightarrow \type{t}\)\\
    The term \(\proc{Set} (a, n, e)\) represents an
    array equal to array \(a\) except that the element at index
    \(n\) is \(e\). If \(n\) is out of bounds, i.e., if \(n\) is
    not between the bounds of \(a\), the result of \proc{Set} is
    unspecified. Same if \(a\) is empty.

  \end{itemize}

  \item \textbf{Functions}
  \begin{itemize}
  
    \item \(\proc{Lower} : \type{t} \rightarrow \type{int}\)\\
    The call \(\proc{Lower} (a)\) is the lower bound of array
    \(a\), i.e., the smallest valid index. If \(a\) is empty, the
    result is unspecified.

    \item \(\proc{Upper} : \type{t} \rightarrow \type{int}\)\\
    The call \(\proc{Upper} (a)\) is the upper bound of array
    \(a\), i.e., the greatest valid index. If \(a\) is empty, the
    result is unspecified.

    \item \(\proc{Get} : \type{t} \times \type{int} \rightarrow
    \proc{Item}.\type{t}\)\\
    The call \(\proc{Get} (a, n)\) denotes the element in
    array \(a\) at index \(n\). If \(n\) is out of bounds or if
    \(a\) is empty, the result of \proc{Get} is unspecified.

  \end{itemize}

\end{itemize}

\paragraph{Question 1.} Give the lower and upper bounds of a number
represented in balanced ternary with \(n\) trits.



\end{document}
