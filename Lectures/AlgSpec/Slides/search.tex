%%-*-latex-*-

% ------------------------------------------------------------------------
%
\begin{frame}
\frametitle{Search algorithms}

Algorithms are a constrained form of rewriting systems.

\bigskip

You may remember that, sometimes, several rewriting rules can be
applied to the same term. This is a kind of \textbf{non-determinism},
i.e., the process requires an arbitrary choice. Like
{\footnotesize
\begin{align*}
  \proc{Not} (\proc{And} (\underline{\proc{Not} (\proc{True})},
  \proc{Not} (\proc{False})))
&\rightarrow_1 \proc{Not} (\proc{And} (\underline{\proc{False}},
  \proc{Not} (\proc{False})))\\
  \proc{Not} (\proc{And} (\proc{False}, \underline{\proc{Not}
  (\proc{False})}))
&\rightarrow_2 \proc{Not} (\proc{And} (\proc{False},
  \underline{\proc{True}}))
\end{align*}
}
or
{\footnotesize
\begin{align*}
  \proc{Not} (\proc{And} (\proc{Not} (\proc{True}),
  \underline{\proc{Not} (\proc{False})}))
&\rightarrow_2 \proc{Not} (\proc{And} (\proc{Not} (\proc{True}),
  \underline{\proc{True}}))\\
  \proc{Not} (\proc{And} (\underline{\proc{Not} (\proc{True})},
  \proc{True}))
&\rightarrow_1 \proc{Not} (\proc{And} (\underline{\proc{False}},
  \proc{True}))
\end{align*}
}

\end{frame}

% ------------------------------------------------------------------------
%
\begin{frame}
\frametitle{Search algorithms (cont)}

It is possible to constrain the situation where the rules are
applied. This is called a \textbf{strategy}. 

\bigskip

For instance, one common strategy, called \textbf{call by value},
consists in rewriting the arguments of a function call into values
before rewriting the function call itself. 

\bigskip

Some further constraints can impose an order on the rewritings of the
arguments, like rewriting them from left to right or from right to
left. Algorithms rely on rewriting systems with strategies, but use a
different language, easier to read and write. The important thing is
that algorithms can always be expressed in terms of rewriting systems,
if we want.

\end{frame}

% ------------------------------------------------------------------------
%
\begin{frame}
\frametitle{Search algorithms (cont)}

The language we introduce now for expressing algorithms is different
from a programming language, in the sense that it is less detailed. 

\bigskip

Since you already have a working knowledge of programming, you will
understand the language itself through examples.

\bigskip

If we start from a rewriting system, the idea consists to gather all
the rules that define the computation of a given function and create
its algorithmic definition.

\end{frame}

% ------------------------------------------------------------------------
%
\begin{frame}
\frametitle{Search/Booleans}

Let us start with a very simple function of the \proc{Bool} specification:
\begin{align*}
\proc{Not} (\proc{True}) &\rightarrow \proc{False}\\
\proc{Not} (\proc{False}) &\rightarrow \proc{True}
\end{align*}
Let us write the corresponding algorithm in the following way:
\begin{columns}
  \column{0.5\textwidth}
    \begin{codebox}
      \Procname{\(\proc{Not} (\id{b})\)}
      \zi	\If \(\id{b} = \proc{True}\)
      \zi	\Then \(\kw{result} \gets \proc{False}\)
      \zi	\Else \(\kw{result} \gets \proc{True}\)
      \zi	\End
    \end{codebox}
    The variable \id{b} is called a \textbf{parameter}.
  \column{0.5\textwidth}
    Writing \(\id{x} \gets A\) means that we \textbf{assign} the value
    of expression \(A\) to the variable \id{x}. Then the value of
    \id{x} is the value of \(A\). Keyword \kw{result} is a special
    variable whose value becomes the result of the function when it
    finishes.
\end{columns}

\end{frame}

% ------------------------------------------------------------------------
%
\begin{frame}
\frametitle{Search/Booleans (cont)}

You may ask: ``Since we are defining the booleans, what is the meaning
of a conditional \(\If  \dots \Then \dots \kw{else} \dots\)?''

\bigskip

We assume built-in booleans \kw{true} and \kw{false} in our
algorithmic language. So, the expression \(\id{b} = \proc{True}\) may
have value \kw{true} or \kw{false}.

\bigskip

The \proc{Bool} specification is \emph{not} the built-in booleans.

\bigskip

Expression \(\id{b} = \proc{True}\) is not \(\id{b} = \kw{true}\) or
even \id{b}.

\end{frame}

% ------------------------------------------------------------------------
%
\begin{frame}
\frametitle{Search/Booleans (cont)}

Let us take the \proc{Bool}.\proc{And} function:

\begin{columns}
  \column{0.5\textwidth}
    \begin{align*}
      \proc{And} (\proc{True}, \proc{True}) &\rightarrow \proc{True}\\
      \proc{And} (\id{x}, \proc{False}) &\rightarrow \proc{False}\\
      \proc{And} (\proc{False}, \id{x}) &\rightarrow \proc{False}
    \end{align*}
  \column{0.5\textwidth}
    \begin{codebox}
      \Procname{\(\proc{And} (\id{b_1}, \id{b_2})\)}
      \zi	\If \id{b_1} = \proc{False} \kw{or} \id{b_2} = \proc{False}
      \zi	\Then \(\kw{result} \gets \proc{False}\)
      \zi	\Else \(\kw{result} \gets \proc{True}\)
      \zi	\End
    \end{codebox}
\end{columns}

\medskip

Because there is an \emph{order} on the operations, we have been able
to gather the three rules into one conditional. Note that \kw{or} is
\textbf{sequential}: if the first argument evaluates to \proc{True}
the second argument is not computed (this can save time and
memory). Hence this test is better than \(\If \,\, (\id{s_1},
\id{s_2}) = (\proc{True}, \proc{True}) \dots\)

\end{frame}

% ------------------------------------------------------------------------
%
\begin{frame}
\frametitle{Search/Booleans (cont)}

The \proc{Or} function, as we defined it is easy to write as an algorithm:
\[
\proc{Or} (\id{b_1}, \id{b_2}) \rightarrow \proc{Not} (\proc{And}
(\proc{Not} (\id{b_1}), \proc{Not} (\id{b_2})))
\]
becomes simply
\begin{codebox}
\Procname{\(\proc{Or} (\id{b_1}, \id{b_2})\)}
\zi	\(\kw{result} \gets \proc{Not} (\proc{And} (\proc{Not} (\id{b_1}),
\proc{Not} (\id{b_2})))\)
\end{codebox}
Remember that \(\gets\) in an algorithm is not a rewriting step but an
assignment. This function is defined in terms of other functions
(\proc{Not} and \proc{Or}) which are called using an underlying
\textbf{call-by-value strategy}, i.e., the \textbf{arguments} are
computed first, then passed associated to the parameters in order to
compute the \textbf{body} of the (called) function.

\end{frame}

% ------------------------------------------------------------------------
%
\begin{frame}
\frametitle{Search/Stacks}

\begin{columns}
  \column{0.5\textwidth}
    Let us consider again the stacks: \quad \(\proc{Pop} (\proc{Push}
    (\id{x})) \rightarrow \id{x}\) \quad becomes
    \begin{codebox}
      \Procname{\(\proc{Pop} (\id{s})\)}
      \zi	\If   \id{s} = \proc{Empty}
      \zi	\Then \Error
      \zi	\Else \(\kw{result} \gets \textbf{???}\)
      \zi	\End
    \end{codebox}
    What is the problem here? 

    \bigskip

    We want to define a projection (here \proc{Pop}) without knowing
    the definition of the corresponding constructor (\proc{Push}).
  \column{0.5\textwidth}
    The reason why we do not define constructors with an algorithm is
    that we do not want to give too much details about the data
    structure, and so leave these details to the implementation (i.e.,
    the program).

    \bigskip

    Because a projection is, by definition, the inverse of a
    constructor, we cannot define them explicitly with an algorithm.
\end{columns}

\end{frame}

% ------------------------------------------------------------------------
%
\begin{frame}
\frametitle{Search/Stacks (cont)}

With the example of this aborted algorithmic definition of projection
\proc{Pop}, we realise that such definitions must be
\textbf{complete}, i.e., they must handle all values satisfying the
type of their arguments.

\bigskip

In the previous example, the type of the argument was
\proc{Stack}(\type{node}).\type{t}, so the case \proc{Empty} had to be
considered for parameter \id{s}.

\end{frame}

% ------------------------------------------------------------------------
%
\begin{frame}
\frametitle{Search/Stacks (cont)}


In the rewriting rules, the erroneous cases are not represented
because we don't want to give too much details at this stage. It is
left to the algorithm to provide error detection and basic handling.

\bigskip

Note that in algorithms, we do not provide a sophisticated error
handling: we just use a magic keyword \kw{error}. This is because we
leave for the program to detail what to do and maybe use some
specific features of its language, like exceptions.

\end{frame}

% ------------------------------------------------------------------------
%
\begin{frame}
\frametitle{Search/Stacks (cont)}

So let us consider the remaining function \proc{Append}:
\begin{align*}
   \proc{Append} (\proc{Empty}, \id{s}) 
&\rightarrow_1 \id{s}\\
   \proc{Append} (\proc{Push} (\id{e}, \id{s_1}), \id{s_2}) 
&\rightarrow_2 \proc{Push} (\id{e}, \proc{Append} (\id{s_1},
   \id{s_2}))
\end{align*}
We gather all the rules into one function and choose a proper order:
{\small
\begin{codebox}
\(\proc{Append} (\id{s_3}, \id{s_2})\)\\
\zi \If \id{s_3} = \proc{Empty}
\zi \Then \(\kw{result} \gets \id{s_2}\) \RComment This is
rule \(\rightarrow_1\)
\zi	\Else   \((\id{e}, \id{s_1}) \gets \proc{Pop} (\id{s_3})\)
                  \RComment This means \(\proc{Push} (\id{e},
                  \id{s_1}) = \id{s_3}\)
\zi		\(\kw{result} \gets \proc{Push} (\id{e},
                   \proc{Append} (\id{s_1}, \id{s_2}))\) 
                   \RComment This is rule \(\rightarrow_2\)
\zi	\End
\end{codebox}
}

\end{frame}

% ------------------------------------------------------------------------
%
\begin{frame}
\frametitle{Search/Queues}

Let us come back to the \proc{Queue} specification:
\begin{align*}
 \proc{Dequeue} (\proc{Enqueue} (\id{e}, \proc{Empty}))
&\rightarrow_1 (\proc{Empty}, \id{e}) & \\
  \proc{Dequeue} (\proc{Enqueue} (\id{e}, \id{q}))
&\rightarrow_2 (\proc{Enqueue} (\id{e}, \id{q_1}), \id{e_1})
\intertext{where \(\id{q} \neq \proc{Empty}\) and where}
\proc{Dequeue} (\id{q}) &\rightarrow_3 (\id{q_1}, \id{e_1}) &
\end{align*}

\end{frame}

% ------------------------------------------------------------------------
%
\begin{frame}
\frametitle{Search/Queues/Dequeuing}

We can write the corresponding algorithmic function as
\begin{codebox}
\(\proc{Dequeue} (\textbf{\id{q_2}})\)\\
\zi	\If \id{q_2} = \proc{Empty}
\zi	\Then \Error
\zi	\Else   \((\id{e}, \id{q}) \gets \proc{Enqueue}^{-1}(\id{q_2})\)
\zi		\If \(\id{q} = \proc{Empty}\)
\zi		\Then \(\kw{result} \gets (\id{q}, \id{e})\)
\RComment Rule \(\rightarrow_1\)
\zi		\Else \((\id{q_1}, \id{e_1}) \gets
                         \proc{Dequeue} (\textbf{\id{q}})\) \RComment
                         Rule \(\rightarrow_3\)
\zi		      \(\kw{result} \gets (\proc{Enqueue}
(\id{e},\id{q_1}), \id{e_1})\) \RComment Rrule \(\rightarrow_2\)
\zi		\End
\zi	\End
\end{codebox}
Termination is due to \((\id{e}, \id{q}) =
\proc{Enqueue}^{-1}(\id{q_2}) \Rightarrow {\cal H} (\id{q_2}) = {\cal H}
(\id{q}) + 1 > {\cal H} (\id{q})\).

\end{frame}

% ------------------------------------------------------------------------
%
\begin{frame}
\frametitle{Search/Binary trees/Left prefix}

Let us come back to the \proc{Bin-tree} specification and the left
prefix traversal:
{\small
\begin{align*}
   \proc{Lpref} (\proc{Empty}) 
&\rightarrow \proc{Stack}(\type{node}).\proc{Empty}\\
   \proc{Lpref} (\proc{Make} (\id{e}, \id{t_1}, \id{t_2})) 
&\rightarrow \proc{Push} (\id{e}, \proc{Append} (\proc{Lpref}
   (\id{t_1}), \proc{Lpref} (\id{t_2})))
\end{align*}
}
We get the corresponding algorithmic function
\begin{codebox}
\(\proc{Lpref} (\id{t})\)\\
\zi	\If \id{t} = \proc{Empty}
\zi	\Then \(\kw{result} \gets \proc{Stack}(\type{node}).\proc{Empty}\)
\zi	\Else \((\id{e}, \id{t_1}, \id{t_2}) \gets
                \proc{Make}^{-1}(\id{t})\)
\zi	      \(\kw{result} \gets \proc{Push} (\id{e},
                                                \proc{Append}
                                                  (\proc{Lpref} (\id{t_1}), 
                                                   \proc{Lpref}
                                                   (\id{t_2})))\)
\zi	\End
\end{codebox}

\end{frame}

% ------------------------------------------------------------------------
%
\begin{frame}
\frametitle{Search/Binary trees/Left infix}

Similarly, we can consider again the left infix traversal:
\begin{align*}
   \proc{Linf} (\proc{Empty})
&\rightarrow \proc{Stack}(\type{node}).\proc{Empty}\\
   \proc{Linf} (\proc{Make} (\id{e}, \id{t_1}, \id{t_2})) 
&\rightarrow \proc{Append} (\proc{Linf} (\id{t_1}), 
   \proc{Push} (\id{e}, \proc{Linf} (\id{t_2})))
\end{align*}
Hence
\begin{codebox}
\(\proc{Linf} (\id{t})\)\\
\zi	\If \id{t} = \proc{Empty}
\zi	\Then \(\kw{result} \gets \proc{Stack}(\type{node}).\proc{Empty}\)
\zi	\Else \((\id{e}, \id{t_1}, \id{t_2}) \gets
                \proc{Make}^{-1}(\id{t})\)
\zi	        \(\kw{result} \gets \proc{Append} (\proc{Linf} (\id{t_1}),
                  \proc{Push} (\id{e}, \proc{Linf} (\id{t_2})))\)
\zi	\End
\end{codebox}

\end{frame}

% ------------------------------------------------------------------------
%
\begin{frame}
\frametitle{Search/Binary trees/Left postfix}

Similarly, we can consider again the left postfix traversal:
{\small
\begin{align*}
   \proc{Lpost} (\proc{Empty}) 
&\rightarrow \proc{Stack}(\type{node}).\proc{Empty}\\
   \proc{Lpost} (\proc{Make} (\id{e}, \id{t_1}, \id{t_2})) 
&\rightarrow
\intertext{\(\proc{Append} (\proc{Lpost} (\id{t_1}),
   \proc{Append} (\proc{Lpost} (\id{t_2}), \proc{Push} (\id{e},
   \proc{Empty})))\)}
\end{align*}
}
\begin{codebox}
\Procname{\(\proc{Lpost} (\id{t})\)}
\zi	\If \id{t} = \proc{Empty}
\zi	\Then \(\kw{result} \gets \proc{Stack}(\type{node}).\proc{Empty}\)
\zi	\Else \((\id{e}, \id{t_1}, \id{t_2}) \gets
                \proc{Make}^{-1}(\id{t})\)
\zi        \(n \gets                   \proc{Append} (\proc{Lpost} (\id{t_2}), 
                                 \proc{Push} (\id{e}, \proc{Empty}))\)
\zi	        \(\kw{result} \gets \proc{Append} (\proc{Lpost} (\id{t_1}), n)\)
\zi	\End
\end{codebox}

\end{frame}

% ------------------------------------------------------------------------
%
\begin{frame}
\frametitle{Search/Binary trees/Breadth-first}

Let us recall the rewrite system of \proc{Roots} (see page~\pageref{roots}):
\begin{align*}
   \proc{Roots} (\proc{Forest}(\type{node}).\proc{Empty}) 
&\rightarrow_1 \proc{Stack} (\type{node}).\proc{Empty}\\
   \proc{Roots} (\proc{Push} (\proc{Empty}, \,\id{f}))
&\rightarrow_2 \proc{Roots} (\,\id{f})\\
   \proc{Roots} (\proc{Push} (\proc{Make} (\id{r}, \id{t_1},
   \id{t_2}), \,\id{f}))
&\rightarrow_3 \proc{Push} (\id{r}, \proc{Roots} (\,\id{f}))
\end{align*}
Rule \(\rightarrow_2\) skips any empty tree in the forest.

\end{frame}

% ------------------------------------------------------------------------
%
\begin{frame}
\frametitle{Search/Binary trees/Breadth-first (cont)}

The corresponding algorithmic definition is
{\small
\begin{codebox}
\Procname{\(\proc{Roots} (\,\id{f_1})\)}
\zi	\If \,\id{f_1} = \proc{Forest}(\type{node}).\proc{Empty}
\zi	\Then \(\kw{result} \gets
                \proc{Stack}(\type{node}).\proc{Empty}\) 
                \RComment Rule \(\rightarrow_1\)
\zi	\Else \((\id{t}, \,\id{f}) \gets \proc{Pop} (\,\id{f_1})\)
              \RComment \(\proc{Push} (\id{t}, \,\id{f}) =
              \id{f_1}\)
\zi		\If \id{t} = \proc{Empty}
\zi		\Then \(\kw{result} \gets \proc{Roots} (\,\id{f})\)
                        \RComment Rule \(\rightarrow_2\)
\zi		\Else \(\kw{result} \gets \proc{Push} (\proc{Root}
                        (\id{t}), \proc{Roots} (\,\id{f}))\)\!\!
                        \RComment\!\!\!Rule \(\rightarrow_3\)
\zi		\End
\zi	\End
\end{codebox}
}

\end{frame}

% ------------------------------------------------------------------------
%
\begin{frame}
\frametitle{Search/Binary trees/Breadth-first (cont)}

Let us recall the rewrite system of \proc{Next} (see
page~\pageref{next_level}): 
\begin{align*}
   \proc{Next} (\proc{Forest}(\type{node}).\proc{Empty}) 
&\rightarrow_1 \proc{Forest}
   (\type{node}).\proc{Empty}\\
   \proc{Next} (\proc{Push} (\proc{Empty}, \,\id{f}))
&\rightarrow_2 \proc{Next} (\,\id{f})\\
   \proc{Next} (\proc{Push} (\proc{Make} (\id{r}, \id{t_1},
   \id{t_2}), \,\id{f}))
&\rightarrow_3 \proc{Push} (\id{t_1}, \proc{Push} (\id{t_2},
   \proc{Next} (\,\id{f})))
\end{align*}

\end{frame}

% ------------------------------------------------------------------------
%
\begin{frame}
\frametitle{Search/Binary trees/Breadth-first (cont)}

The corresponding algorithmic definition is
{\small
\begin{codebox}
\Procname{\(\proc{Next} (\,\id{f_1})\)}
\zi	\If \,\id{f_1} = \proc{Forest}(\type{node}).\proc{Empty}
\zi	\Then \(\kw{result} \gets
                \proc{Stack}(\type{node}).\proc{Empty}\) 
                \RComment This is rule \(\rightarrow_1\)
\zi	\Else \((\id{t}, \,\id{f}) \gets \proc{Pop} (\,\id{f_1})\)
              \RComment This means \(\proc{Push} (\id{t}, \,\id{f}) =
              \id{f_1}\)
\zi		\If \id{t} = \proc{Empty}
\zi		\Then \(\kw{result} \gets \proc{Next} (\,\id{f})\)
                      \RComment This is rule \(\rightarrow_2\)
\zi		\Else \RComment This is rule \(\rightarrow_3\)
\zi		\(\kw{result} \gets \proc{Push} (\proc{Left}
                  (\id{t}), (\proc{Push} (\proc{Right} (\id{t}),
                   \proc{Next} (\,\id{f}))))\)
\zi		\End
\zi	\End
\end{codebox}
}

\end{frame}

% ------------------------------------------------------------------------
%
\begin{frame}
\frametitle{Search/Binary trees/Breadth-first (cont)}

Let us recall finally the rules for function \({\cal B}\):
\begin{eqnarray*}
  {\cal B} (\proc{Forest}(\type{node}).\proc{Empty})
\rightarrow
\proc{Stack}(\type{node}).\proc{Empty}\\
  {\cal B} (\,\id{f})
\rightarrow \proc{Append} (\proc{Roots} (\,\id{f}), {\cal B}
  (\proc{Next} (\,\id{f})))
\end{eqnarray*}
where \(\id{f} \neq \proc{Empty}\).

\end{frame}

% ------------------------------------------------------------------------
%
\begin{frame}
\frametitle{Search/Binary trees/Breadth-first (cont)}

The corresponding algorithmic definition is
\begin{codebox}
\Procname{\({\cal B} (\,\id{f})\)}
\zi	\If \,\id{f} = \proc{Empty}
\zi	\Then \(\kw{result} \gets \proc{Stack}(\type{node}).\proc{Empty}\) 
\zi	\Else \(\kw{result} \gets \proc{Append} (\proc{Roots}
(\,\id{f}), {\cal B} (\proc{Next} (\,\id{f})))\)
\zi	\End
\end{codebox}
And
\begin{codebox}
\Procname{\(\proc{BFS} (\id{t})\)}
\(\kw{result} \gets {\cal B}
     (\proc{Stack}(\type{node}).\proc{Push} (\id{t}, \proc{Empty}))\)
\end{codebox}

\end{frame}

% ------------------------------------------------------------------------
%
\begin{frame}
\frametitle{Search/Binary trees/Breadth-first (cont)}

Let us imagine we want to realise a breadth-first search on tree
\id{t} \emph{up to a given depth \id{d}} and return the encountered
nodes. Let us reuse the name \proc{BFS} to call such a function whose
signature is then
\[
\proc{BFS}: \proc{Bin-Tree}(\type{node}).\type{t}
\times \textbf{\type{int}} \rightarrow
\proc{Stack}(\type{node}).\type{t}
\]
where \type{int} denotes the positive integers. 

\bigskip

A possible defining equation can be:
\[
\proc{BFS}_{\id{d}} (\id{t}) = {\cal B}_{\id{d}}
(\proc{Push} (\id{t}, \proc{Empty}))
\quad \text{with} \; \id{d} \geqslant 0
\]
where \({\cal B}_{\id{d}} (\,\id{f})\) is the stack of
traversed nodes in the forest \,\id{f} up to depth \(\id{d} \geqslant
0\) in a left-to-right breadth-first way.

\end{frame}

% ------------------------------------------------------------------------
%
\begin{frame}
\frametitle{Search/Binary trees/Breadth-first (cont)}

Here are some possible equations defining \({\cal B}_{\id{d}}\):
\begin{align*}
   {\cal B}_{\id{d}} (\proc{Empty})
&= \proc{Empty} 
& \\
   {\cal B}_{0} (\,\id{f})
&= \proc{Roots} (\,\id{f}) 
& \\
   {\cal B}_{\id{d}} (\,\id{f})
&= \proc{Append} (\proc{Roots} (\,\id{f}), {\cal B}_{\id{d}-1}
   (\proc{Next} (\,\id{f})))
& \text{if} \; \id{d} > 0
\end{align*}
The difference between \({\cal B}_{\id{d}}\) and \({\cal B}\) is
the depth limit \id{d}.

\end{frame}

% ------------------------------------------------------------------------
%
\begin{frame}
\frametitle{Search/Binary trees/Breadth-first (cont)}

In order to write the algorithm corresponding to \({\cal
  B}_{\id{d}}\), it is good practice not to use subscripts, like
\id{d}, and use a regular parameter instead:
\begin{codebox}
\Procname{\({\cal B} (\id{d}, \,\id{f})\)}
\zi	\If \id{f} = \proc{Forest}(\type{node}).\proc{Empty}
\zi	\Then \(\kw{result} \gets
                \proc{Stack}(\type{node}).\proc{Empty}\)
\zi	\ElseIf \id{d} = 0
\zi		\Then \(\kw{result} \gets \proc{Roots} (\,\id{f})\)
\zi		\ElseIf \(\id{d} > 0\)
\zi			\Then \(\kw{result} \gets \proc{Append}
                               (\proc{Roots} (\,\id{f}), {\cal B}
                               (\id{d}-1, \proc{Next}
                               (\,\id{f})))\)
\zi			\Else \kw{error}
\zi			\End
\zi		\End
\zi	\End
\end{codebox}

\end{frame}

% ------------------------------------------------------------------------
%
\begin{frame}
\frametitle{Search/Binary trees/Depth-first}

We gave several algorithms for left-to-right depth-first traversals:
prefix (\proc{Lpref}), postfix (\proc{Lpost}) and infix (\proc{Linf}).

\bigskip

What if we want to limit the depth of such traversals, like
\proc{Lpref}?
{\footnotesize
\begin{align*}
   \proc{Lpref}_{\id{d}} (\proc{Empty}) 
&= \proc{Empty}\\
   \proc{Lpref}_{0} (\proc{Make} (\id{e}, \id{t_1}, \id{t}_2))
&= \proc{Push} (\id{e}, \proc{Empty})\\
   \proc{Lpref}_{\id{d}} (\proc{Make} (\id{e}, \id{t_1}, \id{t_2})) 
&= \proc{Push} (\id{e}, \proc{Append} (\proc{Lpref}_{\id{d}-1}
   (\id{t_1}), \proc{Lpref}_{\id{d}-1} (\id{t_2})))
\end{align*}
}
where \(\id{d} > 0\).

\end{frame}
