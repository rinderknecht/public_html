\paragraph{Questions.} Define predicates \texttt{enqueue1/3},
\texttt{dequeue1/3}, \texttt{enqueue2/3} and \texttt{dequeue2/3} for
the two following cases.
\begin{enumerate}

  \item \textbf{A one-stack implementation.} A simple idea to
    implement one queue is to use one stack. In this
    case, \texttt{enqueue1/3} is simply pushing
    and \texttt{dequeue1/3} removes the item at the bottom of the
    stack.

  \item \textbf{A two-stack implementation.} We can implement a
    queue with two stacks instead of one: one for enqueuing, one for
    dequeuing.
    \[
    \begin{array}{r@{\;}cc|c|c|c|c|c|c|cc@{\;}l}
      \cline{3-6}\cline{8-10}
      \texttt{enqueue2/3} & \rightarrow & & a & b & c & & d & e & &
      \rightarrow & \texttt{dequeue2/3}\\
      \cline{3-6}\cline{8-10}
    \end{array}
    \]
    \noindent So \texttt{enqueue2/3} is pushing on the first stack
    and \texttt{dequeue2/3} is popping on the second. If the second
    stack is empty, we swap the stacks and reverse the (new) second:
    \[
    \begin{array}{lr@{\;}cc|c|c|c|c|cc@{\;}l}
      \cline{4-7}\cline{9-9}
      \text{If} & \texttt{enqueue2/3} & \rightarrow & & a & b & c & & &
      \rightarrow & \texttt{dequeue2/3} \, \textbf{???}\\
      \cline{4-7}\cline{9-9}\\
      \cline{4-4}\cline{6-9}
      \text{then} & \texttt{enqueue2/3} & \rightarrow & & & a & b & c & &
      \rightarrow & \texttt{dequeue2/3}\\
      \cline{4-4}\cline{6-9}
    \end{array}
    \]
    \noindent Let the pair \texttt{\{S,T\}} denote the queue where
    \texttt{S} is the stack for enqueuing and \texttt{T} the stack for
    dequeuing.

\end{enumerate}
