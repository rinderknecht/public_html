\paragraph{Answers.} When trying to matching two terms, the procedure
is:
\begin{itemize}

  \item if the two terms are constants, they match if they are
    verbatim the same;

  \item if one of the terms is a variable, then they match;

  \item if the two terms are functors, they match if and only if 
  \begin{itemize}

    \item they are the same (name),
 
    \item they have the same number of arguments (subterms),

    \item if they have arguments, the \(n\)-th argument of one term
      matches the \(n\)-th argument of the other.

  \end{itemize}
    
\end{itemize}
\noindent It is often a good idea to represent the terms to be matched
by trees.

\begin{enumerate}

  \item The figure of the query
    \verb|?- f(g(X,X), h(Y,Y)) = f(g(Z), Z).| is
  \begin{center}
    \includegraphics{tree-1-1}
  \end{center}
  So the answer is \texttt{No}, because the subterm \texttt{g(X,X)}
  does not match \texttt{g(Z)}: the number of arguments of \texttt{g}
  is different;

  \item The figure of query
  \verb|?- f(g(X,X),h(Y,Y)) = f(g(h(W,a),Z),Z).| is
  \begin{center}
    \includegraphics{tree-1-2}
  \end{center}
  The steps are
\begin{align*}
X &= h(W,a) & X &= h(W,a) & h(W,a) &= h(Y,Y) & W &= Y\\
X &= Z      & X &= h(Y,Y) & X      &= h(Y,Y) & Y &= a\\
Z &= h(Y,Y) & Z &= X      & Z      &= X      & X &= h(Y,Y)\\
  &         &   &         &        &         & Z &= X
\end{align*}
  \noindent Finally
\begin{verbatim}
X = h(a,a)
Y = a
W = a
Z = h(a,a)
Yes
\end{verbatim}
  \noindent This answer means that if we take the two trees of the
  query and replace the variables according to the substitution, we
  get the exact same tree. Indeed, we get
  \begin{center}
    \includegraphics{tree-1-2u}
  \end{center}


  \item The figure of 
  \verb|?- f(g(X,X), h(_,_)) = f(g(h(W,a),Z), Z).| is
  \begin{center}
    \includegraphics{tree-1-3}
   \end{center}
  \noindent Note that each underscore has been replaced with a unique
  variable, i.e. different from all the other variables. The steps are
  now
  \begin{align*}
                   X &= h(W,a) &                X &= h(W,a) & X &= h(W,a) & X &= h(\alpha,a)\\
                   X &= Z      &                X &=Z       & Z &= X      & Z &= h(\alpha,a)\\
    h(\alpha, \beta) &= Z      & h(\alpha, \beta) &= h(W,a) & W &= \alpha & W &= \alpha\\
                     &         &                  &         &   &         & \beta &= a
  \end{align*}
  \noindent Finally
\begin{alltt}
X = h(\(\alpha\),a)
W = \(\alpha\)
Z = h(\(\alpha\),a) 
Yes
\end{alltt}
  \noindent This means that if we replace the variables in the two
  initial trees, we get the same. Indeed, we get
  \begin{center}
    \includegraphics{tree-1-3u}
  \end{center}

  \item The figure of \verb|?- f(x(A,B), C) = f(C, x(B,A)).| is
  \begin{center}
    \includegraphics{tree-1-4}
  \end{center}
  \noindent The steps are
  \begin{align*}
    x(A,B) &= C      & x(A,B) &= x(B,A) & A &= B\\
    C      &= x(B,A) & C      &= x(B,A) & C &= x(B,A)
  \end{align*}
  \begin{multicols}{2}
  \noindent Finally
\begin{alltt}
A = \(\alpha\)
B = \(\alpha\)
C = x(\(\alpha\),\(\alpha\)) 
Yes
\end{alltt}
  \par\vfill\columnbreak
  \noindent This means that if we replace these variables by their
  value in the original trees, we get the same tree. Indeed, we get
  \begin{center}
    \includegraphics{tree-1-4u}
  \end{center}
  \end{multicols}

\end{enumerate}
