\noindent \textbf{Answer.} Consider the all the possible cases:
\begin{itemize}
 
  \item If the new rule is first (i.e. number 1), then the relation is
    not what we want since, for instance
{\small
\begin{verbatim}
?- delete(a,[a],[a]).
Yes
\end{verbatim}
}
\noindent instead of \texttt{No}.

  \item If the new rule is second (i.e. number 3), then the relation
    is broken too since, for example
{\small
\begin{verbatim}
?- delete(a,[b,a],[b,a]).
Yes
\end{verbatim}
}
\noindent instead of \texttt{No}.

  \item If the new rule is last (i.e. number 5), then the relation is
    correct. Rule 2 handles the case when \texttt{X} is found on the
    top of \texttt{S}; rule 4 handles the case when \texttt{X} is not
    found on the top of \texttt{S} \emph{but} is below (i.e. in
    \texttt{A}). Therefore, the new rule in last position will handle
    the case where \texttt{S}, perhaps empty, does not contain
    \texttt{X}. Since the heads of rules 2 and 4 match a non empty
    \texttt{S}, \texttt{X} must only match \verb|[]| in the new rule
    5, which can then be further simplified as 
{\small
\begin{verbatim}
delete(_,[],[]).                            % Rule 5
\end{verbatim}
}
\noindent Then this rule can be moved at any position, since rules 2
and 4 do not match an empty \texttt{S}. For example:
%%\PrologIn{delete_bis}
{\small\verbatiminput{delete_bis.pl}}

\end{itemize}
