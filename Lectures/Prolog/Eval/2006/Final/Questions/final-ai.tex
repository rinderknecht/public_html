%%-*-latex-*-

\documentclass[11pt,a4paper]{article}

\input{trace}
\input{prolog}

\title{Answers to the mid-term exam on\\ Introduction to the Internet}

\author{Christian Rinderknecht}
\date{14 December 2006}

\begin{document}

\maketitle
\thispagestyle{empty}

\begin{enumerate}

  \item Let \texttt{delete(X,S,T)} be a relation true when the list
    \texttt{T} contains the same items as the list \texttt{S}, in the
    same order, except the first \texttt{X} in \texttt{S} (starting
    from the top). One possible definition is
    \PrologIn{delete}
%    {\small\verbatiminput{delete.pl}}

    \noindent We want to modify this definition so that the relation
    is true when \texttt{S} does not contain any \texttt{X}.
 
    \paragraph{Question 1.} Give the lower and upper bounds of a number
represented in balanced ternary with \(n\) trits.



  \item
    \noindent\textbf{Question.} Define a relation
    \texttt{catenate(U,V,W)} to be true when list \texttt{W} is made
    of list \texttt{U} followed by list \texttt{V}. For example
{\small 
\begin{verbatim}
?- catenate(U,[3,4],[0,1,2,3,4]).
U = [0,1,2] ;
No
\end{verbatim}
}


  \item 
    \paragraph{Question.} Define the meaning of the pointers
\(\upharpoonleft\), \(\upharpoonright\) and \(\Uparrow\) presented in
class and show how the input is analysed using the transition diagrams
of the previous questions.



  \item 
      \begin{enumerate}

    \item D�finissez les arbres binaires.

    \item Qu'est-ce que le parcours en profondeur et en largeur?

    \item Parcours pr�fixe, postfixe et infixe?

    \item \textbf{D�nombrements sur les arbres binaires.} �tablir:
      \begin{enumerate}

      \item Un arbre binaire � \(n\) n{\oe}uds poss�de \(n+1\) branches
        vides.

      \item Dans un arbre binaire � \(n\) n{\oe}uds, le nombre de
        n{\oe}uds sans fils est inf�rieur ou �gal � \((n+1)/2\). Il y
        a �galit� si et seulement si tous les n{\oe}uds ont z�ro ou
        deux fils.

      \item La hauteur d'un arbre binaire non vide � \(n\) n{\oe}uds
        est comprise entre \(\lfloor\log_2{n}\rfloor\) et \(n-1\).

      \end{enumerate}

    \item Qu'est-ce qu'un arbre binaire �quilibr� en hauteur?

  \end{enumerate}


\end{enumerate}


\end{document}
