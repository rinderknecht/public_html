%%-*-latex-*-

\documentclass[11pt,a4paper]{article}

\input{trace}
\input{prolog}

\title{Final examination on Logic Circuit Design}

\author{Christian Rinderknecht}
\date{14 December 2006}

\begin{document}

\maketitle
\thispagestyle{empty}

\begin{enumerate}

  \item Let \texttt{delete(X,S,T)} be a relation true when the list
    \texttt{T} contains the same items as the list \texttt{S}, in the
    same order, except the first \texttt{X} in \texttt{S} (starting
    from the top). One possible definition is
    \PrologIn{delete}
%    {\small\verbatiminput{delete.pl}}

    \noindent We want to modify this definition so that the relation
    is true when \texttt{S} does not contain any \texttt{X}.
 
    \paragraph{Question.} Convert from decimal to 2-complement 8-bit
binary numbers.
\begin{center}
\begin{tabular}{>{\sf(}c<{)}>{$}r<{$}>{\sf(}c<{)}>{$}r<{$}>{\sf(}c<{)}>{$}r<{$}}
a & 105 & b & -28 & c & 76\\ d & -25 & e & -101 & f & 127
\end{tabular}
\end{center}


  \item
    \paragraph{Question.} Which of the one-stack or two-stack
implementation is the most efficient for dequeuing? What are the best
and worst cases for each?


  \item 
    %%-*-latex-*-

\paragraph{III. Question.}

Consider sending a large file of \(F\) bits from host \(A\) to host
\(B\). There are two links (and one switch) between them and the links
are uncongested (that is, no queuing delays). Host \(A\) segments the
file into segments of \(S\) bits each and adds 40 bits of header to
each segment, forming packets of \(L = 40 + S\) bits. Each link has a
transmission rate of \(R\) bit/s. Assuming that \(F/S\) is an integer,
find the value of \(S\) that minimises the delay of moving the file
from host \(A\) to host \(B\). Disregard propagation delay.


  \item 
    \paragraph{Question.} State any overflow or underflow.
\[
\renewcommand\arraystretch{0.85}
\begin{array}{lr@{\;}c@{\;}c@{\;}c@{\;}c@{\;}cc@{\;}c@{\;}c@{\;}cc@{}c@{\;}c@{\;}c@{\;}c@{\;}cc@{\;}c@{\;}c@{\;}c@{}}
\textsf{(a)} &   & 0&1&1&1 & & 0&1&0&1 & & 1&0&0&1 & & 0&0&0&0\\
             & + & 0&0&1&0 & & 1&0&0&1 & & 0&0&1&1 & & 0&1&0&1\\
\cline{3-21}\\
\textsf{(b)} &   & 0&0&1&1 & & 0&1&1&0 & & 0&0&0&1 & & 0&1&1&1\\
             & + & 0&0&0&1 & & 0&1&0&0 & & 0&1&1&0 & & 1&0&0&0\\
\cline{3-21}\\
\textsf{(c)} &   & 1&0&0&1 & & 0&1&0&1 & & 1&0&0&0 & & 0&1&1&1\\
             & + & 0&0&1&1 & & 0&1&0&1 & & 0&1&1&0 & & 0&0&1&0\\
\cline{3-21}\\
\end{array}
\]



\end{enumerate}


\end{document}
