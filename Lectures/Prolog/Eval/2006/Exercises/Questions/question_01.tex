\paragraph{Question.} Define a relation \texttt{diff} with
three arguments, the first being the function to be differentiated,
the second the variable used for differentiation and the third is the
derivative. For example \texttt{diff(F,var(x),DF)} means that
\texttt{DF} is the derivative of \texttt{F} on variable \texttt{x}.

\paragraph{Hint.} You will need the goal \verb|\+(X=Y)| which
means ``\texttt{X} and \texttt{Y} do not match.''

\paragraph{Example.} Now you can make queries like the following: 
{\small
\begin{verbatim}
?- diff(plus(times(var(x),var(x)),var(y)), var(x), D).
D = plus(plus(times(var(x),num(1)),times(var(x),num(1))),num(0));
No
\end{verbatim}
}
\noindent which proves that
\[
\frac{d}{dx}(x \times x + y) = (x \times 1) + (x \times 1) + 0
\]

