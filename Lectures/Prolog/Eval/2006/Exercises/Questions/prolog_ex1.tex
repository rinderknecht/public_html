%%-*-latex-*-

\documentclass[11pt,a4paper]{article}

\usepackage{amsmath,amssymb}
\usepackage{verbatim}
\usepackage{graphicx}

\input{trace}
\input{prolog}

\title{Answers to the mid-term exam on\\ Introduction to the Internet}

\author{Christian Rinderknecht}
\date{12 October 2006}

\begin{document}

\maketitle
\thispagestyle{empty}

\noindent The following equations define the formal differentiation of
a function:
\begin{gather*}
\dfrac{dx}{dx}    = 1 \qquad
\dfrac{dy}{dx}    = 0 \quad \text{where} \;y \neq x\; \text{or}\;
                            y \in\mathbb{N}\\
\dfrac{d}{dx}(f + g) = \dfrac{df}{dx} + \dfrac{dg}{dx}
\qquad
\dfrac{d}{dx}(f - g) = \dfrac{df}{dx} - \dfrac{dg}{dx}\\
\dfrac{d(-f)}{dx} = -{\dfrac{df}{dx}}
\qquad
\dfrac{d}{dx}(f \times g) = f \times {\dfrac{dg}{dx}} +
g \times {\dfrac{df}{dx}} 
\end{gather*}

\noindent Let us call \texttt{plus}, \texttt{minus} and \texttt{times}
the relations for, respectively, addition (\(+\)), subtraction
(\(-\)) and multiplication (\(\times\)). Variables are defined by the
predicate \texttt{var} and numbers by \texttt{num}. For instance,
\texttt{minus(var(x),var(y))} denotes \(x-y\); \texttt{minus(var(x))}
denotes \(-x\) (note that subtraction can have one or two arguments);
\texttt{times(num(3),plus(var(x),var(x)))} means \(3(x + x)\), etc.

\paragraph{Question 1.} Give the lower and upper bounds of a number
represented in balanced ternary with \(n\) trits.


\noindent\textbf{Question.} Define a relation
    \texttt{catenate(U,V,W)} to be true when list \texttt{W} is made
    of list \texttt{U} followed by list \texttt{V}. For example
{\small 
\begin{verbatim}
?- catenate(U,[3,4],[0,1,2,3,4]).
U = [0,1,2] ;
No
\end{verbatim}
}


\end{document}
