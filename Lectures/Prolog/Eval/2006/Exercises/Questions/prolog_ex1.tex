%%-*-latex-*-

\documentclass[11pt,a4paper]{article}

\usepackage{amsmath,amssymb}
\usepackage{verbatim}
\usepackage{graphicx}

\input{trace}
\input{prolog}

\title{Final examination on Logic Circuit Design}

\author{Christian Rinderknecht}
\date{12 October 2006}

\begin{document}

\maketitle
\thispagestyle{empty}

\noindent The following equations define the formal differentiation of
a function:
\begin{gather*}
\dfrac{dx}{dx}    = 1 \qquad
\dfrac{dy}{dx}    = 0 \quad \text{where} \;y \neq x\; \text{or}\;
                            y \in\mathbb{N}\\
\dfrac{d}{dx}(f + g) = \dfrac{df}{dx} + \dfrac{dg}{dx}
\qquad
\dfrac{d}{dx}(f - g) = \dfrac{df}{dx} - \dfrac{dg}{dx}\\
\dfrac{d(-f)}{dx} = -{\dfrac{df}{dx}}
\qquad
\dfrac{d}{dx}(f \times g) = f \times {\dfrac{dg}{dx}} +
g \times {\dfrac{df}{dx}} 
\end{gather*}

\noindent Let us call \texttt{plus}, \texttt{minus} and \texttt{times}
the relations for, respectively, addition (\(+\)), subtraction
(\(-\)) and multiplication (\(\times\)). Variables are defined by the
predicate \texttt{var} and numbers by \texttt{num}. For instance,
\texttt{minus(var(x),var(y))} denotes \(x-y\); \texttt{minus(var(x))}
denotes \(-x\) (note that subtraction can have one or two arguments);
\texttt{times(num(3),plus(var(x),var(x)))} means \(3(x + x)\), etc.

\paragraph{Question.} Convert from decimal to 2-complement 8-bit
binary numbers.
\begin{center}
\begin{tabular}{>{\sf(}c<{)}>{$}r<{$}>{\sf(}c<{)}>{$}r<{$}>{\sf(}c<{)}>{$}r<{$}}
a & 105 & b & -28 & c & 76\\ d & -25 & e & -101 & f & 127
\end{tabular}
\end{center}

\paragraph{Question.} Which of the one-stack or two-stack
implementation is the most efficient for dequeuing? What are the best
and worst cases for each?


\end{document}
