\noindent A \emph{queue} is a like stack where items are pushed on one
end and popped on the other end. Adding an item in a queue is called
\emph{enqueuing}, whereas removing one is called
\emph{dequeuing}. Compare the following figures:
\[
\begin{array}{rr@{\;}cc|c|c|c|c|c|cc@{\;}l}
\cline{4-10}
\text{Queue}: & \textsc{Enqueue} & \rightarrow & & a & b & c & d & e &
& \rightarrow & \textsc{Dequeue}\\
\cline{4-10}\\
\cline{4-9}
\text{Stack}: & \textsc{Push}, \textsc{Pop} & \leftrightarrow & & a &
b & c & d & e &&&\\
\cline{4-9}
\end{array}
\]
\noindent If \texttt{enqueue(I,In,Out)} is proved, then \texttt{Out}
is the queue \texttt{In} in which item \texttt{I} has been enqueued;
if \texttt{dequeue(In,Out,I)} is proved, then \texttt{Out} is the
queue \texttt{In} from which item \texttt{I} has been dequeued.

