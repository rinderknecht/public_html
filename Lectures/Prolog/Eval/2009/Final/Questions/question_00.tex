A \emph{binary search tree} (BST) is a binary tree whose nodes contain
items who can be totally ordered and dispersed such that all node, the
values of the left subtree are smaller than the value of the node,
which is, in turn, smaller than the values of the right subtree. See
figure~\ref{fig:bst} for an example.
A node is added to BST as a leaf. For instance, adding 10 leads to the
BST shown in figure~\ref{fig:bst_10}.
\begin{figure}[!h]
\centering
\subfloat[A BST\label{fig:bst}]{%
\pstree[nodesep=2pt,levelsep=20pt,treesep=14pt]{\erlnode{5}}{
  \pstree{\erlnode{1}}{
    \erlnode{0}
    \pstree{\erlnode{3}}{
      \erlnode{2}
      \erlnode{4}
    }
  }
  \pstree{\erlnode{7}}{
    \erlnode{6}
    \pstree{\erlnode{9}}{
      \erlnode{8}
      \Tn
    }
  }
}
}
\qquad
\subfloat[Addition of 10.\label{fig:bst_10}]{
\pstree[nodesep=2pt,levelsep=20pt,treesep=14pt]{\erlnode{5}}{
  \pstree{\erlnode{1}}{
    \erlnode{0}
    \pstree{\erlnode{3}}{
      \erlnode{2}
      \erlnode{4}
    }
  }
  \pstree{\erlnode{7}}{
    \erlnode{6}
    \pstree{\erlnode{9}}{
      \erlnode{8}
      \erlnode{10}
    }
  }
}
}
\end{figure}
Write a function \erlcode{add/2}, such that the call
\erlcode{add(\(N\),\(T\))} evaluates into a BST containing the nodes
of BST \(T\) plus \(N\), by adding \(N\) as a leaf.
