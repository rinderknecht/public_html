%%-*-latex-*-

\documentclass{article}

\usepackage{vaucanson-g}
\usepackage{pst-eps}
\usepackage{xcolor}

\begin{document}

\TeXtoEPS
\SetEdgeLineColor{blue} 
\VCDraw{%
\begin{VCPicture}{(0,0)(13,3)}

\StateVar[\textcolor{blue}{1},\textcolor{red}{1}]{(0,0)}{p1}
\nput{180}{p1}{\Large \((0,0)\)}

\StateVar[\textcolor{blue}{1},\textcolor{red}{0}]{(3,0)}{p2}
\nput{0}{p2}{\Large \((1,0)\)}

\StateVar[\textcolor{blue}{0},\textcolor{red}{0}]{(3,3)}{p3}
\nput{0}{p3}{\Large \((1,1)\)}

\StateVar[\textcolor{blue}{1},\textcolor{red}{0}]{(0,3)}{p4}
\nput{180}{p4}{\Large \((0,1)\)}

\State[?]{(10,0)}{q1}
\nput{180}{q1}{\Large \((1,1) \rightarrow (\textcolor{blue}{0}, \textcolor{red}{0})\)}

\State[?]{(13,0)}{q2} \nput{0}{q2}{\Large \((\textcolor{blue}{1},\textcolor{red}{0})
\leftarrow \lbrace(1,0),(0,1)\rbrace\)}

\State[?]{(13,3)}{q3}
\nput{0}{q3}{\Large \((\textcolor{blue}{1},\textcolor{red}{1}) \leftarrow (0,0)\)}

%% Mapping

\EdgeLineDouble\VCurveL{angleA=-30,angleB=135,ncurv=0.5}{p3}{q1}{}
\EdgeBorder
\EdgeLineDouble\ArcL{p1}{q3}{}
\EdgeLineDouble\VCurveL{angleA=-30,angleB=135,ncurv=0.5}{p4}{q2}{}
\EdgeLineDouble\ArcL{p2}{q2}{}
\EdgeBorderOff

\end{VCPicture}
}
\endTeXtoEPS

\end{document}
