%%-*-latex-*-

\documentclass{article}

\usepackage{vaucanson-g}
\usepackage{pst-eps}

\begin{document}

\TeXtoEPS
% Settings
%
%\ShowGrid
\SetEdgeLabelColor{blue}
\SetEdgeLabelScale{1.2}

% Automaton
%
\VCDraw{%
\begin{VCPicture}{(0,0)(5,4)}

%% Input layer
%
\VSState{(0,3)}{x}
\nput{180}{x}{\huge \(x\)}

\VSState{(0,1)}{y}
\nput{180}{y}{\huge \(y\)}

%% Hidden layer
%
\StateVar[\theta=-1.5]{(3,4)}{H1}
\StateVar[\theta=-0.5]{(3,0)}{H2}

%% Output layer
%
\StateVar[\theta=0.5]{(6,2)}{Out}
\Final{Out}

%% From input to hidden layer
%
\EdgeL[0.6]{y}{H1}{-1}
\EdgeR{y}{H2}{-1}
\EdgeBorder
\EdgeL{x}{H1}{-1}
\EdgeR[0.6]{x}{H2}{-1}
\EdgeBorderOff

%% From hidden layer to output
%
\EdgeL{H1}{Out}{1}
\EdgeR{H2}{Out}{-1}

% Output signal
%
% Note: It is not possible to use the ``interface'' of Vaucanson-g to
% put something pointed by the ending edge of a final state. But, the
% implementation shows that there is an invisible vertex, whose name
% is the catenation of the finale state label (here `Out') and the
% direction of the out-going egde  (here `e', standing for
% `east'). Then we can short-circuit Vaucanson-g and call the PSTricks
% macro `nput'.
%
\nput{0}{Oute}{\huge \(x \oplus y\)}
\end{VCPicture}
}
\endTeXtoEPS

\end{document}
