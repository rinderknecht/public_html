%%-*-latex-*-

% ------------------------------------------------------------------------
%
\begin{frame}[containsverbatim]
\frametitle{Declarative versus procedural meaning}

There are two ways of understanding a \Prolog program.

\bigskip

For example, consider the abstract query

\bigskip

{\small \(\cal P\) \verb|:-| \(\cal Q\), \(\cal R\).}

\bigskip

where \(\cal P\), \(\cal Q\) and \(\cal R\) are objects.

\bigskip

This clause can be read in two ways:
\begin{enumerate}

  \item \(\cal P\) is true if \(\cal Q\) and \(\cal R\) are true.

  \item To prove \(\cal P\), \emph{first} prove \(\cal Q\) and
    \emph{next} prove \(\cal R\).

\end{enumerate}

\end{frame}

% ------------------------------------------------------------------------
%
\begin{frame}
\frametitle{Declarative versus procedural meanings (cont)}

Note the difference in the wording and the ordering.

\bigskip

In the first case, we speak about \textbf{truth} and the order in
which the truth values are obtained is actually not completely
defined. For example, in this context, ``\(\cal Q\) and \(\cal R\) are
true'' is equivalent to ``\(\cal Q\) and \(\cal R\) are true'',
because \(\cal Q\) must be true and, separately, \(\cal R\) too.

\bigskip

In the second case, we speak about the \textbf{process} of obtaining
the truth values in details. In this context, ``\(\cal P\) and \(\cal
Q\) are true'' may \textbf{not} be equivalent to ``\(\cal Q\) and
\(\cal P\) are true'' because one may be more efficient (e.g. if
\(\cal P\) is false, there is no need to prove \(\cal Q\)) or one may
not terminate (e.g. if \({\cal P} = {\cal Q}\)).

\end{frame}

%% % ------------------------------------------------------------------------
%% %
%% \begin{frame}
%% \frametitle{Declarative versus procedural meanings / 
%%        Conjunction and disjunction}

%% A clause of the kind 
%% {\small
%% \begin{semiverbatim}
%% \(\cal P\) :- \(\cal Q\), \(\cal R\).
%% \end{semiverbatim}
%% }
%% is called a \textbf{conjunction}.

%% Dually, there are clause which are a \textbf{disjunction}:
%% {\small
%% \begin{semiverbatim}
%% \(\cal P\) :- \(\cal Q\); \(\cal R\).
%% \end{semiverbatim}
%% }
%% read: ``\(\cal P\) is true if \(\cal Q\) is true \emph{or} \(\cal R\)
%% is true'' and equivalent to
%% {\small
%% \begin{semiverbatim}
%% \(\cal P\) :- \(\cal Q\).
%% \(\cal P\) :- \(\cal R\).
%% \end{semiverbatim}
%% }

%% \end{frame}

% ------------------------------------------------------------------------
%
\begin{frame}
\frametitle{Declarative meaning}

Informally, the declarative meaning of a \Prolog program is as
follows.

{\it
\begin{quote}
A goal \(\cal G\) is true (or logically follows from the program) if
\begin{enumerate}

  \item there is a clause \(C\) in the program

  \item of which an instance \(I\) can be deduced such that
    \begin{enumerate}

      \item the head of \(I\) is identical to \(\cal G\),

      \item all the goals of the body of \(I\) are true.

    \end{enumerate}

\end{enumerate}

\end{quote}
}
Note that it is not said how to find \(C\) and \(I\), and no ordering
of the goals is imposed (they just all must be true).

\end{frame}

% ------------------------------------------------------------------------
%
\begin{frame}
\frametitle{Procedural meaning}

{\it
Given a list \(\cal G\) of goals, a list \(\cal C\) of clauses and
the identity substitution \(\sigma\),
\begin{enumerate}
 
  \item if \(\cal G\) is empty, then end with \(\sigma\) and
    \textbf{success};

  \item if \(\cal C\) is empty then \textbf{fail}; let \(G_1\) be the
    first goal and \(C_1\) the first clause;

  \item let \(\overline{C}_1\) be an instance of \(C_1\) containing
    no variable in common with \({\cal G}\);

  \item if the head of \(\overline{C}_1\) does not match the head of
    \(G_1\), restart with the remaining clauses;
    
  \item let \(\sigma'\) be the resulting substitution, \({\cal G}'\)
    the remaining goals and \(\cal B\) the goals in the body of
    \(\overline{C}_1\); restart with the list of goals made of
    \(\sigma'({\cal G}')\) and \(\sigma'({\cal B})\) and substitution
    \(\sigma' \circ \sigma\);

  \item if it failed, restart with \(\cal G\) and the remaining
    clauses in \(\cal C\) \textnormal{(backtracking)}.

\end{enumerate}
}

\end{frame}

% ------------------------------------------------------------------------
%
\begin{frame}[containsverbatim]
\frametitle{Procedural meaning/Example}

The declarative meaning can be seen as an \emph{abstraction} of the
procedural meaning, i.e. it hides certain aspects of it. Consider

\PrologIn{animal}

{\small
\begin{verbatim}
?- dark(X), big(X).   % What is dark and big?
\end{verbatim}
}

\end{frame}

% ------------------------------------------------------------------------
%
\begin{frame}
\frametitle{Procedural meaning/Example}

The list of goals is \({\cal G} = (G_1, G_2)\), where \(G_1 =
\texttt{dark(X)}\) and \(G_2 = \texttt{big(X)}\).

\bigskip

There is no match between \(G_1\) and the facts, until clause 7.

\bigskip

Clause 7 has no variable in common with \(G_1\).

\bigskip

The matching between the head of clause 7, \texttt{dark(Z)}, and
\(G_1\) succeeds with substitution \(\sigma' = \{\texttt{X = } \alpha,
\texttt{Z = } \alpha\}\).

\bigskip

Let us start again with the list of goals
\((\texttt{black(}\alpha\texttt{)}, \texttt{big(}\alpha\texttt{)})\),
and the substitution \(\sigma'\). 

\bigskip

Actually, it is enough to restrict \(\sigma'\) to the variables in
\(G_1\), so let us take instead \[\sigma' = \{\texttt{X = }
\alpha\}\]

\end{frame}

% ------------------------------------------------------------------------
%
\begin{frame}
\frametitle{Procedural meaning/Example (cont)}

Now the list of goals is \({\cal G} = (G_1, G_2)\), where \(G_1 =
\texttt{black(}\alpha\texttt{)}\) and \(G_2 =
\texttt{big(}\alpha\texttt{)}\), and \(\sigma = \{\texttt{X = }
\alpha\}\).

\bigskip

The first clause whose head matches
\(\texttt{black(}\alpha\texttt{)}\) is clause 5, which contains no
variable. The substitution resulting of the matching is \(\sigma' =
\{\alpha \texttt{ = cat}\}\). 

\bigskip

Let us start again with the list of goals reduced to
\(\texttt{big(cat)}\) (since clause 5 is a fact, so has no body) and
substitution \(\sigma' \circ \sigma = \{\texttt{X = cat}\}\).

\bigskip

But no clause has a head matching \texttt{big(cat)}, so let us
backtrack and try another match below clause 5. 

\bigskip

There is none. 

\end{frame}

% ------------------------------------------------------------------------
%
\begin{frame}
\frametitle{Procedural meaning/Example (cont)}

So we must backtrack further and reconsider the list of
goals is \({\cal G} = (G_1, G_2)\), where \(G_1 = \texttt{dark(X)}\)
and \(G_2 = \texttt{big(X)}\), and the identity substitution \(\sigma\).

\bigskip

The clause after clause 7, whose head matches \(G_1\) is clause 8. It
does not have common variables with \(G_1\). The resulting
substitution is \(\sigma' = \{\texttt{X = } \beta\}\).

\bigskip

Let us start again with the list of goals
\((\texttt{brown(}\beta\texttt{)}, \texttt{big(}\beta\texttt{)})\)
and the substitution \[\sigma' \circ \sigma = \sigma'\]

\end{frame}

% ------------------------------------------------------------------------
%
\begin{frame}
\frametitle{Procedural meaning/Example (cont)}

Now the list of goals is \({\cal G} = (G_1, G_2)\), where \(G_1 =
\texttt{brown(}\beta\texttt{)}\) and \(G_2 =
\texttt{big(}\beta\texttt{)}\), and \(\sigma = \{\texttt{X = }
\beta\}\).

\bigskip

The first clause whose head matches \(G_1\) is clause 4, which
contains not variable (it is a fact). 

\bigskip

The matching leads to substitution \(\sigma' = \{\beta \texttt{ =
  bear}\}\).

\bigskip

Let us start again with the list of goals
\texttt{big(bear)} and the substitution 
\[
\sigma' \circ \sigma = \{\texttt{X = bear}\}
\]

\end{frame}

% ------------------------------------------------------------------------
%
\begin{frame}[containsverbatim]
\frametitle{Procedural meaning/Example (cont)}

Now the the list of goals is \({\cal G} = (G_1)\), where \(G_1 =
\texttt{big(bear)}\), and the substitution \[\sigma = \{\texttt{X =
  bear}\}\]

\bigskip

The first clause whose head matches \(G_1\) is clause 1. It has no
variable. The resulting substitution is the identity. Since it is a
fact, it is proven and there is no new (sub-)goals.

\bigskip

This means that the interpreters ends with the positive result
{\small
\begin{verbatim}
?- dark(X), big(X).   % What is dark and big?
X = bear
Yes
\end{verbatim}
}

\end{frame}

% ------------------------------------------------------------------------
%
\begin{frame}
\frametitle{Procedural meaning/Example (cont)}

\begin{center}
\includegraphics[scale=0.95,bb=-2 550 287 721,clip=false]{trace_bear}
\end{center}

\end{frame}

% ------------------------------------------------------------------------
%
\begin{frame}
\frametitle{Procedural meaning/Example (cont)}

The corresponding proof tree is
\begin{mathpar}
\inferrule
{
 \inferrule*[Left=\(\langle{8\textnormal{,}\{\texttt{Z = bear}\}}\rangle\)]
 {
  \inferrule*[lab=\(\langle{4}\rangle\)]{}{\texttt{brown(bear)}}
 }
 {\texttt{dark(bear)}}
 \and
 \inferrule*[lab=\(\langle{1}\rangle\)]{}{\texttt{big(bear)}}
}
{\texttt{dark(bear),big(bear)}}
\end{mathpar}

\end{frame}

% ------------------------------------------------------------------------
%
\begin{frame}[containsverbatim]
\frametitle{Declarative versus procedural meaning (resumed)}

Given a \Prolog program, it is possible to provide several
programs which have the same declarative meaning but potentially
different procedural meanings by playing on the order of the clauses
and the order of the goals in the bodies. For example, consider again
the relation \texttt{ancestor} page~\pageref{ancestor}:
{\small
\begin{verbatim}
ancestor(X,Y) :- parent(X,Y).                     % Version 1
ancestor(X,Y) :- parent(X,Z), ancestor(Z,Y).
\end{verbatim}
}
{\small
\begin{verbatim}
ancestor(X,Y) :- parent(X,Z), ancestor(Z,Y).      % Version 2
ancestor(X,Y) :- parent(X,Y).
\end{verbatim}
}
{\small
\begin{verbatim}
ancestor(X,Y) :- parent(X,Y).                     % Version 3
ancestor(X,Y) :- ancestor(Z,Y), parent(X,Z).
\end{verbatim}
}
{\small
\begin{verbatim}
ancestor(X,Y) :- ancestor(Z,Y), parent(X,Z).      % Version 4
ancestor(X,Y) :- parent(X,Y).
\end{verbatim}
}
\end{frame}

% ------------------------------------------------------------------------
%
\begin{frame}[containsverbatim]
\frametitle{Declarative versus procedural meaning (resumed)}

Let \(x\) and \(y\) be some data object. Given a query
{\small
\begin{semiverbatim}
?- ancestor(\(x\), \(y\)).
\end{semiverbatim}
}
The procedural behaviours of the different versions are as follows:
\begin{itemize}

  \item versions 1 and 2 always allow an answer to be found;

  \item versions 3 and 4 always loop forever.
\end{itemize}
Consider the examples:
{\small
\begin{verbatim}
?- ancestor(liz,jim).
?- ancestor(tom,pat).
\end{verbatim}
}

\end{frame}

% ------------------------------------------------------------------------
%
\begin{frame}[containsverbatim]
\frametitle{Declarative versus procedural meaning (resumed)}

What about the following variations?
{\small
\begin{verbatim}
ancestor(X,Y) :- parent(X,Y).                % Version 3bis
ancestor(X,Y) :- ancestor(X,Z), parent(Z,Y).
\end{verbatim}
}
{\small
\begin{verbatim}
ancestor(X,Y) :- ancestor(X,Z), parent(Z,Y). % Version 4bis
ancestor(X,Y) :- parent(X,Y).
\end{verbatim}
}

\end{frame}
