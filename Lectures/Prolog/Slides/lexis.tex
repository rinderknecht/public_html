%-*-latex-*-

% ------------------------------------------------------------------------
%
\begin{frame}[containsverbatim]
\frametitle{Data objects/Atoms}

Some objects are only identified by their name, called an
\textbf{atom}. Thus, it is correct to say that, in such a case, an
object \emph{is} an atom. For example, we saw the atom \texttt{bob}.

\bigskip

An atom starts with a lower-case letter which can be followed by a
string of characters out of lower-case letters, upper-case letters,
digits and the underscore character (`\_').

\bigskip

For example, the following are valid atoms:
{\small
\begin{verbatim}
anna         alpha_beta_proc
x25          call_Java
x_25         x_
x_25AB       x____y
\end{verbatim}
}

\end{frame}

% ------------------------------------------------------------------------
%
\begin{frame}[containsverbatim]
\frametitle{Data objects/Numbers}

\textbf{Numbers} in \Prolog include integer numbers and floating-point
numbers. The syntax of integer is as expected, for example
{\small
\begin{verbatim}
1    1234    0    -97
\end{verbatim}
}
The lower and larger integers are limited by the actual \Prolog system
in use.

\bigskip

Floating-point numbers follow the usual syntax too, like
{\small
\begin{verbatim}
3.14    -0.06    100.5
\end{verbatim}
}
The general syntax is not given here because \Prolog is primarily
intended for symbolic computations, not numerical computations.

\bigskip

Atoms and numbers define the group of \textbf{constants}.

\end{frame}

% ------------------------------------------------------------------------
%
\begin{frame}[containsverbatim]
\frametitle{Data objects/Variables}

A \textbf{variable} is a name for an object, but, contrary to atoms,
variables do not define any object. So a variable can denote several
objects. They must start with an upper-case letter or an underscore
and may be followed by any number of letters, digit and underscores,
in any order. For example, the following are valid variables:
{\small
\begin{verbatim}
X    Obj_List    _23    Object2    Result    ObjList    _x23
\end{verbatim}
}
If a variable appears only once in the body of a clause and not in the
head, it is an \textbf{unknown variable} and can be replaced by an
underscore. For example
{\small
\begin{verbatim}
has_a_child(X) :- parent(X,Y).
\end{verbatim}
}
can be replaced by the equivalent
{\small
\begin{verbatim}
has_a_child(X) :- parent(X,_).
\end{verbatim}
}

\end{frame}

% ------------------------------------------------------------------------
%
\begin{frame}[containsverbatim]
\frametitle{Data objects/Variables (cont)}

Several underscores can occur in a body. In this case, each of them
should be considered an absolutely unique, despite unknown, variable.

\bigskip

For example, consider
{\small
\begin{verbatim}
someone_has_a_child :- parent(_,_).
\end{verbatim}
}
Each underscore could be replaced by a unique variable, like
{\small
\begin{verbatim}
someone_has_a_child :- parent(X,Y).
\end{verbatim}
}
But certainly \textbf{not}
{\small
\begin{verbatim}
someone_has_a_child :- parent(X,X).
\end{verbatim}
}

\end{frame}

% ------------------------------------------------------------------------
%
\begin{frame}[containsverbatim]
\frametitle{Data objects/Variables and lexical scopes}

If an unknown variable appears in a query, then its value, found by
the interpreter, is not displayed in the result.

\bigskip

For example, if one wants to know who are the people who have
children, the \Prolog query is
{\small
\begin{verbatim}
?- parent(X,_).
\end{verbatim}
}
and only all the possible values of \texttt{X} that satisfy the query
will be displayed.

\bigskip

Given an occurrence of a variable, the part of the program where
this variable is usable is called the \textbf{scope}. 

\end{frame}

% ------------------------------------------------------------------------
%
\begin{frame}[containsverbatim]
\frametitle{Data objects/Variables and lexical scopes (cont)}

\Prolog uses \textbf{lexical scoping}, which means
\begin{itemize}
  
  \item the same variable always represents the same object inside a clause;

  \item the same variable in two different clauses represent different
    objects, in general.

\end{itemize}
For example, in clause
{\small
\begin{verbatim}
ancestor(X,Y) :- parent(X,Y)
\end{verbatim}
}
the tow occurrences of variable \texttt{X} represent the same object.
But, if we also have
{\small
\begin{verbatim}
ancestor(X,Y) :- parent(X,Z), ancestor(Z,Y).
\end{verbatim}
}
the occurrences of \texttt{X} denote, in general, objects different
from the \texttt{X} in the previous clause.

\end{frame}

% ------------------------------------------------------------------------
%
\begin{frame}[containsverbatim]
\frametitle{Data objects/Functors and structures}

Structured objects, or \textbf{structures}, are objects that are
composed of several objects. These sub-objects are boxed together by
means of a \textbf{functor}. For example
{\small
\begin{verbatim}
date(9,july,2006)
\end{verbatim}
}
is a structure composed, in order, of the integer constant \texttt{1},
the atom \texttt{july}, the number \texttt{2006} and tied together by
the functor \texttt{date}. The sub-objects are called
\textbf{arguments} and the number of arguments is the \textbf{arity}.

\bigskip

The syntax of functor is the same as the atoms. Functors are used to
define facts (i.e. instances of a relation). For example,
{\small
\begin{verbatim}
female(pam).
\end{verbatim}
}

\end{frame}

% ------------------------------------------------------------------------
%
\begin{frame}
\frametitle{Data objects/Functors and structures (cont)}
\label{structures_as_trees}

Just as it is possible to make a graphical representation of a
successful run of the \Prolog interpreter (see
page~\pageref{proof_trees}), it is possible to draw a tree which
represents a \Prolog structure.

\bigskip

The idea here is to map a functor to a node whose label is the
functor, and its arguments to sub-trees, recursively in the same
order. The consequence is that the component objects that are not
structures are mapped to the leaves. For example
\begin{center}
\includegraphics[bb=71 683 161 721]{struct_tree}
\hspace{2cm}
\includegraphics[bb=71 657 132 721]{expr_tree}
\end{center}

\end{frame}

% ------------------------------------------------------------------------
%
\begin{frame}
\frametitle{Data objects/Summary}

It is also handy to represent all the kinds of \Prolog data objects we
reviewed until now in a tree. The idea consists in mapping a set of
lexical elements to a node whose label is this set name, and the
subsets are mapped to subtrees.
\begin{center}
\includegraphics[bb=71 597 237 721]{lexicon_tree}
\end{center}

\end{frame}
