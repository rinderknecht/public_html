%-*-latex-*-
 
% ------------------------------------------------------------------------
%
\begin{frame}
\frametitle{How \Prolog answers queries}

To answer a query, the \Prolog interpreter tries to satisfy all the
goals.

\bigskip

Satisfying a goal means proving that a goal logically follows from the
facts and rules in the program.

\bigskip

If the query contains variables, the interpreter must find particular
objects in place of the the variables that entail the goal.

\bigskip

If it cannot prove the goal, the interpreter answers \texttt{No}.

\end{frame}

% ------------------------------------------------------------------------
%
\begin{frame}[containsverbatim]
\frametitle{How \Prolog answers queries (cont)}

For example, consider the famous syllogism about the philosopher
Socrates. Given
\begin{quote}
All men are fallible [a rule],
Socrates is a man [a fact].
\end{quote}
a logical consequence is that
\begin{quote}
Socrates is fallible.
\end{quote}
In \Prolog, this is written
{\small
\begin{verbatim}
fallible(X) :- man(X).
man(socrates).
\end{verbatim}
}
Then we have
{\small
\begin{verbatim}
?- fallible(socrates).
Yes
\end{verbatim}
}

\end{frame}

% ------------------------------------------------------------------------
%
\begin{frame}[containsverbatim]
\frametitle{How \Prolog answers queries (cont)}

This query was answered by the interpreter by first looking up some
fact that would match the goal \texttt{fallible(socrates)}. 

\bigskip

Since there is none, the interpreter looked for rules such that the
goal is an instance of the head, i.e. such that the goal can be formed
by replacing variables in the head by some object.

\bigskip

If we set \texttt{X = socrates}, then the rule
{\small
\begin{verbatim}
fallible(X) :- man(X).
\end{verbatim}
}
is instantiated into
{\small
\begin{verbatim}
fallible(socrates) :- man(socrates).
\end{verbatim}
}
whose head matches exactly the query. 

\end{frame}

% ------------------------------------------------------------------------
%
\begin{frame}[containsverbatim]
\frametitle{How \Prolog answers queries (cont)}

Now the interpreter tries to prove the body, i.e. the sub-goal
{\small
\begin{verbatim}
?- man(socrates).
\end{verbatim}
}
just as it tried to prove the initial query.

\bigskip

It searches first for a fact which would be the sub-goal
\texttt{man(socrates)}, and, indeed, there is such a fact in the
program.

\bigskip

Therefore the sub-goal is true, so is the goal and the query is
positively answered.

\end{frame}

% ------------------------------------------------------------------------
%
\begin{frame}[containsverbatim]
\frametitle{How \Prolog answers queries (cont)}

Consider a query about the family tree page~\pageref{family_tree} like
{\small
\begin{verbatim}
?- ancestor(tom,pat).
\end{verbatim}
}
Let us recall the definition
{\small
\begin{verbatim}
ancestor(X,Y) :- parent(X,Y).                % Rule [anc1]
ancestor(X,Y) :- parent(X,Z), ancestor(Z,Y). % Rule [anc2]
\end{verbatim}
} 
where what follows a \texttt{\%} until the end of the line is a commentary.

\bigskip

First, the interpreter tries to instantiate the first rule,
\texttt{[anc1]}, in such a way that the instance's head matches the
goal. This can be achieved by letting \texttt{X=tom} and
\texttt{Y=pat}. The instantiated rule is {\small
\begin{verbatim}
ancestor(tom,pat) :- parent(tom,pat). % Instance of [anc1]
\end{verbatim}
}

\end{frame}

% ------------------------------------------------------------------------
%
\begin{frame}[containsverbatim]
\frametitle{How \Prolog answers queries (cont)}

Next, the interpreter tries to prove the body of the rule's instance,
i.e.
{\small
\begin{verbatim}
?- parent(tom,pat).
\end{verbatim}
}
It searches among the facts defining the \texttt{parent} relation but
finds not match. Since there is no rule for \texttt{parent}, the
interpreter fails and \texttt{parent(tom,pat)} is false.

\bigskip

Hence, \texttt{ancestor(tom,pat)} cannot be proven using rule
\texttt{[anc1]}. 

\bigskip

Before giving up, the interpreter tries again with the last remaining
rule, \texttt{[anc2]}. The variable bindings are the same as before,
and the rule instance is 
{\small
\begin{verbatim}
ancestor(tom,pat) :- parent(tom,Z),
                     ancestor(Z,pat). % Instance of [anc2]
\end{verbatim}
}

\end{frame}

% ------------------------------------------------------------------------
%
\begin{frame}[containsverbatim]
\frametitle{How \Prolog answers queries (cont)}

First, the interpreter tries to prove the sub-goal
{\small
\begin{verbatim}
?- parent(tom,Z).
\end{verbatim}
}
It searches again the database defining \texttt{parent} and finds two
matches: \texttt{Z=bob} and \texttt{Z=liz}.

\bigskip

For each binding of \texttt{Z}, the interpreter substitutes \texttt{Z}
by the associated object into the second sub-goal and tries to prove
it. First, it gets to prove
{\small
\begin{verbatim}
?- ancestor(bob,pat).
\end{verbatim}
}

\end{frame}

% ------------------------------------------------------------------------
%
\begin{frame}[containsverbatim]
\frametitle{How \Prolog answers queries (cont)}

The process for proving this goal is the same as before. Rule
\texttt{[anc1]} is considered first. The variable binding
\texttt{X=bob} and \texttt{Y=pat} leads to the following instance of
\texttt{[anc1]}:
{\small
\begin{verbatim}
ancestor(bob,pat) :- parent(bob,pat). % Instance of [anc1]
\end{verbatim}
}
whose head matches the current sub-goal. Now, the interpreter tries to
prove
{\small
\begin{verbatim}
?- parent(bob,pat).
\end{verbatim}
} 
It searches the facts about the \texttt{parent} relation and finds a
match. Therefore the sub-goal \texttt{ancestor(bob,pat)} is true, and,
since
{\small
\begin{verbatim}
ancestor(tom,pat) :- parent(tom,bob), ancestor(bob,pat).
\end{verbatim}
}
it proves the initial goal \texttt{ancestor(tom,pat)}.

\end{frame}

% ------------------------------------------------------------------------
%
\begin{frame}
\frametitle{How \Prolog answers queries (cont)}

The execution is over, even if we left suspended the binding
\texttt{Z=liz}, and in spite that \Prolog interpreters always offer
the possibility to find \emph{all the solutions}.

\bigskip

The reason is that the initial goal contained no variable, so the
interpreter will try to prove it only once, if there is at least one
proof.

\bigskip

The technique that consists, when finding that a goal is false, to go
back in history and try to prove an alternative goal is called
\textbf{backtracking}.

\bigskip

Backtracking is also used in case of success but the user wants more
solutions, if any.

\end{frame}

% ------------------------------------------------------------------------
%
\begin{frame}[containsverbatim]
\frametitle{How \Prolog answers queries (cont)}

Let us imagine now what would have happened if the interpreter had
chosen to try the binding \texttt{Z=liz}, before \texttt{Z=bob}.

\bigskip

So it tries to prove
{\small
\begin{verbatim}
?- ancestor(liz,pat).
\end{verbatim}
}
It uses the same strategy, and instantiate rule \texttt{[anc1]} with
the bindings \texttt{X=liz} and \texttt{Y=pat}:
{\small
\begin{verbatim}
ancestor(liz,pat) :- parent(liz,pat).     % Instance of [anc1]
\end{verbatim}
}
Since there is no fact \texttt{parent(liz,pat)}, it fails and
backtracks.

\end{frame}

% ------------------------------------------------------------------------
%
\begin{frame}[containsverbatim]
\frametitle{How \Prolog answers queries (cont)}

It tries now with rule \texttt{[anc2]}, with the same variable
bindings: 
{\small
\begin{verbatim}
ancestor(liz,pat) :- parent(liz,Z), ancestor(Z,pat).
\end{verbatim}
}
It searches all the facts of the shape \texttt{parent(liz,Z)} and
finds none. Therefore it is useless to try to prove the second
sub-goal \texttt{parent(Z,pat)}, because the conjunction of false and
any other boolean value is always false. In other words, for all \(x\),
\[\texttt{false} \, \wedge \, x = x\]

\bigskip

Therefore, the interpreter backtracks further, because the binding
\texttt{Z=liz} only leads to falsity, and then tries to prove the
query with \texttt{Z=bob}, as we did in the first presentation.

\end{frame}

% ------------------------------------------------------------------------
%
\begin{frame}
\frametitle{How \Prolog answers queries/Proof trees}

\label{proof_trees}

There is a graphical representation of proofs that helps a lot to
understand how the \Prolog interpreter works. It is called a
\textbf{proof tree}.

\bigskip

The idea consists in making a tree whose root is the goal to prove and
the sub-trees correspond to the proofs of the sub-goals. 

\bigskip

In other words, the inner nodes are made from rule instances and the
leaves consist of facts.

\end{frame}

% ------------------------------------------------------------------------
%
\begin{frame}[containsverbatim]
\frametitle{How \Prolog answers queries/Proof trees (cont)}

For example, the successful proof of 
{\small
\begin{verbatim}
?- ancestor(tom,pat).
\end{verbatim}
} 
can be graphically represented as the following proof tree.
\begin{mathpar}
\inferrule*[right=anc\(_2\)]
{\texttt{parent(tom,bob)}\\
 \inferrule*[right=anc\(_1\)]
 {\texttt{parent(bob,pat)}}
 {\texttt{ancestor(bob,pat)}}
}
{\texttt{ancestor(tom,pat)}}
\end{mathpar}
Note that all the leaves, \texttt{parent(tom,bob)} and
\texttt{parent(bob,pat)}, are facts; the name of the instantiated rule
appears on the right of each inner node (horizontal line).

\end{frame}

% ------------------------------------------------------------------------
%
\begin{frame}
\frametitle{How \Prolog answers queries/Proof trees (cont)}

The \Prolog interpreter starts from the root and tries to grow
branches that all end in leaves which are facts. If not, it backtracks
to try another rule instance, and if none matches the knot, it fails.

\bigskip

For instance, we saw that it tried first
\begin{mathpar}
\inferrule*[right=anc\(_1\)]
{\texttt{parent(tom,pat)}}
{\texttt{ancestor(tom,pat)}}
\end{mathpar}
but the leaf was not a fact, so it tried next
\begin{mathpar}
\inferrule*[right=anc\(_2\)]
{\texttt{parent(tom,bob)}\\
 \texttt{ancestor(bob,pat)}
}
{\texttt{ancestor(tom,pat)}}
\end{mathpar}

\end{frame}
