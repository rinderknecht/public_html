%%-*-latex-*-

% ------------------------------------------------------------------------
%
\begin{frame}[containsverbatim]
\frametitle{The eight queens problem}

Let us now use \Prolog to solve a more difficult kind of problem. One
of these famous problems is \textbf{the eight queens problem}.

\bigskip

It consists in placing on a (European) chess board eight queens such
that they do not attack each other.

\bigskip

We would like to define a predicate \texttt{solution} such that
{\small
\begin{verbatim}
?- solution(Pos).
\end{verbatim}
}
returns a substitution for \texttt{Pos} which corresponds to a
chess board position satisfying the problem's constraints.

\end{frame}

% ------------------------------------------------------------------------
%
\begin{frame}
\frametitle{The eight queens problem (cont)}

\begin{figure}
\centering
\includegraphics[scale=0.85,bb=212 480 380 650]{queens}
\caption{A solution to the eight queens problem.}
\end{figure}

\end{frame}

% ------------------------------------------------------------------------
%
\begin{frame}[containsverbatim]
\frametitle{The eight queens problem (cont)}

First, we have to choose a representation for the board positions. One
possibility is to model a square by two coordinates, the leftmost,
down-most square being \((1,1)\) and the rightmost, uppermost
\((8,8)\).

\bigskip

The example in the previous slide can be modeled by the list of
queens
{\small
\begin{verbatim}
[.(1,4),.(2,2),.(3,7),.(4,3),.(5,6),.(6,8),.(7,5),.(8,1)]
\end{verbatim}
}
Keeping this idea, we choose a template solution  of the form
{\footnotesize
\begin{verbatim}
template([.(1,Y1),.(2,Y2),.(3,Y3),.(4,Y4),
          .(5,Y5),.(6,Y6),.(7,Y7),.(8,Y8)]).
\end{verbatim}
}
because there must be a queen on each column.

\end{frame}

% ------------------------------------------------------------------------
%
\begin{frame}
\frametitle{The eight queens problem (cont)}

There are two cases:
\begin{enumerate}

  \item the list of queens is empty: the empty list is certainly a
    solution since there is no attack;

  \item the list of queens is not empty: then it has the shape
    \texttt{\small [.(X,Y) | Others]}, that is, the first queen is on
    the square \texttt{.(X,Y)} and the others in the sub-list
    \texttt{Others}. If this is a solution, then the following
    constraints must hold.
    \begin{enumerate}

      \item there must be no attack between the queens in
        \texttt{Others}, i.e. \texttt{Others} must be a solution;

      \item \texttt{X} and \texttt{Y} must be integers between 1 and
        8;

      \item a queen at square \texttt{.(X,Y)} must not attack any of
        the queens in the list \texttt{Others}.

    \end{enumerate}
    
\end{enumerate}

\end{frame}

% ------------------------------------------------------------------------
%
\begin{frame}[containsverbatim]
\frametitle{The eight queens problem (cont)}

This is written in \Prolog
{\small
\begin{verbatim}
solution([]).

solution([.(X,Y) | Others]) :-
  solution(Others),
  member(Y, [1,2,3,4,5,6,7,8]),
  no_attack(.(X,Y), Others).

member(Item, [Item | _]).

member(Item, [_ | Tail]) :- member(Item, Tail).
\end{verbatim}
}
It remains to define the relation \texttt{no\_attack}.

\end{frame}

% ------------------------------------------------------------------------
%
\begin{frame}
\frametitle{The eight queens problem (cont)}

Given \texttt{no\_attack(Q, Qlist)}, there are two cases.
\begin{enumerate}

  \item if the list of queens \texttt{Qlist} is empty, then the
    relationship is true because there is no queen to attack or to be
    attacked by;

  \item if the list \texttt{Qlist} is not empty, it must be of the
    shape \texttt{[Q1 | Qsublist]}, with the following conditions
    holding:
    \begin{enumerate}

      \item the queen at \texttt{Q} must not attack the queen at
        \texttt{Q1},

      \item the queen at \texttt{Q} must not attack the queens in
        \texttt{Qsublist}.

    \end{enumerate}

\end{enumerate}

\end{frame}

% ------------------------------------------------------------------------
%
\begin{frame}[containsverbatim]
\frametitle{The eight queens problem (cont)}

A queen does not attack another queen if they are on different
columns, rows and diagonals. 

\bigskip

Finally:
{\small
\begin{verbatim}
no_attack(_, []).

no_attack(.(X,Y), [.(X1,Y1) | Others]) :-
  Y =\= Y1,
  Y1 - Y =\= X1 - X,
  Y1 - Y =\= X - X1,
  no_attack(.(X,Y), Others).
\end{verbatim}
}

\end{frame}

% ------------------------------------------------------------------------
%
\begin{frame}[containsverbatim]
\frametitle{The eight queens problem (cont)}

The query has the shape
{\small
\begin{verbatim}
?- template(S), solution(S).
\end{verbatim}
}

\bigskip

Note that 
{\small
\begin{verbatim}
?- solution(S), template(S).
\end{verbatim}
}
is \textbf{wrong}. Why?

\end{frame}
