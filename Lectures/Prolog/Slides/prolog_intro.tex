%%-*-latex-*-

% ------------------------------------------------------------------------
%
\begin{frame}[containsverbatim]
\frametitle{Defining relations with facts}

\Prolog is a programming language for symbolic (non-numeric)
computation. Its name means \textbf{programming in logic}.

\bigskip

Let us approach it as if it were a \textbf{deductive database}. First,
a database is populated by \textbf{relations}. A relation is what
links two or more objects. This way we learn something about their
relative meaning. For example, in ``Tom is the parent of Bob,'' is a
statement, the relation is parenthood and relates Tom and Bob.

\bigskip

In \Prolog, the simplest way of defining a relation is by enumerating
\textbf{facts} (statements). For example
{\small 
\begin{verbatim}
parent(tom,bob).
\end{verbatim}
}
is a fact.

\end{frame}

% ------------------------------------------------------------------------
%
\begin{frame}
\frametitle{Defining relations with facts (cont)}
\label{family_tree}


\begin{columns}
  \column{0.5\textwidth}
    \textbf{A family tree encoded in \Prolog}
    \begin{center}
       \includegraphics[bb=60 608 185 732,scale=0.9]{parenthood}
    \end{center}
  \column{0.5\textwidth}
    \PrologIn{parent}
\end{columns}

\end{frame}

% ------------------------------------------------------------------------
%
\begin{frame}[containsverbatim]
\frametitle{Queries}

This program is composed of six \textbf{clauses}, each declaring a
fact about the \texttt{parent} relation. As we shall see later,
a clause may not be a fact.

\bigskip

We say that, for example, \texttt{parent(tom,bob)} is an
\textbf{instance} of the \texttt{parent} relation.

\bigskip

In general, a relation is defined as the set of all its instances.

\bigskip

After loading this relation, the \Prolog interpreter allows the user
to ask some questions about it. For example, the query ``Is Bob a
parent of Pat?'' is written 
{\small
\begin{verbatim}
?- parent(bob,pat).
\end{verbatim}
}
Having found that this a fact, the interpreter answers
{\small
\begin{verbatim}
Yes
\end{verbatim}
}

\end{frame}

% ------------------------------------------------------------------------
%
\begin{frame}[containsverbatim]
\frametitle{Ground queries}

Another query can be
{\small
\begin{verbatim}
?- parent(liz,pat).
\end{verbatim}
}
The interpreter answers
{\small
\begin{verbatim}
No
\end{verbatim}
}
The same answers is given to query
{\small
\begin{verbatim}
?- parent(tom,ben).
\end{verbatim}
}
because the interpreter never heard of \texttt{ben}.

\bigskip

A \textbf{query} starts with `\verb+?-+', is followed by a
\textbf{goal} and ended by a period.

\end{frame}

% ------------------------------------------------------------------------
%
\begin{frame}[containsverbatim]
\frametitle{Existential queries}

More interesting is asking ``Who are the parents of Liz?'' In this
case what we do not know is a set of persons. The usual way of
speaking about an unknown person is to name it with an arbitrary
\textbf{variable}, e.g. \texttt{X}. The corresponding \Prolog query is
then
{\small
\begin{verbatim}
?- parent(X,liz).
\end{verbatim}
}
In this case, we expect \emph{all} the parents of Liz to be found --- or
none if no instance were defined. The interpreter answers
{\small
\begin{verbatim}
X = tom
Yes
\end{verbatim}
}
A \textbf{ground query} is a query with no variables.

\end{frame}

% ------------------------------------------------------------------------
%
\begin{frame}[containsverbatim]
\frametitle{Existential queries on one variable}

If asked who are Bob's children:
{\small
\begin{verbatim}
?- parent(bob,X).
\end{verbatim}
}
we expect several answers. The interpreter gives one and wait for the
user to either type in a semi-colon for more answers or the return
key to stop:
{\small
\begin{verbatim}
X = ann ;
X = pat 
Yes
\end{verbatim}
}
If we had exhausted all the solutions, the interpreter would print
{\small
\begin{verbatim}
No
\end{verbatim}
}

\end{frame}

% ------------------------------------------------------------------------
%
\begin{frame}[containsverbatim]
\frametitle{Existential queries on two variables}

We can also query on two variables, like asking for all \texttt{X} and
\texttt{Y} such that \texttt{X} is a parent of \texttt{Y}. In other
words, find all the instances of the \texttt{parent} relation. This is
formally written as
{\small
\begin{verbatim}
?- parent(X,Y).
\end{verbatim}
}
The answer starts as
{\small
\begin{verbatim}
X = pam
Y = bob ;
X = tom
Y = bob ;
X = tom
Y = liz ;
\end{verbatim}
}

\end{frame}

% ------------------------------------------------------------------------
%
\begin{frame}[containsverbatim]
\frametitle{Conjunctive queries and shared variables}

We could ask: ``Who is the grand-parent of Jim?'' But, since, we did
not define the \texttt{grandparent} relation, we have to break this
question in two parts:
\begin{enumerate}

  \item Who is a parent of Jim? Assume that (s)he is named \texttt{Y}.

  \item Who is a parent of \texttt{Y}? Assume that (s)he is named
    \texttt{X}.

\end{enumerate}
The answer to the original question is \texttt{X}. Formally, this is
written
{\small
\begin{verbatim}
?- parent(Y,jim), parent(X,Y).
\end{verbatim}
}
The answer is (final \texttt{Yes} omitted):
{\small
\begin{verbatim}
X = bob
Y = pat
\end{verbatim}
}

\end{frame}

% ------------------------------------------------------------------------
%
\begin{frame}[containsverbatim]
\frametitle{Conjunctive queries and shared variables (cont)}

Note that the \Prolog interpreter returns the values for all the
variables involved in the query (here \texttt{X} and \texttt{Y}), even
if we are only interested in some of them (here \texttt{X}).

\bigskip

The previous query is made by tying two queries sharing a common
variable, \texttt{Y}. It is a \textbf{conjunctive query}.

\bigskip

If we change the order of composition of the two queries, the logical
meaning remains the same:
{\small
\begin{verbatim}
?- parent(X,Y), parent(Y,jim).
\end{verbatim}
}
\noindent produces the same result. Similarly, we ask: ``Who are Tom's
grand-children?'' by
{\small
\begin{verbatim}
?- parent(tom,X), parent(X,Y).
\end{verbatim}
}

\end{frame}

% ------------------------------------------------------------------------
%
\begin{frame}[containsverbatim]
\frametitle{Conjunctive queries and shared variables (cont)}

Let us ask: ``Do Ann and Pat have a common parent?'' Again, the
question can be broken down in two sub-questions:
\begin{enumerate}

  \item Who is a parent \texttt{X} of Ann?

  \item Is (the same) \texttt{X} a parent of Pat?

\end{enumerate}
In \Prolog, this conjunctive query is written
{\small
\begin{verbatim}
?- parent(X,ann), parent(X,pat).
\end{verbatim}
}
The \Prolog queries are usually called \textbf{goals}.

\end{frame}

% ------------------------------------------------------------------------
%
\begin{frame}
\frametitle{Defining relations by rules}

We can extend our example in many interesting ways. Let us first add
the information about the sex of the people. This can be easily done
by simply adding the following facts to our program:

\PrologIn{sex}

The relations introduced are \texttt{male} and \texttt{female}. These
relations only have one argument, and are called \textbf{predicates}.

\end{frame}

% ------------------------------------------------------------------------
%
\begin{frame}[containsverbatim]
\frametitle{Defining relations by rules (cont)}

Another way consists in defining a \textbf{binary relation} for the
sex, i.e. a relation between two objects, like the \texttt{parent}
relation. For example:
{\small
\begin{verbatim}
sex(pam,feminine).
sex(tom,masculine).
sex(bob,masculine).
...
\end{verbatim}
}
Now let us introduce the \textbf{inverse relation} of \texttt{parent},
we call \texttt{offspring}. We can add new facts again. For example
{\small
\begin{verbatim}
offspring(liz,tom).
\end{verbatim}
}

\end{frame}

% ------------------------------------------------------------------------
%
\begin{frame}[containsverbatim]
\frametitle{Defining relations by rules (cont)}

However, the \Prolog language offers a more compact and elegant way to
define the relation \texttt{offspring} in terms of the already defined
relation \texttt{parent}.

\bigskip

Indeed, what we want to say is: 
\begin{quote}
For all \texttt{X} and \texttt{Y},
\texttt{Y} is an offspring of \texttt{X} if \texttt{X} is a parent of
\texttt{Y}.
\end{quote}
In \Prolog this is expressed by means of a clause
{\small
\begin{verbatim}
offspring(Y,X) :- parent(X,Y).
\end{verbatim}
}
Such clauses are called \textbf{rules}.

\end{frame}

% ------------------------------------------------------------------------
%
\begin{frame}[containsverbatim]
\frametitle{Defining relations by rules (cont)}

There is an important difference between facts and rules. A fact like
{\small
\begin{verbatim}
parent(tom,liz).
\end{verbatim}
} 
is something that is unconditionally true. On the other hand, rules
specify things that are true \textbf{if} some condition is satisfied.

Rules have a \emph{condition} part (right-part of the rule) and a
\emph{conclusion} (left-hand side). The conclusion is also called the
\textbf{head} and the condition the \textbf{body}. For example
\[
\underbrace{\texttt{\small offspring(Y,X)}}_{\textrm{\normalsize head}} \;
\texttt{:-} \; \underbrace{\texttt{\small
    parent(X,Y)}}_{\textrm{\normalsize body}}\texttt{.}
\]

\end{frame}

% ------------------------------------------------------------------------
%
\begin{frame}[containsverbatim]
\frametitle{Defining relations by rules (cont)}

Let us ask now whether Liz is an offspring of Tom:
{\small
\begin{verbatim}
?- offspring(liz,tom).
\end{verbatim}
} 
Since there are no facts about offsprings in the program and there
is a rule whose head is an instance of \texttt{offspring}, the
\Prolog interpreter tries to use the latter.

The head of this rule is \texttt{offspring(Y,X)}, implicitly
for all \texttt{X} and \texttt{Y}. So \texttt{X}
can be replaced by \texttt{tom} and \texttt{Y} by
\texttt{liz}. This process is called \textbf{instantiation}. We
get a special case of the rule:
{\small
\begin{verbatim}
offspring(liz,tom) :- parent(tom,liz).
\end{verbatim}
}

\end{frame}

% ------------------------------------------------------------------------
%
\begin{frame}[containsverbatim]
\frametitle{Defining relations by rules (cont)}

Now the \Prolog interpreter tries to prove the body
\texttt{parent(tom,liz)}. This is actually a fact, so it answers
\texttt{Yes}.

\bigskip

It does not give any variable information, like \texttt{X =
  tom}, because the original goal contained no variable.

\bigskip

Let us add more twists to our program by considering a
\texttt{mother} relation. It could be informally defined by
\begin{quote}
For all \texttt{X} and \texttt{Y}, \texttt{X} is the mother of
\texttt{Y} if \texttt{X} is a parent of \texttt{Y} and \texttt{X} is a
female.
\end{quote}
This is written in \Prolog
{\small
\begin{verbatim}
mother(X,Y) :- parent(X,Y), female(X).
\end{verbatim}
}

\end{frame}

% ------------------------------------------------------------------------
%
\begin{frame}[containsverbatim]
\frametitle{Defining relations by rules (cont)}

The grand-parent relation can be defined as
{\small
\begin{verbatim}
grandparent(X,Y) :- parent(X,Z), parent(Z,Y).
\end{verbatim}
}
The interesting point here is that there is a variable, \texttt{Z},
that is shared between the two sub-goals in the body and that does not
appear in the head.

\bigskip

The interpreter, when asked a question about two persons, on one being
the grand-parent of the other, tries to find another person
(\texttt{Z}) who is the child of the first and the parent of the
second.

\end{frame}

% ------------------------------------------------------------------------
%
\begin{frame}[containsverbatim]
\frametitle{Defining relations by rules (cont)}

The sister-hood relation can be defined as
{\small
\begin{verbatim}
sister(X,Y) :- parent(Z,X), parent(Z,Y), female(X).
\end{verbatim}
}
This definition seems right but something strange happens if we query
{\small
\begin{verbatim}
?- sister(X,pat).
\end{verbatim}
}
which corresponds to the question: "Who are the sisters of Pat?" The
answers are:
{\small
\begin{verbatim}
X = ann;
X = pat
\end{verbatim}
}
Pat is her own sister!

\end{frame}

% ------------------------------------------------------------------------
%
\begin{frame}[containsverbatim]
\frametitle{Defining relations by rules (cont)}

To fix this, we need a binary relation \texttt{different} and use it
in the rule defining the relation \texttt{sister}:
{\small
\begin{verbatim}
sister(X,Y) :- parent(Z,X), parent(Z,Y),
               female(X), different(X,Y).
\end{verbatim}
}
The relation \texttt{different} can be defined by listing all the
pairs of different persons:
{\small 
\begin{verbatim}
different(bob,pat).
different(bob,ann).
...
\end{verbatim}
}
This approach does not scale. We will see later how to define a
general \texttt{different}, based on equality.

\end{frame}
